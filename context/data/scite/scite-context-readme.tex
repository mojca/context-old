\starttext

The following files are needed for the lpeg based lexer:

\starttyping
scite-ctx.lua
scite-context.properties
scite-pragma.properties
scite-ctx.properties
scite-ctx-context.properties
scite-ctx-example.properties
lexers/scite-context-lexer-tex.lua
lexers/scite-context-lexer-mps.lua
lexers/scite-context-lexer-cld.lua
lexers/scite-context-lexer.lua
lexers/context/mult-def.lua
lexers/context/mult-prm.lua
lexers/context/mult-low.lua
lexers/context/mult-mps.lua
lexers/themes/scite-context-theme.lua
\stoptyping

On windows you can copy al files including the subpaths to the path where
the scite binary lives.

If the \type {mult-*.lua} files are not in the archive but you can copy them
from the \CONTEXT\ distribution, where they live in:

\starttyping
<contextroot>/tex/texmf-context/tex/context/base
\stoptyping

Because property files can only be loaded from the same path you need to copy
the following files:

\starttyping
scite-context.properties
scite-pragma.properties
scite-ctx.properties
scite-ctx-context.properties
scite-ctx-example.properties
\stoptyping

to (on windows):

\starttyping
c:/Users/YourName
\stoptyping

Next you need to add this to:

\starttyping
import scite-context
import scite-pragma
\stoptyping

to the file:

\starttyping
SciTEUser.properties
\stoptyping

If you want to have spellchecking, you need have files with correct words on each
line. The first line of a file determines the language:

\starttyping
% language=uk
\stoptyping

In this case the following file is needed:

\starttyping
spell-uk.txt
\stoptyping

This file is searched on the the path determined by the environment variable:

\starttyping
CTXSPELLPATH
\stoptyping

\stoptext
