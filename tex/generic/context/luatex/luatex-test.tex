%D \module
%D   [       file=luatex-test,
%D        version=2009.12.01,
%D          title=\LUATEX\ Support Macros,
%D       subtitle=Simple Test File,
%D         author=Hans Hagen,
%D           date=\currentdate,
%D      copyright={PRAGMA ADE \& \CONTEXT\ Development Team}]

%D See \type {luatex-plain.tex} (or on my machine \type {luatex.tex}
%D for how to make a format.

\pdfoutput=1

% \directlua{generic_context.caches.compilemethod = "both" } % none luac dump both

\font\testa=file:lmroman10-regular                  at 12pt \testa \input tufte \par
\font\testb=file:lmroman12-regular:+liga;           at 24pt \testb effe flink fietsen \par
\font\testc=file:lmroman12-regular:mode=node;+liga; at 24pt \testc effe flink fietsen \par
\font\testd=name:lmroman10bold                      at 12pt \testd a bit bold \par

\font\oeps=cmr10

\font\oeps=[lmroman12-regular]:+liga at 30pt \oeps crap
\font\oeps=[lmroman12-regular]       at 40pt \oeps more crap

\font\cidtest=adobesongstd-light

\font\mathtest=cambria(math) {\mathtest 123}

\font\gothic=msgothic(ms-gothic) {\gothic whatever}

\font\testy=file:IranNastaliq.ttf:mode=node;script=arab;language=dflt;+calt;+ccmp;+init;+isol;+medi;+fina;+liga;+rlig;+kern;+mark;+mkmk at 14pt
\testy این یک متن نمونه است با قلم ذر که درست آمده است.

\pdfprotrudechars2 \pdfadjustspacing2

\font\testb=file:lmroman12-regular:+liga;extend=1.5         at 12pt \testb \input tufte \par
\font\testb=file:lmroman12-regular:+liga;slant=0.8          at 12pt \testb \input tufte \par
\font\testb=file:lmroman12-regular:+liga;protrusion=default at 12pt \testb \input tufte \par

\setmplibformat{plain}

\mplibcode
    beginfig(1) ;
        draw fullcircle
            scaled 10cm
            withcolor red
            withpen pencircle xscaled 4mm yscaled 2mm rotated 30 ;
    endfig ;
\endmplibcode

\font\mine=file:luatex-fonts-demo-vf-1.lua at 12pt

\mine \input tufte \par


% \font\mine=file:luatex-fonts-demo-vf-2.lua at 12pt \mine [abab] \par
% \font\mine=file:luatex-fonts-demo-vf-3.lua at 12pt \mine [abab] \par

\end
