%D \module
%D   [       file=strc-not,
%D        version=2008.10.20,
%D          title=\CONTEXT\ Structure Macros,
%D       subtitle=Note Handling,
%D         author=Hans Hagen,
%D           date=\currentdate,
%D      copyright=PRAGMA-ADE / Hans Hagen]
%C
%C This module is part of the \CONTEXT\ macro||package and is
%C therefore copyrighted by \PRAGMA. See mreadme.pdf for
%C details.

\writestatus{loading}{ConTeXt Structure Macros / Note Handling}

\registerctxluafile{strc-not}{1.001}

\unprotect

% obsolete

\let\autopostponenotes\relax

% removed:
%
% \pushsomestates
%
% core-ins -> obsolete
%
% saveinsertiondata
% restoreinsertiondata
% saveinsertionbox
% eraseinsertionbackup
% restoreinsertionbackup
%
% \def\doprocessnotescs#1#2% #1 == \cs that takes arg
%   {\def\currentnote{#2}\@EA#1\csname\??vn:\currentnote\endcsname}
% \def\processnotescs#1{\processcommacommand[\noteinsertions]{\doprocessnotescs#1}}
% \def\noteinsertion     #1{\csname\??vn:#1\endcsname}

\def\savenotedata      {\writestatus{todo}{save    note data}}
\def\restorenotedata   {\writestatus{todo}{restore note data}}
\def\savenotecontent   {\writestatus{todo}{save    note content}}
\def\restorenotecontent{\writestatus{todo}{restore note content}}
\def\erasenotebackup   {\writestatus{todo}{erase   note backup}}

% page-set:

\def\enablenotes    {\writestatus{todo}{enable  notes}}
\def\disablenotes   {\writestatus{todo}{disable notes}}
\def\savenotes      {\writestatus{todo}{save    notes}}
\def\flushsavednotes{\writestatus{todo}{flush   notes}}

% experiment: (compare scope=text and scope=page)
%
% \definenote[mynote][way=bytext,location=text,width=\leftmarginwidth,scope=page,rule=,before=,after=,factor=0]
% \setuptexttexts[margin][\vbox to \textheight{\placenotes[mynote]\vfill}][]

%D Footnotes are can be characterized by three components:
%D
%D \startitemize[packed]
%D \item a small number \footnote {a footnote number} or
%D      symbol {\setupfootnotes [conversion=set 2]\footnote
%D      {a footnote}}
%D \item and a similar mark at the bottom of the page
%D \item followed by some additional text
%D \stopitemize
%D
%D Because footnotes are declared at the location of their
%D reference they can be seen as a special kind of
%D floating bodies. Their placement is postponed but has to be
%D taken into account in the pagebreak calculations. This kind
%D of calculations are forced by using \type{\insert}s and dealing
%D with all cases is not trivial.

%D \macros
%D   {notesenabled}
%D
%D We need a couple of states because at some moments we don't want
%D to mess around with inserts at all. Take for instance a table
%D of contents. And so we can temporary disable footnotes by saying
%D
%D \starttyping
%D \notesenabledfalse
%D \stoptyping

\newif\ifnotesenabled \notesenabledtrue

\appendtoks \notesenabledfalse \to \everymarking

%D Often we need to process the whole set of notes and to make that
%D fast, we use a token register:

\newtoks\tobeprocessednotes

\def\processnotes#1% #1: \macro that uses \currentnote
  {\def\doprocesssomenote##1{\edef\currentdescription{##1}\edef\currentnote{##1}#1}%
   \the\tobeprocessednotes}

%D Notes have their own paremater handlers. The complication here
%D is that we use descriptions to typeset the note, so we have several
%D resolvers.

\let\currentnote\v!footnote

\def\noteparameter    #1{\csname\donoteparameter{\??vn\currentnote}#1\endcsname}
\def\noteparameterhash#1{\donoteparameterhash   {\??vn\currentnote}#1}

\def\donoteparameter    #1#2{\ifcsname#1#2\endcsname#1#2\else\expandafter\donoteparentparameter    \csname#1\s!parent\endcsname#2\fi}
\def\donoteparameterhash#1#2{\ifcsname#1#2\endcsname  #1\else\expandafter\donoteparentparameterhash\csname#1\s!parent\endcsname#2\fi}

\def\donoteparentparameter    #1#2{\ifx#1\relax\s!empty\else\donoteparameter    #1#2\fi}
\def\donoteparentparameterhash#1#2{\ifx#1\relax        \else\donoteparameterhash#1#2\fi}

\def\detokenizednoteparameter#1{\detokenize\expandafter\expandafter\expandafter{\csname\??vn#1\endcsname}}

\def\dosetnoteattributes#1#2% style color
  {\edef\fontattributehash {\noteparameterhash#1}%
   \edef\colorattributehash{\noteparameterhash#2}%
   \ifx\fontattributehash \empty\else\dosetfontattribute \fontattributehash #1\fi
   \ifx\colorattributehash\empty\else\dosetcolorattribute\colorattributehash#2\fi}

%D \macros
%D   {setupnote,setupnotedefinition}
%D
%D We can influence footnote typesetting with the setup
%D command:
%D
%D \showsetup{setupnotes}
%D \showsetup{setupnote}
%D
%D The definition command indicate that we can frame the footnote
%D area. The footnotes themselves are treated as descriptions.
%D
%D \showsetup{definenote}
%D
%D It's sort of a custom to precede footnotes by a horizontal
%D rule and although fancy rules like
%D
%D \starttyping
%D \hbox to 10em{\hskip-3em\dotfill}
%D \stoptyping
%D
%D Are quite ligitimate, we default to a simple one 20\% of the
%D text width.

\def\setupnotes
  {\dodoubleargument\getparameters[\??vn]}

\setupnotes
  [\c!location=\v!page,
   \c!way=\v!by\v!part,
   \c!sectionnumber=\v!no,
  %\c!conversion=,
   \c!rule=\v!on,
   \c!before=\blank,
   \c!bodyfont=\v!small,
  %\c!style=,
  %\c!color=,
  %\c!after=,
  %\c!rulecolor=,
   \c!rulethickness=\linewidth,
   \c!frame=\v!off,
   \c!margindistance=.5em,
   \c!columndistance=1em,
   \c!distance=.125em,
   \c!align=\v!normal,
   \c!tolerance=\v!tolerant,
   \c!split=\v!tolerant,
  %\c!width=\makeupwidth,
  %\c!width=\ifdim\hsize<\makeupwidth\hsize\else\makeupwidth\fi,
   \c!width=\defaultnotewidth,
   \c!height=\textheight,
   \c!numbercommand=\high,
   \c!command=\noteparameter\c!numbercommand, % downward compatible
   \c!separator=,% \@@koseparator,
   \c!textcommand=\high,
   \c!textstyle=\tx,
  %\c!textcolor=,
   \c!interaction=\v!yes,
  %\c!factor=,
  %\c!scope=, % \v!text \v!page
\c!prefixconnector=.,
\c!prefix=\v!no,
   \c!next=\autoinsertnextspace, % new, experimental with startnotes
   \c!n=1]

\def\@@defaultnotedefloc{\v!inleft}
\def\@@defaultnotedefdis{\!!zeropoint}

% also s!root
%
% \definedescription
%   [\??vn\??vn]
%   [\c!location=\@@defaultnotedefloc,
%    \c!distance=\@@defaultnotedefdis,
%    \c!width=\v!fit,
%    \c!headstyle=\noteparameter\c!style,
%    \c!headcolor=\noteparameter\c!color,
%    \c!before=,
%    \c!after=]

\def\startnotedef{\resetdescriptions\csname\e!start\??vn\??vn\currentnote\endcsname}
\def\stopnotedef                   {\csname\e!stop \??vn\??vn\currentnote\endcsname}

\def\currentnoteins{\csname\??vn:\currentnote\endcsname}

\newtoks \everysetupnote

\def\definenote
  {\dodoubleempty\dodefinenote}

\def\dodefinenote[#1][#2]%
  {\edef\currentnote{#1}%
   \ifcsname\??vn:\currentnote\endcsname\else
     \@EA\installinsertion\csname\??vn:\currentnote\endcsname\relax
     \appendtoks\doprocesssomenote{#1}\to\tobeprocessednotes
   \fi
   \defineenumeration % description
     [\currentnote]
     [\c!location=\@@defaultnotedefloc,
      \c!distance=\@@defaultnotedefdis,
      \c!width=\v!fit,
      \c!headstyle=\noteparameter\c!style,
      \c!headcolor=\noteparameter\c!color,
\s!handler=\v!note,
      \c!text=,
      \c!before=,
      \c!after=]%
   \setupenumerations
     [\currentnote]
     [\s!parent=\??vn\currentnote,
      \c!number=\v!yes] % no inheritance from decriptions which is okay
   \presetlocalframed
     [\??vn\currentnote]%
   \getparameters
     [\??vn\currentnote]
     [\s!parent=\??vn,#2]%
   \definestructurecounter
     [\currentnote]%
   \ctxlua{structure.notes.define("\currentnote","insert",\number\csname\??vn:\currentnote\endcsname)}%
   \the\everysetupnote}

\let\setupnotedefinition\setupenumerations

\appendtoks
    \setupenumerations[\currentnote][]%
\to \everysetupnote

\def\setupnote
  {\dodoubleempty\dosetupnote}

\def\dosetupnote[#1][#2]%
  {\edef\currentnote{#1}%
   \ifsecondargument
     \getparameters[\??vn\currentnote][#2]%
     \the\everysetupnote
   \fi
   \dochecknote}

\appendtoks
  \letvalue{\??vn\c!rule:\currentnote}\normalnoterule % hm
\to \everysetupnote

\appendtoks
  \processaction
    [\noteparameter\c!rule]
    [     \v!on=>\letvalue{\??vn\c!rule:\currentnote}\normalnoterule,
         \v!off=>\letvalue{\??vn\c!rule:\currentnote}\relax,
     \s!default=>\letvalue{\??vn\c!rule:\currentnote}\relax,
     \s!unknown=>\setvalue{\??vn\c!rule:\currentnote}{\noteparameter\c!rule}]%
\to \everysetupnote

\appendtoks
  \processaction % todo
    [\noteparameter\c!split]
    [  \v!tolerant=>\notepenalty\zeropoint,
         \v!strict=>\notepenalty9999,
     \v!verystrict=>\notepenalty\maxdimen,
        \s!default=>\notepenalty\zeropoint,
        \s!unknown=>\notepenalty\commalistelement]%
\to \everysetupnote

%D The following switch can be used to disable limiting the
%D height of the footnote area, something that is needed in
%D multi column balancing. Use this switch with care.

\newif\ifnotelimit \notelimittrue % shared

% bottomnotes endnotes
% clevernotes

\appendtoks
  \doifsomething{\noteparameter\c!factor}
    {\ifnum\noteparameter\c!factor<\zerocount\else
       \count\currentnoteins\noteparameter\c!factor
     \fi}%
\to \everysetupnote

% compatibility (will go away)

\newif\ifendnotes
\newif\ifbottomnotes

% locations:

\def\s!noteloc{nodeloc} % 1=page 2=columns 3=lastcolumn 4=firstcolumn 5=none
\def\s!notepos{nodepos} % 0=nothing 1=high 2=bottom
\def\s!notefmt{nodefmt} % 1 text
\def\s!notecol{nodecol}

\def\clevernotes % compatibility hack
  {\numexpr\ifcase\noteparameter\s!noteloc\or0\or2\or2\or1\else0\fi\relax}

\def\setnotelocation  #1{\expandafter\chardef\csname\??vn\currentnote\s!noteloc\endcsname#1\relax}
\def\setnoteposition  #1{\expandafter\chardef\csname\??vn\currentnote\s!notepos\endcsname#1\relax}
\def\setnoteformatting#1{\expandafter\chardef\csname\??vn\currentnote\s!notefmt\endcsname#1\relax}
\def\setnotecolumns   #1{\expandafter\chardef\csname\??vn\currentnote\s!notecol\endcsname#1\relax}

\def\currentnofcolumns{\@@kln}

\def\dochecknote
  {% node states
   \setnotelocation\plusone
   \setnoteposition\plustwo
   \processallactionsinset
     [\noteparameter\c!location]
     [       \v!page=>\setnotelocation  \plusone,
          \v!columns=>\setnotelocation  \plustwo,
      \v!firstcolumn=>\setnotelocation  \plusthree,
       \v!lastcolumn=>\setnotelocation  \plusfour,
             \v!none=>\setnotelocation  \plusfive,
             \v!text=>\setnotelocation  \plusfive
                      \setnoteformatting\plusone, % test
             \v!high=>\setnoteposition  \plusone,
           \v!bottom=>\setnoteposition  \plustwo]%
   % compatibility hack
   \ifnum\noteparameter\s!noteloc=\plusfive \endnotestrue    \else \endnotesfalse    \fi
   \ifnum\noteparameter\s!notepos=\plustwo  \bottomnotestrue \else \bottomnotesfalse \fi
   % set column multiplier
   \edef\currentnotenofcolumns{\noteparameter\c!n}%
   \ifx\currentnotenofcolumns\empty
     \let\currentnotenofcolumns\!!plusone
   \fi
   \ifcase\noteparameter\s!noteloc\or
     % page
     \scratchcounter \currentnotenofcolumns
   \or
     % columns
     \scratchcounter\ifnum\currentnofcolumns=\zerocount \plusone \else \currentnotenofcolumns \fi \relax
   \or
     % firstcolumn
     \scratchcounter\plusone
   \or
     % lastcolumn
     \scratchcounter\plusone
   \or
     % text
     \scratchcounter\currentnotenofcolumns
   \fi
   % column factor
   \global\count\currentnoteins\plusthousand
   \global\count\currentnoteins\numexpr\plusthousand/\scratchcounter\relax
   % maximize height
   \ifnotelimit
     \global\dimen\currentnoteins\dimexpr\noteparameter\c!height*\scratchcounter\relax
   \fi
   % distance
   \begingroup
   \setbox\scratchbox\vbox
     {\forgetall
      \noteparameter\c!before
      \placenoterule
      \noteparameter\c!after}%
   \global\skip\currentnoteins\ht\scratchbox
   \endgroup
   % play safe
   \ifnum\noteparameter\s!noteloc=\plusfive
     \ctxlua{structure.notes.setstate("\currentnote","store")}%
     % text notes (e.g. end notes) but we don't use inserts anyway
     \global\dimen\currentnoteins\maxdimen
     \global\count\currentnoteins\zerocount
     \global\skip \currentnoteins\zeropoint
   \fi}

\def\checknotes
  {\processnotes\dochecknote}

% D When \type{n} exceeds~1, footnotes are typeset in
% D multi||columns, using the algoritm presented on page~397
% D of \TEX book. Footnotes can be places on a per page basis
% D or whereever suitable. When we set~\type{n} to~0, we get a
% D rearanged paragraph, typeset by the algoritms on pages 398
% D and~389 (at least in \MKII). We definitely did not reinvent
% D that wheel.

% Example of using factor:
%
% \definenote[mynote][way=bypage,location=text,width=\marginwidth,rule=,before=,factor=0]
% \setuplayout[backspace=5cm,margin=3cm,margindistance=.5cm,width=middle]
% \setuptexttexts[margin][\vbox to \textheight{\placenotes[mynote]\vfill}][]
% \starttext
% \dorecurse{10}{test \mynote{one one one one one one} \input zapf \mynote{one one one one one one} }
% \stoptext

%D The noterule can be a graphic and therefore calling this
%D setup macro at every skipswitch is tricky (many many MP
%D runs). Let's just reserve a few points, that probably match
%D those of the stretch component.

\def\placenoterule
  {\getvalue{\??vn\c!rule:\currentnote}}

\def\normalnoterule
  {\ifvmode
     \color
       [\noteparameter\c!rulecolor]
       {\hrule\!!width .2\hsize\!!height\noteparameter\c!rulethickness\!!depth \zeropoint}%
     \kern\strutdepth
   \fi}

\ifx\setnotehsize\undefined

  \def\setnotehsize{\hsize\noteparameter\c!width\relax} % can be overloaded

\fi

%D The formatting depends on the width of the table, so we
%D have to set \type {n} to zero.
%D
%D \starttyping
%D \startbuffer
%D \bTABLE
%D \bTR \bTD one \footnote{\dorecurse{10}{abcd }} \eTD \bTD two \eTD \eTR
%D \bTR \bTD three fout five six seven eight nine \eTD \bTD ten \eTD \eTR
%D \eTABLE
%D \stopbuffer
%D
%D \startlocalfootnotes[n=0,location={text,none}]
%D \placelegend[n=2]{\getbuffer}{\placelocalfootnotes}
%D \stoplocalfootnotes
%D \stoptyping

%D \macros
%D   {footnote}
%D
%D A footnote can have a reference as optional argument and
%D therefore its formal specification looks like:
%D
%D \showsetup{footnote}
%D
%D This command has one optional command: the reference. By
%D saying \type{[-]} the number is omitted. The footnote
%D command is not that sensitive to spacing, so it's quite
%D legal to say:
%D
%D \startbuffer
%D Users of \CONTEXT\ must keep both feet \footnote{Given they
%D have two.} on the ground and not get confused \footnote{Or
%D even crazy.} by all those obscure \footnote{But fortunately
%D readable.} parameters.
%D \stopbuffer
%D
%D \typebuffer
%D
%D When setting the \type{conversion} to \type{set 2} we get
%D something like:
%D
%D \bgroup
%D \startnarrower
%D \setupfootnotes[conversion=set 1]
%D \getbuffer
%D \stopnarrower
%D \egroup
%D
%D Typesetting footnotes is, at least for the moment, disabled
%D when reshaping boxes.
%D
%D The additional macro \type {\footnotetext} and the
%D associated \type {\note} macro were implemented at
%D request of users on the mailing list and a suggestion by
%D taco to split of the symbol placement. I decided to
%D merge this functionality with the existing \type {\note}
%D functionality.

%D The next implementation runs on top of enumerations (only in \MKIV).

% TODO: \ifnotesenabled

\newif\ifnotesymbol \notesymboltrue

\def\setnote    [#1]{\getvalue{#1}}
\def\setnotetext[#1]{\global\settrue\skipnoteplacement\getvalue{#1}}

\def\domovednote#1#2#3#4%
  {\ifcase\ctxlua{structure.notes.deltapage("#1",#2)}\or\symbol[#3]\or\symbol[#4]\fi}

\setvalue{\??dd:\v!note:\s!handler         }{\@@doenumerationhandler}
\setvalue{\??dd:\v!note:\s!handler:\s!do   }{\@@somenotedescription}
\setvalue{\??dd:\v!note:\s!handler:\s!start}{\@@startsomenotedescription}

\def\@@somenotedescription     {\@@notemakedescription}
\def\@@startsomenotedescription{\@@notemakedescription}

\newconditional\skipnoteplacement

\def\@@notemakedescription[#1]#2#3% todo ... proper [key=value] etc
  {\begingroup
   \doenumerationcheckconditions
   \let\currentnote\currentdescriptionmain
   \dodescriptioncomponent[\c!reference=#1,\c!label={\descriptionparameter\c!text},\c!title={#3},\c!bookmark=,][]%
   \xdef\currentnotenumber{\ctxlua{structure.notes.store("\currentnote",\currentdescriptionnumberentry)}}%
   \settrue\processingnote
   \ifconditional\skipnoteplacement
     \globallet\lastnotesymbol\dolastnotesymbol
   \else
     \iftypesettinglines % otherwise problems with \type <crlf> {xxx}
       \ignorelines % makes footnotes work in \startlines ... \stoplines
     \fi
     \ifnotesymbol
       \dolastnotesymbol
     \else
       \unskip\unskip
       \globallet\lastnotesymbol\dolastnotesymbol
     \fi
   \fi
   \ifconditional\postponingnotes
     \global\settrue\postponednote
   \else
     \handlenoteinsert\currentnote\currentnotenumber
   \fi
   \ifconditional\skipnoteplacement \else
     \kern\notesignal\relax % \relax is needed to honor spaces
     \iftrialtypesetting \else \global\setfalse\skipnoteplacement \fi
   \fi
   \endgroup}

\def\dolastnotesymbol{\typesetsomenotesymbol\currentnote\currentnotenumber}

\def\dotypesetsomenotesymbol#1#2%
  {\dodonotesymbol
     {\synchronizesomenotesymbol{#1}{#2}%
      \ctxlua{structure.notes.number("\currentnote",\currentnotenumber)}% \currentdescriptionnumberentry
      \domovednote{#1}{#2}\v!previouspage\v!nextpage}}

\def\typesetsomenotesymbol#1#2%
  {\removeunwantedspaces
   \doifitalicelse\/\donothing % Charles IV \footnote{the fourth}
   \ifdim\lastkern=\notesignal
     \dodonotesymbol{\kern\noteparameter\c!distance}% gets the font right, hack !
   \fi
   \nobreak
   \doifelse{\noteparameter\c!interaction}\v!no
     {\dotypesetsomenotesymbol{#1}{#2}}
     {\gotobox{\dotypesetsomenotesymbol{#1}{#2}}[page(\ctxlua{structure.notes.getnumberpage("#1",\number#2)})]}% f:
   \globallet\lastnotesymbol\relax}

\def\currentnotedescriptiontext % todo: can be other number
  {\ctxlua{structure.notes.title("\currentnote",\currentdescriptionnumberentry)}}

\def\currentnoteenumerationfullnumber
  {\doifelse{\noteparameter\c!interaction}\v!no
     {\docurrentnoteenumerationfullnumber}%
     {\gotobox
        {\docurrentnoteenumerationfullnumber}%
        [page(\ctxlua{structure.notes.getsymbolpage("\currentnote",\currentdescriptionnumberentry)})]}}

\def\docurrentnoteenumerationfullnumber
  {\noteparameter\c!numbercommand
     {\ctxlua{structure.notes.number("\currentnote",\currentdescriptionnumberentry)}%
      \domovednote\currentdescription\currentdescriptionnumberentry\v!nextpage\v!previouspage}}

\def\synchronizesomenotesymbol#1#2% called more often than needed
  {\expanded{\noexpand\ctxlatelua{structure.notes.setsymbolpage("#1",#2)}}}

\def\handlenoteinsert#1#2%
  {\begingroup
   \edef\currentnote{#1}%
   \the\everybeforenoteinsert
   \insert\currentnoteins\bgroup
     \the\everyinsidenoteinsert
     \handlenoteitself{#1}{#2}%
   \egroup
   \the\everyafternoteinsert
   \endgroup}

\def\handlenoteitself#1#2% tg, id
  {\edef\currentdescription{#1}%
   \edef\currentnote{#1}%
   \edef\currentdescriptionnumberentry{#2}%
   \let\currentdescriptiontext\currentnotedescriptiontext
   \let\currentenumerationfullnumber\currentnoteenumerationfullnumber
   \dostartstoreddescription\begstrut\currentnotedescriptiontext\endstrut\dostopstoreddescription}

\def\dostartstoreddescription
  {\bgroup\@@dostartdescriptionindeed}

\def\dostopstoreddescription
  {\@@stopdescription}

%D The main typesetting routine is more or less the same as the
%D \PLAIN\ \TEX\ one, except that we only handle one type while
%D \PLAIN\ also has something \type{\v...}. In most cases
%D footnotes can be handled by a straight insert, but we do so
%D by using an indirect call to the \type{\insert} primitive.

%D Making footnote numbers active is not always that logical,
%D Making footnote numbers active is not always that logical,
%D especially when we keep the reference and text at one page.
%D On the other hand we need interactivity when we refer to
%D previous notes or use end notes. Therefore we support
%D interactive footnote numbers in two ways \footnote{This
%D feature was implemented years after we were able to do so,
%D mainly because endnotes had to be supported.} that is,
%D automatically (vise versa) and by user supplied reference.

\newcount\internalnotereference

\let\startpushnote=\relax
\let\stoppushnote =\relax

\newsignal\notesignal
\newcount \notepenalty

\notepenalty=0 % needed in order to split in otrset

\newconditional\processingnote
\newconditional\postponednote

\newtoks\everybeforenoteinsert
\newtoks\everyinsidenoteinsert
\newtoks\everyafternoteinsert

\appendtoks
   \let\doflushnotes\relax
   \let\postponenotes\relax
   \forgetall
\to \everybeforenoteinsert

\appendtoks
  \doif{\noteparameter\c!scope}\v!page{\floatingpenalty\maxdimen}% experiment
  \penalty\notepenalty
  \forgetall
  \setnotebodyfont
  \redoconvertfont % to undo \undo calls in in headings etc
  \splittopskip\strutht  % not actually needed here
  \splitmaxdepth\strutdp % not actually needed here
  \leftmargindistance\noteparameter\c!margindistance
  \rightmargindistance\leftmargindistance
  \ifnum\noteparameter\c!n=\zerocount % no ifcase new 31-07-99 ; always ?
    \doifnotinset{\noteparameter\c!width}{\v!fit,\v!broad}\setnotehsize % ?
  \fi
\to \everyinsidenoteinsert

\let\lastnotesymbol\relax

%D \macros
%D   {note}
%D
%D Refering to a note is accomplished by the rather short
%D command:
%D
%D \showsetup{note}
%D
%D This command is implemented rather straightforward as:

\def\notesymbol
  {\dodoubleempty\donotesymbol}

\def\donotesymbol[#1][#2]%
  {\bgroup
   \ifnotesenabled
     \edef\currentnote{#1}%
     \ifsecondargument
       \unskip
       \dodonotesymbol{\in[#2]}%
     \else
       \dodonotesymbol\lastnotesymbol
     \fi
   \fi
   \egroup}

\def\dodonotesymbol#1%
  {\noteparameter\c!textcommand{\dosetnoteattributes\c!textstyle\c!textcolor#1}}

%D Normally footnotes are saved as inserts that are called upon
%D as soon as the pagebody is constructed. The footnote
%D insertion routine looks just like the \PLAIN\ \TEX\ one,
%D except that we check for the end note state.

% testcase for split bottom alignment see (a) below
%
% \dorecurse{6}{\input tufte\footnote{\input ward \input tufte \relax}}

\def\placenoteinserts
  {\processnotes\doplacenoteinserts}

\def\unvboxed {\ifvmode\unvbox \else\box \fi}
\def\unvcopied{\ifvmode\unvcopy\else\copy\fi}

\def\doplacenoteinserts
  {\relax\ifdim\ht\currentnoteins>\zeropoint\relax
     \ifnum\noteparameter\s!noteloc=\plusfive
     \else
        \endgraf
        \ifvmode
          \whitespace
          \noteparameter\c!before
        \fi
       \placenoterule  % alleen in ..mode
       \bgroup
       \setnotebodyfont
       \setbox\scratchbox\hbox
         {% this should be checked, smells like a mix-up
          % does not split: \ifcase\noteparameter\c!n\unvbox\else\box\fi\currentnoteins
          \ifcase\noteparameter\c!n\relax
            \iftrialtypesetting\unvcopied\else\unvboxed\fi\currentnoteins
          \or
            \iftrialtypesetting\copy\else\box\fi\currentnoteins
            \obeydepth % (a) added , since split footnotes will not align properly
          \else
            \iftrialtypesetting\unvcopied\else\unvboxed\fi\currentnoteins
          \fi}%
       \setbox\scratchbox\hbox
         {\localframed
            [\??vn\currentnote]
            [\c!width=\v!fit,
             \c!height=\v!fit,
             \c!strut=\v!no,
             \c!offset=\v!overlay]
            {\ifdim\dp\scratchbox=\zeropoint         % this hack is needed because \vadjust
               \hbox{\lower\strutdp\box\scratchbox}% % in margin number placement
             \else                                   % hides the (always) present depth
               \box\scratchbox
             \fi}}%
       \setbox\scratchbox\hbox{\lower\strutdepth\box\scratchbox}%
       \dp\scratchbox\strutdepth % so we know that it has the note bodyfont depth
       \box\scratchbox
       \egroup
       \endgraf
       \ifvmode
         \noteparameter\c!after
       \fi
     \fi
   \fi}

%D Supporting end notes is surprisingly easy. Even better, we
%D can combine this feature with solving the common \TEX\
%D problem of disappearing inserts when they're called for in
%D deeply nested boxes. The general case looks like:
%D
%D \starttyping
%D \postponenotes
%D \.box{whatever we want with footnotes}
%D \flushnotes
%D \stoptyping
%D
%D This alternative can be used in headings, captions, tables
%D etc. The latter one sometimes calls for notes local to
%D the table, which can be realized by saying
%D
%D \starttyping
%D \setlocalfootnotes
%D some kind of table with local footnotes
%D \placelocalfootnotes
%D \stoptyping
%D
%D Postponing is accomplished by simply redefining the (local)
%D insert operation. A not too robust method uses the
%D \type{\insert} primitive when possible. This method fails in
%D situations where it's not entirely clear in what mode \TEX\
%D is. Therefore the auto method can is to be overruled when
%D needed.

\newconditional\postponingnotes

% we need a proper state: normal, postponing, flushing

\def\postponenotes
  {\ifconditional\postponingnotes\else
     \global\settrue\postponingnotes
     \ctxlua{structure.notes.postpone()}%
   \fi}

% \def\flushnotes
%   {\ifconditional\processingnote \else
%      \ifconditional\postponednote
%        \ifinner \else
%          \ifinpagebody \else
%            %ifvmode % less interference, but also less secure
%              \doflushnotes
%            %fi
%          \fi
%        \fi
%      \fi
%    \fi}

\def\flushnotes
  {\ifconditional\postponednote
     \flushnotesindeed
   \fi}

\def\flushnotesindeed
  {\ifconditional\processingnote \else
     \ifinner \else
       \ifinpagebody \else
         %ifvmode % less interference, but also less secure
           \doflushnotes
         %fi
       \fi
     \fi
   \fi}

\def\doflushnotes % also called directly, \ifvoid is needed !
  {\begingroup
   \let\doflushnotes\relax
   \let\postponenotes\relax
   \ifconditional\processingnote \else
     \ifconditional\postponednote
       \processnotes\dodoflushnotes
       \global\setfalse\postponednote
       \setfalse\postponingnotes
     \fi
   \fi
   \endgroup}

\def\dodoflushnotes % per class, todo: handle endnotes here
  {%\writestatus{notes}{flushing \currentnote}%
   \global\setfalse\postponingnotes
   \ctxlua{structure.notes.flush("\currentnote","postpone")}}

%D \macros
%D   {startlocalfootnotes,placelocalfootnotes}
%D
%D The next two macros can be used in for instance tables, as
%D we'll demonstrate later on.
%D
%D \showsetup{startlocalfootnotes}
%D \showsetup{placelocalfootnotes}

% todo: compatibility mode: when first arg is assignment or missing, then all

\newtoks\everyplacelocalnotes

\appendtoks
    \let\doflushnotes\relax
    \let\postponenotes\relax
\to \everyplacelocalnotes

\def\defaultnotewidth{\makeupwidth} % {\ifdim\hsize<\makeupwidth\hsize\else\makeupwidth\fi}

\def\startlocalnotes
  {\dosingleempty\dostartlocalnotes}

\def\dostartlocalnotes[#1]%
  {\def\localnoteslist{#1}%
   \processcommacommand[\localnoteslist]\dodostartlocalnotes}

\def\stoplocalnotes
  {\processcommacommand[\localnoteslist]\dodostoplocalnotes}

\def\dodostartlocalnotes#1%
  {\savestructurecounter[#1]%
   \resetstructurecounter[#1]%
   \ctxlua{structure.notes.save("#1","store")}}

\def\dodostoplocalnotes#1%
  {\restorestructurecounter[#1]%
   \ctxlua{structure.notes.restore("#1")}}

\def\placelocalnotes
  {\dodoubleempty\doplacelocalnotes}

\def\doplacelocalnotes[#1][#2]%
  {\doif{\ctxlua{structure.notes.getstate("#1")}}{store}{\dodoplacelocalnotes{#2}{#1}}}

\def\dodoplacelocalnotes#1#2% settings note
  {\begingroup
   \the\everyplacelocalnotes
   % beware, we cannot trust setting \currentnote here
   \getparameters[\??vn#2][\c!width=\v!fit,\c!height=\v!fit,\c!strut=\v!no,\c!offset=\v!overlay,#1]% we only need a selective one
   \donotealternative{#2}%
   \endgroup
   \dochecknote} % we need to restore the old state

%D These commands can be used like:
%D
%D \startbuffer
%D \startlocalnotes[width=.3\hsize,n=0]
%D   \placetable
%D     {Some Table}
%D     \placeontopofeachother
%D       {\starttable[|l|r|]
%D        \HL
%D        \VL Nota\footnote{Bene} \VL Bene\footnote{Nota} \VL\SR
%D        \VL Bene\footnote{Nota} \VL Nota\footnote{Bene} \VL\SR
%D        \HL
%D        \stoptable}
%D       {\placelocalnotes}
%D \stoplocalnotes
%D \stopbuffer
%D
%D \typebuffer
%D
%D Because this table placement macro expect box content, and
%D thanks to the grouping of the local footnotes, we don't need
%D additional braces.
%D
%D \getbuffer

%D \macros
%D   {placefootnotes}
%D
%D We still have no decent command for placing footnotes
%D somewhere else than at the bottom of the page (for which no
%D user action is needed). Footnotes (endnotes) can be
%D placed by using
%D
%D \showsetup{placefootnotes}

\def\placebottomnotes
  {\processnotes\placenoteinserts}

\def\placenotes
  {\dodoubleempty\doplacenotes}

\def\doplacenotes[#1][#2]%
  {\processcommalist[#1]{\dodoplacenotes{#2}}}

\def\dodoplacenotes#1#2% settings note
  {\edef\currentnote{#2}%
   \doifelse{\ctxlua{structure.notes.getstate("#1")}}{store}
     \dodoplacelocalnotes
     \dodoplaceglobalnotes
       {#1}{#2}}

\def\dodoplaceglobalnotes#1#2%
  {\begingroup
   \setupnote[#2][#1]%
   \doplacenoteinserts
   \endgroup
   \the\everysetupnote} % to be checkes

%D Placement

\long\def\installnotealternative#1#2%
  {\setvalue{\??vn:\c!alternative:#1}{#2}}

\def\doifnotescollected#1%
  {\ctxlua{structure.notes.doifcontent("#1")}}

\def\donotealternative#1%
  {\edef\currentnote{#1}%
   \doifnotescollected\currentnote
     {\endgraf
      \ifvmode
        \whitespace
        \noteparameter\c!before
      \fi
      \begingroup
      \setnotebodyfont
      \getvalue{\??vn:\c!alternative:\noteparameter\c!alternative}%
      \endgroup
      \ifvmode
        \noteparameter\c!after
      \fi}}

\setvalue{\??vn:\c!alternative:}{\getvalue{\??vn:\c!alternative:\v!none}}

%D A stupid alternative is also provided:
%D
%D \starttyping
%D \setupfootnotes[location=text,alternative=none]
%D \stoptyping

\def\flushlocalnotes#1{\ctxlua{structure.notes.flush("#1","store")}}

\installnotealternative \v!none
  {\flushlocalnotes\currentnote}

\installnotealternative \v!grid % test if n > 0
  {\snaptogrid\hbox
    {\localframed
       [\??vn\currentnote]
       {\flushlocalnotes\currentnote}}}

\installnotealternative \v!fixed % test if n > 0
  {\localframed
     [\??vn\currentnote]
     {\flushlocalnotes\currentnote}}

\installnotealternative \v!columns % redundant
  {\localframed
     [\??vn\currentnote]
     {\edef\currentnotewidth{\noteparameter\c!width}%
      \doifdimensionelse\currentnotewidth\donothing
        {\edef\currentnotewidth{\the\hsize}}%
%       \setupinmargin[\c!align=\v!left]%
      \startsimplecolumns[\c!distance=\noteparameter\c!columndistance,\c!n=\noteparameter\c!n,\c!width=\currentnotewidth]%
        \flushlocalnotes\currentnote
      \stopsimplecolumns}}

%D \macros
%D   {fakenotes}

    \def\fakenotes
      {\ifhmode\endgraf\fi\ifvmode
         \calculatetotalclevernoteheight
         \ifdim\totalnoteheight>\zeropoint \kern\totalnoteheight \fi
       \fi}

    \def\fakepagenotes
      {\ifhmode\endgraf\fi\ifvmode
         \calculatetotalpagenoteheight
         \ifdim\totalnoteheight>\zeropoint \kern\totalnoteheight \fi
       \fi}

    \newdimen\totalnoteheight

    \def\doaddtototalnoteheight#1%
      {\ifdim\ht#1>\zeropoint
         \advance\totalnoteheight\ht  #1%
         \advance\totalnoteheight\skip#1%
       \fi}

    \def\docalculatetotalnoteheight
      {\ifcase\clevernotes % tricky here ! ! ! to be sorted out ! ! !
         \doaddtototalnoteheight\currentnoteins
       \else
         \doaddtototalnoteheight\currentbackupnoteins
       \fi}

    \def\docalculatetotalclevernoteheight
      {\ifcase\clevernotes \else % tricky here ! ! ! to be sorted out ! ! !
         \doaddtototalnoteheight\currentnoteins
       \fi}

    \def\docalculatetotalpagenoteheight
      {\doaddtototalnoteheight\currentnoteins}

    \def\calculatetotalnoteheight      {\totalnoteheight\zeropoint\processnotes\docalculatetotalnoteheight}
    \def\calculatetotalclevernoteheight{\totalnoteheight\zeropoint\processnotes\docalculatetotalclevernoteheight}
    \def\calculatetotalpagenoteheight  {\totalnoteheight\zeropoint\processnotes\docalculatetotalpagenoteheight}

    \newif\ifnotespresent

    \def\dochecknotepresence
      {\ifdim\ht\currentnoteins>\zeropoint
         \notespresenttrue
       \fi}

    \def\checknotepresence
      {\notespresentfalse
       \processnotes\dochecknotepresence}

%D Now how can this mechanism be hooked into \CONTEXT\ without
%D explictly postponing footnotes? The solution turned out to
%D be rather simple:
%D
%D \starttyping
%D \everypar  {...\flushnotes...}
%D \neverypar {...\postponenotes}
%D \stoptyping
%D
%D and
%D
%D \starttyping
%D \def\ejectinsert%
%D   {...
%D    \flushnotes
%D    ...}
%D \stoptyping
%D
%D We can use \type{\neverypar} because in most commands
%D sensitive to footnote gobbling we disable \type{\everypar}
%D in favor for \type{\neverypar}. In fact, this footnote
%D implementation is the first to use this scheme.

%D This is a nasty and new secondary footnote flusher. It
%D can be hooked into \type {\everypar} like:
%D
%D \starttyping
%D \appendtoks \synchronizenotes \to \everypar
%D \stoptyping

    % \def\dosynchronizenotes
    %   {\ifvoid\currentnoteins\else\insert\currentnoteins{\unvbox\currentnoteins}\fi}
    %
    % \def\synchronizenotes
    %   {\processnotes\dosynchronizenotes}

\let\synchronizenotes\relax

%D When typesetting footnotes, we have to return to the
%D footnote specific bodyfont size, which is in most cases derived
%D from the global document bodyfont size. In the previous macros
%D we already used a footnote specific font setting macro.

\def\setnotebodyfont
  {\let\setnotebodyfont\relax
   \restoreglobalbodyfont
   \switchtobodyfont[\noteparameter\c!bodyfont]%
   \setuptolerance[\noteparameter\c!tolerance]%
   \setupalign[\noteparameter\c!align]}

%D The footnote mechanism defaults to a traditional one
%D column way of showing them. By default we precede them by
%D a small line.

\ifx\v!endnote\undefined \def\v!endnote{endnote} \fi

\definenote [\v!footnote]
\definenote [\v!endnote ] [\c!location=\v!none] % else no break

%D Compatibility macros:

            \def\setupfootnotedefinition{\setupnotedefinition                [\v!footnote]}
            \def\setupfootnotes         {\setupnote                          [\v!footnote]}
%unexpanded \def\footnote               {\setnote                            [\v!footnote]}
\unexpanded \def\footnotetext           {\setnotetext                        [\v!footnote]}
            %def\note                   {\dodoubleempty\notesymbol           [\v!footnote]} %  alleen footnote
            \def\placefootnotes         {\dodoubleempty\doplacefootnotes     [\v!footnote]}
            \def\placelocalfootnotes    {\dodoubleempty\doplacelocalfootnotes[\v!footnote]}
            \def\startlocalfootnotes    {\startlocalnotes                    [\v!footnote]} %  alleen footnote
            \def\stoplocalfootnotes     {\stoplocalnotes }
            \def\flushfootnotes         {\flushnotes}
            \def\doflushfootnotes       {\doflushnotes}

\def\doplacefootnotes     [#1][#2]{\ifsecondargument\placenotes     [#1][#2,\c!height=\textheight]\else\placenotes     [#1]\fi}
\def\doplacelocalfootnotes[#1][#2]{\ifsecondargument\placelocalnotes[#1][#2,\c!height=\textheight]\else\placelocalnotes[#1]\fi}

\def\note{\dodoubleempty\donote}

\def\donote[#1][#2]{\ifsecondargument\donotesymbol[#1][#2]\else\secondargumenttrue\donotesymbol[\v!footnote][#1]\fi}

%D New trickery:

\def\ownnotesymbol#1% #1 gets number passed
  {\executeifdefined{\??vn::\currentnote}\empty}

\def\setnotesymbol[#1]#2#3%
  {\prewordbreak % prevent lookback
   \setgvalue{\??vn::#1}{#3}
   \dolastnotesymbol}

\def\ownnote[#1]#2#3#4%
  {\setnotesymbol[#1]{#2}{#3}%
   \setnotetext  [#1]{#4}}

\defineconversion
  [ownnote]
  [\ownnotesymbol]

\protect \endinput
