%D \module
%D   [       file=char-utf,
%D        version=2006.12.05,
%D          title=\CONTEXT\ Character Support,
%D       subtitle=Unicode UTF,
%D         author=Hans Hagen,
%D           date=\currentdate,
%D      copyright=PRAGMA]
%C
%C This module is part of the \CONTEXT\ macro||package and is
%C therefore copyrighted by \PRAGMA. See mreadme.pdf for
%C details.

\writestatus{loading}{ConTeXt Character Support / Unicode UTF}

%D After a bit of experimenting we reached a clean state where \UTF\
%D 8, 16 and 32 were supported as well as collapsing (combining
%D sequences). Writing the code was a relaxed experience, not in the
%D last place because it was accompanied by listening to those nice
%D Vienna Teng cd's (who decided that making music was more fun than
%D programming, but then, she may not know \TEX\ and \LUA).

\unprotect

\registerctxluafile{char-utf}{1.001}

%D We enable collapsing (combining characters) by default, but
%D since the source files are rather simple, we postpone the
%D initialization till runtime.

\appendtoks
    \ctxlua {
        characters.filters.utf.collapsing = true
        resolvers.install_text_filter('utf',characters.filters.utf.collapse)
    }%
\to \everyjob

%D The next one influences input parsing.
%D
%D \starttyping
%D \definecomposedutf 318 108 126 % lcaron
%D \stoptyping

\def\definecomposedutf#1 #2 #3 %
  {\ctxlua{characters.filters.utf.add_grapheme("#1","#2","#3")}}

\protect
