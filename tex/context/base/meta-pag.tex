%D \module
%D   [       file=meta-pag,
%D        version=1999.07.10,
%D          title=\METAPOST\ Graphics,
%D       subtitle=Initialization,
%D         author=Hans Hagen,
%D           date=\currentdate,
%D      copyright={PRAGMA / Hans Hagen \& Ton Otten}]
%C
%C This module is part of the \CONTEXT\ macro||package and is
%C therefore copyrighted by \PRAGMA. See mreadme.pdf for
%C details.

%D These definitions used to be part of the old \type
%D {core-mps} file, later changed into \type {meta-ini}, but
%D keeping them separate is cleaner.

\writestatus{loading}{MetaPost Graphics / Page Data Management}

\unprotect

\startMPextensions
  if unknown context_page: input mp-page; fi;
\stopMPextensions

%D The next few macros tell \METAPOST\ how the \CONTEXT\
%D pagebody looks.

\startMPextensions
  boolean PageStateAvailable,OnRightPage,InPageBody;
  PageStateAvailable:=true;
\stopMPextensions

\startMPinitializations
  OnRightPage:=true;
  InPageBody:=\ifinpagebody true \else false \fi;
\stopMPinitializations

\startMPinitializations
  def LoadPageState =
    OnRightPage:=\MPonrightpage;
    OnOddPage:=\MPonoddpage;
    RealPageNumber:=\the\realpageno;
    PageNumber:=\the\pageno;
    NOfPages:=\lastpage;
    PaperHeight:=\the\paperheight;
    PaperWidth:=\the\paperwidth;
    PrintPaperHeight:=\the\printpaperheight;
    PrintPaperWidth:=\the\printpaperwidth;
    TopSpace:=\the\topspace;
    BottomSpace:=\the\bottomspace;
    BackSpace:=\the\backspace;
    CutSpace:=\the\cutspace;
    MakeupHeight:=\the\makeupheight;
    MakeupWidth:=\the\makeupwidth;
    TopHeight:=\the\topheight;
    TopDistance:=\the\topdistance;
    HeaderHeight:=\the\headerheight;
    HeaderDistance:=\the\headerdistance;
    TextHeight:=\the\textheight;
    FooterDistance:=\the\footerdistance;
    FooterHeight:=\the\footerheight;
    BottomDistance:=\the\bottomdistance;
    BottomHeight:=\the\bottomheight;
    LeftEdgeWidth:=\the\leftedgewidth;
    LeftEdgeDistance:=\the\leftedgedistance;
    LeftMarginWidth:=\the\leftmarginwidth;
    LeftMarginDistance:=\the\leftmargindistance;
    TextWidth:=\the\textwidth ;
    RightMarginDistance:=\the\rightmargindistance;
    RightMarginWidth:=\the\rightmarginwidth;
    RightEdgeDistance:=\the\rightedgedistance;
    RightEdgeWidth:=\the\rightedgewidth;
    InnerMarginDistance:=\the\innermargindistance;
    InnerMarginWidth:=\the\innermarginwidth;
    OuterMarginDistance:=\the\outermargindistance;
    OuterMarginWidth:=\the\outermarginwidth;
    InnerEdgeDistance:=\the\inneredgedistance;
    InnerEdgeWidth:=\the\inneredgewidth;
    OuterEdgeDistance:=\the\outeredgedistance;
    OuterEdgeWidth:=\the\outeredgewidth;
    PageOffset:=\the\pageoffset;
    PageDepth:=\the\pagedepth;
    LayoutColumns:=\the\layoutcolumns;
    LayoutColumnDistance:=\the\layoutcolumndistance;
    LayoutColumnWidth:=\the\layoutcolumnwidth;
  enddef;
\stopMPinitializations

\def\MPonrightpage{true}
\def\MPonoddpage  {true}

\def\freezeMPpagelayout
  {\doifbothsides
     {\def\MPonrightpage{true}}
     {\def\MPonrightpage{true}}
     {\def\MPonrightpage{false}}%
   \edef\MPonoddpage{\doifoddpageelse{true}{false}}}

\let\freezeMPlayout\relax % obsolete

%D We need to freeze the pagelayout before the backgrounds
%D are build, because the overlay will temporarily become
%D zero (overlay).

\appendtoks
  \freezeMPpagelayout
\to \everybeforepagebody

%D By freezing these value every graphic, we can use layout
%D variables that change halfways a page, whatever use that
%D has.

\prependtoks
  \calculatereducedvsizes % this is really needed
  \freezeMPpagelayout
  \freezeMPlayout % to be used grouped
\to \everyMPgraphic

%D The next feature provides information about for instance
%D column positions. This is an experimental feature,
%D introduced when we needed backgrounds in columns (fill||in
%D questions as implemented in a private module).
%D
%D See \type {mp-page.mp} for the definition of the macros:
%D
%D \starttabulate[|tl|l|p|]
%D \NC ResetTextAreas        \NC no arguments \NC
%D     reset areas on page                    \NC \NR
%D \NC RegisterTextArea      \NC x, y, w, h   \NC
%D     adds area to the list                  \NC \NR
%D \NC TextAreaX,Y,W,H,XY,WH \NC x and/or y   \NC
%D     reports offsets and dimensions         \NC \NR
%D \stoptabulate
%D
%D The \type {TextArea*} macros can be used to determine
%D overlap.

\newcount\currentMPtextareadata

\newtoks\MPsavedtextareadata
\newtoks\MPtextareadata
\newtoks\MPlocaltextareadata

% optimaliseren voor herhaling

\def\registerMPtextarea#1%
  {\ifpositioning
     \bgroup
     \global\advance\currentMPtextareadata\plusone
    %\hpos{gbd:\the\currentMPtextareadata}{#1}%
     \hpos{gbd:\the\currentMPtextareadata}%
       {\iftracetextareas\boxrulewidth1.5pt\ruledhbox\fi{#1}}%
     \edef\!!stringa{gbd:\the\currentMPtextareadata}%
     \edef\!!stringa{RegisterTextArea(%
       \MPx\!!stringa,\MPy\!!stringa,%
       \MPw\!!stringa,\MPh\!!stringa,\MPd\!!stringa);}%
     \@EA \doglobal \@EA \appendtoks \!!stringa \to \MPtextareadata
     \egroup
   \else
     \hbox{#1}%
   \fi}

\def\registerMPlocaltextarea#1%
  {\ifpositioning
     \bgroup
     \global\advance\currentMPtextareadata\plusone
    %\hpos{gbd:\the\currentMPtextareadata}{#1}%
     \hpos{gbd:\the\currentMPtextareadata}%
       {\iftracetextareas\boxrulewidth3pt\ruledhbox\fi{#1}}%
     \edef\!!stringa{gbd:\the\currentMPtextareadata}%
     \edef\!!stringa{RegisterLocalTextArea(%
       \MPx\!!stringa,\MPy\!!stringa,%
       \MPw\!!stringa,\MPh\!!stringa,\MPd\!!stringa);}%
     \global\MPlocaltextareadata\@EA{\!!stringa}%
     \egroup
   \else
     \hbox{#1}%
   \fi}

% better, so that we can force a key and share with e.g. renumbering
%
% \let\namedtextarea\empty
%
% \def\registerMPlocaltextarea#1%
%   {\ifpositioning
%      \bgroup
%      \ifx\namedtextarea\empty
%        \global\advance\currentMPtextareadata\plusone
%        \edef\namedtextarea{gbd:\the\currentMPtextareadata}%
%      \fi
%      \hpos\namedtextarea{\iftracetextareas\boxrulewidth3pt\ruledhbox\fi{#1}}%
%      \edef\ascii{RegisterLocalTextArea(%
%        \MPx\namedtextarea,\MPy\namedtextarea,%
%        \MPw\namedtextarea,\MPh\namedtextarea,\MPd\namedtextarea);}%
%      \global\MPlocaltextareadata\@EA{\ascii}%
%      \egroup
%    \else
%      \hbox{#1}%
%    \fi}

\def\resetMPlocaltextarea
  {\global\MPlocaltextareadata\emptytoks}

\startMPextensions
   path PlainTextArea;
\stopMPextensions

\startMPinitializations
  ResetTextAreas;
  \the\MPsavedtextareadata;
  SaveTextAreas;
  ResetTextAreas;
  \the\MPtextareadata;
  \the\MPlocaltextareadata;
  PlainTextArea:=boundingbox(\MPxy{text:\realfolio}--\MPxy{text:\realfolio}
    shifted (\MPw{text:\realfolio},\MPh{text:\realfolio}));
\stopMPinitializations

\appendtoks
  \global\MPsavedtextareadata\MPtextareadata
  \global\MPtextareadata     \emptytoks
  \global\MPlocaltextareadata\emptytoks
\to \everyshipout

\protect \endinput
