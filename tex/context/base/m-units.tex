%D \module
%D   [       file=m-units,
%D        version=1997.03.19,
%D          title=\CONTEXT\ Extra Modules,
%D       subtitle=Scientific Units,
%D         author={Hans Hagen \& Ton Otten},
%D           date=\currentdate,
%D      copyright={PRAGMA / Hans Hagen \& Ton Otten}]
%C
%C This module is part of the \CONTEXT\ macro||package and is
%C therefore copyrighted by \PRAGMA. See mreadme.pdf for
%C details.

%D Scientific units can be typeset in math mode pretty well,
%D but occasionally one has to take care of spacing.
%D Furthermore, entering these units is not that natural as
%D wanted. Therefore this module presents a more natural way of
%D doing things, like:
%D
%D \starttyping
%D 1.23 \Cubic \Meter \Per \Second
%D \stoptyping
%D
%D This example shows that we use the order in which we say
%D things, instead of typeset things. There is a separate
%D manual for this module.

%D Message number~1 deals with overruling the \type {\Degrees}
%D macro defined in the core modules. Let's say that this is
%D an upward compatibility issue.

\startmessages  dutch  library: units
  title: eenheden
      1: gebruik \string\Degrees\space\string\Celsius\space in plaats van \string\Celsius !
\stopmessages

\startmessages  english  library: units
  title: units
      1: use \string\Degrees\space\string\Celsius\space instead of \string\Celsius !
\stopmessages

\startmessages  german  library: units
  title: Einheiten
      1: Verwende \string\Degrees\space\string\Celsius\space statt \string\Celsius !
\stopmessages

\startmessages  italian  library: units
  title: unita
      1: usare \string\Degrees\space\string\Celsius\space invece di \string\Celsius !
\stopmessages

\startmessages  norwegian  library: units
  title: enheter
      1: bruk \string\Degrees\space\string\Celsius\space istedenfor \string\Celsius !
\stopmessages

\startmessages  romanian  library: units
  title: unitati
      1: folositi \string\Degrees\space\string\Celsius\space in locul \string\Celsius !
\stopmessages

\startmessages  french  library: units
  title: unitas
      1: utilisez \string\Degrees\space\string\Celsius\space A la place de \string\Celsius !
\stopmessages

\unprotect

%D This runtime loadable module implements a way of defining
%D units. The core macro is \type {\dimension}, a rather clever
%D one that is able to cooperate with some other dimension
%D related macros. As said, this module enables user to enter:
%D
%D \startbuffer
%D some 10 \Square \Meter \Per \Second or more
%D \stopbuffer
%D
%D \getbuffer
%D
%D as:
%D
%D \typebuffer
%D
%D The units itself are implemented as synonyms.
%D
%D \starttyping
%D \definesynonyms [unit] [units] [\unitmeaning]
%D \setupsynonyms [unit] [textstyle=\dimension]
%D \stoptyping
%D
%D This definition means that we can ask for the meaning of a
%D unit using \type {\unitmeaning} and get a list of used
%D units by saying \type {\placelistofunits}
%D
%D We have to use the command \type {\unitmeaning} instead
%D of \type {\meaning}, simply because the latter is a \TEX\
%D primitive we don't want to loose. We use the label text
%D mechanism for translations.

%D \macros
%D   {dimension}
%D
%D The core of this module is the low level macro \type
%D {\dimension}. Before presenting this macro, it's best to
%D look at some applications, because it's supposed to show
%D some intelligence that can beter be understood from the
%D context.
%D
%D The next useless examples show some of the cases we want
%D to handle in a proper way.
%D
%D \startbuffer
%D ... 10 \Square \Meter \Per \Volt \
%D ... 10 \Square \Meter \Volt \
%D ... 10 \Meter \Volt \
%D ... 10 \Milli \Square \Meter \Per \Volt \
%D ... 10 \Square \Milli \Meter \Per \Volt \
%D ... 10 \Meter \Times \Meter \
%D ... 10 \Square \Meter \Times \Meter \
%D ... 10 \Square \Milli \Meter \Times \Meter \
%D \stopbuffer
%D
%D \typebuffer
%D
%D Becomes:
%D
%D \getbuffer

%D \macros
%D   {mathematicstypeface, dimensiontypeface,
%D    dimensionhalfspace, dimensionbackspace}
%D
%D There are some low level constants, that can be changed
%D when needed. Some day I will write a decent setup command.

\def\mathematicstypeface{\rm}
\def\dimensiontypeface  {\tf}

\def\dimensionhalfspace {\ifmmode\,\else\hskip+.1em\relax\fi}
\def\dimensionbackspace {\ifmmode\!\else\hskip-.1em\relax\fi}

%D \macros
%D   {smashdimensionpower}
%D
%D Sometimes the baseline distance is not enough to provide
%D for superscripts, so we smash their height by default.

\newif\ifsmashdimensionpower \smashdimensionpowertrue

%D The dimension mechanism uses a lot of signals to keep
%D track if the current state.

\newsignal\dimensionsignal
\newsignal\dimensionpowersignal
\newsignal\dimensionmidfixsignal
\newsignal\dimensionaddfixsignal

\let\thedimensionprefix = \empty
\let\thedimensionpower  = \empty

%D \macros
%D   {spaceddimensions,textdimensions}
%D
%D The actual definition of \type {\dimension} overruled the
%D one in the core modules. The boolean can be used to
%D force spacing between units. Vergelijk {\Newton \Meter}
%D eens met {\spaceddimensionstrue \Newton \Meter}. The
%D rather ugly test prevents problems with nested dimensions.

\newif\ifspaceddimensions \spaceddimensionsfalse % user switch
\newif\iftextdimensions   \textdimensionsfalse   % user switch

%D You can see the consequence of forcing text dimensions
%D when you compare the following code:
%D
%D \starttyping
%D {\rm test \Square \Meter \Per \Second\ ziezo\Degrees} \par
%D {\ss test \Square \Meter \Per \Second\ ziezo} \par
%D {\tt test \Square \Meter \Per \Second\ ziezo}
%D
%D \textdimensionstrue
%D
%D {\rm test \Square \Meter \Per \Second\ ziezo} \par
%D {\ss test \Square \Meter \Per \Second\ ziezo} \par
%D {\tt test \Square \Meter \Per \Second\ ziezo}
%D \stoptyping

\newif\ifnesteddimension  \nesteddimensionfalse  % local switch

\def\dodimensionpower#1%
  {\iftextdimensions\expandafter\high\else\expandafter^\fi{#1}}

\def\ustartmathmode      {\iftextdimensions\else\expandafter\startmathmode      \fi}
\def\ustopmathmode       {\iftextdimensions\else\expandafter\stopmathmode       \fi}
\def\umathematicstypeface{\iftextdimensions\else\expandafter\mathematicstypeface\fi}

%D In forced text mode, we ignore spacing in monospaced fonts.

\def\udimensionhalfspace {\dodimensionspace\dimensionhalfspace}
\def\udimensionbackspace {\dodimensionspace\dimensionbackspace}

\def\dodimensionspace
  {\iftextdimensions
     \begingroup
     \setbox0\hbox{i}%
     \setbox2\hbox{m}%
     \ifdim\wd0=\wd2
       \endgroup
       \@EAEAEA\gobbleoneargument
     \else
       \endgroup
     \fi
   \fi}

\unexpanded\def\dimension#1%
  {\begingroup
   \global\let\savedthedimensionprefix\thedimensionprefix
   \global\let\savedthedimensionpower\thedimensionpower
   \unexpanded\def\dimension##1{\global\nesteddimensiontrue}%
   \let\dimensionprefix\dimension
   \let\dimensionmidfix\dimension
   \let\dimensionsuffix\dimension
   \let\dimensionpower \dimension
   \global\nesteddimensionfalse
   \setbox\scratchbox\hbox{\ustartmathmode#1\ustopmathmode}% pre-roll
   \global\let\thedimensionprefix\savedthedimensionprefix
   \global\let\thedimensionpower \savedthedimensionpower
   \endgroup
   \ifnesteddimension#1\else\dodimension{#1}\fi}

\def\dodimension#1%
  {\dontbreakdimension
   \ifdim\scratchdimen=\zeropoint\relax
     \ifmmode
       \udimensionhalfspace
       \udimensionhalfspace
     \fi
     \ustartmathmode\dimensiontypeface
   \else
     \ustartmathmode\dimensiontypeface
     \ifspaceddimensions
       \ifdim\scratchdimen=\dimensionsignal\relax
         \udimensionhalfspace
\else\ifdim\scratchdimen=\dimensionpowersignal\relax
  \udimensionhalfspace
\fi
       \fi
     \fi
   \fi
   \umathematicstypeface\thedimensionprefix#1%
   \ifx\thedimensionpower\empty
   \else\ifsmashdimensionpower
     \setbox\scratchbox=\hbox
       {\iftextdimensions
          \tx\thedimensionpower
        \else
          $\scriptstyle\thedimensionpower$%
        \fi}%
     \ht\scratchbox=\zeropoint
     \dodimensionpower{\box\scratchbox}%
   \else
     \dodimensionpower{\thedimensionpower}%
   \fi\fi
   \ustopmathmode
   % otherwise nobreak before space in 2 \Milli \Meter\ blabla
   \doifnotmode{atpragma}{\nobreak}% this was always \nobreak
   % only test this at pragma
   \ifx\thedimensionpower\empty
     \hskip\dimensionsignal
   \else
     \hskip\dimensionpowersignal
   \fi
   \global\let\thedimensionprefix\empty
   \global\let\thedimensionpower\empty}


%D \macros
%D   {dontbreakdimension,
%D    dimensionprefix, dimensionaddfix,
%D    dimensionnopfix, dimensionmidfix,
%D    dimensionpower}
%D
%D Here are some auxilliary macros.

\def\dontbreakdimension
  {\scratchdimen\lastskip
   \unskip
   \nobreak
   \hskip\scratchdimen
   \nobreak}

\def\dimensionprefix#1%
  {\gdef\thedimensionprefix{#1}}

\def\dimensionaddfix#1%
  {\unskip
  %\mathematics{\umathematicstypeface#1}%
   \ustartmathmode\umathematicstypeface#1\ustopmathmode
   \nobreak
   \hskip\dimensionaddfixsignal}

\def\dimensionnopfix#1%
  {\dontbreakdimension
   \ifdim\scratchdimen=\dimensionpowersignal\relax
     \ustartmathmode
   \else
     \ustartmathmode
     \udimensionhalfspace
     \nobreak
   \fi
   \umathematicstypeface#1%
   \ustopmathmode
   \nobreak
   \hskip\dimensionsignal}

\def\dimensionmidfix#1%
  {\dontbreakdimension
   \ifdim\scratchdimen=\dimensionpowersignal\relax
     \ustartmathmode
     \udimensionbackspace
     \nobreak
   \else
     \ustartmathmode
   \fi
   \umathematicstypeface#1%
   \ustopmathmode
   \nobreak
   \hskip\dimensionmidfixsignal}

\def\dimensionpower#1%
  {\gdef\thedimensionpower{#1}}

%D \macros
%D   {SIunits, noSI, doSI}
%D
%D Some low level unit switching macros:

\newif\ifSIunits \SIunitstrue

\def\noSI#1{\begingroup\SIunitsfalse#1\endgroup}
\def\doSI#1{\begingroup\SIunitstrue #1\endgroup}

%D \macros
%D   {Degrees}
%D
%D We can fake the degrees symbol with:

\def\Degrees{\dimensionaddfix{\mathematics{^\circ}}}

%D \macros
%D   {Unit, NoUnit}
%D
%D When a dimension has no leading number, we can use \type
%D {\Unit}, and when no unit is appended, \type {\NoUnit} is
%D to be used, just to prevent the prefix migrating to the
%D next occasion.

\def\Unit   {\hskip\dimensionsignal}
\def\NoUnit {\dimension{}}

%D The mechanism described at the top of this module, depends
%D on several dimensional components, like prefixes:

\def\Atto    {\dimensionprefix{a}}
\def\Femto   {\dimensionprefix{f}}
\def\Pico    {\dimensionprefix{p}}
\def\Nano    {\dimensionprefix{n}}
\def\Micro   {\dimensionprefix{\iftextdimensions u\else\mu\fi}}
\def\Milli   {\dimensionprefix{m}}
\def\Centi   {\dimensionprefix{c}}
\def\Deci    {\dimensionprefix{d}}
\def\Hecto   {\dimensionprefix{h}}
\def\Kilo    {\dimensionprefix{k}}
\def\Mega    {\dimensionprefix{M}}
\def\Giga    {\dimensionprefix{G}}
\def\Terra   {\dimensionprefix{T}} % for old times sake
\def\Tera    {\dimensionprefix{T}}
\def\Peta    {\dimensionprefix{P}}
\def\Exa     {\dimensionprefix{E}}

%D and operators:

\def\Times   {\dimensionnopfix{\iftextdimensions.\else\cdot\fi}}
\def\Solidus {\dimensionmidfix{/}}
\def\Per     {\dimensionmidfix{/}}
\def\OutOf   {\dimensionnopfix{:}}

%D  and suffixes:

\def\Linear  {\dimensionpower{1}}
\def\Square  {\dimensionpower{2}}
\def\Cubic   {\dimensionpower{3}}

\def\Inverse {\dimensionpower{-1}}
\def\ILinear {\dimensionpower{-1}}
\def\ISquare {\dimensionpower{-2}}
\def\ICubic  {\dimensionpower{-3}}

%D Apart from these components, the units themselves are
%D defined using the synonym mechanism. First we define some
%D length and volume related units.

\getvalue{\v!unit} [Meter]  {m}             {meter}
\getvalue{\v!unit} [pMeter] {\Pico  \Meter} {picometer}
\getvalue{\v!unit} [nMeter] {\Nano  \Meter} {nanometer}
\getvalue{\v!unit} [uMeter] {\Micro \Meter} {micrometer}
\getvalue{\v!unit} [mMeter] {\Milli \Meter} {millimeter}
\getvalue{\v!unit} [cMeter] {\Centi \Meter} {centimeter}
\getvalue{\v!unit} [dMeter] {\Deci  \Meter} {decimeter}
\getvalue{\v!unit} [hMeter] {\Hecto \Meter} {hectometer}
\getvalue{\v!unit} [kMeter] {\Kilo  \Meter} {kilometer}

%D After some discussion on the \CONTEXT\ mailing list in
%D february 2002 it was decided to go from L to l for liters
%D (Karel Wesselings alternative: \mathematics{\ell}).

\getvalue{\v!unit} [Liter]  {l}             {liter}
\getvalue{\v!unit} [mLiter] {\Milli \Liter} {milliliter}
\getvalue{\v!unit} [cLiter] {\Centi \Liter} {centiliter}
\getvalue{\v!unit} [dLiter] {\Deci  \Liter} {deciliter}

%D Next we define time related units (\type {\ifSI} still dutch only).

\getvalue{\v!unit} [Sec]   {s}                       {\labeltext{u:sec}}
\getvalue{\v!unit} [fSec]  {\Femto \Sec}             {\labeltext{u:fsec}}
\getvalue{\v!unit} [pSec]  {\Pico \Sec}              {\labeltext{u:psec}}
\getvalue{\v!unit} [nSec]  {\Nano \Sec}              {\labeltext{u:nsec}}
\getvalue{\v!unit} [uSec]  {\Micro \Sec}             {\labeltext{u:usec}}
\getvalue{\v!unit} [mSec]  {\Milli \Sec}             {\labeltext{u:msec}}
\getvalue{\v!unit} [Year]  {\ifSIunits a \else j\fi} {\labeltext{u:year}}
\getvalue{\v!unit} [Month] {m}                       {\labeltext{u:month}}
\getvalue{\v!unit} [Week]  {w}                       {\labeltext{u:week}}
\getvalue{\v!unit} [Day]   {d}                       {\labeltext{u:day}}
\getvalue{\v!unit} [Hour]  {\ifSIunits h \else u\fi} {\labeltext{u:hour}}
\getvalue{\v!unit} [Min]   {min}                     {\labeltext{u:min}}

\setuplabeltext
  [\s!nl]
  [u:sec=seconde,
   u:psec=picoseconde,
   u:fsec=femtoseconde,
   u:nsec=nanoseconde,
   u:usec=microseconde,
   u:msec=milliseconde,
   u:year=jaar,
   u:month=maand,
   u:week=week,
   u:day=dag,
   u:hour=uur,
   u:min=minuten]

\setuplabeltext
  [\s!en]
  [u:sec=second,
   u:fsec=femtosecond,
   u:psec=picosecond,
   u:nsec=nanosecond,
   u:usec=microsecond,
   u:msec=millisecond,
   u:year=year,
   u:month=month,
   u:week=week,
   u:day=day,
   u:hour=hour,
   u:min=minutes]

\setuplabeltext
  [\s!de]
  [u:sec=Sekunde,
   u:fsec=Femtosekunde,
   u:psec=Picosekunde,
   u:nsec=Nanosekunde,
   u:usec=Microsekunde,
   u:msec=Millisekunde,
   u:year=Jahr,
   u:month=Monat,
   u:week=Woche,
   u:day=Tag,
   u:hour=Stunde,
   u:min=Minuten]

\setuplabeltext
  [\s!it]
  [u:sec=secondo,
   u:fsec=femtosecondo,
   u:psec=picosecondo,
   u:nsec=nanosecondo,
   u:usec=microsecondo,
   u:msec=millisecondo,
   u:year=anno,
   u:month=mese,
   u:week=settimana,
   u:day=giorno,
   u:hour=ora,
   u:min=minuti]

%D Then we define some angles.

\getvalue{\v!unit} [Rad] {rad}                    {\labeltext{u:rad}}
\getvalue{\v!unit} [Deg] {{\mathematics{^\circ}}} {\labeltext{u:deg}}

\setuplabeltext
  [\s!nl]
  [u:rad=hoek radialen,
   u:deg=hoek graden]

\setuplabeltext
  [\s!en]
  [u:rad=angle radians,
   u:deg=angle degrees]

\setuplabeltext
  [\s!de]
  [u:rad=Bogenma\SS,
   u:deg=Gradma\SS]

\setuplabeltext
  [\s!it]
  [u:rad=radianti,
   u:deg=angoli sessagesimali]

%D Rotation and frequency related units are defined by:

\getvalue{\v!unit} [Hertz]     {Hz}            {Hertz}
\getvalue{\v!unit} [kHertz]    {\Kilo  \Hertz} {kilo Hertz}
\getvalue{\v!unit} [MHertz]    {\Mega  \Hertz} {mega Hertz}
\getvalue{\v!unit} [GHertz]    {\Giga  \Hertz} {giga Hertz}
\getvalue{\v!unit} [THertz]    {\Tera  \Hertz} {tera Hertz}
\getvalue{\v!unit} [mHertz]    {\Milli \Hertz} {milli Hertz}

\getvalue{\v!unit} [RevPerSec] {RPS} {\labeltext{u:rps}}
\getvalue{\v!unit} [RevPerMin] {RPM} {\labeltext{u:rpm}}

\setuplabeltext
  [\s!nl]
  [u:rps=omwentelingen per seconde,
   u:rpm=omwentelingen per minuut]

\setuplabeltext
  [\s!en]
  [u:rps=revolutions per second,
   u:rpm=revolutions per minute]

\setuplabeltext
  [\s!de]
  [u:rps=Umdrehungen pro Sekunde,
   u:rpm=Umdrehungen pro Minute]

\setuplabeltext
  [\s!it]
  [u:rps=giri al secondo,
   u:rpm=giri al minuto]

%D Mass and force:

\getvalue{\v!unit} [Gram]    {g}              {gram}
\getvalue{\v!unit} [uGram]   {\Micro \Gram}   {microgram}
\getvalue{\v!unit} [mGram]   {\Milli \Gram}   {milligram}
\getvalue{\v!unit} [kGram]   {\Kilo  \Gram}   {kilogram}
\getvalue{\v!unit} [Atom]    {u}              {\labeltext{u:u}}

\getvalue{\v!unit} [Newton]  {N}              {Newton}
\getvalue{\v!unit} [kNewton] {\Kilo \Newton}  {kilo Newton}

\getvalue{\v!unit} [Pascal]  {Pa}             {Pascal}
\getvalue{\v!unit} [mPascal] {\Milli \Pascal} {milli Pascal}
\getvalue{\v!unit} [kPascal] {\Kilo  \Pascal} {kilo Pascal}

\setuplabeltext
  [\s!nl]
  [u:u=atomaire massa eenheid]

\setuplabeltext
  [\s!en]
  [u:u=atom mass unit]

\setuplabeltext
  [\s!de]
  [u:u=Atomare Masseneinheit]

\setuplabeltext
  [\s!it]
  [u:u=unit\`a di massa atomica]

%D Energy units comes in two alternatives:

\getvalue{\v!unit} [Joule]  {J}             {Joule}
\getvalue{\v!unit} [mJoule] {\Milli \Joule} {milli Joule}
\getvalue{\v!unit} [kJoule] {\Kilo  \Joule} {kilo Joule}
\getvalue{\v!unit} [MJoule] {\Mega  \Joule} {mega Joule}
\getvalue{\v!unit} [GJoule] {\Giga  \Joule} {giga Joule}

\getvalue{\v!unit} [Watt]  {W}              {Watt}
\getvalue{\v!unit} [mWatt] {\Milli \Watt}   {milli Watt}
\getvalue{\v!unit} [kWatt] {\Kilo  \Watt}   {kilo Watt}
\getvalue{\v!unit} [MWatt] {\Mega  \Watt}   {mega Watt}
\getvalue{\v!unit} [GWatt] {\Giga  \Watt}   {giga Watt}
\getvalue{\v!unit} [TWatt] {\Tera  \Watt}   {tera Watt}

%D Although Celsius is no longer permitted, we define it by
%D saying:

\getvalue{\v!unit} [Celsius]    {C} {Celsius}
\getvalue{\v!unit} [Kelvin]     {K} {Kelvin}
\getvalue{\v!unit} [Fahrenheit] {F} {Fahrenheit}

%D Some chemic related units are:

\getvalue{\v!unit} [Mol]         {mol}                {mol}
\getvalue{\v!unit} [mMol]        {\Milli \Mol}        {millimol}
\getvalue{\v!unit} [kMol]        {\Kilo  \Mol}        {kilomol}
\getvalue{\v!unit} [Molair]      {M}                  {molair (\Mol \Per \Liter)}
\getvalue{\v!unit} [Equivalent]  {eq}                 {equivalent}
\getvalue{\v!unit} [mEquivalent] {\Milli \Equivalent} {milli equivalent}

%D There are quite a lot units related to electricity and
%D magnetism:

\getvalue{\v!unit} [Farad]   {F}             {Farad}
\getvalue{\v!unit} [pFarad]  {\Pico  \Farad} {pico Farad}
\getvalue{\v!unit} [nFarad]  {\Nano  \Farad} {nano Farad}
\getvalue{\v!unit} [uFarad]  {\Micro \Farad} {micro Farad}
\getvalue{\v!unit} [mFarad]  {\Milli \Farad} {milli Farad}

\getvalue{\v!unit} [Ohm]     {\Omega}        {Ohm}
\getvalue{\v!unit} [kOhm]    {\Kilo \Ohm}    {kilo Ohm}

\getvalue{\v!unit} [Siemens] {S}              {Siemens}

\getvalue{\v!unit} [Ampere]  {A}              {Amp\`ere}
\getvalue{\v!unit} [mAmpere] {\Milli \Ampere} {milli Amp\`ere}

\getvalue{\v!unit} [Coulomb] {C}              {Coulomb}

\getvalue{\v!unit} [Volt]    {V}              {Volt}
\getvalue{\v!unit} [mVolt]   {\Milli \Volt}   {milli Volt}
\getvalue{\v!unit} [kVolt]   {\Kilo  \Volt}   {kilo Volt}
\getvalue{\v!unit} [eVolt]   {eV}             {electronvolt}
\getvalue{\v!unit} [keVolt]  {\Kilo \eVolt}   {kilo electronvolt}
\getvalue{\v!unit} [MeVolt]  {\Mega \eVolt}   {mega electronvolt}

\getvalue{\v!unit} [Tesla]   {T}              {Tesla}

\getvalue{\v!unit} [VoltAC]  {V_{\xbox{ac}}}  {\labeltext{u:vac}}
\getvalue{\v!unit} [VoltDC]  {V_{\xbox{dc}}}  {\labeltext{u:vdc}}

\setuplabeltext
  [\s!nl]
  [u:vac=wisselspanning,
   u:vdc=gelijkspanning]

\setuplabeltext
  [\s!en]
  [u:vac=alternating current,
   u:vdc=direct current]

\setuplabeltext
  [\s!de]
  [u:vac=Wechselspannung,
   u:vdc=Gleichspannung]

\setuplabeltext
  [\s!it]
  [u:vac=corrente alternata,
   u:vdc=corrente continua]

%D Computer memory is specified in Bytes:

\getvalue{\v!unit} [Baud]  {Baud}        {Baud (Bit/s)}
\getvalue{\v!unit} [Bit]   {Bit}         {Bit}

\getvalue{\v!unit} [Byte]  {Byte}        {Byte}
\getvalue{\v!unit} [kByte] {\Kilo \Byte} {kilo Byte}
\getvalue{\v!unit} [MByte] {\Mega \Byte} {mega Byte}
\getvalue{\v!unit} [GByte] {\Giga \Byte} {giga Byte}
\getvalue{\v!unit} [TByte] {\Tera \Byte} {tera Byte}

%D Some radiation related units:

\getvalue{\v!unit} [Bequerel]  {Bq}              {Bequerel}
\getvalue{\v!unit} [MBequerel] {\Mega \Bequerel} {Bequerel}
\getvalue{\v!unit} [Sievert]   {Sv}              {Sievert}
\getvalue{\v!unit} [mSievert]  {\Milli \Sievert} {milli Sievert}

%D Light:

\getvalue{\v!unit} [Candela] {cd} {Candela}

%D and some sound ones:

\getvalue{\v!unit} [Bell]  {B}           {Bell}
\getvalue{\v!unit} [dBell] {\Deci \Bell} {decibel}

%D We also define some non||regular, sometimes even forbidden,
%D units:

\getvalue{\v!unit} [At]    {at}         {\labeltext{u:at}}
\getvalue{\v!unit} [Atm]   {atm}        {\labeltext{u:atm}}
\getvalue{\v!unit} [Bar]   {bar}        {bar (100 \Kilo \Pascal)}
\getvalue{\v!unit} [EVolt] {eV}         {electronvolt}
\getvalue{\v!unit} [Foot]  {ft}         {\labeltext{u:ft}}
\getvalue{\v!unit} [Inch]  {inch}       {inch}
\getvalue{\v!unit} [Cal]   {cal}        {\labeltext{u:cal}}
\getvalue{\v!unit} [Force] {f}          {\labeltext{u:f}}
\getvalue{\v!unit} [kCal]  {\Kilo \Cal} {\labeltext{u:kcal}}
\getvalue{\v!unit} [Lux]   {lux}        {lux}


\def\xPercent {\dimensionaddfix{\percent }}
\def\xPromille{\dimensionaddfix{\promille}}

\getvalue{\v!unit} [Percent]  {\xPercent } {percent}
\getvalue{\v!unit} [Permille] {\xPromille} {promille}
\getvalue{\v!unit} [Promille] {\xPromille} {promille}

%D Some more, thanks to Tobias:

\getvalue{\v!unit} [Gray]     {Gr}  {Gray}
\getvalue{\v!unit} [Weber]    {Wb}  {Weber}
\getvalue{\v!unit} [Henry]    {H}   {Henry}
\getvalue{\v!unit} [Sterant]  {sr}  {Sterant}
\getvalue{\v!unit} [Angstrom] {\hbox{\Aring}} {\Aring ngstr\"om}
\getvalue{\v!unit} [Gauss]    {G}   {Gauss}

\setuplabeltext
  [\s!nl]
  [u:at=technische atmosfeer,
   u:atm=fysische atmosfeer,
   u:ft=voet,
   u:cal=calorie,
   u:f=kracht (force),
   u:kcal=kilocalorie]

\setuplabeltext
  [\s!en]
  [u:at=technical atmosferic pressure,
   u:atm=physical atmosferic pressure,
   u:ft=foot,
   u:cal=calory,
   u:f=force,
   u:kcal=kilocalory]

\setuplabeltext
  [\s!de]
  [u:at=Technischer atmosph\"arischer Druck,
   u:atm=physkalischer atmosph\"arischer Druck
   u:ft=Fu\SS,
   u:cal=Kalorien,
   u:f=Force,
   u:kcal=Kilokalorien]

\setuplabeltext
  [\s!it]
  [u:at=pressione atmosferica tecnica,
   u:atm=pressione atmosfera fisica,
   u:ft=piede,
   u:cal=caloria,
   u:f=forza,
   u:kcal=chilocaloria]

%D Here are some old ones, still there for compatibility
%D reasons. These will probably be obsolete in a few years.

\def\MeterTwee    {\Square        \Meter}
\def\mMeterTwee   {\Square \Milli \Meter}
\def\cMeterTwee   {\Square \Centi \Meter}
\def\dMeterTwee   {\Square \Deci  \Meter}
\def\kMeterTwee   {\Square \Kilo  \Meter}

\def\MeterDrie    {\Cubic         \Meter}
\def\mMeterDrie   {\Cubic  \Milli \Meter}
\def\cMeterDrie   {\Kubic  \Centi \Meter}
\def\dMeterDrie   {\Cubic  \Deci  \Meter}
\def\kMeterDrie   {\Cubic  \Kilo  \Meter}

\def\LiterTwee    {\Square        \Liter}
\def\SecTwee      {\Square        \Sec}
\def\SecMinEen    {\Inverse       \Sec}

%D To make ourselves happy, we define some dutch specific
%D units:

\startinterface dutch

  \getvalue{\v!unit} [PaardenKracht] {pk}                 {paardenkracht}
  \getvalue{\v!unit} [Duits]         {D}                  {duits}
  \getvalue{\v!unit} [Kwik]          {Hg}                 {kwikkolom}
  \getvalue{\v!unit} [Hectare]       {ha}                 {hectare}
  \getvalue{\v!unit} [kGramForce]    {\Kilo \Gram \Force} {kilogramforce}
  \getvalue{\v!unit} [kWattUur]      {\Kilo \Watt \Uur}   {kilowattuur}
  \getvalue{\v!unit} [MeterKwik]     {\Meter \Kwik}       {meter kwikkolom}
  \getvalue{\v!unit} [Waterkolom]    {WK}                 {waterkolom}
  \getvalue{\v!unit} [MeterWater]    {\Meter \Waterkolom} {meter waterkolom}
  \getvalue{\v!unit} [DrogeStof]     {ds}                 {droge stof}
  \getvalue{\v!unit} [Normaal]       {N}                  {normaal}

  \getvalue{\v!unit} [Ton]           {t}                  {ton}
  \getvalue{\v!unit} [kTon]          {\Kilo \Ton}         {kiloton}

  \let \OmwPerSec    \RevPerSec
  \let \OmwPerMin    \RevPerMin
  \let \Graden       \Deg
  \let \PaardeKracht \PaardenKracht
  \let \Atoom        \Atom
  \let \Heure        \Hour
  \let \Jaar         \Year
  \let \Maand        \Month
  \let \Dag          \Day
  \let \Uur          \Hour

\stopinterface

%D Finally we define some equivalents. By using \type {\let}
%D we can be sure that they don't end up double in the lists of
%D units.

\let \Second \Sec
\let \Kubic  \Cubic
\let \IKubic \ICubic

%D Option:

% \def\Micro{\dimensionprefix{\iftextdimensions\mathematics\mu  \else\mu  \fi}}
% \def\Times{\dimensionnopfix{\iftextdimensions\mathematics\cdot\else\cdot\fi}}

\protect \endinput
