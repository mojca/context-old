%D \module
%D   [       file=core-uti,
%D        version=1997.03.31,
%D          title=\CONTEXT\ Core Macros,
%D       subtitle=Utility File Handling,
%D         author=Hans Hagen,
%D           date=\currentdate,
%D      copyright={PRAGMA / Hans Hagen \& Ton Otten}]
%C
%C This module is part of the \CONTEXT\ macro||package and is
%C therefore copyrighted by \PRAGMA. See mreadme.pdf for
%C details.

\writestatus{loading}{Context Core Macros / Utility File Handling}

\unprotect

% todo : safe lan etc too
% todo : load all commands at once (tok)
% todo : merge status info patch into tui file (language, encoding, etc),

% Utility-file
%
% De onderstaande macro's ondersteunen het gebruik van de
% zogeheten utility-file. Alle extern onder te brengen
% informatie wordt opgeslagen in de file \jobname.tui, tenzij
% er selectief pagina's worden gezet. In dat geval wordt de
% file \jobname.tmp gebruikt. Informatie wordt ingelezen uit
% de file \jobname.tuo, welke door TeXUtil wordt aangemaakt.

\edef\utilityversion{1998.07.07} % was: 1996.03.15  % status variables
\edef\utilityversion{1998.12.20} % was: 1998.07.07  % index attributes
\edef\utilityversion{2003.07.19} % was: 1998.12.20  % object pages
\edef\utilityversion{2006.06.23} % was: 2003.07.19  % -- instead of :
\edef\utilityversion{2006.09.21} % pt in pos

% Bepaalde commando's worden als string weggeschreven. Deze
% zijn aan het eind van deze file gedefinieerd.

% Om een opbouw van spaties te voorkomen (???) moet ^^M een
% andere betekenis krijgen:
%
% \catcode`\^^M=14 (comment)
%
% read file
%
% \catcode`\^^M=5  (end of line)

\newwrite\utility@tui
\newif\ifutilitydone

\def\@@utilityerrormessage
  {\showmessage\m!systems8\empty
   \globallet\@@utilityerrormessage\relax}

\def\thisisutilityversion#1%
  {\doifelse\utilityversion{#1}%
     {\checksectionseparator}
     {\@@utilityerrormessage\resetutilities\endinput}}

\def\checksectionseparator                  % catches backward compatibility conflict
  {}% \doifnot\sectionseparator:\endinput}  % this dependency may go in a few years

\def\dosplitofffoliopart[#1--#2--#3]{#3}

\def\thisissectionseparator#1%
  {\bgroup
   \globallet\checksectionseparator\relax
   \convertcommand \sectionseparator\to\asciia
   \convertargument               #1\to\asciib
   \expanded{\gdef\noexpand\dosplitofffoliopart[####1\sectionseparator
     \sectionseparator####2\sectionseparator\sectionseparator####3]{####3}}%
   \ifx\asciia\asciib
     \egroup
   \else
     \egroup
     % todo \@@utilityerrormessage
     \resetutilities
     \endinput
   \fi}

\def\writeutility         {\write\utility@tui}
\def\writeutilitycommand#1{\write\utility@tui{c \string#1}}

% less tokens
%
% \def\immediatewriteutility       {\immediate\writeutility}
% \def\immediatewriteutilitycommand{\immediate\writeutilitycommand}
%
% more flexible (for overloading)

\def\immediatewriteutility         {\immediate\write\utility@tui}
\def\immediatewriteutilitycommand#1{\immediate\write\utility@tui{c \string#1}}

% as in:

\def\cwriteutility#1%
  {\write\utility@tui{\noexpand\checkedutility{\number\nofshipouts}{#1}}}

\def\cwriteutilitycommand#1%
  {\write\utility@tui{\noexpand\checkedutility{\number\nofshipouts}{c \string#1}}}

\let\checkedutility\secondoftwoarguments

\def\docheckedutility#1#2{\ifnum#1=\nofshipouts#2\else\letterpercent\fi}

\prependtoks
  \let\checkedutility\docheckedutility
\to \everybeforeshipout

% Better use marks.

\newtoks \everyopenutilities
\newtoks \everycloseutilities
\newtoks \everycheckutilities

\def\openutilities {\the\everyopenutilities }
\def\closeutilities{\the\everycloseutilities}
\def\checkutilities{\the\everycheckutilities}

\appendtoks
  \let\writeutility                \cwriteutility
  \let\writeutilitycommand         \cwriteutilitycommand
 %\let\immediatewriteutility       \cimmediatewriteutility
 %\let\immediatewriteutilitycommand\cimmediatewriteutilitycommand
  \let\checkutilities              \relax
\to \everycheckutilities

\appendtoks
  \immediate\openout\utility@tui\jobname.\f!inputextension
  \immediatewriteutilitycommand{\thisissectionseparator{\sectionseparator}}% for the moment
  \immediatewriteutilitycommand{\thisisutilityversion  {\utilityversion  }}% in this order
\to \everyopenutilities

\appendtoks
%   \immediate\closeout\utility@tui % niet echt nodig
  \reportutilityproblems
  % should be a message :
  \let\writeutilitycommand         \gobbleoneargument
  \let\writeutility                \gobbleoneargument
  \let\immediatewriteutilitycommand\gobbleoneargument
  \let\immediatewriteutility       \gobbleoneargument
\to \everycloseutilities

% \def\reopenutilities
%   {\immediate\closeout\utility@tui
%    \openutilities}

\def\abortutilitygeneration
  {\immediatewriteutilitycommand\utilitygenerationaborted
   \immediatewriteutility{q {quit}}}

\def\utilitygenerationaborted
  {\showmessage\m!systems{21}\empty
   \globallet\utilitygenerationaborted\endinput
   \gdef\reportutilityproblems{\showmessage\m!systems{22}\empty}%
   \endinput}

\let\savecurrentvalue  \gobbletwoarguments % mkii/mkiv
\let\initializevariable\gobbletwoarguments % mkii/mkiv

\appendtoks
    \globallet\initializevariable\gobbletwoarguments
\to \everyafterutilityread

\let\reportutilityproblems\relax
\let\utilityresetlist     \empty

\newtoks\utilityresetlist

\def\addutilityreset#1%
  {\@EA\appendtoks\csname\s!reset#1\endcsname\to\utilityresetlist}

\def\resetutilities
  {\the\utilityresetlist}

% #1=type
% #2=file
% #3=melding

% #4=voor
% #5=na

% Er wordt gegroepeerd. Als binnen een lijst (bijvoorbeeld) de
% \leftskip is aangepast, maar nog geen \par is gegeven, dan
% geldt buiten de groep de oude \leftskip. Aan #5 kan dan
% ook \par worden meegegeven om de paragraaf af te sluiten.

\newif\ifdoinpututilities
\newif\ifunprotectutilities   % voor't geval er \v!xxxxxx's zijn

% no longer needed, since texutil is now multi platform
%
% \def\utilitycheckmessage
%   {\showmessage\m!systems{12}{}%
%    \globallet\utilitycheckmessage=\relax}
%
% \def\saveutilityline#1 #2\txen% tricky maar ok, want achter \command
%   {\if     #1c% commands      % in \ascii staat een spatie; #1 kan
%      \write\scratchwrite{#2}% % \par in stringvorm zijn (eof)!
%    \else\if#1s% synoniems
%      \utilitycheckmessage
%    \else\if#1r% registers
%      \utilitycheckmessage
%    \fi\fi\fi}
%
% \def\checkutilityfile
%   {\doiflocfileelse{\jobname.\f!outputextension}
%      {}
%      {\doiflocfileelse{\jobname.\f!inputextension}
%         {\bgroup
%          \showmessage\m!systems{11}{}%
%          \openout\scratchwrite\jobname.\f!outputextension
%          \openlocin\scratchread{\jobname.\f!inputextension}%
%          \def\doprocessline%
%            {\ifeof\scratchread
%               \def\doprocessline{\closein\scratchread}%
%             \else
%               \read\scratchread to \ascii
%               \convertcommand\ascii\to\ascii
%               \expandafter\saveutilityline\ascii\txen
%             \fi
%             \doprocessline}%
%          \doprocessline
%          \closeout\scratchwrite
%          \egroup}
%         {}}}

\def\currentutilityfilename{\jobname}

% \long\def\doutilities#1#2#3#4#5% % introduceren in utility file
%   {\restorecatcodes
%    \resetutilities
%    % more than one utility thing can be handled in one pass,
%    % for instance lists, so we process ##1 as list
%    \def\douticommand##1{\csname\s!set##1\endcsname}%
%    \processcommacommand[#1]\douticommand
%    \begingroup
%    \def\currentutilityfilename{#2}%
%    \notesenabledfalse
%    \doinpututilitiestrue
%    \global\utilitydonefalse
%    \catcode`\\=\@@escape
%    \catcode`\{=\@@begingroup
%    \catcode`\}=\@@endgroup
%    \catcode`\%=\@@comment\relax
%    \pushendofline % geeft problemen zodra andere file wordt ingelezen
%    \ifunprotectutilities % nog nodig ?
%      \unprotect
%    \fi
%    \ifnum\catcode`\@=\@@active \else
%      \catcode`\@=\@@letter % permits expanded commands with \@'s
%    \fi
%    \ifnum\catcode`\!=\@@active \else
%      \catcode`\!=\@@letter % permits multilingual constants
%    \fi
%    #4%
%    \the\everybeforeutilityread
%    \readjobfile{#2.\f!outputextension}\donothing\donothing
%    \the\everyafterutilityread
%    #5%
%    \relax
%    \ifunprotectutilities
%      \protect
%    \fi
%    \popendofline
%    \ifutilitydone\else
%      \doifsomething{#3}
%        {\showmessage\m!systems9{{#3}}%
%         \doifconcepttracing
%           {\blank
%            \type{[\currentmessagetext]}%
%            \blank}}%
%    \fi
%    \endgroup}

% we need to pop and push, else problems with reading
% utility files (toc) in xml mode and (e.g.) in a toc
% entry doing a doifmode
%
% the following is not ok because we have no way to signal
% xml content (yet), so for the moment we use this:

\appendtoks
    \ifprocessingXML
        \processingXMLfalse
        \enableXML
        \catcode`\\=\@@escape
        \catcode`\{=\@@begingroup
        \catcode`\}=\@@endgroup
        \catcode`\%=\@@comment\relax
    \fi
\to \everybeforeutilityread

\long\def\doutilities#1#2#3#4#5% % introduceren in utility file
  {\resetutilities
   % more than one utility thing can be handled in one pass,
   % for instance lists, so we process ##1 as list
   \def\douticommand##1{\csname\s!set##1\endcsname}%
   \processcommacommand[#1]\douticommand
   \begingroup
   \def\currentutilityfilename{#2}%
   \notesenabledfalse
   \doinpututilitiestrue
   \global\utilitydonefalse
   \pushendofline % geeft problemen zodra andere file wordt ingelezen
   \pushcatcodetable
   \setcatcodetable\ctxcatcodes
   \ifunprotectutilities % nog nodig ?
     \unprotect
   \fi
   #4%
   \the\everybeforeutilityread
   \readjobfile{#2.\f!outputextension}\donothing\donothing
   \the\everyafterutilityread
   \popcatcodetable
   #5%
   \relax
   \ifunprotectutilities
     \protect
   \fi
   \popendofline
   \ifutilitydone\else
     \doifsomething{#3}
       {\showmessage\m!systems9{{#3}}%
        \doifconcepttracing
          {\blank
           \type{[\currentmessagetext]}%
           \blank}}%
   \fi
   \endgroup}

% Saving the sort vector:

% \def\savesortkeys
%   {\startnointerference
%    \def\flushsortkey##1##2##3##4%
%      {\convertargument{##1}{##2}{##3}{##4}\to\ascii
%       \immediatewriteutility{k {\currentlanguage}{\currentencoding}\ascii}}%
%    \let\definesortkey\flushsortkey
%    \flushsortkeys
%    \globallet\savesortkeys\relax
%    \stopnointerference}
%
% \prependtoks \savesortkeys \to \everystarttext

% Default-instellingen (verborgen)

\prependtoks \resetutilities \to \everyjob

% Experiment
%
% \installprogram{Hello World}
% \installprogram[hw]{Hello World}
% \installedprogram[hw]

\def\installprogram
  {\dosingleempty\doinstallprogram}

\def\doinstallprogram[#1]#2%
  {\doifelsenothing{#1}
     {\dodoinstallprogram{#2}}
     {\setvalue{\??up#1}{\dodoinstallprogram{#2}}}}

% \def\doinstallprogram[#1][#2]% less code
%   {\doifsomething{#1}{\setvalue{\??up#1}}{\dodoinstallprogram{#2}}}

\def\dodoinstallprogram#1%
  {\immediatewriteutility{e p {#1}}}

\def\installedprogram[#1]%
  {\getvalue{\??up#1}}

% \writeplugindata{texutil}{{alpha}}
% \writeplugindata{texutil}{{beta}}
% \writeplugindata{texutil}{{gamma}}
% \writeplugindata{texutil}{{delta}}
%
% \loadplugindata {plugintest}

\def\immediatewriteplugindata#1#2%
  {\immediatewriteutility{p u {#1} #2}}

\def\writeplugindata#1#2%
  {\writeutility{p u {#1} #2}}

\def\loadplugindata#1%
  {\doutilities{#1}\jobname\empty\relax\relax}

% \plugincommand{\command{}{}{}}
%
% this way we can catch undefined commands

\long\def\plugincommand#1%
  {\doplugincommand#1\relax}

\long\def\doplugincommand#1%
  {\ifx#1\undefined
     \expandafter\noplugincommand
   \else
     \expandafter#1%
   \fi}

% shorter:
%
% \long\def\doplugincommand#1%
%   {\ifx#1\undefined\expandafter\noplugincommand\fi#1}

\long\def\noplugincommand#1\relax
  {}

% \addutilityreset{plugintest}
%
% \def\resetplugintest{\let\plugintest\gobbletwoarguments}
% \def\setplugintest  {\let\plugintest\writestatus}
%
% \installplugin
%   {plugintest}
%   {\let\plugintest\gobbletwoarguments}
%   {\let\plugintest\writestatus}

\long\def\installplugin#1#2#3%
  {\addutilityreset          {#1}%
   \long\setvalue{\s!reset#1}{#2}%
   \long\setvalue{\s!set  #1}{#3}}

% plugins

\loadmarkfile{core-uti}

\protect \endinput
