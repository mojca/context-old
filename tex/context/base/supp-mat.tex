%D \module
%D   [       file=supp-mat,
%D        version=1998.09.10,
%D          title=\CONTEXT\ Support Macros,
%D       subtitle=Math,
%D         author=Hans Hagen,
%D           date=\currentdate,
%D      copyright={PRAGMA / Hans Hagen \& Ton Otten}]
%C
%C This module is part of the \CONTEXT\ macro||package and is
%C therefore copyrighted by \PRAGMA. See mreadme.pdf for
%C details.

%D For practical reasons, I decided to move some math things to
%D a support module. There is nothing spectacular here. It may move
%D back to math-ini.

\writestatus{loading}{Context Support Macros / Math}

\unprotect

%D \macros
%D   {mathematics, math, nomathematics, startmathmode}
%D
%D The \type{$} can be both an begin and end math character.
%D This can lead to confusing and errorprone situations when
%D macros insert \type{$}. When for instance we have defined:
%D
%D \starttyping
%D \def\MyPlus{$\,+\,$}
%D \stoptyping
%D
%D the sequence \type{$x^2 \MyPlus y^2 = r^2$} will expand to:
%D
%D \starttyping
%D $x^2 $\,+\,$ y^2 = r^2$
%D \stoptyping
%D
%D Here the \type{\,} are given outside math mode and \TEX\ will
%D definitely complain about this. A more save definition would
%D have been:
%D
%D \starttyping
%D \def\MyPlus{\mathematics{\,+\,}}
%D \stoptyping
%D
%D Which is implemented as:

\long\def\mathematics#1{\relax\ifmmode#1\else $#1$\fi} % lookahead bug reported by brooks
\long\def\displaymath#1{\relax\ifmmode#1\else$$#1$$\fi}
\long\def\inlinemath #1{\dontleavehmode\relax\ifmmode#1\else $#1$\fi}

\let\stopmathmode\relax

\def\startmathmode % cannot be used nested
  {\relax\ifmmode
     \let\stopmathmode\relax
   \else
     $\def\stopmathmode{$}%  \let\stopmathmode=$
   \fi}

\def\startmathmode % nested variant
  {\relax\ifmmode
     \begingroup\let\stopmathmode\endgroup
   \else
     $\let\stopmathmode$%
   \fi}

\def\nomathematics#1%
  {\relax\ifmmode\hbox{#1}\else#1\fi}

\let\math\mathematics

%D \macros
%D   {displaymathematics,inlinemathematics,automathematics}
%D
%D An example of usage of the following can be found in the MathML module:

\long\def\displaymathematics#1{\relax\ifmmode#1\else\dostartformula{}#1\dostopformula\fi}
     \def\inlinemathematics   {\dontleavehmode\mathematics}
     \def\automathematics     {\relax\ifhmode\@EA\inlinemathematics\else\@EA\displaymathematics\fi}

% better, esp when used in bTABLE ... eTABLE

\def\automathematics
  {\relax
   \ifhmode
     \expandafter\inlinemathematics
   \else\ifintable
     \expandafter\expandafter\expandafter\inlinemathematics
   \else
     \expandafter\expandafter\expandafter\displaymathematics
   \fi\fi}

%D \macros
%D   {dimension, nodimension}
%D
%D The next few macros are used for typesetting dimensions in
%D such a way that spacing is acceptable. I won't spend much
%D words on these macros, because they will be overloaded in
%D the units module.

\newsignal\dimensionsignal

\def\dimensiontypeface  {\tf}
\def\dimensionhalfspace {\,}

\unexpanded\def\dimension#1%
  {\def\dodimensionsignal{\kern\dimensionsignal}%
   \ifdim\lastskip=\zeropoint\relax
     \ifdim\lastkern=\zeropoint\relax
       \ifmmode
         \mathematics{\dimensionhalfspace\dimensionhalfspace\dimensiontypeface#1}%
       \else
         \mathematics{\dimensiontypeface#1}%
       \fi
     \else\ifdim\lastkern=\dimensionsignal
       \mathematics{\dimensionhalfspace\dimensiontypeface#1}%
     \else
       \unkern\mathematics{\dimensionhalfspace\dimensionhalfspace\dimensiontypeface#1}%
     \fi\fi
   \else
     \unskip\mathematics{\dimensionhalfspace\dimensionhalfspace\dimensiontypeface#1}%
   \fi
   \dodimensionsignal}

\unexpanded\def\nodimension#1%
  {\unskip#1\global\let\dodimensionsignal\relax}

%D \macros
%D   {super, suber}
%D
%D \TEX\ uses \type{^} and \type{_} for entering super- and
%D subscript mode. We want however a bit more control than
%D normally provided, and therefore provide \type {\super}
%D and \type{\suber} (\type {\sub} is already taken).

\global\let\normalsuper=^
\global\let\normalsuber=_

\newcount\supersubmode

\newevery\everysupersub \EverySuperSub

\appendtoks \advance\supersubmode 1\relax \to \everysupersub

% \def\dodosuper#1{\normalsuper{\the\everysupersub#1}}
% \def\dodosuber#1{\normalsuber{\the\everysupersub#1}}
%
% \def\dosuper{\ifx\next\bgroup\expandafter\dodosuper\else\normalsuper\fi}
% \def\dosuber{\ifx\next\bgroup\expandafter\dodosuber\else\normalsuber\fi}
%
% \def\super{\futurelet\next\dosuper}
% \def\suber{\futurelet\next\dosuber}

\def\super#1{\normalsuper{\the\everysupersub#1}}
\def\suber#1{\normalsuber{\the\everysupersub#1}}

%D \macros
%D   {enablesupsub}
%D
%D We can let \type {^} and \type {_} act like \type {\super}
%D and \type {\suber} by saying \type {\enablesupsub}.

\bgroup
\catcode`\^=\@@active
\catcode`\_=\@@active
\gdef\enablesupsub
  {\catcode`\^=\@@active
   \def^{\ifmmode\expandafter\super\else\expandafter\normalsuper\fi}%
   \catcode`\_=\@@active
   \def_{\ifmmode\expandafter\suber\else\expandafter\normalsuber\fi}}
\egroup

%D \macros
%D   {restoremathstyle}
%D
%D We can pick up the current math style by calling \type
%D {\restoremathstyle}.

\def\restoremathstyle
  {\ifmmode
     \ifcase\supsubmode
       \textstyle
     \or
       \scriptstyle
     \else
       \scriptscriptstyle
     \fi
   \fi}

%D \macros
%D   {cramped}
%D
%D \TEX provides no primitive to force in cramped math mode. Here is
%D a macro that does so. It is based on a solution by Don Knuth (\useurl
%D {http://www.ctan.org/tex-archive/digests/tex-implementors/042}).

\def\cramped
  {\mathpalette\docramped}

%D We force a cramped style by issuing a non||existing radical (box).
%D After that we make sure the height is correct. (Comment taken from
%D Morten Høgholm's mathtools package.) Macro suggested by AM and
%D cleaned up a bit by HH.

% original
%
% \def\docramped#1#2%
%   {\setbox\scratchbox\hbox
%     {\nulldelimiterspace\zeropoint
%      $\m@th#1\radical\zerocount{#2}$}%
%    \ifx#1\displaystyle
%       \scratchdimen\fontdimen8\textfont3
%       \advance\scratchdimen .25\fontdimen5\textfont2
%    \else
%       \scratchdimen1.25\fontdimen8
%       \ifx#1\textstyle\textfont
%       \else
%         \ifx#1\scriptstyle\scriptfont\else\scriptscriptfont\fi
%       \fi
%       3
%     \fi
%     \advance\scratchdimen-\ht\scratchbox
%     \ht\scratchbox-\scratchdimen
%     \box\scratchbox}

\def\docramped#1#2%
  {\begingroup % added HH, made even more cramped
   \setbox\scratchbox\hbox
     {\nulldelimiterspace\zeropoint
      $\m@th#1\radical\zerocount{#2}$}%
   \ht\scratchbox-\dimexpr
     \ifx#1\displaystyle
       \fontdimen8\textfont3
       +.25\fontdimen5\textfont2
     \else
        1.25\fontdimen8
          \ifx#1\textstyle
            \textfont
          \else\ifx#1\scriptstyle
            \scriptfont
          \else
            \scriptscriptfont
          \fi \fi
          \plusthree
     \fi
     -\ht\scratchbox
   \relax
   \box\scratchbox
   \endgroup}

%D \macros
%D    {cramedllap, crampedrlap, crampedclap}
%D
%D In many cases, one wants to use the \tex {clap} macro in
%D math mode while using subscripts. Normally, subscripts are
%D in cramped mode, and these macros allow you to go to
%D cramped mode. For example:
%D
%D \startbuffer
%D \startformula
%D   \sum_{a^2 < b^2 < c^2} \quad
%D   \sum_{\clap{a^2 < b^2 < c^2}} \quad
%D   \sum_{\crampedclap{a^2 < b^2 < c^2}}
%D \stopformula
%D \stopbuffer
%D
%D \start
%D   \typebuffer
%D   \switchtobodyfont[24pt]
%D   \getbuffer
%D \stop
%D
%D Notice that \type {\clap{\cramped ... }} would also work,
%D but it requires \TEX\ to typeset it arguments 16 times.

\def\docrampedllap#1#2{{}\dollap{\docramped #1{#2}}}
\def\docrampedrlap#1#2{{}\dorlap{\docramped #1{#2}}}
\def\docrampedclap#1#2{{}\doclap{\docramped #1{#2}}}

\def\crampedllap{\mathpalette\docrampedllap}
\def\crampedrlap{\mathpalette\docrampedrlap}
\def\crampedclap{\mathpalette\docrampedclap}

\protect \endinput
