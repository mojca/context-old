%D \module
%D   [      file=s-fnt-30,
%D        version=2006.05.10, % abou tthen, quite old already
%D          title=\CONTEXT\ Style File,
%D       subtitle=Showing Character Data,
%D         author=Hans Hagen,
%D           date=\currentdate,
%D      copyright={PRAGMA / Hans Hagen \& Ton Otten}]
%C
%C This module is part of the \CONTEXT\ macro||package and is
%C therefore copyrighted by \PRAGMA. See mreadme.pdf for
%C details.

\startluacode
function document.show_character_data(n)
    local n = characters.number(n)
    local d = characters.data[n]
    local NC, NR = context.NC, context.NR
    if d then
        local function entry(label,name)
            NC() context(label)
            NC() context(d[name])
            NC() NR()
        end
        context.starttabulate { "|Tl|Tl|]" }
            entry("unicode index" , "unicodeslot")
            entry("context name"  , "contextname")
            entry("adobe name"    , "adobename")
            entry("category"      , "category")
            entry("description"   , "description")
            entry("uppercase code", "uccode")
            entry("lowercase code", "lccode")
            entry("specials"      , "specials")
        context.stoptabulate()
    end
end
\stopluacode

\def\ShowCharacterData#1%
  {\ctxlua{document.show_character_data(#1)}}

% \ShowCharacterData{123}
% \ShowCharacterData{0x7B}

% \dostepwiserecurse{`A}{`Z}{1}{\ShowCharacterData{#1}}
