%D \module
%D   [       file=strc-mar,
%D        version=2008.10.20,
%D          title=\CONTEXT\ Structure Macros,
%D       subtitle=Markings,
%D         author=Hans Hagen,
%D           date=\currentdate,
%D      copyright=PRAGMA-ADE / Hans Hagen]
%C
%C This module is part of the \CONTEXT\ macro||package and is
%C therefore copyrighted by \PRAGMA. See mreadme.pdf for
%C details.

\writestatus{loading}{ConTeXt Structure Macros / Markings}

\registerctxluafile{strc-mar}{1.001}

\unprotect

%D Old stuff.

\newtoks \listofmarks

\let \getmarks         \gobbleoneargument
\let \getallmarks      \relax
\let \getsplitmarks    \gobbleoneargument
\let \getallsplitmarks \relax

%D \macros
%D   {expandmarks}
%D
%D We can force expansion of marks with the following switch.

% Synchronizing marks is a rather tricky and messy business. When
% setting a mark, a node is added to the list in order for to \TEX\
% be able to figure out the 3 current marks when a page is made
% (last mark on previous page, first on current page, last on
% current page; in \LUATEX\ we might at one point have the first on
% the next page as well).

% Resetting a mark is not easy. An empty one will not erase the last
% one on the previous page for instance. In \LUATEX\ we can clear a
% marks state register with \type {\clearmarks} but since this is an
% immediate operation it might have unwanted side effects when \TEX\
% has collected several pages of text and finishing off these pages
% uses marks.

% In \MKIV\ we provide a model that permits some control over the
% way marks are used. It is not entirely compatible with \MKII\ but
% in practice this is not a real problem. Quality has a price.

% In fact we define multiple marks per visible mark and define
% additional ones on the fly. This has some price in terms of used
% mark registers but given the way that we fill marks in \MKIV\
% their accumulated content is not really the issue. Also,
% periodically we cleanup any leftovers.

\newif\ifexpandmarks \expandmarkstrue

\def\marksomecs #1#2{\csname\string#1:m:\number#2\endcsname}
\def\markautocs   #1{\csname\string#1:m:\number\csname\string#1:s\endcsname\endcsname}
\def\markmaincs   #1{\csname\string#1:m\endcsname}
\def\markresetcs  #1{\csname\string#1:r\endcsname}
\def\markstatecs  #1{\csname\string#1:s\endcsname}
\def\markcurrentcs#1{\csname\string#1:c\endcsname}
\def\marktokscs   #1{\csname\string#1:t\endcsname}

\def\renewmarks#1%
  {\ifx#1\relax
   % \writestatus\m!systems{defining low level mark: \string#1}%
     \newmarks#1%
   \else
     \clearmarks#1%
   \fi}

\def\definenewmark#1%
  {\ifcsname\string#1:m\endcsname\else
     \@EA\@EA\@EA\newcount  \markstatecs  #1\global\markstatecs#1\plusone
     \@EA\@EA\@EA\renewmarks\markautocs   #1%
     \@EA\@EA\@EA\renewmarks\markmaincs   #1%
     \@EA\@EA\@EA\renewmarks\markresetcs  #1%
     \@EA\@EA\@EA\newtoks   \marktokscs   #1%
     \@EA\@EA\@EA\let       \markcurrentcs#1\empty
     \listofmarks\expandafter{\the\listofmarks\checkmark#1}%
   \fi}

\long\def\setmark#1#2% marks expand
  {\@EA\@EA\@EA\xdef \markcurrentcs#1{\ifexpandmarks#2\else\normalunexpanded{#2}\fi}%
               \marks\markautocs   #1{\ifexpandmarks#2\else\normalunexpanded{#2}\fi}% we could expand current one level
               \marks\markmaincs   #1{\ifexpandmarks#2\else\normalunexpanded{#2}\fi}% we could expand current one level
               \marks\markresetcs  #1{\number\markstatecs#1}}

\def\resetmark#1%
  {\global\advance\markstatecs#1\plusone
   \@EA\@EA\@EA\glet\markcurrentcs#1\empty
   \@EA\@EA\@EA\renewmarks\markautocs#1%
   \the\marktokscs#1\relax}

\def\addmarkreset#1#2%
  {\global\marktokscs#2\@EA{\the\marktokscs#2\resetmark#1}}

% already there: \prependtoks \getallmarks \to \everybeforepagebody
%
% \def\getallmarks{\the\listofmarks}

\let\checkmark\gobbleoneargument

\prependtoks \clearmarkswhenemptypage \to \everybeforepagebody

\def\clearmarkswhenemptypage
  {\iffalse % check if page is empty
     \clearallmarks
   \fi}

\def\clearallmarks
  {\let\checkmark\clearmarkchain
   \the\listofmarks
   \let\checkmark\gobbleoneargument}

\def\clearmarkchain#1%
  {\@EA\@EA\@EA\clearmarks\markmaincs#1%
   \@EA\@EA\@EA\clearmarks\markresetcs#1%
   \@EA\doclearmarkchain\@EA{\number\csname\string#1:s\endcsname}#1%
   \@EA\@EA\@EA\glet\markcurrentcs#1\empty
   \global\markstatecs#1\plusone}

\def\doclearmarkchain#1#2%
  {\@EA\@EA\@EA\clearmarks\marksomecs#1{#2}%
   \@EA\doclearmarkchain\@EA#1\@EA{\number\numexpr#2+\minusone}}

% Fetching (expandable versions, so no intermediate counter):

\def\currenttopmarknumber  #1{\number0\topmarks  \markresetcs#1}
\def\currentfirstmarknumber#1{\number0\firstmarks\markresetcs#1}
\def\currentbotmarknumber  #1{\number0\botmarks  \markresetcs#1}

\def\checkedcurrentmarkrange#1{[\currenttopmarknumber#1,\currentfirstmarknumber#1,\currentbotmarknumber#1]}

\def\checkedcurrentmarks{\markcurrentcs} % #1 shared current mark

\let\currentsplittopmarknumber\currenttopmarknumber
\let\normalsplittopmarks      \normaltopmarks

\def\uncheckedautotopmark        {\normaltopmarks       \markautocs} % #1
\def\uncheckedautofirstmark      {\normalfirstmarks     \markautocs} % #1
\def\uncheckedautobotmark        {\normalbotmarks       \markautocs} % #1
\def\uncheckedautosplittopmark   {\normalsplittopmarks  \markautocs} % #1
\def\uncheckedautosplitfirstmark {\normalsplitfirstmarks\markautocs} % #1
\def\uncheckedautosplitbotmark   {\normalsplitbotmarks  \markautocs} % #1

\def\uncheckedmaintopmark        {\normaltopmarks       \markmaincs} % #1
\def\uncheckedmainfirstmark      {\normalfirstmarks     \markmaincs} % #1
\def\uncheckedmainbotmark        {\normalbotmarks       \markmaincs} % #1
\def\uncheckedmainsplittopmark   {\normalsplittopmarks  \markmaincs} % #1
\def\uncheckedmainsplitfirstmark {\normalsplitfirstmarks\markmaincs} % #1
\def\uncheckedmainsplitbotmark   {\normalsplitbotmarks  \markmaincs} % #1

\def\checkedpagetopmarks       #1{\ifcase\currentbotmarknumber       #1\else\normaltopmarks       \marksomecs#1{\currentbotmarknumber       #1}\fi}
\def\checkedpagefirstmarks     #1{\ifcase\currentbotmarknumber       #1\else\normalfirstmarks     \marksomecs#1{\currentbotmarknumber       #1}\fi}
\def\checkedpagebotmarks       #1{\ifcase\currentbotmarknumber       #1\else\normalbotmarks       \marksomecs#1{\currentbotmarknumber       #1}\fi}
\def\checkedpagesplittopmarks  #1{\ifcase\currentsplitbotmarknumber  #1\else\normalsplittopmarks  \marksomecs#1{\currentsplitbotmarknumber  #1}\fi}
\def\checkedpagesplitfirstmarks#1{\ifcase\currentsplitbotmarknumber  #1\else\normalsplitfirstmarks\marksomecs#1{\currentsplitbotmarknumber  #1}\fi}
\def\checkedpagesplitbotmarks  #1{\ifcase\currentsplitbotmarknumber  #1\else\normalsplitbotmarks  \marksomecs#1{\currentsplitbotmarknumber  #1}\fi}

\def\checkedfulltopmarks       #1{\ifcase\currenttopmarknumber       #1\else\normaltopmarks       \marksomecs#1{\currenttopmarknumber       #1}\fi}
\def\checkedfullfirstmarks     #1{\ifcase\currentfirstmarknumber     #1\else\normalfirstmarks     \marksomecs#1{\currentfirstmarknumber     #1}\fi}
\def\checkedfullbotmarks       #1{\ifcase\currentbotmarknumber       #1\else\normalbotmarks       \marksomecs#1{\currentbotmarknumber       #1}\fi}
\def\checkedfullsplittopmarks  #1{\ifcase\currentsplittopmarknumber  #1\else\normalsplittopmarks  \marksomecs#1{\currentsplittopmarknumber  #1}\fi}
\def\checkedfullsplitfirstmarks#1{\ifcase\currentsplitfirstmarknumber#1\else\normalsplitfirstmarks\marksomecs#1{\currentsplitfirstmarknumber#1}\fi}
\def\checkedfullsplitbotmarks  #1{\ifcase\currentsplitbotmarknumber  #1\else\normalsplitbotmarks  \marksomecs#1{\currentsplitbotmarknumber  #1}\fi}

% Interface macros:

\def\getcurrentmark    {\checkedcurrentmarks       }
\def\gettopmark        {\checkedfulltopmarks       }
\def\getfirstmark      {\checkedfullfirstmarks     }
\def\getbotmark        {\checkedfullbotmarks       }
\def\getsplittopmark   {\checkedfullsplittopmarks  }
\def\getsplitfirstmark {\checkedfullsplitfirstmarks}
\def\getsplitbotmark   {\checkedfullsplitbotmarks  }

\def\getbottommark     {\getbotmark}
\def\getsplitbottommark{\getsplitbotmark}

%D Some of these will go away (in the process of rewriting).

\let \newmark              \definenewmark
\let \newpersistentmark    \newmarks
\let \normalsetmark        \setmark
\let \rawnewmark           \newmarks
\let \rawdefinemark        \newmarks
\let \rawsetmark           \normalmarks
\let \rawgettopmark        \normaltopmarks
\let \rawgetfirstmark      \normalfirstmarks
\let \rawgetbotmark        \normalbotmarks
\let \rawgetsplitbotmark   \normalsplitbotmarks
\let \rawgetsplitfirstmark \normalsplitfirstmarks
\let \rawgetsplittopmark   \normalsplitfirstmarks

\let \noninterferingmarks \relax % old color interference related hack

%D Next comes the layer around the previous mechanism.
%D
%D Parameters

\def\markingparameter        #1#2{\csname\domarkingparameter{\??mk#1}#2\endcsname}
\def\domarkingparameter      #1#2{\ifcsname#1#2\endcsname#1#2\else\@EA\domarkingparentparameter\csname#1\s!parent\endcsname#2\fi}
\def\domarkingparentparameter#1#2{\ifx#1\relax\s!empty\else\domarkingparameter#1#2\fi}
\def\markingcoupling           #1{\ifcsname\??mk#1\c!coupling\endcsname\@EA\markingcoupling\csname\??mk#1\c!coupling\endcsname\else#1\fi}

\let\mainmarking\markingcoupling % compatibility

\def\doifelsemarking#1%
  {\ifcsname\??mk#1\c!coupling\endcsname
     \expandafter\firstoftwoarguments
   \else
     \expandafter\secondoftwoarguments
   \fi}

\def\dowithmarkcommandone#1#2% \command {mark}
  {\expandafter#1\csname\??mk:\markingcoupling{#2}\endcsname}

\def\dowithmarkcommandtwo#1#2#3% \command {mark} {mark}
  {\expandafter#1\csname\??mk:\markingcoupling{#2}\expandafter\endcsname\csname\??mk:\markingcoupling{#3}\endcsname}

\def\setupmarking
  {\dodoubleargument\dosetupmarking}

\def\dosetupmarking[#1][#2]%
  {\def\docommand##1{\getparameters[\??mk##1][#2]}%
   \processcommalist[#1]\docommand}

%D The filtercommand key is used to hook in a filtering command. Users are
%D adviced not to misuse this key.

\getparameters
  [\??mk]
  [\c!expansion=\v!no, % saves a macro
   \c!separator={\space\emdash\space},
   \c!limittext=\@@kolimittext,
   \c!filtercommand=\firstofoneargument,
   \c!state=\v!start]

\let\alldefinedmarks\empty

\def\definemarking
  {\dodoubleempty\dodefinemarking}

\def\dodefinemarking[#1][#2]%
  {\doifelsenothing{#2}\donormaldefinemarking\docloneddefinemarking[#1][#2]}

\def\donormaldefinemarking[#1][#2]% #2 empty
  {\ifcsname\??mk#1\s!parent\endcsname
     % already defined
  \else
     \letgvalue{\??mk#1\s!parent}\??mk
     \dowithmarkcommandone\newmark{#1}%
     \doglobal\addtocommalist{#2}\alldefinedmarks
     \ifproductionrun\showmessage\m!systems{13}{#1,[#1]}\fi
  \fi}

\def\docloneddefinemarking[#1][#2]%
  {\ifcsname\??mk#1\s!parent\endcsname \else \ifcsname\??mk#2\s!parent\endcsname
     \doifnot{#1}{#2}%
       {\setxvalue{\??mk#1\s!parent}{\??mk#2}%
        \setxvalue{\??mk#1\c!coupling}{#2}%
        \ifproductionrun\showmessage\m!systems{13}{#1,[#2]}\fi}%
   \fi \fi}

\def\decouplemarking[#1]%
  {\letbeundefined{\??mk#1\c!coupling}}

\def\couplemarking[#1]#2[#3]% couple 1 to 2 (this macro is not really needed)
  {\setvalue{\??mk#1\c!coupling}{#3}}

\def\relatemarking[#1]#2[#3]% define 1 as child of 2
  {\dowithmarkcommandtwo\addmarkreset{#1}{#3}}

\def\definerawmarking[#1]%
  {\dododefinemarking[#1][#1]%
   \getgparameters[\??mk#1][\c!limittext=]} % global !

% \decouplemarking[#1]%  % no coupling with sections

\def\fastresetmarker#1%
  {\ifcsname\??mk#1\s!parent\endcsname
     \dowithmarkcommandone\resetmark{#1}%
   \fi}

\def\fastresetmarkerlist[#1]%
  {\normalexpanded{\noexpand\rawprocesscommalist[#1]}\fastresetmarker}

\def\resetmarking
  {\dosingleargument\doresetmarking}

\def\doresetmarking[#1]%
  {\processcommalist[#1]\fastresetmarker}

%D Used elsewhere:

\let\nomarking\empty

%D Basic fetching:

\letvalue{\??mk::\??mk::\v!previous}\firstoffourarguments
\letvalue{\??mk::\??mk::\v!first   }\secondoffourarguments
\letvalue{\??mk::\??mk::\v!last    }\thirdoffourarguments
\letvalue{\??mk::\??mk::\v!current }\fourthoffourarguments

\letvalue{\??mk\??mk\v!previous}\gettopmark
\letvalue{\??mk\??mk\v!first   }\getfirstmark
\letvalue{\??mk\??mk\v!last    }\getbotmark
\letvalue{\??mk\??mk\v!current }\getcurrentmark

\letvalue{\??mk\??mk\v!column:\v!first}\getsplitfirstmark
\letvalue{\??mk\??mk\v!column:\v!last }\getsplitbottommark

\def\fetchmark[#1]#2[#3]% % expandable / never use \unexpanded
  {\ifcsname\??mk::#1\endcsname % saved mark
     \markingparameter{#1}\c!filtercommand{\csname\??mk::\??mk::#3\@EA\@EA\@EA\endcsname\csname\??mk::#1\endcsname}%
   \else\ifcsname\??mk#1\s!parent\endcsname % real mark
     \markingparameter{#1}\c!filtercommand{\expandafter\dowithmarkcommandone\csname\??mk\??mk#3\endcsname{#1}}%
   \fi\fi}

\def\fetchtwomarks[#1]%
  {\dofetchtwomarks[#1][#1]}

\def\fetchallmarks[#1]%
  {\dofetchallmarks[#1][#1]}

\def\dofetchtwomarks[#1][#2]% class class:tag
  {\doifsomething{\fetchmark[#2][\v!first]}
     {\fetchmark[#2][\v!first]%
      \doifsomething{\fetchmark[#2][\v!last]}
        {\doifnot{\fetchmark[#2][\v!first]}{\fetchmark[#2][\v!last]}
            {\markingparameter{#1}\c!separator\fetchmark[#2][\v!last]}}}}

\def\dofetchallmarks[#1][#2]%
  {\doifsomething{\fetchmark[#2][\v!first]}
     {\doifsomething{\fetchmark[#2][\v!previous]}
        {\doifnot{\fetchmark[#2][\v!previous]}{\fetchmark[#2][\v!first]}
           {\fetchmark[#2][\v!previous]\markingparameter{#1}\c!separator}}}%
      \fetchtwomarks[#1][#2]}

% \newtoks \everymarking

% \def\Interesting{\doifmodeelse{*\v!marking}{Interesting}{Boring}}
% \setupheadertexts[chapter]
% \starttext
%     \chapter{This Is \Interesting}
% \stoptext

\def\markingnomarking#1{\splitsequence{\markingparameter{#1}\c!limittext}} % #2

\def\dogetmarking[#1][#2][#3]%
  {\doif{\markingparameter{#1}\c!state}\v!start
     {\bgroup
      \setsystemmode\v!marking
      \the\everymarking
      \def\nomarking{\markingnomarking{#1}}% just for good old times, might disappear
      \ifthirdargument
        \dodogetmarking{#3}{#1}{#1:#2}{#3}%
      \else
        \dodogetmarking{#2}{#1}{#1}{#2}%
      \fi
      \egroup}}

\def\dodogetmarking#1#2#3#4% to be made faster
  {\processaction % slow
     [#1]
     [   \v!both=>{\dofetchtwomarks[#2][#3]},
          \v!all=>{\dofetchallmarks[#2][#3]},
      \s!default=>{\fetchmark[#3][\v!first]},
      \s!unknown=>{\fetchmark[#3][#4]}]}

\def\nogetmarking[#1][#2][#3]%
  {}

\unexpanded\def\getmarking
  {\dotripleargument\dogetmarking}

\let\setsomemark\setmark

\def\setmarking
  {\dosingleargument\dosetmarking}

\def\dosetmarking[#1]#2%
  {\ifcsname\??mk#1\s!parent\endcsname
     \begingroup
     \doifelse{\markingparameter{#1}\c!expansion}\v!yes\expandmarkstrue\expandmarksfalse
     \dowithmarkcommandone\setsomemark{#1}{#2}%
     \endgroup
   \fi}

\let\marking\setmarking

% to be adapted for mkiv:
%
% this version can be used when a page is built up in steps without
% feedback of the otr'd list to the mvl (i.e.\ a page made of pages,
% as in column sets where content is buffered)

% reset at begin
% preset before page
% bubble in column
% refresh at end

% marks is a kind of toks, so maybe we need a low level \the\marks
%
% use \normalunexpanded here

\def\refreshsavedmark[#1][#2]% mark tag (packing saves many hash entries)
  {\setxvalue{\??mk::#1:#2}%
     {{\@EA\ifx\csname\??mk::#1:pp\endcsname\relax
         % empty
       \else
         \csname\??mk::#1:pp\endcsname
       \fi}%
      {\@EA\ifx\csname\??mk::#1:ff\endcsname\relax
         \fetchmark[#1][\v!first]%
       \else
         \csname\??mk::#1:ff\endcsname
       \fi}%
      {\fetchmark[#1][\v!last]}%
      {\fetchmark[#1][\v!current]}}%
    \setxvalue{\??mk::#1:pp}{\fetchmark[#1][\v!first]}%
    \letgvalue{\??mk::#1:ff}\relax
    }

\def\bubblesavedmark[#1][#2]% no packing (not now, maybe make a six-pack later)
  {\@EA\ifx\csname\??mk::#1:ff\endcsname\relax
     \setxvalue{\??mk::#1:ff}{\fetchmark[#1][\v!first]}%
   \fi}

\def\resetsavedmark[#1][#2]% mark tag
  {\doifelsenothing{\fetchmark[#1][\v!previous]}
     {\letgvalue{\??mk::#1:pp}\relax}
     {\setxvalue{\??mk::#1:pp}{\fetchmark[#1][\v!previous]}}%
   \doifelsenothing{\fetchmark[#1][\v!first]}
     {\letgvalue{\??mk::#1:ff}\relax}
     {\setxvalue{\??mk::#1:ff}{\fetchmark[#1][\v!first]}}%
   \letgvalue{\??mk::#1:#2}\emptysavedmark}

\def\presetsavedmark[#1][#2]% mark tag
  {\letgvalue{\??mk::#1:#2}\emptysavedmark}

\def\emptysavedmark{{}{}{}{}}

% new (can be used in column sets)
%
% \getsavedmarking[M][previous]
% \getsavedmarking[M][first]
% \getsavedmarking[M][last]

\def\getsavedmarking
  {\dodoubleargument\dogetsavedmarking}

\def\dogetsavedmarking[#1][#2]%
  {\doifelse{#2}\v!previous
     {\getmarking[#1][1][\v!previous]}
     {\doifelse{#2}\v!first
        {\getmarking[#1][1][\v!first]}
        {\getmarking[#1][\v!last]}}}

%D And then \unknown\ we had a chaptertitle packaged in a
%D makeup environment. And we don't want to loose marks there!

\newbox\collectedmarks

\def\flushmarks % use with care to avoid empty pages
  {\ifvoid\collectedmarks\else\unhbox\collectedmarks\fi}

\def\postponemarks
  {\let\setsomemark\postponemark}

\def\postponemark#1#2%
  {\global\setbox\collectedmarks\hbox{\unhbox\collectedmarks\setmark{#1}{#2}}}

\protect \endinput

% todo: make it work in balancing
%
% \definemarking[vers][]
% \setupheadertexts
%   [\doiftext{\getmarking[vers][first]}
%      {\doiftextelse{\getmarking[vers][column:last]}
%         {\getmarking[vers][first] -- \getmarking[vers][column:last]}
%         {\getmarking[vers][first]}}]
% \starttext
%   \startcolumns[n=2,balance=no]
%     \dorecurse{10}{\normalexpanded{\noexpand\marking[vers]{\recurselevel}} \recurselevel:\dorecurse{4}{\input ward } \endgraf}
%   \stopcolumns
% \stoptext
