%D \module
%D   [       file=back-pdf,
%D        version=2009.04.15,
%D          title=\CONTEXT\ Backend Macros,
%D       subtitle=\PDF,
%D         author=Hans Hagen,
%D           date=\currentdate,
%D      copyright=\PRAGMA]
%C
%C This module is part of the \CONTEXT\ macro||package and is
%C therefore copyrighted by \PRAGMA. See mreadme.pdf for
%C details.

\writestatus{loading}{ConTeXt Backend Macros / PDF}

\registerctxluafile{back-pdf}{1.001}

\unprotect

%D When dealing with resources, we share the resource dictionaries
%D between all xforms. This is inefficent in the sense that when no
%D resources are used, redundant entries take space, but on the other
%D hand we save redundant dictionaries so it's a nice compromise. Maybe
%D that in \LUATEX\ I will reimplement most of the code here anyway.

%D Initialization of fields is tricky. If a field has no
%D value, it is kind of not there. If ResetForm is used, the
%D default is assigned, but pushbuttons are spoiled. Adding a
%D \type {/MK} dictionary helps, but gives ugly down
%D appearances (displaced with background). What a mess.
%D Also, in order to get at least something, the \type {/AS}
%D key should be provided.

%D A couple of variables:

\newtoks \everybackendshipout
\newtoks \everylastbackendshipout

\let\lastPDFaction\empty

\ifdefined\everyPDFximage \else \newtoks\everyPDFximage \fi
\ifdefined\everyPDFxform  \else \newtoks\everyPDFxform  \fi
\ifdefined\everygoto      \else \newtoks\everygoto      \fi
\ifdefined\everysetfield  \else \newtoks\everysetfield  \fi

%D A few helpers:

\let\PDFcode        \pdfliteral
\def\PDFcontentcode{\pdfliteral}
\def\PDFdirectcode {\pdfliteral direct}

%D \macros
%D   {PDFobjref}
%D
%D Just a shortcut.

% Watch out, \def\PDFobjref#1{\purenumber#1 0 R} also works, but not when
% #1 == \the\whatever

\def\PDFobjref#1{\purenumber{#1} 0 R}

%D \macros
%D   {PDFswapdir}

\let\PDFswapdir\empty \def\PDFswapdir{\ifcase\inlinedirection\or\or-\fi}

% the pdf spec changed cq. viewers started behaving differently / 5+

\chardef\overcomePDFpage\plusone   % page numbers/ beware: optimizers remove this one
\chardef\overcomePDFpage\plustwo   % page:number
\chardef\overcomePDFpage\plusthree % pdftex page ref feature

%D \macros
%D   {setPDFdestination}
%D
%D \PDF\ destinations should obey the specifications laid down
%D in the \PDF\ reference manual. The next macro strips illegal
%D characters from the destination name.

\def\setPDFdestination  #1{\xdef\PDFdestination{\ctxlua{pdf.cleandestination("\luaescapestring{#1}")}}}
\def\hexifiedPDFstring  #1{\ctxlua{pdf.hexify("\luaescapestring{#1}")}}
\def\sanitizePDFencoding#1\to#2{\xdef#2{\ctxlua{pdf.hexify("\luaescapestring{#1}")}}}

%D

\def\appendtopdfpageresources  #1{\normalexpanded{\global\pdfpageresources{#1\the\pdfpageresources}}}
\def\appendtopdfpageattributes #1{\normalexpanded{\global\pdfpageattr     {#1\the\pdfpageattr     }}}
\def\appendtopdfpagesattributes#1{\normalexpanded{\global\pdfpagesattr    {#1\the\pdfpagesattr    }}}
\def\appendtopdfcatalog          {\pdfcatalog}
\def\appendtopdfinfo             {\pdfinfo}

\def\resetpdfpageattributes{\global\pdfpageattr\emptytoks}
\def\resetpdfpageresources {\global\pdfpageresources\emptytoks}

%D Due to the fact that \PDFTEX\ has a different concept of
%D page attributes, we need:

\appendtoksonce
  \resetpdfpageattributes
  \resetpdfpageresources
\to \everyaftershipout

%D \macros
%D   {insertpdfaction,
%D    insertpdfannotation,
%D    insertpdfannotationobject,
%D    createpdfdictionaryobject,
%D    createpdfarrayobject,
%D    defaultobjectreference,
%D    doPDFgetobjectreference}
%D
%D This module deals with \PDF\ support, including fill||in
%D forms. Before we present the largely unreadable bunch of
%D macros, we introduce the here||not||defined low level
%D interface macros. These must be provided by the special
%D drivers \type{pdf} (\ACROBAT) and \type{tpd} (\PDFTEX).
%D
%D \starttyping
%D \insertpdfaction             #1#2#3     width height action
%D \insertpdfannotation         #1#2#3     width height data
%D \createpdfannotationobject   #1#2#3#4#5 class name width height data
%D \createpdfdictionaryobject   #1#2#3     class name data
%D \createpdfarrayobject        #1#2#3     class name data
%D
%D \defaultobjectreference      #1#2       class name
%D \doPDFgetobjectreference     #1#2#3     class name \PDFobjectreference
%D \doPDFgetobjectpagereference #1#2#3     class name \PDFobjectreference
%D \stoptyping
%D
%D The keywords reflect their use. For the moment we stick to
%D keywords, because that way at we get an indication of what
%D we're doing.

\def\createpdfdictionaryobject#1#2#3%
  {\flushatshipout
     {\immediate\pdfobj{<< #3 >>}%
      \dosetobjectreference{#1}{#2}{\the\pdflastobj}}}

\def\createpdfarrayobject#1#2#3%
  {\flushatshipout
     {\immediate\pdfobj{[ #3 ]}%
      \dosetobjectreference{#1}{#2}{\the\pdflastobj}}}

\def\createpdfannotationobject#1#2#3#4#5%
  {\insertpdfannotation{#3}{#4}{#5}%
   \dosetobjectreference{#1}{#2}{\the\pdflastannot}}

\def\createpdfactionobject#1#2#3#4#5%
  {\insertpdfaction{#3}{#4}{#5}%
   \dosetobjectreference{#1}{#2}{\the\pdflastannot}}

%D \macros
%D   {insertpdfaction,insertpdfannotation,ifsharePDFactions}
%D
%D Next we handle annotations. All link annotations are
%D implemented using the action dictionary. This enables us to
%D use multiple actions. The second macro is for instance
%D used for movie inclusion.

\newif\ifsharePDFactions \sharePDFactionstrue

\def\insertpdfaction#1#2#3%
  {\xdef\lastPDFcontent{#3}%
   \ifcollectreferenceactions
     \global\let\lastPDFaction\lastPDFcontent
   \else
     \ifsharePDFactions
       \ifcase\similarreference\relax
         \xdef\lastPDFaction{<<\lastPDFcontent>>}%
       \or
         \immediate\pdfobj{<<\lastPDFcontent>>}%
         \xdef\lastPDFaction{\PDFobjref\pdflastobj}%
       \else
         % leave \lastPDFaction untouched
       \fi
     \else
       \xdef\lastPDFaction{<<\lastPDFcontent>>}%
     \fi
     \pdfannot
       width #1 height #2 depth \zeropoint
         {/Subtype /Link
          /Border [0 0 0]
          \ifhighlighthyperlinks \else /H /N \fi
          /A \lastPDFaction}%
   \fi}

\def\insertpdfannotation#1#2#3%
  {\pdfannot width #1 height #2 depth \zeropoint{#3}}

%D \macros
%D   {doPDFbookmark}
%D
%D Well, isn't the next one ugly? Thanks to the \PDF\
%D standard.

\def\doPDFbookmark#1#2#3#4#5% to be renamed
  {\doPDFgetpagereference{#4}\PDFobjectreference
   \pdfoutline
     user {<</S /GoTo /D [\PDFobjectreference\space\PDFpageviewwrd]>>}%
     \ifcase#2 \else count \ifcase#5-\fi#2 \fi
     {#3}}

%D For special (\METAPOST) effects, we need to build
%D resource dictionaries. Here is the framework.

\let\docuPDFextgstates \empty
\let\docuPDFcolorspaces\empty
\let\docuPDFshades     \empty

\def\checkPDFextgstates
  {\ifx\docuPDFextgstates\empty \else
     \ifnum\realpageno=\lastpage\relax
       \createpdfdictionaryobject{FDF}{docuextgstates}{\docuPDFextgstates}%
     \fi
     \doPDFgetobjectreference{FDF}{docuextgstates}\PDFobjectreference
     \appendtopdfpageresources{/ExtGState \PDFobjectreference}%
   \fi}

\def\checkPDFcolorspaces
  {\ifx\docuPDFcolorspaces\empty \else
     \ifnum\realpageno=\lastpage\relax
       \createpdfdictionaryobject{FDF}{colorspaces}{\docuPDFcolorspaces}%
     \fi
     \doPDFgetobjectreference{FDF}{colorspaces}\PDFobjectreference
     \appendtopdfpageresources{/ColorSpace \PDFobjectreference}%
   \fi}

\def\checkPDFshades
  {\ifx\docuPDFshades\empty \else
     \ifnum\realpageno=\lastpage\relax
       \createpdfdictionaryobject{FDF}{docushades}{\docuPDFshades}%
     \fi
     \doPDFgetobjectreference{FDF}{docushades}\PDFobjectreference
     \appendtopdfpageresources{/Shading \PDFobjectreference}%
   \fi}

\def\appendtoPDFdocumentextgstates #1{\xdef\docuPDFextgstates {\docuPDFextgstates \space#1}}
\def\appendtoPDFdocumentcolorspaces#1{\xdef\docuPDFcolorspaces{\docuPDFcolorspaces\space#1}}
\def\appendtoPDFdocumentshades     #1{\xdef\docuPDFshades     {\docuPDFshades     \space#1}}

%D Page actions:

\let\lastpdfopenaction \empty
\let\lastpdfcloseaction\empty

\def\dosetupopenaction {\appendtopdfcatalog{/OpenAction  <<\lastPDFaction>>}}
\def\dosetupcloseaction{\appendtopdfcatalog{/CloseAction <<\lastPDFaction>>}}

\def\dosetupopenpageaction {\glet\lastpdfopenaction \lastPDFaction}
\def\dosetupclosepageaction{\glet\lastpdfcloseaction\lastPDFaction}

\def\checkPDFpageactions
  {\iflocation % important since direct
     \donefalse
     \ifx\lastpdfopenaction \empty\!!doneafalse\else\donetrue\!!doneatrue\fi
     \ifx\lastpdfcloseaction\empty\!!donebfalse\else\donetrue\!!donebtrue\fi
     \ifdone
       \appendtopdfpageattributes
         {/AA <<\if!!donea/O <<\lastpdfopenaction >> \fi
                \if!!doneb/C <<\lastpdfcloseaction>> \fi>>}%
     \fi
     \glet\lastpdfopenaction \empty
     \glet\lastpdfcloseaction\empty
   \fi}

%D \macros
%D   {ifPDFstrokecolor}
%D
%D We can reduce the filesize a bit by setting the next switch
%D to false. The amount of reduction depends on the use of
%D color, but don't expect more than a few percent. Zip
%D compression is already rather efficient in itself.

\newif\ifPDFstrokecolor \PDFstrokecolortrue

%D When submitting forms, we need to communicate the format.

\chardef\submitoutputformat=0 % 0=unknown 1=HTML 2=FDF 3=XML

\def\setsubmitoutputformat#1%
  {\doifinsetelse{#1}{FDF,fdf}
     {\chardef\submitoutputformat2}
     {\doifinsetelse{#1}{XML,xml}
        {\chardef\submitoutputformat3}
        {\chardef\submitoutputformat1}}%
   \relax}

%D Handy to have this available asap:

\ifdefined\everyPDFxform  \newtoks\everyPDFxform  \fi
\ifdefined\everyPDFximage \newtoks\everyPDFximage \fi

% once we can be sure that the latest versions of pdftex are
% available we can use:
%
% \pdfobj reserveobjnum \edef\one{\the\pdflastobj}
% \pdfobj reserveobjnum \edef\two{\the\pdflastobj}
%
% \pdfobj useobjnum \one {x}
% \pdfobj useobjnum \two {x}
%
% we then can rewrite part of spec-fdf because the other drivers
% already support symbolic references

%D \macros
%D   {jobsuffix}
%D
%D Being one of the first typographical systems able to support
%D advances \PDF\ support, \TEX\ is also one of the first
%D systems to produce high quality \PDF\ code directly. Thanks
%D to Han The Thanh c.s. the \TEX\ community can leap forward
%D once again.
%D
%D One important characteristic of \PDFTEX\ is that is can
%D produce standard \DVI\ code as well as \PDF\ code. This
%D enables us to use one format file to support both output
%D formats.

%D All modules in this group use specials to tell drivers what
%D non||\TEX\ actions to take. Because from the \TEX\ point of
%D view, there is no difference between \DVI\ and \PDF, we
%D therefore only have to bend the \DVI\ driver support into
%D \PDF\ support. Technically spoken, specials no longer serve
%D a purpose, except from ending up as comment in the \PDF\
%D file.
%D
%D Before we continue we need to make sure if indeed those
%D \PDFTEX\ primitives are permitted. If no primitives are
%D available, we just stop reading any further.

\pdfoutput              =   1
\pdfhorigin             =   1 true in
\pdfvorigin             =   1 true in
\pdfimageresolution     = 300
\pdfpkresolution        = 600
\pdfdecimaldigits       =  10
\pdfinclusionerrorlevel =   0
\pdfminorversion        =   5
%pdfuniqueresname       =   1

\def\PDFversion{1.\number\pdfminorversion}

%D For some internal testing we need to know the output
%D suffix.

\setjobsuffix{pdf}

%D \macros
%D   {dosetuppaper}
%D
%D If we don't set the paper size, \PDFTEX\ will certainly do
%D it in a way we don't want, therefore we need:

\def\dosetuppaper#1#2#3%
  {\global\pdfpagewidth #2\relax
   \global\pdfpageheight#3\relax}

%D \macros
%D   {doloadmapfile,doloadmapline,doresetmapfilelist}

\def\doresetmapfilelist
  {\global\let\doresetmapfilelist\relax
   \pdfmapfile{original-empty.map}}

\def\doloadmapfile #1#2{\pdfmapfile{#1#2}}
\def\doloadmapline #1#2{\pdfmapline{#1#2}}

%D nasty but needed

\appendtoksonce \loadallfontmapfiles \to \everyPDFximage
\appendtoksonce \loadallfontmapfiles \to \everyPDFxform

%D left overs:

        \let\currentmovie\s!unknown

        \def\doPDFinsertmov
          {\bgroup
           \xdef\currentmovie{\@@DriverImageLabel}%
           \PointsToBigPoints\@@DriverImageWidth \width
           \PointsToBigPoints\@@DriverImageHeight\height
           \let\pdf@@options\empty
           \let\pdf@@actions\empty
           \donefalse
           \expanded{\processallactionsinset[\@@DriverImageOptions]}
             [\v!controls=>\donetrue,
                \v!repeat=>\edef\pdf@@actions{\pdf@@actions /Mode /Repeat },
               \v!preview=>\edef\pdf@@options{\pdf@@options /Poster true  }]%
           \edef\pdf@@actions{\pdf@@actions /ShowControls \ifdone true\else false\fi}%
           \insertpdfannotation\@@DriverImageWidth\@@DriverImageHeight
             {/Subtype /Movie
              /Border [0 0 0]
              /T (movie \currentmovie)
              /Movie << /F (\@@DriverImageFile) /Aspect [\width\space\height] \pdf@@options >>
              /A << \pdf@@actions >>}%
           \egroup}

%D \macros
%D   {doinsertsoundtrack}
%D
%D We use numbers instead of labels to keep track of sounds.

\let\currentsound\s!unknown

\def\doinsertsoundtrack#1#2#3%
  {\bgroup
   \xdef\currentsound{#2}%
   \let\pdf@@actions\empty
   \@EA\processallactionsinset\@EA
     [#3]
     [\v!repeat=>\edef\pdf@@actions{\pdf@@actions /Mode /Repeat }]%
   \collectdriverresource
  %\flushatshipout % since it can be buried in a chained box
     {\insertpdfannotation{0pt}{0pt}
        {/Subtype /Movie
         /Border [0 0 0]
         /T (sound \currentsound)
         /Movie <</F (#1)>>%
         \ifx\pdf@@actions\empty\else/A << \pdf@@actions >>\fi}}%
   \egroup}

%D \macros
%D   {doPDFattachfile}

\def\doPDFfilestreamobject#1#2#3#4%
  {}

\def\doPDFfilestreamidentifier#1%
  {0}

\def\doPDFgetfilestreamreference#1#2%
  {0 0 R}

\def\doattachfile#1#2#3#4#5#6#7#8%
  {\bgroup % title width height color symbol file
   \edefconvertedargument\PDFfile{#8}%
   % beware: the symbol may (indirectly) use the file
   % reference when typesetting the object number;
   \presetPDFsymbolappearance{#5}{#6}{#2}{#3}{#4}% sets width/height
   \startPDFsymbolappearance
     \doPDFembedfile\PDFfile{#7}{#8}%
     \doPDFgetembeddedfilereference\PDFfile\PDFobjectreference
     \setFDFlayer\@@DriverAttachmentLayer
     \insertpdfannotation{\width}{\totalheight}
       {/Subtype /FileAttachment
        /FS \PDFobjectreference\space
        /Contents (#1)
        \PDFsymbol
        \FDFlayer
        \PDFattributes}%
   \stopPDFsymbolappearance
   \egroup}

% semi-public

\def\doPDFembedfile#1#2#3% symbolic name | filename | user name
  {\edefconvertedargument\PDFfile{#1}%
   \doifnotflagged{a:\PDFfile}%
     {\doPDFfilestreamobject{PDFEF}{\PDFfile}{#2}{#3}%
      \doglobal\setflag{a:\PDFfile}}}

\def\doPDFgetembeddedfilereference#1#2%
  {\edefconvertedargument\PDFfile{#1}%
   \doPDFgetobjectreference{PDFEF}\PDFfile#2}

\def\doPDFgetembeddedfilestreamreference#1#2%
  {\edefconvertedargument\PDFfile{#1}%
   \doPDFgetfilestreamreference\PDFfile#2} % == \doPDFgetobjectreference{PDFFS}\PDFfile#2

% requested by Jens-Uwe Morawski: permits usage of pdftosrc
% in viewers that don't support attachments:
%
% \definesymbol
%   [ObjectNumber]
% % [object number {\PDFattachmentnumber[xx]}] % named
%   [object number \PDFattachmentnumber]       % current
%
% \useattachment[test][xx][test.tex]
% \setupattachments[symbol=ObjectNumber]
% \attachment[test]

\def\PDFattachmentnumber
  {\dosingleargument\doPDFattachmentnumber}

\def\doPDFattachmentnumber[#1]%
  {\iffirstargument
     \doPDFfilestreamidentifier{#1}%
   \else
     \doPDFfilestreamidentifier\PDFfile
   \fi}

%D \macros
%D   {...}
%D
%D Rather preliminary. We have to wait till the complete specs
%D show up. As usual, we cannot really check it (Acrobat 6.0
%D has a bug that inhibits us to make a test file). Half a day
%D of testing made clear that trying to control the plugin fails
%D in most cases (we need plugin specs -). We also miss a feature
%D to let acrobat wait with proceeding (action processing) till
%D the media clip is ready.

% aiff audio/aiff
% au   audio/basic
% avi  video/avi
% mid  audio/midi
% mov  video/quicktime
% mp3  audio/x-mp3 (mpeg)
% mp4  audio/mp4
% mp4  video/mp4
% mpeg video/mpeg
% smil application/smil
% swf  application/x-shockwave-flash

% beware, this is preliminary code, should be improved

\def\PDFrenderingspecs#1{\executeifdefined{PDFMR:#1}\empty}

\def\PDFexecutestartrendering  {/Rendition /OP 0 \PDFrenderingspecs\argumentA}
\def\PDFexecutestoprendering   {/Rendition /OP 1 \PDFrenderingspecs\argumentA}
\def\PDFexecutepauserendering  {/Rendition /OP 2 \PDFrenderingspecs\argumentA}
\def\PDFexecuteresumerendering {/Rendition /OP 3 \PDFrenderingspecs\argumentA}

% todo : sub files
%
% \doPDFembedfile{pier-39.png}{pier-39.png}{pier-39.png}%
% \doPDFgetembeddedfilestreamreference{pier-39.png}\xPDFobjectreference
% \edef\xxxx{/RF [(pier-39.png) \xPDFobjectreference]}%

% todo: alternative renderings
%
% object_1  -> <</Type /Rendition /S /MR /C << /Type /MediaClip ... >> >>
% object_2  -> <</Type /Rendition /S /MR /C << /Type /MediaClip ... >> >>
% rendering -> <</Type /Rendition /S /MS [objref_1 objref_2]>>

\def\doinsertrendering#1#2#3#4% tag mime file options
  {\ifundefined{PDFMR:#1}%
     \doifinstringelse{://}{#3}\donetrue\donefalse % evt url as keyword
     \createpdfdictionaryobject{PDFMF}{#1}
       {/Type /Rendition
        /S    /MR
        % does not work: /SP << /Type /MediaScreenParam /BE << /B [1 0 0] /O 0.5 >> >>
        /C << /Type /MediaClip
              /S /MCD
              /N (#1)
              /Alt [() (file not found)] % language id + message
              /D << /Type /Filespec
                    /F (#3)
                    \ifdone/FS /URL\fi >>
              /CT (#2) >>}%
     % common code
     \doifobjectreferencefoundelse{PDFMS}{#1}
       {\doPDFgetobjectreference{PDFMS}{#1}\PDFobjectreferenceB}
       {\doPDFgetobjectreference{PDFMU}{#1}\PDFobjectreferenceB}%
     \doPDFgetobjectreference{PDFMF}{#1}\PDFobjectreferenceA
     \setxvalue{PDFMR:#1}% needed /AA actions in /Screen
       {/R  \PDFobjectreferenceA
        /AN \PDFobjectreferenceB}%
     \doifobjectreferencefoundelse{PDFMS}{#1}\donothing
       {\dodoinsertrenderingwindow{PDFMU}{#1}\zeropoint\zeropoint{#4}}%
   \fi}

\def\doinsertrenderingobject#1#2#3#4% tag class objectname options
  {\ifundefined{PDFMR:#1}%
     \doPDFgetobjectreference{#2}{#3}\PDFobjectreference
     \createpdfdictionaryobject{PDFMF}{#1}
       {/Type /Rendition
        /S    /MR
        /C << /Type /MediaClip
              /S /MCD
              /N (#1)
              /D \PDFobjectreference>>}%
     % common code
     \doifobjectreferencefoundelse{PDFMS}{#1}
       {\doPDFgetobjectreference{PDFMS}{#1}\PDFobjectreferenceB}
       {\doPDFgetobjectreference{PDFMU}{#1}\PDFobjectreferenceB}%
     \doPDFgetobjectreference{PDFMF}{#1}\PDFobjectreferenceA
     \setxvalue{PDFMR:#1}% needed /AA actions in /Screen
       {/R  \PDFobjectreferenceA
        /AN \PDFobjectreferenceB}%
     \doifobjectreferencefoundelse{PDFMS}{#1}\donothing
       {\dodoinsertrenderingwindow{PDFMU}{#1}\zeropoint\zeropoint{#4}}%
   \fi}

\def\doinsertrenderingwindow
  {\dodoinsertrenderingwindow{PDFMS}}

\def\dodoinsertrenderingwindow#1#2#3#4#5%
  {\vbox to #4 \bgroup
     \checkPDFscreenactions{#2}{#5}%
     \doPDFgetobjectpagereference{PDFMF}{#2}\PDFobjectreferenceA
     \doPDFgetobjectreference    {PDFMF}{#2}\PDFobjectreferenceB
     \vss
     \hbox to #3 \bgroup
       \createpdfannotationobject{#1}{#2}{#3}{#4}
         {/Subtype /Screen
          /P \PDFobjectreferenceA
          /A \PDFobjectreferenceB
          \PDFattributes
          /Border [0 0 0]}%
       \hss
     \egroup
   \egroup}

\global\let\PDFrenderingopenpageaction \empty
\global\let\PDFrenderingclosepageaction\empty

\def\checkPDFscreenactions#1#2%
  {\let\PDFattributes\empty
   \iflocation % important since direct -)
     % the action can either (already) be set by the window handler
     % or (normally when no window [i.e a zero dimensions one] is present) by keyword
     \doifinset\v!auto{#2}
       {% brrr, here instead of in navigation module, must move and become special
        % now two sided dependency
        \let\checkrendering\gobbleoneargument
        \ifx\PDFrenderingopenpageaction \empty
          \handlereferenceactions{\v!StartRendering{#1}}\dosetuprenderingopenpageaction
        \fi
        \ifx\PDFrenderingclosepageaction\empty
          \handlereferenceactions{\v!StopRendering {#1}}\dosetuprenderingclosepageaction
        \fi
        }%
     \donefalse
     \ifx\PDFrenderingopenpageaction \empty\!!doneafalse\else\donetrue\!!doneatrue\fi
     \ifx\PDFrenderingclosepageaction\empty\!!donebfalse\else\donetrue\!!donebtrue\fi
     \ifdone
       \edef\PDFattributes
         {/AA <<\if!!donea/PO <<\PDFrenderingopenpageaction >> \fi
                \if!!doneb/PC <<\PDFrenderingclosepageaction>> \fi>>}%
     \fi
     \global\let\PDFrenderingopenpageaction \empty
     \global\let\PDFrenderingclosepageaction\empty
   \fi}

\def\dosetuprenderingopenpageaction {\global\let\PDFrenderingopenpageaction \lastPDFaction}
\def\dosetuprenderingclosepageaction{\global\let\PDFrenderingclosepageaction\lastPDFaction}

%D For the moment we don't test for alternatives that
%D themselves have alternatives, especially cylcic
%D dependencies.

% \def\pdfimmediateximage{\immediate\pdfximage}
%
% \def\checkpdfimageattributes
%   {\ifx\PDFfigurereference\empty
%      \global\let\pdfimageattributes\empty
%    \else
%      \immediate\pdfobj
%        {[ << /Image \PDFobjref\PDFfigurereference
%              /DefaultForPrinting true >> ]}%
%      \xdef\pdfimageattributes
%        {attr {/Alternates \PDFobjref\pdflastobj}}%
%    \fi}
%
% \global\let\PDFimagecolorreference\empty
%
% \def\checkpdfimagecolorspecs
%   {\ifx\pdflastximagecolordepth \undefined
%      \global\let\pdfimagecolorspecs\empty
%    \else\ifx\PDFimagecolorreference\empty
%      \global\let\pdfimagecolorspecs\empty
%    \else
%      \xdef\pdfimagecolorspecs{colorspace \PDFimagecolorreference\space}%
%    \fi\fi
%    \global\let\PDFimagecolorreference\empty}

%D \macros
%D   {doregisterfigure}
%D
%D Here is the fuzzy, very special dependant figure
%D registration special. We need to refer to the innermost
%D object (ximage).

        \def\doregisterfigure#1#2%
          {\doifundefined{IM::#1::#2}
             {\setxvalue{IM::#1::#2}{\the\pdflastximage}}%
           \xdef\PDFfigurereference{\getvalue{IM::#1::#2}}}

%D \macros
%D  {doovalbox}
%D
%D Drawing frames with round corners is inherited from the
%D main module.
%D
%D For drawing ovals we use quite raw \PDF\ code. The next
%D implementation does not differ that much from the one
%D implemented in the \POSTSCRIPT\ driver.

\def\doPDFovalcalc#1#2#3%
  {\PointsToBigPoints{\dimexpr#1+#2\relax}#3}

\def\doovalbox#1#2#3#4#5#6#7#8% todo: \scratchdimen/\scatchbox
  {\forcecolorhack
   \bgroup
   \dimen0=#4\divide\dimen0 \plustwo
   \doPDFovalcalc{0pt}{+\dimen0}\xmin
   \doPDFovalcalc{#1}{-\dimen0}\xmax
   \doPDFovalcalc{#2}{-\dimen0}\ymax
   \doPDFovalcalc{-#3}{+\dimen0}\ymin
   \advance\dimen0 by #5%
   \doPDFovalcalc{0pt}{+\dimen0}\xxmin
   \doPDFovalcalc{#1}{-\dimen0}\xxmax
   \doPDFovalcalc{#2}{-\dimen0}\yymax
   \doPDFovalcalc{-#3}{+\dimen0}\yymin
   \doPDFovalcalc{#4}{\zeropoint}\stroke
   \doPDFovalcalc{#5}{\zeropoint}\radius
   \edef\dostroke{#6}%
   \edef\dofill{#7}%
   \edef\mode{\number#8 \space}%
   % no \ifcase, else \relax in pdfcode
   \setbox\scratchbox\hbox
     {\ifnum\dostroke\dofill>\zerocount
        \ifPDFstrokecolor\else\ifnum\dostroke=\plusone
          \writestatus\m!colors{pdf stroke color will fail}\wait
        \fi\fi
        \PDFcode
          {q
           \stroke\space               w
           \ifcase\mode
             \xxmin\space \ymin \space m
             \xxmax\space \ymin \space l
             \xmax \space \ymin \space \xmax \space \yymin\space y
             \xmax \space \yymax\space l
             \xmax \space \ymax \space \xxmax\space \ymax \space y
             \xxmin\space \ymax \space l
             \xmin \space \ymax \space \xmin \space \yymax\space y
             \xmin \space \yymin\space l
             \xmin \space \ymin \space \xxmin\space \ymin \space y
             h
           \or % 1
             \xxmin\space \ymin \space m
             \xxmax\space \ymin \space l
             \xmax \space \ymin \space \xmax \space \yymin\space y
             \xmax \space \ymax \space l
             \xmin \space \ymax \space l
             \xmin \space \yymin\space l
             \xmin \space \ymin \space \xxmin\space \ymin \space y
             h
           \or % 2
             \xxmin\space \ymin \space m
             \xmax \space \ymin \space l
             \xmax \space \ymax \space l
             \xxmin\space \ymax \space l
             \xmin \space \ymax \space \xmin \space \yymax\space y
             \xmin \space \yymin\space l
             \xmin \space \ymin \space \xxmin\space \ymin \space y
             h
           \or % 3
             \xmin \space \ymin \space m
             \xmax \space \ymin \space l
             \xmax \space \yymax\space l
             \xmax \space \ymax \space \xxmax\space \ymax \space y
             \xxmin\space \ymax \space l
             \xmin \space \ymax \space \xmin \space \yymax\space y
             \xmin \space \ymin \space l
             h
           \or % 4
             \xmin \space \ymin \space m
             \xxmax\space \ymin \space l
             \xmax \space \ymin \space \xmax \space \yymin\space y
             \xmax \space \yymax\space l
             \xmax \space \ymax \space \xxmax\space \ymax \space y
             \xmin \space \ymax \space l
             \xmin \space \ymin\space l
             h
           \or % 5
             \xmin \space \ymin \space m
             \xmax \space \ymin \space l
             \xmax \space \yymax\space l
             \xmax \space \ymax \space \xxmax\space \ymax \space y
             \xmin \space \ymax \space l
             \xmin \space \ymin \space l
             h
           \or % 6
             \xmin \space \ymin \space m
             \xxmax\space \ymin \space l
             \xmax \space \ymin \space \xmax \space \yymin\space y
             \xmax \space \ymax \space l
             \xmin \space \ymax \space l
             \xmin \space \ymin \space l
             h
           \or
             \xxmin\space \ymin \space m
             \xmax \space \ymin \space l
             \xmax \space \ymax \space l
             \xmin \space \ymax \space l
             \xmin \space \yymin\space l
             \xmin \space \ymin \space \xxmin\space \ymin \space y
             h
           \or
             \xmin \space \ymin \space m
             \xmax \space \ymin \space l
             \xmax \space \ymax \space l
             \xxmin\space \ymax \space l
             \xmin \space \ymax \space \xmin \space \yymax\space y
             \xmin \space \ymin \space l
             h
           \or % 9 top open
             \xmin \space \ymax \space m
             \xmin \space \yymin\space l
             \xmin \space \ymin \space \xxmin\space \ymin \space y
             \xxmax\space \ymin \space l
             \xmax \space \ymin \space \xmax \space \yymin\space y
             \xmax \space \ymax \space l
           \or % 10 right open
             \xmax \space \ymax \space m
             \xxmin\space \ymax \space l
             \xmin \space \ymax \space \xmin \space \yymax\space y
             \xmin \space \yymin\space l
             \xmin \space \ymin \space \xxmin\space \ymin \space y
             \xmax\space  \ymin \space l
           \or % 11 bottom open
             \xmax \space \ymin \space m
             \xmax \space \yymax\space l
             \xmax \space \ymax \space \xxmax \space \ymax\space y
             \xxmin\space \ymax \space l
             \xmin \space \ymax \space \xmin \space \yymax\space y
             \xmin \space \ymin \space l
           \or % 12 left open
             \xmin \space \ymax \space m
             \xxmax\space \ymax \space l
             \xmax \space \ymax \space \xmax \space \yymax\space y
             \xmax \space \yymin\space l
             \xmax \space \ymin \space \xxmax\space \ymin \space y
             \xmin \space \ymin \space l
           \or % 13
             \xmin \space \ymax \space m
             \xxmax\space \ymax \space l
             \xmax \space \ymax \space \xmax \space \yymax\space y
             \xmax\space  \ymin \space l
           \or % 14
             \xmax \space \ymax \space m
             \xmax \space \yymin\space l
             \xmax \space \ymin \space \xxmax\space \ymin \space y
             \xmin \space \ymin \space l
           \or % 15
             \xmax \space \ymin \space m
             \xxmin\space \ymin \space l
             \xmin \space \ymin \space \xmin \space \yymin\space y
             \xmin \space \ymax \space l
           \or % 16
             \xmin \space \ymin \space m
             \xmin \space \yymax\space l
             \xmin \space \ymax \space \xxmin\space \ymax \space y
             \xmax \space \ymax \space l
           \or % 17
             \xxmax\space \ymax \space m
             \xmax \space \ymax \space \xmax \space \yymax\space y
           \or % 18
             \xmax \space \yymin\space m
             \xmax \space \ymin \space \xxmax\space \ymin \space y
           \or % 19
             \xxmin\space \ymin \space m
             \xmin \space \ymin \space \xmin \space \yymin\space y
           \or % 20
             \xmin \space \yymax\space m
             \xmin \space \ymax \space \xxmin\space \ymax \space y
           \or % 21
             \xxmax\space \ymax \space m
             \xmax \space \ymax \space \xmax \space \yymax\space y
             \xmin \space \yymax\space m
             \xmin \space \ymax \space \xxmin\space \ymax \space y
           \or % 22
             \xxmax\space \ymax \space m
             \xmax \space \ymax \space \xmax \space \yymax\space y
             \xmax \space \yymin\space m
             \xmax \space \ymin \space \xxmax\space \ymin \space y
           \or % 23
             \xmax \space \yymin\space m
             \xmax \space \ymin \space \xxmax\space \ymin \space y
             \xxmin\space \ymin \space m
             \xmin \space \ymin \space \xmin \space \yymin\space y
           \or % 24
             \xxmin\space \ymin \space m
             \xmin \space \ymin \space \xmin \space \yymin\space y
             \xmin \space \yymax\space m
             \xmin \space \ymax \space \xxmin\space \ymax \space y
           \or % 25
             \xxmax\space \ymax \space m
             \xmax \space \ymax \space \xmax \space \yymax\space y
             \xmax \space \yymin\space m
             \xmax \space \ymin \space \xxmax\space \ymin \space y
             \xxmin\space \ymin \space m
             \xmin \space \ymin \space \xmin \space \yymin\space y
             \xmin \space \yymax\space m
             \xmin \space \ymax \space \xxmin\space \ymax \space y
           \or % 26
             \xmax \space \yymin\space m
             \xmax \space \ymin \space \xxmax\space \ymin \space y
             \xmin \space \yymax\space m
             \xmin \space \ymax \space \xxmin\space \ymax \space y
           \or % 27
             \xxmax\space \ymax \space m
             \xmax \space \ymax \space \xmax \space \yymax\space y
             \xxmin\space \ymin \space m
             \xmin \space \ymin \space \xmin \space \yymin\space y
           \or % 28
           \fi
           \ifnum\mode>8
                                       S
           \else
             \ifnum\dostroke=\plusone  S \fi
             \ifnum\dofill  =\plusone  f \fi
           \fi
           Q}%
      \fi}%
   \wd\scratchbox#1\ht\scratchbox#2\dp\scratchbox#3\box\scratchbox
   \egroup}

%D \macros
%D   {dostartgraymode,dostopgraymode,
%D    dostartrgbcolormode,dostartcmykcolormode,dostartgraycolormode,
%D    dostopcolormode,
%D    dostartrotation,dostoprotation,
%D    dostartscaling,dostopscaling,
%D    dostartmirroring,dostopmirroring,
%D    dostartnegative,dostopnegative,
%D    dostartoverprint,dostopoverprint}

\def\dostartrotation#1% grouped
  {\setcalculatedcos\cos{#1}%
   \setcalculatedsin\sin{#1}%
   \forcecolorhack
   \PDFcode{q \cos\space\sin\space\negated\sin\space\cos\space0 0 cm}}

\def\dostoprotation
  {\PDFcode{Q}}

\def\@@PDFzeroscale{.0001}

\def\dostartscaling#1#2% the test is needed because acrobat is bugged!
  {\forcecolorhack
   \PDFcode{q \ifdim#1\points=\zeropoint\@@PDFzeroscale\else#1\fi\space 0 0
              \ifdim#2\points=\zeropoint\@@PDFzeroscale\else#2\fi\space 0 0 cm}}

\def\dostopscaling
  {\PDFcode{Q}}

\def\dostartmirroring{\PDFcode{-1 0 0 1 0 0 cm}}
\def\dostopmirroring {\PDFcode{-1 0 0 1 0 0 cm}}

\def\dostartnegative {\ifdefined\initializePDFnegative \initializePDFnegative \PDFcode{/GSnegative  gs}\fi}
\def\dostopnegative  {\ifdefined\initializePDFnegative \initializePDFnegative \PDFcode{/GSpositive  gs}\fi}
\def\dostartoverprint{\ifdefined\initializePDFoverprint\initializePDFoverprint\PDFcode{/GSoverprint gs}\fi}
\def\dostopoverprint {\ifdefined\initializePDFoverprint\initializePDFoverprint\PDFcode{/GSknockout  gs}\fi} % wrong

%D \macros
%D   {doPDFstartgraymode,doPDFstopgraymode,
%D    doPDFstartrgbcolormode,doPDFstartcmykcolormode,doPDFstartgraycolormode,
%D    doPDFstopcolormode}
%D
%D In \PDF\ there are two color states, one for strokes and one
%D for fills. This means that we have to set the color in a
%D rather redundant looking way. Unfortunately this makes the
%D \PDF\ file much larger than needed. We can save few bytes
%D by not setting the stroke color. Due to zip compression we
%D only save a few percent.

\def\dostartgraymode           #1{\PDFcode{#1 g\ifPDFstrokecolor\space#1 G\fi}}
\def\dostopgraymode              {\PDFcode{0  g\ifPDFstrokecolor\space 0 G\fi}}
\def\dostartrgbcolormode   #1#2#3{\PDFcode{#1 #2 #3   rg\ifPDFstrokecolor\space#1 #2 #3   RG\fi}}
\def\dostartcmykcolormode#1#2#3#4{\PDFcode{#1 #2 #3 #4 k\ifPDFstrokecolor\space#1 #2 #3 #4 K\fi}}
\def\dostartgraycolormode      #1{\PDFcode{#1 g\ifPDFstrokecolor\space#1 G\fi}}
\def\dostopcolormode             {\PDFcode{0  g\ifPDFstrokecolor\space0  G\fi}}

\def\dostartspotcolormode#1#2% redefining spotcolors is not possible anyway
  {\ifundefined{pdf:scs:#2}%
     \bgroup
     \getcommacommandsize[#2]%
     \ifcase\commalistsize\or
       \setxvalue{pdf:scs:#2}{#2 SCN #2 scn}% \setxvalue{pdf:scs:#2}{#2 SC #2 sc}%
     \else
       \let\PDFspotcolorspecs\empty
       \def\dospotcolorcommand##1{\edef\PDFspotcolorspecs{\PDFspotcolorspecs##1\space}}%
       \processcommacommand[#2]\dospotcolorcommand
       \setxvalue{pdf:scs:#2}{\PDFspotcolorspecs SCN \PDFspotcolorspecs scn}%
     \fi
     \egroup
   \fi
   \PDFcode{/#1 cs /#1 CS \PDFgetspotcolorspec{#2}}}

\def\PDFgetspotcolorspec#1%
  {\executeifdefined{pdf:scs:#1}\empty} % better no default than one with too less args

\def\dostartnonecolormode
  {\PDFcode{/None CS 1 SC /None cs 1 sc}}

%D We need to register the spot colors and their fallbacks.

% we cannot use /DeviceN since GS <=7.21 breaks on it
% and Jaws does not handle it at all {[/DeviceN [/All|/None]
% /Device#2 \PDFobjref\pdflastobj]} so we use separation
% colors that work and print ok

\def\doPDFregistersomespotcolor#1#2#3#4% implemented in the driver
  {\writestatus\m!systems{missing spot color definition}\wait}

\def\doregisternonecolor % internal command
  {\doregistergrayspotcolor{None}{1}%
   \globallet\doregisternonecolor\relax}

\def\dodoPDFregisterrgbspotcolor#1#2#3#4#5#6#7% name noffractions names p's r g b
  {\doPDFregistersomespotcolor{#1}{#2}{#3}{#4}{RGB}{0.0 1.0 0.0 1.0 0.0 1.0}%
     {\ifcase#2\or dup #5 mul exch dup #6 mul exch #7 mul\else#5 #6 #7\fi}}

\def\dodoPDFregistercmykspotcolor#1#2#3#4#5#6#7#8% name noffractions names p's c m y k
  {\doPDFregistersomespotcolor{#1}{#2}{#3}{#4}{CMYK}{0.0 1.0 0.0 1.0 0.0 1.0 0.0 1.0}%
     {\ifcase#2\or dup #5 mul exch dup #6 mul exch dup #7 mul exch #8 mul\else #5 #6 #7 #8\fi}}

\def\dodoPDFregistergrayspotcolor#1#2#3#4#5% name noffractions names p's s
  {\doPDFregistersomespotcolor{#1}{#2}{#3}{#4}{Gray}{0.0 1.0}%
     {\ifcase#2\or #5 mul\else #5\fi}}

\def\doregisterrgbspotcolor#1#2#3#4#5#6#7% name noffractions names p's r g b
  {\ifRGBsupported
     \dodoPDFregisterrgbspotcolor{#1}{#2}{#3}{#4}{#5}{#6}{#7}%
   \else
     \edef\@@cl@@r{#5}\edef\@@cl@@g{#6}\edef\@@cl@@b{#7}%
     \ifCMYKsupported
       \convertRGBtoCMYK\@@cl@@r\@@cl@@g\@@cl@@b
       \dodoPDFregistercmykspotcolor{#1}{#2}{#3}{#4}\@@cl@@c\@@cl@@m\@@cl@@y\@@cl@@k
     \else
       \convertRGBtoGRAY\@@cl@@r\@@cl@@g\@@cl@@b
       \dodoPDFregistergrayspotcolor{#1}{#2}{#3}{#4}\@@cl@@s
     \fi
   \fi}

\def\doregistercmykspotcolor#1#2#3#4#5#6#7#8% name noffractions names p's c m y k
  {\ifCMYKsupported
     \dodoPDFregistercmykspotcolor{#1}{#2}{#3}{#4}{#5}{#6}{#7}{#8}%
   \else
     \edef\@@cl@@c{#5}\edef\@@cl@@m{#6}\edef\@@cl@@y{#7}\edef\@@cl@@k{#8}%
     \ifRGBsupported
       \convertCMYKtoRGB\@@cl@@c\@@cl@@m\@@cl@@y\@@cl@@k
       \dodoPDFregisterrgbspotcolor{#1}{#2}{#3}{#4}\@@cl@@r\@@cl@@g\@@cl@@b
     \else
       \convertCMYKtoGRAY\@@cl@@c\@@cl@@m\@@cl@@y\@@cl@@k
       \dodoPDFregistergrayspotcolor{#1}{#2}{#3}{#4}\@@cl@@s
     \fi
   \fi}

\def\doregistergrayspotcolor{\dodoPDFregistergrayspotcolor}

%D New and very experimental.

\def\doregistercmykindexcolor#1#2#3#4#5#6#7#8% name noffractions names p's c m y k
  {\doPDFregistersomeindexcolor{#1}{#2}{#3}{#4}{CMYK}{0.0 1.0 0.0 1.0 0.0 1.0 0.0 1.0}%
     {dup #5 mul exch dup #6 mul exch dup #7 mul exch #8 mul}}

\def\doregisterrgbindexcolor#1#2#3#4#5#6#7% name noffractions names p's r g b
  {\doPDFregistersomeindexcolor{#1}{#2}{#3}{#4}{RGB}{0.0 1.0 0.0 1.0 0.0 1.0}%
     {dup #5 mul exch dup #6 mul exch #7 mul}}

\def\doregistergrayindexcolor#1#2#3#4#5% name noffractions names p's s
  {\doPDFregistersomeindexcolor{#1}{#2}{#3}{#4}{Gray}{0.0 1.0}%
     {pop}}

\let\checkpredefinedcolor\predefineindexcolor % we need an index in order to negate bitmaps

\def\doregisterfigurecolor#1% always an index color
  {\dogetobjectreference{PDFIX}{\internalspotcolorname{#1}}\PDFimagecolorreference}

\def\doregisterspotcolorname#1#2% no need for escape in luatex
  {\bgroup
   \let\ascii\empty
   \def\docommand##1%
     {\edef\ascii{\ascii
      \ifx\nexthandledtoken\space
        \letterhash20%
      \else\ifx\nexthandledtoken\blankspace
        \letterhash20%
      \else
        ##1%
      \fi\fi}}%
   \expanded{\handletokens#2}\with\docommand
   \letgvalue{@@pdf@@scn@@#1}\ascii
   \egroup}

\def\doPDFregistersomespotcolor#1#2#3#4#5#6#7% name fractions names p's space domain function
  {\bgroup
   \let\spotpops\empty
   \ifcase#2\or
     %def\PDFspotcolornames{/Separation /#1}%
     \edef\PDFspotcolornames{/Separation /\executeifdefined{@@pdf@@scn@@#1}{#1}}%
     \def\PDFspotcolordomain{0.0 1.0}%
   \else
     \dorecurse{#2}{\edef\spotpops{\spotpops pop }}%
     \let\PDFspotcolornames \empty
     \let\PDFspotcolordomain\empty
     \def\dospotcolorcommand##1%
       {\edef\PDFspotcolornames {\PDFspotcolornames/\executeifdefined{@@pdf@@scn@@##1}{##1}\space}%
        \edef\PDFspotcolordomain{\PDFspotcolordomain 0.0 1.0\space}}%
     \processcommacommand[#3]\dospotcolorcommand
     \edef\PDFspotcolornames{/DeviceN [\PDFspotcolornames]}%
   \fi
   \immediate \pdfobj stream attr
     {/FunctionType 4 /Domain [\PDFspotcolordomain] /Range [#6]}{{\spotpops#7}}%
   \immediate \pdfobj
     {[\PDFspotcolornames\space /Device#5 \PDFobjref\pdflastobj]}%
   \dosetobjectreference{PDFCS}{#1}{\the\pdflastobj}%
   \appendtoPDFdocumentcolorspaces{/#1 \PDFobjref\pdflastobj}%
   \egroup}

%D New and very experimental.

\def\doPDFregistersomeindexcolor#1#2#3#4#5#6#7% name fractions names p's space domain function
  {\bgroup
   \let\spotpops\empty
   \dorecurse{#2}{\edef\spotpops{\spotpops exch pop\space}}%
   \let\PDFspotcolornames \empty
   \let\PDFspotcolordomain\empty
   \def\docommand##1%
     {%\edef\PDFspotcolornames {\PDFspotcolornames/##1\space}%
      \edef\PDFspotcolornames{\PDFspotcolornames/\executeifdefined{@@pdf@@scn@@##1}{##1}\space}%
      \edef\PDFspotcolordomain{\PDFspotcolordomain 0.0 1.0\space}}%
   \processcommacommand[#3,None]\docommand
   \let\PDFcolorindexvector\empty
   \def\docommand##1%
     {\scratchdimen##1\points
      \scratchdimen\recurselevel\scratchdimen
      \scratchcounter\scratchdimen
      \divide\scratchcounter \maxcard
      \edef\PDFcolorindexvector{\PDFcolorindexvector\uchexnumbers\scratchcounter}}%
   %\dostepwiserecurse\zerocount{255}\plusone
   \dostepwiserecurse{255}\zerocount\minusone % we need to negate
     {\rawprocesscommacommand[#4,1]\docommand
      \xdef\PDFcolorindexvector{\PDFcolorindexvector\space}}%
   \immediate \pdfobj stream attr
     {/FunctionType 4 /Domain [\PDFspotcolordomain] /Range [#6]}{{\spotpops#7}}%
   \immediate \pdfobj
     {[/Indexed
         [/DeviceN [\PDFspotcolornames] /Device#5 \the\pdflastobj\space0 R] %
         255 <\PDFcolorindexvector>]}%
   \dosetobjectreference{PDFIX}{#1}{\the\pdflastobj}%
   \appendtoPDFdocumentcolorspaces{/#1_INDEXED \the\pdflastobj\space0 R}%
   \egroup}

%D \macros
%D   {dostarttransparency,dostoptransparency}
%D
%D For transparency, we need to implement a couple of
%D auxiliary macros. If needed, we will generalize them later.

\def\@@PDT{@PDT@}

\ifx\PDFcurrenttransparency\undefined
  \newcount\PDFcurrenttransparency \PDFcurrenttransparency=0 % -1
\fi

\def\assignPDFtransparency#1#2%
  {\edef\PDFtransparencyidentifier{/Tr#1}%
   \edef\PDFtransparencyreference{\PDFobjref{#2}}}

\def\presetPDFtransparency#1#2%
  {\initializePDFtransparency
   \executeifdefined{\@@PDT#1:#2}{\dopresetPDFtransparency{#1}{#2}}}

\def\dopresetPDFtransparency#1#2%
  {\global\advance\PDFcurrenttransparency \plusone
   \immediate\pdfobj{\PDFtransparancydictionary{#1}{#2}{}}%
   \edef\PDFtransparencyidentifier{/Tr\the\PDFcurrenttransparency}%
   \edef\PDFtransparencyreference {\PDFobjref\pdflastobj}%
   \setxvalue{\@@PDT#1:#2}%
     {\noexpand\assignPDFtransparency{\the\PDFcurrenttransparency}{\the\pdflastobj}}%
   \appendtoPDFdocumentextgstates
     {\PDFtransparencyidentifier\space
      \PDFtransparencyreference\space}}

\def\initializePDFtransparency
  {\immediate\pdfobj{\PDFtransparancydictionary{1}{1}{/AIS false}}%
   \xdef\PDFtransparencyresetidentifier{/Tr0}%
   \xdef\PDFtransparencyresetreference{\PDFobjref\pdflastobj}%
   \setxvalue{\@@PDT0:0}%
     {\noexpand\assignPDFtransparency{0}{\the\pdflastobj}}%
   \appendtoPDFdocumentextgstates
     {\PDFtransparencyresetidentifier\space
      \PDFtransparencyresetreference\space}%
   \global\let\initializePDFtransparency\relax}

%D Transparency support:

\def\PDFtransparancydictionary#1#2#3% type fraction extras
  {<</Type /ExtGState
     /ca #2 /CA #2
     /BM /\ifcase#1 Normal\or Normal\or Multiply\or Screen\or
          Overlay\or SoftLight\or HardLight\or ColorDodge\or
          ColorBurn\or Darken\or Lighten\or Difference\or
          Exclusion\else Compatible\fi
     #3>>}

\def\dodoPDFstarttransparency#1#2%
  {\presetPDFtransparency{#1}{#2}%
   \PDFcode{\PDFtransparencyidentifier\space gs }}

\def\dodoPDFstoptransparency
  {\PDFcode{/Tr0 gs }}

\def\dostarttransparency
  {\global\let\dostarttransparency\dodoPDFstarttransparency
   \global\let\dostoptransparency \dodoPDFstoptransparency
   \initializetransparency
   \dostarttransparency}

% This is tricky: because a text stream is handled before
% the page body is built, we can run into stops that will
% match an outer start; however, the stop is needed in case
% of a text color: [text color text] [other color text] on a
% first page combined with color splitting will go wrong if
% we stick to the relaxing method.

% \def\dostoptransparency
%   {\initializetransparency
%    \dodoPDFstoptransparency}

%D These use:

\let\initializetransparency\relax

\let\PDFtransparencyresetreference \empty
\let\PDFtransparencyresetidentifier\empty

\let\PDFtransparencyreference \empty
\let\PDFtransparencyidentifier\empty

%D New trickery:

\def\dostartgraphicgroup{\PDFcode{q}}
\def\dostopgraphicgroup {\PDFcode{Q}}

%D \macros
%D   {dostartclipping,dostopclipping}
%D
%D Clipping in \PDFTEX\ is rather trivial. We can even hook
%D in \METAPOST\ without problems.

\def\dostartclipping#1#2#3%
  {\PointsToBigPoints{#2}\width
   \PointsToBigPoints{#3}\height
   \grabMPclippath{#1}{1}\width\height
     {0 0 m \width\space 0 l \width \height l 0 \height l}%
   \pdfliteral % PDFcode ?
     {q 0 w \MPclippath\space W n}}

\def\dostopclipping
  {\pdfliteral{Q n}} % PDFcode

%D \macros
%D   {dosetupinteraction}
%D
%D Nothing special is needed to enable \PDF\ commands and
%D interaction. We stick with a message.

\def\dosetupinteraction
  {\showmessage\m!interactions{21}{pdftex}}

%D \macros
%D   {doresetgotowhereever,
%D    dostartthisisrealpage,dostartthisislocation,
%D    dostartgotorealpage,dostartgotolocation,dostartgotoJS}
%D
%D The interactions macros are the core of this module. We
%D support both page destinations and named ones. We don't
%D need the \type{\stop}||alternatives. We also don't need
%D to set the special that sets the real page number.

%D In the goto specials we took care of secondary references.
%D Here we define the macros used.

\def\doresetgotowhereever
  {\global\let\secondaryPDFreferences\empty}

\doresetgotowhereever % just to be sure

% we can (in etex) share more by testing on this

\def\savesecondaryPDFreference#1%
  {\@EA\xdef\csname PDF-SR:\the\nofsecondaryreferences\endcsname{#1}}

\def\savesecondaryPDFreference % #1 == \action
  {\global\@EA\let\csname PDF-SR:\the\nofsecondaryreferences\endcsname}

% test should happen in core-ref

\def\getsecondaryPDFreferences
  {\ifcase\nofsecondaryreferences\else
     \ifcsname PDF-SR:\the\nofsecondaryreferences\endcsname
       \xdef\secondaryPDFreferences{/Next <<\csname PDF-SR:\the\nofsecondaryreferences\endcsname\space\secondaryPDFreferences>>}%
     \fi
     \global\advance\nofsecondaryreferences \minusone
     \expandafter\getsecondaryPDFreferences
   \fi}

%D \macros
%D   {dostartthisislocation}
%D
%D Next we define the macros that deal with hyperreferencing,
%D graphic inclusion and general document features. These are
%D the olderst ones. I won't comment much because one needs
%D knowledge of \PDF\ itself, and explaning \PDF\ is beyond
%D this documentation.

\def\dostartthisislocation#1%
  {\bgroup
   \setPDFdestination{#1}%
   \ifx\PDFdestination\empty \else
     \pdfdest name {\PDFdestination}\PDFpageviewkey
   \fi
   \egroup}

\def\locationfilesuffix{pdf}

\def\dostartgotolocation#1#2#3#4#5#6%
  {\bgroup
   \doifelsenothing{#3}
     {\setPDFdestination{#5}%
      \doifelsenothing\PDFdestination
        {\let\action\empty}
        {\doifelsenothing{#4}
           {\let\PDFfile\empty}
           {\expanded{\beforesplitstring#4}\at.\to\PDFfile
            \doifparentfileelse\PDFfile % {#4}
              {\let\PDFfile\empty}
             %{\setreferencefilename#4.\locationfilesuffix\to\PDFfile
              {\@EA\setreferencefilename\PDFfile.\locationfilesuffix\to\PDFfile
               \edef\PDFfile
                 {R /F (\PDFfile)\ifgotonewwindow\space/NewWindow true \fi}}}%
         \edef\action%
           {/S /GoTo\PDFfile\space /D (\PDFdestination)}}}
     {\doifelsenothing{#4}
        {\let\PDFfile\empty
         \let\PDFdestination\empty}
        {\setreferencefilename/#4\to\PDFfile
         \setPDFdestination{#5}%
         \doifsomething\PDFdestination
           {\edef\PDFdestination{\letterhash\PDFdestination}}}%
      \edef\action{/S /URI /URI (#3\PDFfile\PDFdestination)}}%
   \ifx\action\empty\else
     \ifsecondaryreference
       \savesecondaryPDFreference\action
     \else
       \getsecondaryPDFreferences
       \insertpdfaction{\PDFswapdir#1}{#2}{\action \secondaryPDFreferences}%
     \fi
   \fi
   \egroup}

\def\PDFgotonewwindow{\ifgotonewwindow\space/NewWindow true \fi}

% optimization in tpd driver
%
% \edef\PDFdestination{(page:\the\scratchcounter)}%
%
%   ==>
%
% \advance\scratchcounter 1
% \edef\PDFdestination{[\pdfpageref \PDFobjref\scratchcounter\PDFpageviewwrd]}%
%
% \doPDFgetpagedestination#1#2% pagenumber macro % % fuzzy hack

\def\dostartgotorealpage#1#2#3#4#5% watch the R append trick
  {\bgroup
   \doifelsenothing{#3}% #1 = url
     {\scratchcounter0#5\relax
      \ifnum\scratchcounter>0
        \doifelsenothing{#4}
          {\let\PDFfile\empty}
          {\expanded{\beforesplitstring#4}\at.\to\PDFfile
           \doifparentfileelse\PDFfile % {#4}
             {\let\PDFfile\empty}
            %{\setreferencefilename#4.\locationfilesuffix\to\PDFfile
             {\@EA\setreferencefilename\PDFfile.\locationfilesuffix\to\PDFfile
              \edef\PDFfile{R /F (\PDFfile)\PDFgotonewwindow}}}%
        \ifx\PDFfile\empty
          \ifcase\overcomePDFpage
          \or % pdf starts numbering at zero
            \advance\scratchcounter \minusone
            \edef\PDFdestination{[\the\scratchcounter\space\PDFpageviewwrd]}%
          \or % pdf starts numbering at zero
            \advance\scratchcounter \minusone
            \edef\PDFdestination{(page:\the\scratchcounter)}%
          \or % pdftex starts numbering at one
            \edef\PDFdestination{[\pdfpageref\scratchcounter\space0 R \PDFpageviewwrd]}%
          \fi
        \else % across files it's a page number / pdf starts numbering at zero
          \advance\scratchcounter \minusone
          \edef\PDFdestination{[\the\scratchcounter\space\PDFpageviewwrd]}%
        \fi
        \edef\action{/S /GoTo\PDFfile\space /D \PDFdestination}%
      \else
        \let\action\empty
      \fi}
     {\doifelsenothing{#4}
        {\let\PDFfile\empty}
        {\setreferencefilename/#4\to\PDFfile}%
      \edef\action{/S /URI /URI (#3\PDFfile)}}%
   \ifx\action\empty\else
     \ifsecondaryreference
       \savesecondaryPDFreference\action
     \else
       \getsecondaryPDFreferences
       \insertpdfaction{\PDFswapdir#1}{#2}{\action \secondaryPDFreferences}%
     \fi
   \fi
   \egroup}

\let\lastfakedPDFpage\!!zerocount

\def\fakePDFpagedestination % as in pdf, we start numbering at zero
  {\iflocation \ifarrangingpages \ifnum\overcomePDFpage=\plustwo \else
     \ifnum\lastfakedPDFpage<\realpageno
       \bgroup
         \xdef\lastfakedPDFpage{\realfolio}%
         \advance\realpageno \minusone % is \expanded needed ?
         \normalexpanded{\noexpand\pdfdest name {page:\realfolio}\PDFpageviewkey}%
       \egroup
     \fi
   \fi \fi \fi}

\def\dostartgotoJS#1#2#3%
  {\bgroup
   \doPSsanitizeJScode#3\to\sanitizedJScode
   \edef\action{/S /JavaScript /JS (\sanitizedJScode)}%
   \ifsecondaryreference
     \savesecondaryPDFreference\action
   \else
     \getsecondaryPDFreferences
     \insertpdfaction{\PDFswapdir#1}{#2}{\action \secondaryPDFreferences}%
   \fi
   \egroup}

%D When going to a location, we obey the time and space saving
%D boolean \type{\ifusepagedestination}. Named destinations are
%D stripped and made robust. This all happens in the macros
%D called for.

%D \macros
%D   {doflushJSpreamble}
%D
%D It does not make sense to duplicate common \JAVASCRIPT\
%D functions, and therefore they can be predefined and must be
%D output separately. Currently this special is not shared
%D with the \ACROBAT\ one, simply because \DISTILLER\ does not
%D yet support something \type{\pdfnames}.

% \oneJSpreamblefalse  % buggy in acrobat

\def\doflushJSpreamble#1%
  {\bgroup
   \let\compositeJScode\empty
   \def\docommand##1%
     {\edef\sanitizedJScode{\getJSpreamble{##1}}%
      \@EA\doPSsanitizeJScode\sanitizedJScode\to\sanitizedJScode
      \immediate\pdfobj {<< /S /JavaScript /JS (\sanitizedJScode) >>}%
      \edef\compositeJScode
        {\compositeJScode\space (##1) \PDFobjref\pdflastobj}}%
   \processcommalist[#1]\docommand
   \immediate\pdfobj{<< /Names [ \compositeJScode ] >>}%
   \pdfnames{/JavaScript \PDFobjref\pdflastobj}%
   \egroup}

%D \macros
%D   {dostarthide,dostophide}
%D
%D Hiding parts of the document for printing is not yet
%D supported by \PDF\ and therefore \PDFTEX.

\let\dostarthide\donothing
\let\dostophide \donothing

%D \macros
%D   {doPDFsetupscreen,doPDFsetupidentity}
%D
%D Opposite to \DVI\ drivers, \PDF\ ones must know which what
%D page dimensions they are dealing. We also use the
%D opportunity to launch full screen (1) or show bookmarks (2).
%D
%D Setting of the screen boundingbox involves some
%D calculations. Here we also take care of (non) full screen
%D startup. The dimensions are rounded. Because \PDFTEX\ and
%D \ACROBAT\ handle setting the page dimensions in a
%D different way, we do not share this special.

\def\dosetupscreen{\doPDFsetupscreen\pdfpageheight}

\let\currentPDFpagemode   \empty % document catalog
\let\currentPDFviewerprefs\empty % document catalog

\let\currentPDFcropbox    \empty % page attributes
\let\currentPDFbleedbox   \empty % page attributes
\let\currentPDFartbox     \empty % page attributes
\let\currentPDFtrimbox    \empty % page attributes

\def\doPDFsetupscreen#1#2#3#4#5#6% watch the extra argument
  {\bgroup
   \xdef\currentPDFpagemode
     {\ifnum#6=4
        /PageLayout /TwoColumnRight
      \else
        /PageMode \ifcase#6
        /UseNone\or/FullScreen\or/UseOutlines\else/UseNone\fi
      \fi}%
   \xdef\currentPDFviewerprefs % space after #6 needed, else \relax
     {\ifcase#6 \or\or\else /ViewerPreferences << /FitWindow true >>\fi}%
   \egroup}

\def\addPDFdocumentinfo
  {\appendtopdfcatalog{\currentPDFpagemode\currentPDFviewerprefs}%
   \appendtopdfcatalog{/Version \ifdim\PDFversion00\points>100\points 1.\fi\PDFversion}%
   \appendtopdfinfo{/Trapped /False}%
   \appendtopdfinfo{/ConTeXt.Version (\contextversion)}%
   \appendtopdfinfo{/ConTeXt.Time    (\number\normalyear.\twodigits\normalmonth.\twodigits\normalday\space \twodigits\currenthour:\twodigits\currentminute)}%
   \appendtopdfinfo{/ConTeXt.Jobname (\jobname)}%
   \appendtopdfinfo{/ConTeXt.Url     (www.pragma-ade.com)}%
   \glet\addPDFdocumentinfo\relax}

\def\PDFversion{1.5}

\appendtoksonce
  \def\PDFversion{1.5}%
\to \everyresetspecials

\def\doPDFsetupwhateverbox#1#2#3#4#5#6% watch the extra arguments
  {\bgroup
   \!!widtha \dimexpr#5+#3\relax
   \!!heightb\dimexpr#2-#4\relax
   \!!heighta\dimexpr\!!heightb-#6\relax
   % sometimes whole values give better results
   % \PointsToWholeBigPoints{#3}\left
   % \PointsToWholeBigPoints\!!heighta\bottom
   % \PointsToWholeBigPoints\!!widtha \width
   % \PointsToWholeBigPoints\!!heightb\height
   % but since pdf/x does not round when checking if
   % the boxes fit inside the media box ...
   \PointsToBigPoints{#3}\left
   \PointsToBigPoints\!!heighta\bottom
   \PointsToBigPoints\!!widtha \width
   \PointsToBigPoints\!!heightb\height
   \xdef#1{[\left\space\bottom\space\width\space\height]}%
   \egroup}

\gdef\currentPDFtrimbox{\currentPDFcropbox} % default, needed for pdf/x

\def\dosetupartbox  {\doPDFsetupwhateverbox\currentPDFartbox  \pdfpageheight}
\def\dosetupcropbox {\doPDFsetupwhateverbox\currentPDFcropbox \pdfpageheight}
\def\dosetupbleedbox{\doPDFsetupwhateverbox\currentPDFbleedbox\pdfpageheight}
\def\dosetuptrimbox {\doPDFsetupwhateverbox\currentPDFtrimbox \pdfpageheight}

\def\flushPDFpageboxes
  {\edef\currentPDFtrimbox{\currentPDFtrimbox}%
   \ifx\currentPDFartbox  \empty\else\appendtopdfpageattributes{/ArtBox   \currentPDFartbox  }\fi
   \ifx\currentPDFcropbox \empty\else\appendtopdfpageattributes{/CropBox  \currentPDFcropbox }\fi
   \ifx\currentPDFbleedbox\empty\else\appendtopdfpageattributes{/BleedBox \currentPDFbleedbox}\fi
   \ifx\currentPDFtrimbox \empty\else\appendtopdfpageattributes{/TrimBox  \currentPDFtrimbox }\fi}

%D \macros
%D   {dostartexecutecommand}
%D
%D \PDF\ viewers enable us to navigate using menus and shortcut
%D keys. These navigational tools can also be accessed by using
%D annotations. The next special takes care of inserting them.
%D
%D At the cost of much auxiliary placeholders, we can pretty
%D fast convert the command asked for. This is how the \PDF\
%D code looks like.

\def\PDFmoviecode#1#2#3%
  {/Movie
   /T (\ifcase#1movie \else sound \fi\ifx\argumentA\empty#2\else\argumentA\fi)
   /Operation /\ifcase#3Play\or Stop\or Pause\or Resume\fi\space}

\def\PDFexecutestartmovie  {\PDFmoviecode0\currentmovie0}
\def\PDFexecutestopmovie   {\PDFmoviecode0\currentmovie1}
\def\PDFexecutepausemovie  {\PDFmoviecode0\currentmovie2}
\def\PDFexecuteresumemovie {\PDFmoviecode0\currentmovie3}

\def\PDFexecutestartsound  {\PDFmoviecode1\currentsound0}
\def\PDFexecutestopsound   {\PDFmoviecode1\currentsound1}
\def\PDFexecutepausesound  {\PDFmoviecode1\currentsound2}
\def\PDFexecuteresumesound {\PDFmoviecode1\currentsound3}

\def\PDFformcode#1%
  {\doiffieldset{#1}{/Field [\dogetfieldset{#1}]}}

% bit 3 = html
% bit 6 = xml
% bit 4 = get

\ifx\PDFsubmitfiller\undefined \let\PDFsubmitfiller\empty \fi

\chardef\PDFformmethod=1 % 0=GET 1=POST

\def\PDFformflag#1#2{\ifcase\PDFformmethod#1\else#2\fi}

\def\PDFexecuteimportform  {/Named /N /AcroForm:ImportFDF}
\def\PDFexecuteexportform  {/Named /N /AcroForm:ExportFDF}
\def\PDFexecuteresetform   {/ResetForm  \PDFformcode\argumentA}
\def\PDFexecutesubmitform  {/SubmitForm \PDFformcode\argumentB
                            /Flags \ifcase\submitoutputformat\space
                                         \PDFformflag{12} {4} % 0=unknown
                                   \or   \PDFformflag{12} {4} % 1=HTML
                                   \or   \PDFformflag {8} {0} % 2=FDF
                                   \or   \PDFformflag{40}{32} % 3=XML
                                   \else \PDFformflag{12} {4} % ?=unknown
                                   \fi
                            /F (\argumentA)\PDFsubmitfiller}

% urifill permits url substitution

\def\PDFexecutehide        {/Hide /T (\argumentA) /H true}
\def\PDFexecuteshow        {/Hide /T (\argumentA) /H false}

\def\PDFexecutefirst       {/Named /N /FirstPage}
\def\PDFexecuteprevious    {/Named /N /PrevPage}
\def\PDFexecutenext        {/Named /N /NextPage}
\def\PDFexecutelast        {/Named /N /LastPage}
\def\PDFexecutebackward    {/Named /N /GoBack}
\def\PDFexecuteforward     {/Named /N /GoForward}
\def\PDFexecuteprint       {/Named /N /Print}
\def\PDFexecuteexit        {/Named /N /Quit}
\def\PDFexecuteclose       {/Named /N /Close}
\def\PDFexecutesave        {/Named /N /Save}
\def\PDFexecutesavenamed   {/Named /N /SaveAs}
\def\PDFexecuteopennamed   {/Named /N /Open}
\def\PDFexecutehelp        {/Named /N /HelpUserGuide}
\def\PDFexecutetoggle      {/Named /N /FullScreen}
\def\PDFexecutesearch      {/Named /N /Find}
\def\PDFexecutesearchagain {/Named /N /FindAgain}
\def\PDFexecutegotopage    {/Named /N /GoToPage}
\def\PDFexecutequery       {/Named /N /AcroSrch:Query}
\def\PDFexecutequeryagain  {/Named /N /AcroSrch:NextHit}
\def\PDFexecutefitwidth    {/Named /N /FitWidth}
\def\PDFexecutefitheight   {/Named /N /FitHeight}

\let\PDFobjectclass\empty
\let\PDFobjectname \empty

\def\dostartexecutecommand#1#2#3#4%
  {\doifdefined{PDFexecute#3}
     {\bgroup
      \edef\argument{#4}%
      \ifx\argument\empty
        \let\argumentA\empty
        \let\argumentB\empty
      \else
        \@EA\dogetcommalistelement\@EA1\@EA\from#4\to\argumentA
        \@EA\dogetcommalistelement\@EA2\@EA\from#4\to\argumentB
      \fi
      \edef\action%
        {/S \getvalue{PDFexecute#3}}%
      \ifsecondaryreference
        \savesecondaryPDFreference\action
      \else
        \getsecondaryPDFreferences
%        \ifx\PDFobjectclass\empty
%          \let\next\insertpdfaction
%        \else
%          \edef\next{\createpdfactionobject{\PDFobjectclass}{\PDFobjectname}}%
%          \globalletempty\PDFobjectclass
%          \globalletempty\PDFobjectname
%        \fi
%        \next
        \insertpdfaction{\PDFswapdir#1}{#2}{\action \secondaryPDFreferences}%
      \fi
      \egroup}}

%D \macros
%D   {dosetupidentity}
%D
%D Documents can be tagged with an application accessible title
%D and subtitle, the authorname, a date, the creator, keywords
%D etc. For the moment \PDFTEX\ only supports the first three
%D of these.

\def\dosetupidentity#1#2#3#4#5#6%
  {\normalexpanded{\noexpand\appendtopdfinfo
     {/Title   <\hexifiedPDFstring{#1}>
      /Subject <\hexifiedPDFstring{#2}>
      /Author  <\hexifiedPDFstring{#3}>
      /Creator <\hexifiedPDFstring{#4}>
      /ModDate (#4)
      /ID (\jobname.#5) % needed for pdf/x
      /Keywords <\hexifiedPDFstring{#6}>}}}

%D \macros
%D   {dostartrunprogam}
%D
%D We can run a program form within a document, although this
%D feature is rather weak, due to path problems and buggy
%D argument passing.

\def\dostartrunprogram#1#2#3#4% new: #3 => #3#4
  {\bgroup
  %\edef\string{#3}%
  %\@EA\beforesplitstring\string\at{ }\to\program
  %\@EA\aftersplitstring \string\at{ }\to\parameters
  %\edef\action%
  %  {/S /Launch /F (\program) /P (\parameters) /D (.)}%
   \edef\action
     {/S /Launch /F (#3) /P (#4) /D (.)}%
   \ifsecondaryreference
     \savesecondaryPDFreference\action
   \else
     \getsecondaryPDFreferences
     \insertpdfaction{\PDFswapdir#1}{#2}{\action \secondaryPDFreferences}%
   \fi
   \egroup}

%D \macros
%D   {dostartgotoprofile, dostopgotoprofile,
%D    dobeginofprofile, doendofprofile}
%D
%D \CONTEXT\ user profiles and version control fall back on
%D \PDF\ article threads. Unfortunately one cannot influence
%D the view yet in an (for me) acceptable way.

\def\dostartgotoprofile#1#2#3% to be done: file
  {\bgroup
   \setPDFdestination{#3}%
   \doifsomething\PDFdestination
     {\edef\action
        {/S /Thread /D (\PDFdestination)}%
      \ifsecondaryreference
        \savesecondaryPDFreference\action
      \else
        \getsecondaryPDFreferences
        \insertpdfaction{\PDFswapdir#1}{#2}{\action \secondaryPDFreferences}%
      \fi}%
   \egroup}

%D Some day, I'll reimplement threading in a useful way.
%D Currently the viewers handle threads rather diffuse.

\def\dobeginofprofile#1#2#3#4%
  {\setPDFdestination{#1}%
   \doifsomething\PDFdestination
     {\pdfthread
        width #2 height #3
        attr {/Title (\PDFdestination)} % can be omitted
        name {\PDFdestination}}}

\def\doendofprofile
  {}

%D \macros
%D  {doinsertbookmark}
%D
%D In \PDF\ bookmarks are the building blocks of a viewer
%D provided sort of table of contents. \TEX\ has to provide
%D the entry as well as the number of child entries. Strings
%D need to be sanatized as good as possible to suit the default
%D encoding. In \CONTEXT\ users can overrule this string by
%D supplying an alternative one. Look at the macro called for
%D to see how funny these bookmarks are defined.

\def\doinsertbookmark#1#2#3#4#5% level sublevels text page open=1
  {\bgroup
   \doPDFgetpagereference{#4}\PDFobjectreference
   \pdfoutline
     user {<</S /GoTo /D [\PDFobjectreference\space\PDFpageviewwrd]>>}%
     \ifcase#2 \else count \ifcase#5-\fi#2 \fi
%      {<\hexifiedPDFstring{#3}>}% goes wrong
     {<#3>}%
   \egroup}

%D \macros
%D  {dostartobject,dostopobject,doinsertobject}
%D
%D Due to \PDF's object oriented character, we can include and
%D reuse objects. These can be compared with \TEX's boxes. The
%D \TEX\ counterpart is defined in the module \type{spec-dvi}.
%D We don't use the dimensions here.
%D
%D The next solution is not that beautiful. Because objects are
%D containers for whatever kind of content, graphics can be
%D part of this content, and a graphic object can be part of
%D the more general type. In practice this means that an ximage
%D would be embedded in an xform, which in itself is not that
%D big a problem, apart from a few bytes overhead. However, for
%D reasons unknown to me alternative images must be pure
%D ximages |<|indeed, somehow one cannot use a vector graphic
%D as alternative|>| that are not embedded into forms, so this
%D is why the object handler treats them different. This
%D implies knowledge of the calling routines, especially the
%D \type{FIG} trigger, that signals that we just embedded an
%D image. Alternatively I could have introduced a dual object
%D system, but the overhead in duplicate specials is currently
%D not what we want. I'd rather implement a more mature
%D object support system from scratch.

\let\currentPDFresources\empty
\let\PDFimageattributes \empty
\let\PDFfigurereference \empty
\let\PDFimagereference  \empty

\def\dostartobject#1#2#3#4#5%
  {\bgroup
   \setbox\nextbox\vbox\bgroup
   \def\dodostopobject
     {\egroup
      \ifx\PDFimagereference\empty
        % We also flush page resources, since shared
        % resources end up there; otherwise transparencies
        % won't work in xforms; some day I will optimize
        % this.
        \the\everyPDFxform
        \finalizeobjectbox\nextbox
        \immediate\pdfxform
          resources {\currentPDFresources\the\pdfpageresources}%
          \nextbox
        \global\let\currentPDFresources\empty
        \dosetobjectreference{#1}{#2}{\the\pdflastxform}%
      \else
        \dosetobjectreference{#1}{#2}{-\PDFimagereference}%
        \global\let\PDFimagereference\empty
      \fi}}

\def\dostopobject
  {\dodostopobject
   \egroup}

\def\doresetobjects
  {\global\let\PDFimagereference\empty}

\def\doinsertobject#1#2%
  {\bgroup
   \doifobjectreferencefoundelse{#1}{#2}
     {\dogetobjectreference{#1}{#2}\PDFobjectreference
      \ifnum\PDFobjectreference<0
        \@EA\@EA\@EA\pdfrefximage\@EA\gobbleoneargument\PDFobjectreference
      \else
        \pdfrefxform\PDFobjectreference
      \fi}%
     {}%
   \egroup}

\appendtoksonce
    \collectPDFresources
    \global\let\currentPDFresources\collectedPDFresources
\to \everyPDFxform

%D \macros
%D   {dosetpagetransition}
%D
%D Page transitions only make sence in presentations. They are
%D passed as raw \PDF\ code to the page object. Take a look
%D at the implementation to get an impression of the rubish
%D passed on.
%D
%D This array holds a reasonable selection of transitions
%D (watch out: \type{replace} is not in this list). Most of
%D the transitions look awful anyway. By the way, \CONTEXT\ is
%D able to select transitions randomly.

\def\pagetransitions
  {{split,in,vertical},{split,in,horizontal},
   {split,out,vertical},{split,out,horizontal},
   {blinds,horizontal},{blinds,vertical},
   {box,in},{box,out},
   {wipe,east},{wipe,west},{wipe,north},{wipe,south},
   dissolve,
   {glitter,east},{glitter,south},
   {fly,in,east},{fly,in,west},{fly,in,north},{fly,in,south},
   {fly,out,east},{fly,out,west},{fly,out,north},{fly,out,south},
   {push,east},{push,west},{push,north},{push,south},
   {cover,east},{cover,west},{cover,north},{cover,south},
   {uncover,east},{uncover,west},{uncover,north},{uncover,south},
   fade}

%D Again, we use macros as placeholders for \PDF\ key||value
%D pairs.

\def\PDFpagesplit    {/S /Split    }
\def\PDFpageblinds   {/S /Blinds   }
\def\PDFpagebox      {/S /Box      }
\def\PDFpagewipe     {/S /Wipe     }
\def\PDFpagedissolve {/S /Dissolve }
\def\PDFpageglitter  {/S /Glitter  }
\def\PDFpagereplace  {/S /R        }

\def\PDFpagefly      {/S /Fly      } % 1.5
\def\PDFpagepush     {/S /Push     } % 1.5
\def\PDFpagecover    {/S /Cover    } % 1.5
\def\PDFpageuncover  {/S /Uncover  } % 1.5
\def\PDFpagefade     {/S /Fade     } % 1.5

\def\PDFpagehorizontal {/Dm /H  }
\def\PDFpagevertical   {/Dm /V  }
\def\PDFpagein         {/M  /I  }
\def\PDFpageout        {/M  /O  }
\def\PDFpageeast       {/Di   0 }
\def\PDFpagenorth      {/Di  90 }
\def\PDFpagewest       {/Di 180 }
\def\PDFpagesouth      {/Di 270 }

\def\dodoPDFsetpagetransition#1%
  {\doifdefined{PDFpage#1}
     {\edef\PDFpagetransitions{\PDFpagetransitions\getvalue{PDFpage#1}}}}

\def\dosetpagetransition#1#2%
  {\let\PDFpagetransitions\empty
   \processcommalist[#1]\dodoPDFsetpagetransition
   \appendtopdfpageattributes
    %{\ifnum#2>0 /Dur #2 \fi
     {\ifnum0<0#2 /Dur #2 \fi
      \ifx\PDFpagetransitions\empty\else/Trans <<\PDFpagetransitions>>\fi}}

%D The expansion is needed because else the \type{\pdfpageattr}
%D token list flushes an unexpanded \type{\csname}. The
%D \type{\global} is needed because the assignment can take
%D place deeply buried (for instance in the \type{\shipout}
%D box.

%D \macros
%D   {doinsertcomment, doflushcomments}
%D
%D Text annotation, or comments, are provided too:

%D \macros
%D   {dopresetlinefield,dopresettextfield,
%D    dopresetchoicefield,dopresetpopupfield,dopresetcombofield,
%D    dopresetpushfield,dopresetcheckfield,
%D    dopresetradiofield,dopresetradiorecord}
%D
%D \PDF\ offers extensive field support. The next bunch of
%D definitions map the specials.

%D \macros
%D   {dodefinefieldset,dogetfieldset,doiffieldset}
%D
%D Field sets, needed for reset and submit handling, are
%D taken care of by:

%D The next section of this module is dedicated to form
%D support. These macros are complicated by the fact that
%D cloning is possible.

%D \macros
%D   {FDFflag...,FDFplus...}
%D
%D The \type{/FT} key determines the type of field: text,
%D button or choice. The latter two come in several disguises,
%D which are set by flipping bits in the \type{/Ff}. Other bits
%D are used to set states. Personally I hate this bitty way of
%D doing things. The next six bit determine the field sub type:

\def\FDFflagMultiLine          {4096} % 13
\def\FDFflagNoToggleToOff     {16384} % 15
\def\FDFflagRadio             {32768} % 16
\def\FDFflagPushButton        {65536} % 17
\def\FDFflagPopUp            {131072} % 18
\def\FDFflagEdit             {262144} % 19

% bugged anyway, so we need to drop it:

\def\FDFflagRadiosInUnison {33554432} % 26

%D A few more (pdf 1.4) flags, what the spell check one: for
%D obscure reasons for Adobe downward compatibility means
%D enabling features that harm old applications like testing.

\def\FDFflagDoNotSpellCheck {4194304} % 23
\def\FDFflagDoNotScroll     {8388608} % 24

%D The next bits (watch how strange the bits are organized)
%D take care of the states:

\def\FDFflagReadOnly              {1} %  1
\def\FDFflagRequired              {2} %  2
\def\FDFflagNoExport              {4} %  3
\def\FDFflagPassword           {8192} % 14
\def\FDFflagSort             {524288} % 20
\def\FDFflagFileSelect      {1048576} % 21

%D There is a second, again bitset oriented, \type{/F} flag:

\def\FDFplusInvisible             {1} %  1
\def\FDFplusHidden                {2} %  2
\def\FDFplusPrintable             {4} %  3

%def\FDFplusNoView               {32} %  6
%def\FDFplusToggleNoView        {256} %  9

\def\FDFplusAutoView            {256} % {288} %  6+9

%D \macros
%D   {setFDFswitches}
%D
%D The non||type bits are mapped onto user||interface
%D swithes, to be used later on:

\def\@@FDFflag{FDFflag}
\def\@@FDFplus{FDFplus}

\letvalue    {\@@FDFflag\v!readonly}=\FDFflagReadOnly
\letvalue    {\@@FDFflag\v!required}=\FDFflagRequired
\letvalue   {\@@FDFflag\v!protected}=\FDFflagPassword
\letvalue      {\@@FDFflag\v!sorted}=\FDFflagSort
\letvalue {\@@FDFflag\v!unavailable}=\FDFflagNoExport
\letvalue     {\@@FDFflag\v!nocheck}=\FDFflagDoNotSpellCheck
\letvalue       {\@@FDFflag\v!fixed}=\FDFflagDoNotScroll
\letvalue        {\@@FDFflag\v!file}=\FDFflagFileSelect

\letvalue      {\@@FDFplus\v!hidden}=\FDFplusHidden
\letvalue   {\@@FDFplus\v!printable}=\FDFplusPrintable

\letvalue        {\@@FDFplus\v!auto}=\FDFplusAutoView

%D A set of switches is collected into the flags we mentioned
%D before by the next macro (we don't handle negations yet,
%D but do take care of redundancy):

\def\FDFflag{0}
\def\FDFplus{0}

\def\setFDFswitches[#1]%
  {\bgroup
   \!!counta\zerocount
   \!!countb\zerocount
   \def\docommand##1%
     {\doifsomething{##1}
        {\advance\!!counta 0\getvalue{\@@FDFflag##1}%
         \setvalue{\@@FDFflag##1}{0}%
         \advance\!!countb 0\getvalue{\@@FDFplus##1}%
         \setvalue{\@@FDFplus##1}{0}}}%
   \processcommacommand[#1]\docommand
   \xdef\FDFflag{\the\!!counta}%
   \xdef\FDFplus{\the\!!countb}%
   \egroup}

%D \macros
%D   {setFDFvalues}
%D
%D Menu items are passed as an array of \type{(string)}'s and
%D the content of this array is build with:

\let\FDFvalues      \empty
\let\FDFfirstvalues \empty
\let\FDFsecondvalues\empty
\let\FDFkidlist     \empty
\let\FDFdefaultindex\!!zerocount
\let\FDFdefaultvalue\empty

% Why do we need to tweak this mechanism each time acrobat updates ...
% it would make sense to have version specific sections in pdf files
% since my guess is that it never will be done right since each year
% new programmers have new ideas about what is supposed to happen with
% kids. So .. best is not to trust this feature esp not for radio
% widgets. (new flags, different interpretation of AS etc etc)

\def\setFDFvalues[#1][#2]% #1 = list (item=>value) #2 = default
  {\let\FDFvalues      \empty
   %when radio opt works ok
   %\let\FDFfirstvalues \empty
   %\let\FDFsecondvalues\empty
   \let\FDFkidlist     \empty
   %\let\FDFdefaultindex\!!zerocount
   %\let\FDFdefaultvalue\empty
   %\scratchcounter\zerocount
   \def\dodocommand##1=>##2=>##3\end
     {\addtocommalist{##1}\FDFkidlist
      %\edef\FDFfirstvalues{\FDFfirstvalues(##1)}%
      %\doif{##1}{#2}{\edef\FDFdefaultindex{\the\scratchcounter}}%
      %\advance\scratchcounter\plusone
      \doifelsenothing{##2}
        {\doif{##1}{#2}{\edef\FDFdefaultvalue{##1}}%
         %\edef\FDFsecondvalues{\FDFsecondvalues(##1)}%
         \edef\FDFvalues{\FDFvalues [(##1)(##1)] }}
        {\doif{##1}{#2}{\edef\FDFdefaultvalue{##2}}%
         %\edef\FDFsecondvalues{\FDFsecondvalues(##2)}%
         \edef\FDFvalues{\FDFvalues [(##2)(##1)] }}}% ! ##1 is shown
   \def\docommand##1%
     {\dodocommand##1=>=>\end}%
   \expanded{\processcommalist[#1]}\docommand}

%D This macro accepts comma separated \type{visual=>result}
%D pairs.

%D \macros
%D   {setFDFalignment}
%D
%D Text and line fields can be entered and showed in three
%D alternative alingments, indicated by a digit:

\def\FDFalign{0}

\def\setFDFalignment[#1]%
  {\processaction
     [#1]
     [  \v!left=>\edef\FDFalign{2},    % raggedleft
      \v!middle=>\edef\FDFalign{1},    % raggedcenter
       \v!right=>\edef\FDFalign{0}]}   % raggedright

%D \macros
%D   {setFDFattributes}
%D
%D The weak part of (at least version 2.1 \PDF) is that only
%D default fonts are handled well. Another restriction is that
%D the encoding vector must be the standard \PDF\ document one.
%D Although the \PDF\ reference explictly states that one could
%D use the normal text operators, leading is not yet handled.
%D
%D For the moment the current \CONTEXT\ font is mapped onto
%D one best suitable default font. The color attribute is
%D less problematic and is directly derived from the \CONTEXT\
%D color.

\def\FDFattributes{/Helv 12 Tf 0 g 14.4 TL}

\def\FDFrm  {TiRo} \def\FDFss  {Helv} \def\FDFtt  {Cour}
\def\FDFrmtf{TiRo} \def\FDFsstf{Helv} \def\FDFtttf{Cour}
\def\FDFrmbf{TiBo} \def\FDFssbf{HeBo} \def\FDFttbf{CoBo}
\def\FDFrmit{TiIt} \def\FDFssit{HeOb} \def\FDFttit{CoOb}
\def\FDFrmsl{TiIt} \def\FDFsssl{HeOb} \def\FDFttsl{CoOb}
\def\FDFrmbi{TiBI} \def\FDFssbi{HeBO} \def\FDFttbi{CoBO}
\def\FDFrmbs{TiBI} \def\FDFssbs{HeBO} \def\FDFttbs{CoBO}

\let\FDFusedfonts=\FDFsstf

\def\setFDFattributes[#1,#2,#3,#4]% style, color, backgroundcolor, framecolor
  {\bgroup % nog interlinie: n TL
   \setbox\scratchbox\hbox
     \bgroup
       \doconvertfont{#1}{}%
       \PointsToBigPoints\bodyfontsize\size % x/xx, so better the actual size
       \doifdefinedelse{FDF\fontstyle\fontalternative}
         {\xdef\FDFattributes{\getvalue{FDF\fontstyle\fontalternative}}}
         {\doifdefinedelse{FDF\fontstyle}
            {\xdef\FDFattributes{\getvalue{FDF\fontstyle}}}
            {\xdef\FDFattributes{\FDFrm}}}%
       \doglobal\addtocommalist\FDFattributes\FDFusedfonts
       \xdef\FDFattributes% move up with "x.y Ts"
         {/\FDFattributes\space\size\space Tf\space\PDFcolor{#2}}%
       \doifelsenothing{#3}
         {\global\let\FDFsurroundings\empty}
         {\xdef\FDFsurroundings{/BG \FDFcolor{#3}}}%
       \doifsomething{#4}
         {\xdef\FDFsurroundings{\FDFsurroundings\space /BC \FDFcolor{#4}}}%
       \ifx\FDFsurroundings\empty \else
         \xdef\FDFsurroundings{/MK << \FDFsurroundings\space>>}%
       \fi
     \egroup
   \egroup}

%D \macros
%D   {setFDFactions}
%D
%D Depending on the type of the field, one can assign
%D \JAVASCRIPT\ code to a mouse event or keystroke. The next
%D preparation macro shows what events are handled.

\let\FDFactions\empty

\def\setFDFactions[#1,#2,#3,#4,#5,#6,#7,#8,%
  {\global\let\FDFactions\empty
   \setFDFaction D#1%  mousedown
   \setFDFaction U#2%  mouseup
   \setFDFaction E#3%  enterregion
   \setFDFaction X#4%  exitregion
   \setFDFaction K#5%  afterkeystroke
   \setFDFaction F#6%  formatresult
   \setFDFaction V#7%  validateresult
   \setFDFaction C#8%  calculatewhatever
   \setFDFactionsmore}

\def\setFDFactionsmore#1,#2]%
  {\setFDFaction{Fo}#1%  focusin
   \setFDFaction{Bl}#2%  focusout % was I (now pdf ref manual explicitly talks about lowercase l)
   \ifx\FDFactions\empty\else
     \xdef\FDFactions{/AA << \FDFactions >>}% since 1.3 no longer inherited
   \fi}

% todo, when new var scheme is implemented
%
%   \setFDFaction{PO}\@@DriverFieldPageOpen
%   \setFDFaction{PC}\@@DriverFieldPageClose
%   \setFDFaction{PV}\@@DriverFieldPageVisible
%   \setFDFaction{PI}\@@DriverFieldPageInVisible

%D The event handler becomes something:
%D
%D \starttyping
%D /AA << /D << /S ... >> ... /C << /S ... >>
%D /A << /S /JavaScript /JS (...) >>
%D \stoptyping

\def\setFDFaction#1#2%
  {\bgroup
   \def\docommand{\xdef\FDFactions{\FDFactions /#1 << \lastPDFaction >> }}%
   \@EA\handlereferenceactions\@EA{#2}\docommand % one level expansion
   \egroup}

%D \macros
%D   {testFDFactions}
%D
%D This rather confusion prone series of script can be tested
%D with:
%D
%D \starttyping
%D \testFDFactions
%D \stoptyping
%D
%D which simply redefined the previous macro to one that prints
%D a message to the console.

\def\testFDFactions
  {\def\setFDFaction##1##2%
     {\doPSsanitizeJScode console.show();console.println("executing:##1"); \to\sanitizedJScode
      \edef\FDFactions{\FDFactions /##1 << /S /JavaScript /JS (\sanitizedJScode) >> }}}

%D \macros
%D   {doregistercalculationset}
%D
%D There is at most one calculation order list, which defines
%D the order in which fields are calculated. The calculation
%D order is defined using:

\let\PDFcalculationset\empty

\def\doregistercalculationset#1%
  {\def\PDFcalculationset{#1}}

%D \macros
%D   {registerFDFobject,everylastshipout}
%D
%D Officially one needs to embed some general datastructures
%D that tell the viewer what fields are present in the file, as
%D well as what resources they use. The next mechanism does that
%D job automatically when one registers the field.

\def\flushFDFnames
  {\ifx\FDFcollection\empty\else
     \defineFDFfonts
     \createpdfarrayobject{FDF}{local:fields}{\FDFcollection}%
     \doPDFgetobjectreference{FDF}{local:fields}\PDFobjectreference
     % The /NeedAppearances is pretty important because
     % otherwise Acrobat 5 blows up on cloned radio widgets
     \createpdfdictionaryobject{FDF}{local:acroform}
       {/Fields \PDFobjectreference\space
        /NeedAppearances true
        \doiffieldset\PDFcalculationset{/CO [\dogetfieldset\PDFcalculationset]}
        /DR << /Font << \FDFfonts >> >>
        /DA (/Helv 10 Tf 0 g)}%
     \doPDFgetobjectreference{FDF}{local:acroform}\PDFobjectreference
     \appendtopdfcatalog
       {/AcroForm \PDFobjectreference}%
     \global\let\FDFcollection\empty
     \global\let\flushFDFnames\relax
   \fi}

\let\FDFcollection\empty

\def\registerFDFobject#1%
  {\ifx\flushFDFnames\relax
     \writestatus{FDF}{second run needed for field list (#1)}%
   \fi
   \doPDFgetobjectreference{FDF}{#1}\PDFobjectreference
   \xdef\FDFcollection{\FDFcollection\space\PDFobjectreference}}

\appendtoksonce \flushFDFnames \to \everylastshipout % test \everybye / was \prependtoksonce

%D \macros
%D   {defineFDFfonts}
%D
%D Another datastruture concerns the fonts used. We only
%D define the fonts we use.

\def\defineFDFfonts
  {\let\FDFfonts\empty
   \processcommacommand[\FDFusedfonts]\defineFDFfont}

\def\defineFDFfont#1%
  {\createpdfdictionaryobject{FDF}{local:#1}
     {/Type /Font
      /Subtype /Type1
      /Name /#1
      /BaseFont /\getvalue{FDFname#1}}%
   \doPDFgetobjectreference{FDF}{local:#1}\PDFobjectreference
   \edef\FDFfonts{\FDFfonts \space/#1 \PDFobjectreference}}

%D Another list of constants:

\def\FDFnameTiRo {Times-Roman}
\def\FDFnameTiBo {Times-Bold}
\def\FDFnameTiIt {Times-Italic}
\def\FDFnameTiBI {Times-BoldItalic}
\def\FDFnameHelv {Helvetica}
\def\FDFnameHeBo {Helvetica-Bold}
\def\FDFnameHeOb {Helvetica-Oblique}
\def\FDFnameHeBO {Helvetica-BoldOblique}
\def\FDFnameCour {Courier}
\def\FDFnameCoBo {Courier-Bold}
\def\FDFnameCoOb {Courier-Oblique}
\def\FDFnameCoBO {Courier-BoldOblique}

%D \macros
%D   {currentFDFmode,currentFDFparent,currentFDFkids,currenrFDFroot}
%D
%D There are three more quasi global interfacing variables
%D that need to be set.

\let\currentFDFmode  \fieldlonermode
\let\currentFDFkids  \empty
\let\currentFDFparent\empty
\let\currentFDFroot  \empty

%D \macros
%D   {dosetfieldstatus}
%D
%D And here comes the special that deals with them.

\def\dosetfieldstatus#1#2#3#4%
  {\chardef\currentFDFmode #1%
   \edef\currentFDFparent {#2}%
   \edef\currentFDFkids   {#3}%
   \edef\currentFDFroot   {#4}}

%D We already dealt with the encoding vector. Conversion from
%D \TEX\ \ASCII\ encoding to the other one, is accomplished by
%D the next few macros. Wach out: we don't group here.

\appendtoksonce
  \simplifycommands
\to \everysetfield

%D \macros
%D   {doPDFinsertcomment}
%D
%D An example its use is the next special, one that deals with
%D text annotations.

\newcounter\nofFDFcomments

\newif\ifPDFpopupcomments \PDFpopupcommentstrue

\def\doflushcomments
  {\box\PDFsymbolbox}

\long\def\doinsertcomment#1#2#3#4#5#6#7#8% % \@@DriverCommentLayer set otherwise
  {\bgroup % title width height color open symbol collect data
   \presetPDFsymbolappearance{#4}{#6}{#2}{#3}\!!zeropoint% sets width/height
   \doifelsenothing{#1}
     {\let\PDFidentifier\empty}
     {\sanitizePDFencoding#1\to\PDFcommenttitle
      \def\PDFidentifier{/T <\PDFcommenttitle>}}%
   \sanitizePDFencoding#8\to\PDFdata
   \setFDFlayer\@@DriverCommentLayer
   \startPDFsymbolappearance
     \ifPDFpopupcomments
       \doglobal\increment\nofFDFcomments
       \doifobjectreferencefoundelse{FDF}{c:\nofFDFcomments}
         {\doPDFgetobjectreference{FDF}{c:\nofFDFcomments}\PDFobjectreference
          \donetrue}
         \donefalse
       \ifdone
         \setbox\scratchbox\hbox
           {\createpdfannotationobject{FDF}{c::\nofFDFcomments}{#2}{#3}% text window, size does not work
              {/Subtype /Popup
               /Parent \PDFobjectreference}}%
         \ifcase#7\relax
           \vbox to \height{\forgetall\vskip#3\box\scratchbox\vss}%
         \else % incredible trial and error hack
           % it's quite a mess, the annot width cannot be set, well, it can
           % but the appearance and text sizes get mixed up
%            \setbox\scratchbox\vbox to \height{\forgetall\vskip#3\box\scratchbox\vss}%
%            \global\setbox\PDFsymbolbox\vbox
%              {\hsize#2%
%               \forgetall
%               \vsmash{\box\PDFsymbolbox}
%               \box\scratchbox}%
           % this may change when acrobat gets less bugged
           \setbox\scratchbox\vbox to #3{\forgetall\vss\box\scratchbox}%
           \wd\scratchbox#2%
           \global\setbox\PDFsymbolbox\vbox
             {\startoverlay{\box\PDFsymbolbox}{\box\scratchbox}\stopoverlay}%
         \fi
       \fi
       % generic
       \doifobjectreferencefoundelse{FDF}{c::\nofFDFcomments}
         {\doPDFgetobjectreference{FDF}{c::\nofFDFcomments}\PDFobjectreference
          \donetrue}
         \donefalse
       \createpdfannotationobject{FDF}{c:\nofFDFcomments}{\width}{\height}
         {/Subtype /Text
          \ifcase#5 \else/Open true\fi
          % pdftex (efficient)
          % \ifdone /Popup \PDFobjref\pdflastannot\fi
          % generic (less efficient)
          \ifdone /Popup \PDFobjectreference\fi
          /Contents <\PDFdata>
          \PDFidentifier
          \FDFlayer
          \PDFsymbol
          \PDFattributes}%
     \else
       \insertpdfannotation{#2}{#3}
         {/Subtype /Text
          \ifcase#5 \else/Open true\fi
          /Contents <\PDFdata>
          \FDFlayer
          \PDFsymbol
          \PDFidentifier
          \PDFattributes}%
     \fi
   \stopPDFsymbolappearance
   \egroup}

% symbols with a reasonable default of 18/24 pt

\newbox\PDFsymbolbox

\def\PDFsymbolNew       {/Insert}
\def\PDFsymbolBalloon   {/Comment}
\def\PDFsymbolAddition  {/NewParagraph}
\def\PDFsymbolHelp      {/Help}
\def\PDFsymbolParagraph {/Paragraph}
\def\PDFsymbolKey       {/Key }

\def\PDFsymbolGraph     {/Graph}
\def\PDFsymbolPaperclip {/Paperclip}
\def\PDFsymbolAttachment{/Attachment}
\def\PDFsymbolTag       {/Tag}

\def\startPDFsymbolappearance
  {\setbox\scratchbox\vbox to \totalheight \bgroup \vfill}

\def\stopPDFsymbolappearance
  {\egroup
   \setbox\scratchbox\hbox{\lower\depth\box\scratchbox}%
   \wd\scratchbox\width
   \ht\scratchbox\height
   \dp\scratchbox\depth
   \box\scratchbox}

\def\presetPDFsymbolappearance#1#2#3#4#5% symbol color width height depth
  {\doifelsenothing{#1}
     {\let\PDFattributes\empty}
     {\def\PDFattributes{/C \FDFcolor{#1}}}%
   \scratchdimen#3\edef\width {\the\scratchdimen}%
   \scratchdimen#4\edef\height{\the\scratchdimen}%
   \scratchdimen#5\edef\depth {\the\scratchdimen}%
   \advance\scratchdimen\height\edef\totalheight{\the\scratchdimen}%
   \doifelsenothing{#2}
     {\let\PDFsymbol\empty}
     {\ifundefined{PDFsymbol#2}%
        \getfromcommacommand[#2][1]\let\PDFsymbolnormalsymbol\commalistelement
        \getfromcommacommand[#2][2]\let\PDFsymboldownsymbol  \commalistelement
        \doifsymboldefinedelse\PDFsymbolnormalsymbol
          {\doifsymboldefinedelse\PDFsymboldownsymbol
             {\dopresetPDFsymbolappearance
                \PDFsymbolnormalsymbol\PDFsymboldownsymbol}
             {\dopresetPDFsymbolappearance
                \PDFsymbolnormalsymbol\PDFsymbolnormalsymbol}}
          {\doifsymboldefinedelse\PDFsymboldownsymbol
             {\dopresetPDFsymbolappearance
                \PDFsymboldownsymbol\PDFsymboldownsymbol}
             {\let\PDFsymbol\empty}}%
      \else
        \def\PDFsymbol{/Name \getvalue{PDFsymbol#2} }%
      \fi}}

\def\dopresetPDFsymbolappearance#1#2%
  {\dopresetfieldsymbol{#1}%
   \dopresetfieldsymbol{#2}%
   \setbox\scratchbox\hbox{\symbol[#1]}%
   \edef\width {\the\wd\scratchbox}%
   \edef\height{\the\ht\scratchbox}%
   \edef\depth {\the\dp\scratchbox}%
   \scratchdimen\height \advance\scratchdimen\depth
   \edef\totalheight{\the\scratchdimen}%
   \doPDFgetobjectreference{SYM}{#1}\FDFsymbolNappearance
   \doPDFgetobjectreference{SYM}{#2}\FDFsymbolDappearance
   \edef\PDFsymbol
     {/AP <</N \FDFsymbolNappearance /D \FDFsymbolDappearance>>}}

%D Hooked into \CONTEXT, this special supports
%D
%D \starttyping
%D \startcomment
%D   hello beautiful\\world
%D \stopcomment
%D
%D \startcomment[hello]
%D   de \'e\'erste keer
%D   the f\'irst time
%D \stopcommen
%D
%D \startcommentaar[hallo][color=green,width=4cm,height=3cm]
%D   first
%D
%D   second
%D \stopcommentaar
%D \stoptyping
%D
%D So, special characters, forced linebreaks using \type{\\}
%D and \type{\par} are handled in the appropriate way.

%D \macros
%D   {dosetuppageview}
%D
%D Because this command will seldom be called, we can permit
%D slow action processing. We need three settings, one for
%D direct \PDF\ inclusion, the other as \PDFTEX\ keyword, an
%D a last one for form. All determine in what way the
%D screen is adapted when going to a destination. Watch the
%D space.

\def\PDFpageviewkey{fit}
\def\PDFpageviewwrd{/Fit}
\def\PDFpageview   {/View [\PDFpageviewwrd] }
\def\PDFpagexyzspec{0 0 0} % hack, pdftex does handle this
\let\PDFpagexyzspec\empty  % hack, pdftex does not accept spec

\def\dosetuppageview#1% watch the v-h swapping here
  {\processaction
     [#1]
     [       \v!fit=>\def\PDFpageviewkey  {fit}\def\PDFpageviewwrd{/Fit},
           \v!width=>\def\PDFpageviewkey {fith}\def\PDFpageviewwrd{/FitH},
          \v!height=>\def\PDFpageviewkey {fitv}\def\PDFpageviewwrd{/FitV},
        \v!minwidth=>\def\PDFpageviewkey{fitbh}\def\PDFpageviewwrd{/FitBH},
       \v!minheight=>\def\PDFpageviewkey{fitbv}\def\PDFpageviewwrd{/FitBV},
        \v!standard=>\def\PDFpageviewkey{xyz \PDFpagexyzspec}\def\PDFpageviewwrd{/XYZ \PDFpagexyzspec},
         \s!unknown=>\def\PDFpageviewkey  {fit}\def\PDFpageviewwrd{/Fit}]%
   \edef\PDFpageview{/View [\PDFpageviewwrd]}}

%D \macros
%D   {setFDFkids}
%D
%D Clones as well as radiofields (which themselves can have
%D cloned components) need a list of kids. The next macro
%D builds one.

\def\setFDFkids[#1][#2]% tag commalist
  {\let\FDFkids\empty
   \def\docommand##1%
     {\doPDFgetobjectreference{FDF}{#1##1}\PDFobjectreference
      \edef\FDFkids{\FDFkids\PDFobjectreference\space}}%
   \@EA\processcommalist\@EA[#2]\docommand
   \ifx\FDFkids\empty\else\edef\FDFkids{/Kids [\FDFkids]}\fi}

%D \macros
%D   {dopresetlinefield,dopresettextfield,
%D    dopresetchoicefield,dopresetpopupfield,dopresetcombofield,
%D    dopresetpushfield,dopresetcheckfield,
%D    dopresetfield,dopresetradiorecord}
%D
%D I would say: read the \PDF\ reference manual first and see
%D what happens here next. Lucky us that they have so much in
%D common.

\def\dopresetlinefield#1#2#3#4#5#6#7#8#9%
  {\bgroup
   \setFDFlayer\@@DriverFieldLayer
   \setFDFswitches[#7]%
   \setFDFattributes[#6]%
   \setFDFalignment[#8]%
   \setFDFactions[#9]%
   \edef\FDFtext{\hexifiedPDFstring{#4}}%
   \ifcase\currentFDFmode
     \createpdfannotationobject{FDF}{#1}{#2}{#3}
       {/Subtype /Widget /T (#1) /FT /Tx
        /MaxLen \ifcase0#5 1000 \else#5 \fi
       %/DV (#4) /V (#4) % value added
        /DV <\FDFtext> /V <\FDFtext>
        /Ff \FDFflag\space
        /F \FDFplus\space
        /DA (\FDFattributes)
        \FDFlayer\space
        \FDFsurroundings\space
        /Q \FDFalign\space
        \FDFactions}%
     \registerFDFobject{#1}%
   \or
     \setFDFkids[kids:][\currentFDFkids]%
     \createpdfdictionaryobject{FDF}{#1}
       {/T (#1) /FT /Tx
        /MaxLen \ifcase0#5 1000 \else#5 \fi
        \FDFkids\space
       %/DV (#4) /V (#4) % value added
        /DV <\FDFtext> /V <\FDFtext>
        /Ff \FDFflag\space
        /F \FDFplus\space
        /DA (\FDFattributes)
        \FDFlayer\space
        \FDFsurroundings\space
        /Q \FDFalign\space
        \FDFactions}%
     \registerFDFobject{#1}%
   \or
     \doPDFgetobjectreference{FDF}\currentFDFparent\PDFobjectreference
    %\global\objectreferencingtrue
     \createpdfannotationobject{FDF}{kids:#1}{#2}{#3}
       {/Subtype /Widget
        /Parent \PDFobjectreference
        /Ff \FDFflag\space
        /F \FDFplus\space
        /DA (\FDFattributes)
        \FDFlayer\space
        \FDFsurroundings\space
        /Q \FDFalign\space
        \FDFactions}%
   \or
     \doPDFgetobjectreference{FDF}\currentFDFparent\PDFobjectreference
    %\global\objectreferencingtrue
     \createpdfannotationobject{FDF}{kids:#1}{#2}{#3}
       {/Subtype /Widget
        /Parent \PDFobjectreference
        /F \FDFplus
        \FDFactions}%
   \fi
   \egroup}

\def\dopresettextfield#1#2#3#4#5#6#7#8#9%
  {\dopresetlinefield{#1}{#2}{#3}{#4}{#5}{#6}{MultiLine,#7}{#8}{#9}}

\def\dopresetchoicefield#1#2#3#4#5#6#7#8%
  {\bgroup
   \setFDFlayer\@@DriverFieldLayer
   \setFDFswitches[#6]%
   \setFDFattributes[#5]%
   \setFDFvalues[#7][#4]%
   \setFDFactions[#8]%
   \ifcase\currentFDFmode
     \createpdfannotationobject{FDF}{#1}{#2}{#3}
       {/Subtype /Widget
        /T (#1) /FT /Ch
        /DV (#4) /V (#4)
        /Ff \FDFflag\space
        /F \FDFplus\space
        /DA (\FDFattributes)
        \FDFlayer\space
        \FDFsurroundings\space
        /Opt [\FDFvalues]
        \FDFactions}%
     \registerFDFobject{#1}%
   \or
     \setFDFkids[kids:][\currentFDFkids]%
     \createpdfdictionaryobject{FDF}{#1}
       {/T (#1) /FT /Ch
        \FDFkids\space
        /DV (#4) /V (#4)
        /Ff \FDFflag\space
        /F \FDFplus\space
        /DA (\FDFattributes)
        \FDFlayer\space
        \FDFsurroundings\space
        /Opt [\FDFvalues]
        \FDFactions}%
     \registerFDFobject{#1}%
   \or
     \doPDFgetobjectreference{FDF}\currentFDFparent\PDFobjectreference
    %\global\objectreferencingtrue
     \createpdfannotationobject{FDF}{kids:#1}{#2}{#3}
       {/Subtype /Widget
        /Parent \PDFobjectreference
        /Ff \FDFflag\space
        /F \FDFplus\space
        /DA (\FDFattributes)
        \FDFlayer\space
        \FDFsurroundings\space
        \FDFactions}%
   \or
     \doPDFgetobjectreference{FDF}\currentFDFparent\PDFobjectreference
    %\global\objectreferencingtrue
     \createpdfannotationobject{FDF}{kids:#1}{#2}{#3}
       {/Subtype /Widget
        /Parent \PDFobjectreference
        /F \FDFplus
        \FDFactions}%
   \fi
   \egroup}

\def\dopresetpopupfield#1#2#3#4#5#6#7#8%
  {\dopresetchoicefield{#1}{#2}{#3}{#4}{#5}{PopUp,#6}{#7}{#8}}

\def\dopresetcombofield#1#2#3#4#5#6#7#8%
  {\dopresetchoicefield{#1}{#2}{#3}{#4}{#5}{PopUp,Edit,#6}{#7}{#8}}

\newif\ifFDFvalues

\def\doFDFpresetpushcheckfield#1#2#3#4#5#6#7#8% in acro<5 (\FDFdefault)
  {\bgroup                                    % in acro>5 /\FDFdefault
   \setFDFlayer\@@DriverFieldLayer
   \ifcase#8\relax\FDFvaluesfalse\else\FDFvaluestrue\fi
   \setFDFswitches[#5]%
   \setFDFactions[#7]%
   \doifelse{#4}{1}
     {\def\FDFdefault{On}}
     {\def\FDFdefault{Off}}%
   \ifcase\currentFDFmode
     \doFDFappearance{On}{#6}{#8}%
     \createpdfannotationobject{FDF}{#1}{#2}{#3}
       {/Subtype /Widget /T (#1) /FT /Btn
        \ifFDFvalues
          /DV /\FDFdefault\space
          /V  /\FDFdefault\space
          /AS /\FDFdefault\space
        \fi
        \FDFlayer
        /Ff \FDFflag\space
        /F \FDFplus\space
        \FDFlayer\space
        \FDFappearance\space
% /IF << /SW /N >> % strange, only works for stupid buttons
        \FDFactions}%
     \registerFDFobject{#1}%
   \or % no appearance and layer ?
     \setFDFkids[kids:][\currentFDFkids]%
     \createpdfdictionaryobject{FDF}{#1}
       {/T (#1) /FT /Btn
        \FDFkids\space
        \ifFDFvalues
          /DV /\FDFdefault\space
          /V  /\FDFdefault\space
          /AS /\FDFdefault\space
        \fi
        /Ff \FDFflag\space
        /F \FDFplus\space
        \FDFactions}%
     \registerFDFobject{#1}%
   \or
     \doFDFappearance{On}{#6}{#8}%
     \doPDFgetobjectreference{FDF}\currentFDFparent\PDFobjectreference
    %\global\objectreferencingtrue
     \createpdfannotationobject{FDF}{kids:#1}{#2}{#3}
       {/Subtype /Widget
        /Parent \PDFobjectreference\space
        \ifFDFvalues
          /DV /\FDFdefault\space
          /V  /\FDFdefault\space
          /AS /\FDFdefault\space
        \fi
        /Ff \FDFflag\space
        /F \FDFplus\space
        \FDFlayer\space
        \FDFappearance\space
        \FDFactions}%
   \or
     \doFDFappearance{On}{#6}{#8}%
     \doPDFgetobjectreference{FDF}\currentFDFparent\PDFobjectreference
    %\global\objectreferencingtrue
     \createpdfannotationobject{FDF}{kids:#1}{#2}{#3}
       {/Subtype /Widget
        /Parent \PDFobjectreference\space
        /F \FDFplus\space
        \ifFDFvalues
          /DV /\FDFdefault\space
          /V  /\FDFdefault\space
          /AS /\FDFdefault\space
        \fi
        \FDFlayer\space
        \FDFappearance
        \FDFactions}%
   \fi
   \egroup}

\def\dopresetpushfield#1#2#3#4#5#6#7%
  {\doFDFpresetpushcheckfield{#1}{#2}{#3}{#4}{PushButton,#5}{#6}{#7}{0}}

\def\dopresetcheckfield#1#2#3#4#5#6#7%
  {\doFDFpresetpushcheckfield{#1}{#2}{#3}{#4}{#5}{#6}{#7}{1}}

\def\dopresetradiofield#1#2#3#4#5#6#7#8%
  {\bgroup
   \setFDFlayer\@@DriverFieldLayer
   \FDFvaluestrue
   \setFDFswitches[#5]%
   \setFDFactions[#8]%
   \doifelsenothing{#4}
     {\def\FDFdefault{Off}}
     {\def\FDFdefault{#4}}%
   \@EA\aftersplitstring\FDFdefault\at=>\to\FDFdefaultvalue
   \ifx\FDFdefaultvalue\empty\else\let\FDFdefault\FDFdefaultvalue\fi
   \ifcase\currentFDFmode
     \doFDFappearance{#1}{#7}{1}%
     \doPDFgetobjectreference{FDF}{#6}\PDFobjectreference
     \createpdfannotationobject{FDF}{#1}{#2}{#3}
       {/Subtype /Widget
        /Parent \PDFobjectreference\space
        /F \FDFplus\space
        /AS /\FDFdefault\space
        \FDFlayer\space
        \FDFappearance\space
        \FDFactions}%
     \registerFDFobject{#1}%
   \or
     \setFDFkids[kids:][\currentFDFkids]%
     \doPDFgetobjectreference{FDF}{#6}\PDFobjectreference
     \createpdfdictionaryobject{FDF}{#1}
       {/Parent \PDFobjectreference\space
        \FDFkids\space
        /F \FDFplus\space
        \FDFactions}%
     \registerFDFobject{#1}%
   \or
     %\doFDFappearance{#1}{#7}{1}%
     \doFDFappearance{\currentFDFparent}{#7}{1}%
     \doPDFgetobjectreference{FDF}\currentFDFparent\PDFobjectreference
    %\global\objectreferencingtrue % nb
     \createpdfannotationobject{FDF}{kids:#1}{#2}{#3}
       {/Subtype /Widget
        /Parent \PDFobjectreference\space
        /AS /\FDFdefault\space
        /F \FDFplus\space
        \FDFlayer\space
        \FDFappearance\space
        \FDFactions}%
   \or
     %\doFDFappearance{#1}{#7}{1}%
     \doFDFappearance{\currentFDFparent}{#7}{1}%
     \doPDFgetobjectreference{FDF}\currentFDFparent\PDFobjectreference
    %\global\objectreferencingtrue
     \createpdfannotationobject{FDF}{kids:#1}{#2}{#3}
       {/Subtype /Widget
        /Parent \PDFobjectreference\space
        /AS /\FDFdefault\space
        /F \FDFplus\space
        \FDFlayer\space
        \FDFappearance\space
        \FDFactions}%
   \fi
   \egroup}

% Beware, RadiosInUnison is really needed in the pre 1.5/6 time this
% was the default but out of a sudden it's no longer the case.  Also
% the NoToggleToOff interferes with kids of kids and both it will
% break older documents, i.e. so much for pdf as standard. With
% features like widgets we can probably best wait till adobe tools
% themselves support it because that's probably the moment that
% functionality gets frozen/becomes definitive. Actually, acrobat
% flattens the kids tree, so that's yet another situation. The
% interesting thing is that it worked ok in acrobat 2/3 but got bugged
% in later versions. [The rationale is in html compatibility, which
% seems to be more important than compatibility of documents, which in
% turn renders acrobat useless for forms.] Anyway, synchronization is
% broken or not depending on the combination pdfversion/acrobatversion.

\def\dopresetradiorecord#1#2#3#4#5%
  {\bgroup
   % < pdf 1.5 (1.5 was broken)
   % \setFDFswitches[Radio,NoToggleToOff,RadiosInUnison,#3]%
   % > pdf 1.5
   \setFDFswitches[Radio,RadiosInUnison,#3]%
   % older, else fatal error
   % \setFDFkids[#4][]%
   % newer
   \setFDFvalues[#4][#2]% inits kidlist
   \expanded{\setFDFkids[][\FDFkidlist]}%
   %
   \setFDFactions[#5]%
   \createpdfdictionaryobject{FDF}{#1}
     {%/Subtype /Widget
      /FT /Btn /T (#1) /Rect [0 0 0 0]
      % used to be this
      % /V (#2)
      % then this
      % /DV (#2)
      % since this bomded in 5
      % /V (#2)
      % and now finally this works
      /H /N
      % /opt is buggy in 5.05, only works once, sigh
      %\ifx\FDFfirstvalues\FDFsecondvalues
        /V /#2
      %\else
      %  /V /\FDFdefaultindex\space
      %  /Opt [\FDFsecondvalues]
      %\fi
      /Ff \FDFflag\space
      /F \FDFplus\space
      \FDFkids\space
      \FDFactions}%
   \egroup}

%D At the cost of some more references, we can save bytes,
%D by sharing appearance dictionaries. This code needs more
%D documentation. Surprise:

\def\dodoFDFappearance#1#2%
  {\ifx#2\empty\else
     \dogetcommacommandelement1\from#2\to\commalistelement
     \ifx\commalistelement\empty\else
       \doPDFgetobjectreference{SYM}\commalistelement\PDFobjectreference
       \edef\N{\ifFDFvalues\N /#1 \fi\PDFobjectreference\space}%
     \fi
     \dogetcommacommandelement2\from#2\to\commalistelement
     \ifx\commalistelement\empty\else
       \doPDFgetobjectreference{SYM}\commalistelement\PDFobjectreference
       \edef\R{\ifFDFvalues\R /#1 \fi\PDFobjectreference\space}%
     \fi
     \dogetcommacommandelement3\from#2\to\commalistelement
     \ifx\commalistelement\empty\else
       \doPDFgetobjectreference{SYM}\commalistelement\PDFobjectreference
       \edef\D{\ifFDFvalues\D /#1 \fi\PDFobjectreference\space}%
       \def\FDFappearance{/H /P }%
     \fi
   \fi}

\def\redoFDFappearance#1%
  {\ifx#1\empty\else
     \dogetcommacommandelement3\from#1\to\commalistelement
     \ifx\commalistelement\empty\else
       \def\FDFappearance{/H /P }%
     \fi
   \fi}

\def\doFDFappearance#1#2#3%
  {\ifcase#3\relax % push only field
     \edef\yes{#2}%
     \let\no\empty
   \else % on / off field
     \dogetcommacommandelement1\from#2,\to\yes
     \dogetcommacommandelement2\from#2,\to\no
   \fi
   \def\FDFappearance{/H /N}%
   \doifobjectfoundelse{FDF}{ap:#1:\yes:\no}
     {\redoFDFappearance\yes
      \redoFDFappearance\no}
     {\presetobject{FDF}{ap:#1:\yes:\no}% funny hack
      \let\N\empty\let\R\empty\let\D\empty
      \dodoFDFappearance{#1}\yes
      \dodoFDFappearance{Off}\no
      \createpdfdictionaryobject{FDF}{ap:#1:\yes:\no}
        {\ifx\N\empty\else/N \ifFDFvalues<<\N>>\else\N\fi\fi
         \ifx\R\empty\else/R \ifFDFvalues<<\R>>\else\R\fi\fi
         \ifx\D\empty\else/D \ifFDFvalues<<\D>>\else\D\fi\fi}}%
   \doPDFgetobjectreference{FDF}{ap:#1:\yes:\no}\PDFobjectreference
   \edef\FDFappearance{\FDFappearance /AP \PDFobjectreference}}

\def\doFDFdefault#1#2%
  {\doifelse{#2}{1}{\def\FDFdefault{On}}{\def\FDFdefault{Off}}}

%D Layer support:

\def\setFDFlayer#1% todo : \ifx\PDFobjectreference\noPDFobjectreference ipv found
  {\letempty\FDFlayer
   \doifsomething{#1}%
     {\checkproperty[#1]% == \dodocheckproperty\@@DriverFieldLayer
      \doifobjectreferencefoundelse{PDLN}{#1}
        {\doPDFgetobjectreference{PDLN}{#1}\!!stringa % we need to avoid a clash with other macros
         \edef\FDFlayer{/OC \!!stringa}}%
        \donothing}}

%D The three appearances {\em normal}, \type{roll over} and
%D \type{push down} are passed as comma separated triplets,
%D that is, the second argument can look like:
%D
%D \starttyping
%D {yes,ok,fine},{no,rubish,awful}
%D \stoptyping

%D \macros
%D   {dodefinefieldset,dogetfieldset,doiffieldset}
%D
%D Field sets, the ones we use in submitting and resetting
%D fields, are implemented using the next low level specials:
%D
%D \starttyping
%D \doFDFdefinefieldset{TAG}{name,name,...}
%D \doFDFgetfieldset{TAG}
%D \doiffieldset{TAG}{sequence}
%D \stoptyping

\def\dodefinefieldset#1#2% tag commalist
  {\let\FDFfieldset\empty
   \def\docommand##1%
     {\doPDFgetobjectreference{FDF}{##1}\PDFobjectreference
      \edef\FDFfieldset{\FDFfieldset\PDFobjectreference\space}}%
   \processcommacommand[#2]\docommand % nb: command
   \setevalue{FDF:set:#1}{\FDFfieldset}}

\def\dogetfieldset#1%
  {\getvalue{FDF:set:#1}}

\def\doiffieldset#1#2%
  {\ifundefined{FDF:set:#1}\else#2\fi}

%D \macros
%D   {defaultobjectreference,doPDFgetobjectreference}
%D
%D Because in \PDFTEX\ we have to construct the object
%D references \type{N 0 R}, we can default to the non existing
%D zero object number.

\def\defaultobjectreference#1#2%
  {0}

\def\doPDFgetobjectreference#1#2#3%
  {\dogetobjectreference{#1}{#2}#3%
   \edef#3{\ifx#3\empty null\else\PDFobjref{#3}\fi}}

\def\doPDFgetobjectnumber#1#2#3%
  {\dogetobjectreference{#1}{#2}#3%
   \edef#3{\ifx#3\empty 0\else#3\fi}}

\def\doPDFgetobjectpage#1#2#3%
  {\dogetobjectreferencepage{#1}{#2}#3%
   \ifx#3\empty\def#3{1}\fi}

\def\doPDFgetobjectpagereference#1#2#3%
  {\dogetobjectreferencepage{#1}{#2}#3%
   \ifx#3\empty
     \doPDFgetpagereference\realfolio#3%
   \else
     \doPDFgetpagereference#3#3% we assume that #3 gets expanded
   \fi}

\def\doPDFgetpagereference#1#2% number macro
  {\edef#2{\ifnum#1>\zerocount\PDFobjref{\pdfpageref#1}\else null\fi}}

\def\thePDFpagereference#1#2% number macro
  {\ifnum#1>\zerocount\PDFobjref{\pdfpageref#1}\else null\fi}

%D \macros
%D   {initializePDFnegative,initializePDFoverprint}
%D
%D Here follow some rather obscure macros. They will only
%D come into action when one wants negated output.

\def\initializePDFnegative
  {\immediate\pdfobj stream attr {/FunctionType 4 /Range [0 1] /Domain [0 1]} {{1 exch sub}}%
   \immediate\pdfobj{<</Type /ExtGState /TR \PDFobjref\pdflastobj>>}%
   \appendtoPDFdocumentextgstates{/GSnegative \PDFobjref\pdflastobj}%
   \immediate\pdfobj{<</Type /ExtGState /TR /Identity>>}%
   \appendtoPDFdocumentextgstates{/GSpositive \PDFobjref\pdflastobj}%
   \global\let\initializePDFnegative\relax}

\def\initializePDFoverprint
  {\immediate\pdfobj{<</Type /ExtGState /OP false /OPM 0>>}% /op defaults to /OP
   \appendtoPDFdocumentextgstates{/GSknockout \PDFobjref\pdflastobj}%
   \immediate\pdfobj{<</Type /ExtGState /OP true /OPM 1>>}% /op defaults to /OP
   \edef\PDFobjectreferenceB{\the\pdflastobj}%
   \appendtoPDFdocumentextgstates{/GSoverprint \PDFobjref\pdflastobj}%
   \global\let\initializePDFoverprint\relax}

%D File embedding. Storing the stream identifier is needed
%D to get access to the number. When typeset, the user can
%D feed this number to \type {pdftosrc} and filter the
%D file from the \PDF\ file.

\let\PDFlaststreamobject   \s!unknown
%def\PDFlaststreamreference{0 0 R}

\def\doPDFfilestreamobject#1#2#3#4%
  {\immediate\pdfobj stream file{#4}%
   \edef\PDFlaststreamobject{\the\pdflastobj}%
   \dosetobjectreference{PDFFS}{#2}{\PDFlaststreamobject}%
   \createpdfdictionaryobject{#1}{#2}{/Type /Filespec /F (#3) /EF <</F \PDFobjref\PDFlaststreamobject>>}}

\def\doPDFgetfilestreamreference#1#2%
  {\doPDFgetobjectreference{PDFFS}{#1}#2}

\def\doPDFfilestreamidentifier#1%
  {\doifsomething{#1}
     {\doPDFgetfilestreamreference{#1}\PDFobjectreference
      \@EA\beforesplitstring\PDFobjectreference\at{ }\to\PDFlaststreamobject
      \PDFlaststreamobject}}

% MP ?

        \def\setMPPDFobject#1#2% resources boxnumber
          {\the\everyPDFxform
           \finalizeobjectbox{#2}%
           \immediate\pdfxform resources{#1}#2%
           \edef\getMPPDFobject{\noexpand\pdfrefxform\the\pdflastxform}}

        \let\getMPPDFobject\relax

        \def\doinsertMPfile#1%
          {\doiffileelse{./#1}{\includeMPasPDF{./#1}}{\message{[MP #1]}}}

%D Even newer trickery:

% resource -> prop -> mc's -> OCG|OCMD (nested)
% ocg:
% /Intent/Design
% ocmd
% /P /AllOn
% kan zelf ocmd bevatten

\let\PDFtextlayers\empty
\let\PDFpagelayers\empty
\let\PDFhidelayers\empty
\let\PDFvidelayers\empty

\def\dostartlayer#1{\PDFcode{/OC /#1 BDC}}
\def\dostoplayer   {\PDFcode        {EMC}}

\def\dodefineviewerlayer#1#2#3#4#5% tag title visible type printable
  {\createpdfdictionaryobject{PDLN}{#1}
     {/Type /OCG
      \ifcase#4 \or
        /Intent /Design % disable layer hiding by user
      \fi
      \ifnum#5=\zerocount
        /Usage << /Print << /PrintState /OFF >> >> % printable or not
      \fi
      /Name (#2)}%
   \doPDFgetobjectreference{PDLN}{#1}\PDFobjectreference
   \xdef\PDFtextlayers{\PDFtextlayers\space\PDFobjectreference}%
   \doifelse{#3}\v!start
     {\xdef\PDFvidelayers{\PDFvidelayers\space\PDFobjectreference}}%
     {\xdef\PDFhidelayers{\PDFhidelayers\space\PDFobjectreference}}%
   \createpdfdictionaryobject{PDLD}{#1}
     {/Type /OCMD
      /OCGs [\PDFobjectreference]}%
   \doPDFgetobjectreference{PDLD}{#1}\PDFobjectreference
   \xdef\PDFpagelayers{\PDFpagelayers\space /#1 \PDFobjectreference}}

\def\flushPDFtextlayers
  {\ifx\PDFtextlayers\empty \else
     \driverreferenced \createpdfarrayobject{PDF}{textlayers}{\PDFtextlayers}%
     \doPDFgetobjectreference{PDF}{textlayers}\!!stringa
     \ifx\PDFvidelayers\empty
       \def\!!stringb{[null]}%
     \else
       \driverreferenced \createpdfarrayobject{PDF}{videlayers}{\PDFvidelayers}%
       \doPDFgetobjectreference{PDF}{videlayers}\!!stringb
     \fi
     \ifx\PDFhidelayers\empty
       \def\!!stringc{[null]}%
     \else
       \driverreferenced \createpdfarrayobject{PDF}{hidelayers}{\PDFhidelayers}%
       \doPDFgetobjectreference{PDF}{hidelayers}\!!stringc
     \fi
     \appendtopdfcatalog
       {/OCProperties
        << % display in menu
           /D  << /Order \!!stringa
                  /ON    \!!stringb
                  /OFF   \!!stringc >>
           % used properties
           /OCGs \!!stringa >>}%
     \globallet\flushPDFtextlayers\relax
   \fi}

\def\flushPDFpagelayers
  {\ifx\PDFpagelayers\empty \else
     \appendtopdfpageresources{/Properties <<\PDFpagelayers>>}%
   \fi}

\def\PDFlayeractionlist{null}

\def\PDFexecutehidelayer   {/SetOCGState /State [/OFF    \PDFlayeractionlist]}
\def\PDFexecutevidelayer   {/SetOCGState /State [/ON     \PDFlayeractionlist]}
\def\PDFexecutetogglelayer {/SetOCGState /State [/Toggle \PDFlayeractionlist]}

\def\domakeviewerlayerlist#1%
  {\bgroup
   \globallet\PDFlayeractionlist\empty
   \def\docommand##1%
     {\doPDFgetobjectreference{PDLN}{##1}\PDFobjectreference
      \xdef\PDFlayeractionlist{\PDFlayeractionlist\space\PDFobjectreference}}%
   \processcommalist[#1]\docommand
   \egroup}

%D Something rather pdf dependent:

% #1 => 1=fill 2=stroke 3=strokedfill 4=invisible
% #2 => linewidth
% #3 => spacing (beware, one needs to set the hsize as well)

\def\dostartfonteffect#1#2#3%
  {\ifdim#2>\zeropoint
     \PointsToBigPoints{#2}\ascii
     \PDFcode{\ascii\space w}%
   \fi
   \ifdim#3\points=\onepoint\else
     \scratchdimen#3\points
     \PDFcode{\withoutpt{\the\scratchdimen}\space Tc}%
   \fi
   \PDFcode{\purenumber#1 Tr}}

\def\dostopfonteffect
  {\PDFcode{1 w 0 Tc 0 Tr}}

%D Handy for the \METAPOST\ to \PDF\ converter:

\appendtoksonce
    \collectPDFresources
    \global\let\currentPDFresources\collectedPDFresources
\to \everyPDFxform

\let\collectedPDFresources\empty

\def\collectPDFresources % suboptimal
  {\doifobjectreferencefoundelse{FDF}{docushades} % redundant, we have an reserved object now
     {\doPDFgetobjectreference{FDF}{docushades}\PDFobjectreference
      \xdef\collectedPDFresources{\collectedPDFresources/Shading \PDFobjectreference}}\donothing
   \doifobjectreferencefoundelse{FDF}{docuextgstates}
     {\doPDFgetobjectreference{FDF}{docuextgstates}\PDFobjectreference
      \xdef\collectedPDFresources{\collectedPDFresources/ExtGState \PDFobjectreference}}\donothing
   \doifobjectreferencefoundelse{FDF}{colorspaces}
     {\doPDFgetobjectreference{FDF}{colorspaces}\PDFobjectreference
      \xdef\collectedPDFresources{\collectedPDFresources/ColorSpace \PDFobjectreference}}\donothing
   \global\let\collectPDFresources\relax}

\appendtoks
    \flushPDFpagelayers
    \flushJSpreamble
    \flushJSpreamble
    \checkPDFextgstates
    \checkPDFcolorspaces
    \checkPDFshades
    \checkPDFpageactions
    \fakePDFpagedestination
    \flushPDFpageboxes
    \addPDFdocumentinfo
\to \everybackendshipout

\appendtoks
    \flushPDFtextlayers
    \finalflushJSpreamble
\to \everylastbackendshipout

%D Temporary hack:

\def\TransparencyHack %  png: /CS /DeviceRGB /I true
  {\appendtoksonce
     \appendtopdfpageattributes{/Group << /S /Transparency /I true /K true>>}%
   \to \everyPDFxform
   \appendtoksonce
     \appendtopdfpageattributes{/Group << /S /Transparency /I true /K true>>}%
   \to \everyshipout}

\protect \endinput
