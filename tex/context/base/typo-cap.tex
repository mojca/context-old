%D \module
%D   [       file=typo-cap,
%D        version=2009.03.27, % code moved from core-spa.mkiv
%D          title=\CONTEXT\ Typesetting Macros,
%D       subtitle=Mirroring,
%D         author=Hans Hagen,
%D           date=\currentdate,
%D      copyright=\PRAGMA]
%C
%C This module is part of the \CONTEXT\ macro||package and is
%C therefore copyrighted by \PRAGMA. See mreadme.pdf for
%C details.

\writestatus{loading}{ConTeXt Typesetting Macros / Caps}

\unprotect

\registerctxluafile{typo-cap}{1.001}

\definesystemattribute[case]

%D \macros
%D   {Word, Words, WORD, WORDS, doprocesswords}
%D
%D This is probably not the right place to present the next set
%D of macros.
%D
%D \starttyping
%D \Word {far too many words}
%D \Words{far too many words}
%D \WORD {far too many words}
%D \WORDS{far too many words}
%D \stoptyping
%D
%D \typebuffer
%D
%D This calls result in:
%D
%D \startvoorbeeld
%D \startlines
%D \getbuffer
%D \stoplines
%D \stopvoorbeeld
%D
%D \showsetup{Word}
%D \showsetup{Words}
%D \showsetup{WORD}
%D \showsetup{WORDS}

% \def\doprocesswords#1 #2\od
%   {\ConvertToConstant\doifnot{#1}{}
%      {\processword{#1} %
%       \doprocesswords#2 \od}}
%
% \def\processwords#1%
%   {\doprocesswords#1 \od\unskip}
%
% \let\processword\relax

% test \WORD{test TEST \TeX} test
% test \word{test TEST \TeX} test
% test \Word{test TEST \TeX} test

\def\setcharactercasing
  {\ctxlua{cases.enabled=true}%
   \gdef\setcharactercasing[##1]{\dosetattribute{case}{\number##1}}%
   \setcharactercasing}

\unexpanded\def\WORD {\groupedcommand{\setcharactercasing[\plusone  ]}{}}
\unexpanded\def\word {\groupedcommand{\setcharactercasing[\plustwo  ]}{}}
\unexpanded\def\Word {\groupedcommand{\setcharactercasing[\plusthree]}{}}
\unexpanded\def\Words{\groupedcommand{\setcharactercasing[\plusfour ]}{}}

\let\WORDS\WORD
\let\words\word

%D \macros
%D   {kap,KAP,Kap,Kaps,nokap,userealcaps,usepseudocaps}
%D
%D We already introduced \type{\cap} as way to capitalize
%D words. This command comes in several versions:
%D
%D \startbuffer
%D \cap {let's put on a \cap{cap}}
%D \cap {let's put on a \nocap{cap}}
%D \CAP {let's put on a \\{cap}}
%D \Cap {let's put on a \\{cap}}
%D \Caps{let's put on a cap}
%D \stopbuffer
%D
%D \typebuffer
%D
%D Note the use of \type{\nocap}, \type{\\} and the nested
%D \type{\cap}.
%D
%D \startvoorbeeld
%D \startlines
%D \getbuffer
%D \stoplines
%D \stopvoorbeeld
%D
%D These macros show te main reason why we introduced the
%D smaller \type{\tx} and \type{\txx}.
%D
%D \starttyping
%D \cap\romannumerals{1995}
%D \stoptyping
%D
%D This at first sight unusual capitilization is completely
%D legal.
%D
%D \showsetup{smallcapped}
%D \showsetup{notsmallcapped}
%D \showsetup{CAPPED}
%D \showsetup{SmallCapped}
%D \showsetup{SmallCaps}
%D
%D The difference between pseudo and real caps is demonstrated
%D below:
%D
%D \startbuffer
%D \usepseudocaps \cap{Hans Hagen}
%D \userealcaps   \cap{Hans Hagen}
%D \stopbuffer
%D
%D \typebuffer
%D
%D \getbuffer
%D
%D The \type {\bgroup} trickery below is needed because of
%D \type {\groupedcommand}.

\let\disablepseudocaps\relax % maybe used elsewhere

\newconditional\pseudocapsenabled

\def\usepseudocaps{\settrue \pseudocapsenabled}
\def\userealcaps  {\setfalse\pseudocapsenabled}

\usepseudocaps

% we use char0 as placeholder for the larger font

\unexpanded\def\pseudosmallcapped{\groupedcommand{\setcharactercasing[\plusone ]\char\zerocount\tx}{}} % all upper
\unexpanded\def\pseudoSmallcapped{\groupedcommand{\setcharactercasing[\plusfive]\char\zerocount\tx}{}} % one upper + font
\unexpanded\def\pseudoSmallCapped{\groupedcommand{\setcharactercasing[\plussix ]\char\zerocount\tx}{}} % some upper + font

\unexpanded\def\realsmallcapped  {\groupedcommand{\sc\setcharactercasing[\plusone  ]}{}} % all lower
\unexpanded\def\realSmallcapped  {\groupedcommand{\sc\setcharactercasing[\plusthree]}{}} % one upper + font
\unexpanded\def\realSmallCapped  {\groupedcommand{\sc\setcharactercasing[\plusfour ]}{}} % some upper

\unexpanded\def\dohandlesmallcaps
  {\ifconditional\pseudocapsenabled
     \expandafter\firstoftwoarguments
   \else
     \expandafter\secondoftwoarguments
   \fi}

\unexpanded\def\smallcapped{\dohandlesmallcaps\pseudosmallcapped\realsmallcapped}
\unexpanded\def\Smallcapped{\dohandlesmallcaps\pseudoSmallcapped\realSmallcapped}
\unexpanded\def\SmallCapped{\dohandlesmallcaps\pseudoSmallCapped\realSmallCapped}

\unexpanded\def\autocap{\ifmmode\expandafter\normalcap\else\expandafter\smallcapped\fi}

\appendtoks
    \let\normalcap\cap % mathmode cap
    \let\cap\autocap
\to \everydump

\let\kap\cap          % for old times sake
\let\Caps\SmallCapped % for old times sake

\let\normalsmallcapped\smallcapped
\let\normalWORD       \WORD
\let\normalword       \word

%D \macros
%D   {setupcapitals}
%D
%D By default we use pseudo small caps in titles. This can be
%D set up with:
%D
%D \showsetup{setupcapitals}

\let\normalsmallcapped\smallcapped

\def\setupcapitals
  {\dosingleempty\dosetupcapitals}

\def\dosetupcapitals[#1]%
  {\getparameters[\??kk][#1]%
   \doifelse\@@kktitle\v!yes
     {\definealternativestyle[\v!capital][\normalsmallcapped][\normalsmallcapped]%
      \definealternativestyle[\v!smallcaps][\sc][\sc]}
     {\definealternativestyle[\v!capital][\normalsmallcapped][\normalWORD]%
      \definealternativestyle[\v!smallcaps][\sc][\normalWORD]}%
   \doifelse\@@kksc\v!yes\userealcaps\usepseudocaps}

\let\uppercased\normalWORD
\let\lowercased\normalword

\setupcapitals
  [\c!title=\v!yes,
   \c!sc=\v!no]

% \definestartstop is not yet in available at core-spa time
%
% \startrandomized \input tufte \stoprandomized
%
% \definestartstop[randomized][\c!before=\dosetattribute{case}{8},\c!after=]

\def\randomizetext{\groupedcommand{\dosetattribute{case}{8}}{}}

\protect \endinput
