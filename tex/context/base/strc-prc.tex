%D \module
%D   [       file=strc-prc,
%D        version=2008.10.20,
%D          title=\CONTEXT\ Structure Macros,
%D       subtitle=Processors,
%D         author=Hans Hagen,
%D           date=\currentdate,
%D      copyright=PRAGMA-ADE / Hans Hagen]
%C
%C This module is part of the \CONTEXT\ macro||package and is
%C therefore copyrighted by \PRAGMA. See mreadme.pdf for
%C details.

\writestatus{loading}{ConTeXt Structure Macros / Processors}

\registerctxluafile{strc-prc}{1.001}

\unprotect

%D Processors are used when we cannot easily associate typesetting directives
%D with (for instance) structural elements. Instead of ending up with numerous
%D additional definitions we can group treatments in so called processors.
%D
%D An example of where processors can be used is in separator sets (these are
%D related to typesetting numbers using structure).
%D
%D \starttyping
%D \defineprocessor[demo][style=\bfb,color=red]
%D \stoptyping
%D
%D This defines a processor named \type {demo}. Such a name ends up as prefix in
%D for instance:
%D
%D \starttyping
%D \definestructureseparatorset [demosep] [demo->!,demo->?,demo->!,demo->?] [demo->@]
%D \stoptyping
%D
%D Here the \type {!} and \type {?} are just the seperator characters that end
%D up between part, chapter, section, etc.\ numbers. The third argument defines the
%D default. When a separator is inserted, the \type{demo} processor will be applied.
%D Here the number will be separated by red slightly bigger than normal bold
%D exclamation marks and questionmarks
%D
%D Valid keys for defining a processor are \type {style}, \type {color}, \type {left},
%D \type {right}, and \type {command} (the given command takes one argument).

\def\defineprocessor
  {\dodoubleargument\dodefineprocessor}

\def\dodefineprocessor[#1][#2]%
  {\ifsecondargument
     \letbeundefined{\??po#1\c!command}%
     \ctxlua{structure.processors.register("#1")}%
     \getparameters[\??po#1][\c!style=,\c!color=,\c!left=,\c!right=,#2]%
   \else
     \letbeundefined{\??po#1\c!style}%
     \ctxlua{structure.processors.reset("#1")}%
   \fi}

%D The following command can be used by users but normally it will be
%D invoked behind the screens. After all, processor prefixes need to
%D be split off first.

\unexpanded\def\applyprocessor#1%
  {\ifcsname\??po#1\c!style\endcsname
     \expandafter\dodoapplyprocessor
   \else
     \expandafter\secondoftwoarguments
   \fi{#1}}

\def\dodoapplyprocessor#1#2%
  {\begingroup
   \dostartattributes{\??po#1}\c!style\c!color
   \csname\??po#1\c!left\endcsname
   \ifcsname\??po#1\c!command\endcsname
     \csname\??po#1\c!command\endcsname{#2}%
   \else
     #2%
   \fi
   \csname\??po#1\c!right\endcsname
   \dostopattributes
   \endgroup}

\protect \endinput
