\setupbodyfont[dejavu,10pt]

\setuplayout
  [width=middle,
   height=middle,
   backspace=1cm,
   topspace=1cm,
   footer=0pt,
   header=1.25cm]

\setuphead
  [subject]
  [style=\bfa,
   page=yes]

\setuppagenumbering
  [location=]

\setupheadertexts
  [\currentdate][MkIV cleanup Status / Page \pagenumber]

\starttext

\startsubject[title=Todo]

\startitemize[packed]
    \startitem currently the new namespace prefixes are not consistent but this
               will be done when we're satisfied with one scheme \stopitem
    \startitem there will be additional columns in the table, like for namespace
               so we need another round of checking then \stopitem
    \startitem the imp modules are not in the list and need checking too \stopitem
    \startitem the s, x, m modules will be checked, redone and reorganized \stopitem
    \startitem the lua code will be cleaned up upgraded as some is quite old
               and experimental \stopitem
    \startitem we need a proper dependency tree and better defined loading order \stopitem
    \startitem all dotag.. will be moved to the tags_.. namespace \stopitem
    \startitem we need to check what messages are gone (i.e.\ clean up mult-mes) \stopitem
    \startitem some commands can go from mult-def (and the xml file) \stopitem
    \startitem check for setuphandler vs simplesetuphandler \stopitem
    \startitem all showcomposition etc can go (we can redo that in lua if needed) \stopitem
    \startitem for the moment we will go for \type {xxxx_} namespaces that (mostly) match
               the filename but later we can replace these by longer names (via a script) so
               module writers should {\bf not} use the core commands with \type{_} in the
               name \stopitem
\stopitemize

\stopsubject

\startsubject[title=Status]

\startluacode

    local coremodules = dofile("status-mkiv.lua")

    if coremodules then

        local function tabelize(loaded,what)

            if loaded then

                local nofunknown = 0
                local nofloaded  = #loaded

                for i=1,nofloaded do
                    loaded[i].order = i
                end

                table.sort(loaded,function(a,b) return a.filename < b.filename end)

                context.starttabulate { "|Tr|Tl|Tl|l|p|" }
                context.NC() -- context.bold("order")
                context.NC() context.bold("file")
                context.NC() context.bold("mark")
                context.NC() context.bold("status")
                context.NC() context.bold("comment")
                context.NC() context.NR()
                for i=1,nofloaded do
                    local module = loaded[i]
                    local status = module.status
                    context.NC() context(module.order)
                    context.NC() context(module.filename)
                    context.NC() context(module.marktype)
                    if status == "unknown" then
                        context.NC() context.bold(status)
                        nofunknown = nofunknown + 1
                    else
                        context.NC() context(status)
                    end
                    context.NC() context(module.comment)
                    context.NC() context.NR()
                end
                context.stoptabulate()

                context.blank()

                context("Of the %s %s modules (so far) in this list %s have the status unknown",nofloaded,what,nofunknown)

            end

        end

        tabelize(coremodules.core, "core")
        tabelize(coremodules.extra,"extra")

    end

    local namespaces = dofile("status-namespaces.lua")

    local valid = table.tohash {
        "toks", "attr", "page", "buff", "font", "colo", "phys", "supp", "typo", "strc",
        "syst", "tabl", "spac", "scrn", "lang", "lxml", "mlib", "java", "pack", "math",
        "symb", "grph", "anch", "luat", "mult", "back", "node", "meta",
        "module",
    }

    context.startsubject { title = "Valid prefixes" }

    for namespace, data in table.sortedhash(namespaces) do
        if valid[namespace] then
            context.type(namespace)
        end
        context.par()
    end

    context.stopsubject()

    context.startsubject { title = "Messy namespaces" }

    for namespace, data in table.sortedhash(namespaces) do
        if valid[namespace] then
        else
            context(namespace)
        end
        context.par()
    end

    context.stopsubject()

    local registers  = dofile("status-registers.lua")

    context.startsubject { title = "Messy registers" }
    for register, data in table.sortedhash(registers) do
        context(register)
        context.par()
        for name in table.sortedhash(data) do
            context.quad()
            context.type(name)
            context.par()
        end
        context.par()
    end

\stopluacode

\stopsubject

\stoptext
