\setupbackend
  [export=export-example.xml,
   css=export-example.css]

\definedescription
  [description]

\settaggedmetadata
  [title=Export Example,
   author=Hans Hagen,
   version=0.1]

\setupbodyfont
  [dejavu]

\starttext

\startchapter[title=Example]

\startparagraph \input zapf (Zapf) \stopparagraph

\placefigure
  {}
  {\externalfigure[hacker.jpg]}

\startparagraph \input zapf (Zapf) \stopparagraph

\placefigure
  {}
  {\externalfigure[mill.png]}

\startparagraph \input tufte (Tufte) \stopparagraph

\placefigure
  {}
  {\externalfigure[cow.pdf]}

\startparagraph \input tufte (Tufte) \stopparagraph

\startitemize[1]
    \startitem \input ward (Ward) \stopitem
    \startitem \input knuth (Knuth) \stopitem
\stopitemize

\startitemize[2]
    \startitem \input zapf (Zapf) \stopitem
    \startitem \input tufte (Tufte) \stopitem
\stopitemize

\startparagraph \input zapf (Zapf) \stopparagraph

\startdescription {Ward} \input ward \stopdescription

\startdescription {Tufte} \input tufte \stopdescription

\startparagraph \input knuth (Knuth) \stopparagraph

\startformula
e = mc^2
\stopformula

\startparagraph
Okay, it's somewhat boring to always use the same formula, so how about
$\sqrt{4} = 2$ or traveling at \unit{120 km/h} instead of $\unit{110 km/h}$.
\stopparagraph

\typefile{export-example.tex}

\stopchapter

\stoptext
