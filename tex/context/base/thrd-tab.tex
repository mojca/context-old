% Since this file is not available in every distribution, we
% have copied the original in this file. The manuals to
% Wichura's PiCTeX and TaBlE packages are not available on
% line and are distributed by respectively the TeX Users Group
% and Personal TeX Inc. Many macros of TaBlE are overloaded
% and/or extended in core-tab.tex. The extensions concern
% splitting over pages, color and consistent spacing.

% TABLE 1.0
% Copyright Michael J. Wichura August 1988

% The TABLE macros are divided into sections, roughly according to
% function:

% Section Name        Function
% a       Allocation  Allocates storage registers for parameters.
% f       Format      Reads format section; builds preamble for \halign;
%                      processes \ReFormat command.
% g       Get Value   Converts "spec"'s (as in spec_{LT}) to
%                      corresponding "values"'s (as in value_{LT}).
% h       Hacks       Utility macros; error messages; miscellaneous commands.
% k       Keys        Definition and scanning of format keys.
% n       Numeric     Macros for TABLE's numeric format.
% s       Struts      Macros for struts.
% t       Tables      Sets up \halign for table; end-of-row processing;
%                      alternate vertical rules; spanning; horizontal
%                      lines; stretching and shrinking; repositioning
%                      commands.

% The name of each internal macro begins with the prefix "\!t", the
% "!" having category code 11, followed by the letter of the section
% in which the macro is defined. For example, a macro beginning "\!th"
% is defined in Section h (Hacks). There a few exceptions:
% the general purpose macros "\!ttemp", "\!ttempa", "\!ttempb", and
% "\!tnext" are repeatedly defined on the spot as the need arises.

% External macros (and active characters) are defined in the following
% sections:
%    Macro                    Section
%    "                        t
%    \-                       t
%    \=                       t
%    \ActivateBarAndQuote     h
%    \AugmentedTableStrut     s
%    \BackSpace               h
%    \BeginFormat             f
%    \BeginTable              t
%    \BeginTableParBox        a
%    \Center                  t
%    \ColumnWidthFactor       a
%    \ColumnWidthUnit         a
%    \DQuote                  h
%    \EndFormat               k  (\EndFormat is actually a key)
%    \EndTable                t
%    \EndTableParBox          a
%    \Enlarge                 s
%    \enlarge                 s
%    \EveryTable              a
%    \EveryTableParBox        a
%    \Expand                  t
%    \InterColumnSpaceFactor  a
%    \InterColumnSpaceUnit    a
%    \JustCenter              t
%    \JustLeft                t
%    \JustRight               t
%    \KernFactor              a
%    \KernUnit                a
%    \Left                    t
%    \LeftTabskip             a
%    \LineThicknessFactor     a
%    \LineThicknessUnit       a
%    \LongLines               t
%    \Lower                   h
%    \MakeStrut               s
%    \NewFormatKey            k
%    \NormalCWU               a
%    \NormalICSU              a
%    \NormalKU                a
%    \NormalLTU               a
%    \NormalSU                a
%    \NormalTableUnits        a
%    \OpenUp                  s
%    \PseudoVrule             t
%    \Raise                   h
%    \ReadFormatKeys          k
%    \ReFormat                f
%    \Right                   t
%    \RightTabskip            a
%    \SetTableToWidth         t
%    \Smash                   h
%    \StandardTableStrut      s
%    \StrutDepthFactor        a
%    \StrutHeightFactor       a
%    \StrutUnit               a
%    \TaBlE                   h
%    \TracingFormats          a
%    \TracingKeys             a
%    \Use                     t
%    \use                     t
%    \VBar                    h
%    \Vspace                  h
%    \VspaceFactor            a
%    \WidenTableBy            t
%    \\                       t
%    \_                       t
%    \|                       t
%    |                        t
%    ~                        t
%

\catcode `\!=11
\catcode `\@=11

\newif\ifh@ % normally in plain tex
\newif\ifv@ % normally in plain tex

% Don't try to read the TABLE macros until after you've read the
% TABLE manual. The internal documentation of the macros is
% sketchy; you need the manual to understand what's going on.
% You should also review the material on \halign s in the TeXbook,
% since TABLE uses an \halign to perform its alignments.

% In studying the TABLE macros, you should start by skimming the
% macros in the "miscellaneous hacks", "error messages", and "loops"
% subsections of Section h, as well as the "\GetValue" macro in
% Section g; these macros are called many times by the other macros.
% To continue with a "bottom-up" approach, read next Sections k,
% f, and t. (Top-downers should reverse the order.) The other
% Sections can be looked at as the need arises.



% *********************************************************************
% SECTION A: ALLOCATION
% *********************************************************************

\let\!tacr=\\ % Save meaning of \\ (Needed if TABLE is used with LaTeX

% *********************************************************************
% TABLE PARAMETERS: Units
% *********************************************************************

\newdimen\LineThicknessUnit
\newdimen\StrutUnit
\newskip \InterColumnSpaceUnit
\newdimen\ColumnWidthUnit
\newdimen\KernUnit

\let\!taLTU=\LineThicknessUnit % Used in preamble
\let\!taCWU=\ColumnWidthUnit   % Used in preamble
\let\!taKU =\KernUnit          % Used in preamble

\newtoks\NormalTLTU
\newtoks\NormalTSU
\newtoks\NormalTICSU
\newtoks\NormalTCWU
\newtoks\NormalTKU

% NOTE: The user should modify the following DEFAULTS to suit his/her
% taste, and output device:
%\def\PixelsPerInch{300}
\NormalTLTU={1in \divide \LineThicknessUnit by 300 }
\NormalTSU ={\normalbaselineskip
  \divide \StrutUnit by 11 }  % 11 = 8+3 = NormalT Height+Depth Factors
\NormalTICSU={.5em plus 1fil minus .25em}  % .5em = width of a digit
\NormalTCWU ={.5em}
\NormalTKU  ={.5em}

\def\NormalTableUnits{%
  \LineThicknessUnit   =\the\NormalTLTU
  \StrutUnit           =\the\NormalTSU
  \InterColumnSpaceUnit=\the\NormalTICSU
  \ColumnWidthUnit     =\the\NormalTCWU
  \KernUnit            =\the\NormalTKU}

\NormalTableUnits

% The user should issue \NormalTableUnits when setting a table
% in a different point size, since the Table...Units themselves
% are static while the Normal...Units vary with the point size.


% *********************************************************************
% TABLE PARAMETERS: Factors
% *********************************************************************

\newcount\LineThicknessFactor
\newcount\StrutHeightFactor
\newcount\StrutDepthFactor
\newcount\InterColumnSpaceFactor
\newcount\ColumnWidthFactor
\newcount\KernFactor
\newcount\VspaceFactor

% DEFAULTS:
\LineThicknessFactor    =2
\StrutHeightFactor      =8
\StrutDepthFactor       =3
\InterColumnSpaceFactor =3
\ColumnWidthFactor      =10
\KernFactor             =1
\VspaceFactor           =2


% *********************************************************************
% DIAGNOSTIC PARAMETERS
% *********************************************************************

\newcount\TracingKeys % >=1 reports new keys, >=2 reports key usage
\newcount\TracingFormats  % >=1 reports templates for columns
                          % >=2 reports \halign preamble


% *********************************************************************
% PARBLOCK PARAMETERS
% *********************************************************************

\def\BeginTableParBox#1{%
  \vtop\bgroup
    \hsize=#1
    \normalbaselines
    \let~=\!ttTie
    \let\-=\!ttDH
    \the\EveryTableParBox}

\def\EndTableParBox{%
    \MakeStrut{0pt}{\StrutDepthFactor\StrutUnit}
  \egroup} % finishes the \vtop begun by \BeginTableParbox

\newtoks\EveryTableParBox
\EveryTableParBox={%
  \parindent=0pt
  \raggedright
  \rightskip=0pt plus 4em %   Provide more stretch
  \relax}


% *********************************************************************
% EVERY TABLE TOKENS
% *********************************************************************

\newtoks\EveryTable
\newtoks\!taTableSpread


% *********************************************************************
% Extreme left- and right- tabskips
% *********************************************************************

\newskip\LeftTabskip
\newskip\RightTabskip


% *********************************************************************
% INTERNAL VARIABLES
% *********************************************************************

\newcount\!taCountA
\newcount\!taColumnNumber
\newcount\!taRecursionLevel % (Initially 0)

\newdimen\!taDimenA  % used by \Enlarge
\newdimen\!taDimenB  % used by \Enlarge
\newdimen\!taDimenC  % used by numeric.tex
\newdimen\!taMinimumColumnWidth

\newtoks\!taToksA

\newtoks\!taPreamble
\newtoks\!taDataColumnTemplate
\newtoks\!taRuleColumnTemplate
\newtoks\!taOldRuleColumnTemplate
\newtoks\!taLeftGlue
\newtoks\!taRightGlue

\newskip\!taLastRegularTabskip

\newif\if!taDigit
\newif\if!taBeginFormat
\newif\if!taOnceOnlyTabskip



% *********************************************************************
% SECTION H: HACKS
% *********************************************************************

% ****************************************************************
% TABLE LOGO
% ****************************************************************
\def\TaBlE{%
  T\kern-.27em\lower.5ex\hbox{A}\kern-.18em B\kern-.1em
    \lower.5ex\hbox{L}\kern-.075em E}


% ****************************************************************
% ACTIVE CHARACTERS
% ****************************************************************

% ACTIVATE BAR AND QUOTE: Makes | and " active if they aren't
% already active (in which case the user will probably have given
% them special meanings); definitions are provided which effectively
% undoes the activeness outside a Table.

{\catcode`\|=13 \catcode`\"=13
  \gdef\ActivateBarAndQuote{%
    \ifnum \catcode`\|=13
    \else
      \catcode`\|=13
      \def|{%
        \ifmmode
          \vert
        \else
          \char`\|
        \fi}%
    \fi
    \ifnum \catcode`\"=13
    \else
      \catcode`\"=13
      \def"{\char`\"}%
    \fi}}

% ****************************************************************
% Macros for | and " having category code 12.
% ****************************************************************
{\catcode `\|=12 \catcode `\"=12
\gdef\VBar{|}
\gdef\DQuote{"}}


% ****************************************************************
% MISCELANEOUS HACKS
% ****************************************************************

% MESSAGE <Message>: Writes out <Message> to terminal and log file.
\def\!thMessage#1{\immediate\write16{#1}\ignorespaces}

% X: Abbreviation for expandafter
\let\!thx=\expandafter

% GOBBLE: Eats next token
\def\!thGobble#1{}

% SPACE TOKEN
\def\\{\let\!thSpaceToken= }\\

% HEIGHT, DEPTH, AND WIDTH
\def\!thHeight{height}
\def\!thDepth{depth}
\def\!thWidth{width}

% TOKSEDEF <token register>=<replacement text>: Places <replacement
% text>, fully expanded a la \edef, in the specified <token register>.
\def\!thToksEdef#1=#2{%
  \edef\!ttemp{#2}%
  #1\!thx{\!ttemp}%
  \ignorespaces}


% ****************************************************************
% ERROR MESSAGES
% ****************************************************************

% STORE ERROR MSG <Control Sequence> <Message>
% Replacement text of <Control Sequence> is a macro with Message
% as its name.  E.g., after \StoreErrorMsg\Help{Type <CR>},
% \Help expands to "\Type <CR>"
\def\!thStoreErrorMsg#1#2{%
  \toks0 =\!thx{\csname #2\endcsname}%
  \edef#1{\the\toks0 }}

% READ ERROR MSG <Control sequence>
% Continuing the above example, \ReadErrorMsg\Help produces "Type <CR>"
\def\!thReadErrorMsg#1{%
  \!thx\!thx\!thx\!thGobble\!thx\string #1}

% ERROR <Error Message> <Error Help>
\def\!thError#1#2{%
  \begingroup
    \newlinechar=`\^^J%
    \edef\!ttemp{#2}%
    \errhelp=\!thx{\!ttemp}%
    \!thMessage{%
      ^^J\!thReadErrorMsg\!thErrorMsgA
      ^^J\!thReadErrorMsg\!thErrorMsgB}%
    \errmessage{#1}%
  \endgroup}

% TEXT FOR ERROR MESSAGE
\!thStoreErrorMsg\!thErrorMsgA{%
  TABLE error; see manual for explanation.}
\!thStoreErrorMsg\!thErrorMsgB{%
  Type \space H <return> \space for immediate help.}

% GET REPLACEMENT <Prompt Message> <Replacement Value>
%  <Replacement Vale> must be a control sequence
\def\!thGetReplacement#1#2{%
   \begingroup
     \!thMessage{#1}
     \endlinechar=-1
     \global\read16 to#2%
   \endgroup}


% ****************************************************************
% LOOP MACRO
% ****************************************************************

% LOOP ... REPEAT macro from TUGboat Vol 8 #2: 1987
% Syntax is like that of plain TeX's \loop ... \repeat macro
\def\!thLoop#1\repeat{%
  \def\!thIterate{%
    #1%
    \!thx \!thIterate
    \fi}%
  \!thIterate
  \let\!thIterate\relax}


% ***************************************************************
% VERTICALLY-CENTERED SMASH
% ***************************************************************

% SMASH: Like TeX's \smash, only the argument
% is centered vertically before its height and depth are smashed to 0pt.
\def\Smash{%
  \relax
  \ifmmode
    \expandafter\mathpalette
    \expandafter\!thDoMathVCS
  \else
    \expandafter\!thDoVCS
  \fi}

% DO VCS
\def\!thDoVCS#1{%
  \setbox\z@\hbox{#1}%
  \!thFinishVCS}

% DO MATH VCS
\def\!thDoMathVCS#1#2{%
  \setbox\z@\hbox{$\m@th#1{#2}$}%
  \!thFinishVCS}

% FINISH VCS
\def\!thFinishVCS{%
  \vbox to\z@{\vss\box\z@\vss}}


% ***************************************************************
% RAISE AND LOWER
% ***************************************************************

% Like TeX's \raise and \lower, except: (1) The first argument
% to these commands is a dimension expressed in TABLE's usual conventions;
% the default is (StrutHeightFactor+StrutDepthFactor)*StrutUnit/2
% (2) like \smash, these commands function in math mode as well
% as horizontal mode; (3) again like \smash, the result is declared
% to have height and depth 0pt

% Examples  \Raise2{Stuff}:  "Stuff" is raised 2*StrutUnit
%           \Raise {Stuff}:  "Stuff" is raised a half-line
%           $\Lower(10pt){\alpha}$:  "$\alpha$" is lowered 10 points

% RAISE
\def\Raise{%
  \def\!thSign{+}%
  \!tgGetValue\!thSetDimen}

% LOWER
\def\Lower{%
  \def\!thSign{-}%
  \!tgGetValue\!thSetDimen}

% SET DIMEN
\def\!thSetDimen{%
  \ifnum \!tgCode=1
    \ifx \!tgValue\empty
      \!taDimenA \StrutHeightFactor\StrutUnit
      \advance \!taDimenA \StrutDepthFactor\StrutUnit
      \divide \!taDimenA 2
    \else
      \!taDimenA \!tgValue\StrutUnit
    \fi
  \else
    \!taDimenA \!tgValue
  \fi
  \!taDimenA=\!thSign\!taDimenA\relax
  %
  % BRANCH ON MODE
  \ifmmode
    \expandafter\mathpalette
    \expandafter\!thDoMathRaise
  \else
    \expandafter\!thDoSimpleRaise
  \fi}

% DO SIMPLE RAISE
\def\!thDoSimpleRaise#1{%
  \setbox\z@\hbox{\raise \!taDimenA\hbox{#1}}%
  \!thFinishRaise} % From Plain TeX: \ht0=0pt \dp0=0pt \box0

% DO MATH RAISE
\def\!thDoMathRaise#1#2{%
  \setbox\z@\hbox{\raise \!taDimenA\hbox{$\m@th#1{#2}$}}%
  \!thFinishRaise}

% FINISH RAISE. This is the same as Plain's \finsm@sh; some macro
% packages redefine \finsm@sh.
\def\!thFinishRaise{%
  \ht\z@\z@
  \dp\z@\z@
  \box\z@}


% ***************************************************************
% BACK SPACE
% ***************************************************************
\def\BackSpace{%
  \!tgGetValue\!thKernBack}

\def\!thKernBack{%
  \kern -
  \ifnum \!tgCode=1
    \ifx \!tgValue\empty
      \the\KernFactor
    \else
      \!tgValue    % user-specified integer
    \fi
    \KernUnit
  \else
    \!tgValue      % user-specified dimension
  \fi
  \ignorespaces}%


% ***************************************************************
% Vspace
% ***************************************************************
\def\Vspace{%
  \noalign
  \bgroup
  \!tgGetValue\!thVspace}

\def\!thVspace{%
  \vskip
    \ifnum \!tgCode=1
      \ifx \!tgValue\empty
        \the\VspaceFactor
      \else
        \!tgValue    % user-specified integer
      \fi
      \StrutUnit
    \else
      \!tgValue      % user-specified skip
    \fi
  \egroup} % Ends the \noalign

% *********************************************************************
% SECTION F: FORMAT
% *********************************************************************

% As explained in Section 3.3 of the manual, TABLE alternates each
% of the user's "data" columns with a "rule" column; moreover, TABLE
% places a "dummy data" column at the left and right of a table.
% A table with  n  nominal data columns therefore actually has a
% total of
%       n        (nominal data columns)
%     +(n+1)     (rule columns)
%     + 2        (dummy data columns)
%     ____
%      2n+3
% columns.

% FORMATs job is to create an \halign preamble for the alignment
% of these (2n+3) columns. The preamble consists of templates
% for the various columns, strung together with &'s and interlaced
% with \tabskip glue specifications.

% FORMAT constructs the template for a nomimal data column according
% to the user-specified format keys. As the keys are read from left
% to right, the template is built up "from the inside out" (as
% illustrated in Section 3.1.9 of the manual), the inner-most part
% being the "#" sign. A "|" in the format terminates template
% building; the completed template is adjoined to preamble along
% with the template for the following rule column.

% Minimum column widths, if specified, are implemented by creating
% an "artificial row" with data entries of the form
%   \hskip <minimum column width>.
% This row has zero height and depth and is completely invisible.


% BEGIN FORMAT
\def\BeginFormat{%
  \catcode`\|=12 % Inhibit expansion if | immediately follows a <number>
  \catcode`\"=12 %  read by \getvalue.
  \!taPreamble={}%
  \!taColumnNumber=0
  \skip0 =\InterColumnSpaceUnit
  \multiply\skip0 \InterColumnSpaceFactor
  \divide\skip0 2
  \!taRuleColumnTemplate=\!thx{%
    \!thx\tabskip\the\skip0 }%
  \!taLastRegularTabskip=\skip0
  \!taOnceOnlyTabskipfalse
  \!taBeginFormattrue % Used to intercept key "]"
  \def\!tfRowOfWidths{}%  Artificial Table Row with horizontal struts
                       %  to enforce specified minimum column widths
  \ReadFormatKeys}

% SET (MINIMUM COLUMN) WIDTH: Invoked by the key "w".
\def\!tfSetWidth{%
  \ifx \!tfRowOfWidths \empty  % true if no prior "w" keys
    \ifnum \!taColumnNumber>0  % true if "w" key is to right of first "|"
      \begingroup              % RowOfWidths={&\omit || n copies of
                               % &\omit&\omit}, where n = number of columns
         \!taCountA=1          % to the left of this one
         \aftergroup \edef \aftergroup \!tfRowOfWidths \aftergroup {%
           \aftergroup &\aftergroup \omit
           \!thLoop
             \ifnum \!taCountA<\!taColumnNumber
             \advance\!taCountA 1
             \aftergroup \!tfAOAO
           \repeat
           \aftergroup }%
      \endgroup
    \fi
  \fi
  \ifx [\!ttemp % \!tgGetValue sets \!ttemp = token after w
    \!thx\!tfSetWidthText
  \else
    \!thx\!tfSetWidthValue
  \fi}

% AOAO = (Apersand Omit Ampersand Omit)
\def\!tfAOAO{%
  &\omit&\omit}

% SET WIDTH TEXT
\def\!tfSetWidthText [#1]{% #1 = specified text
  \def\!tfWidthText{#1}%
  \ReadFormatKeys}

% SET WIDTH VALUE
\def\!tfSetWidthValue{%
  \!taMinimumColumnWidth =
    \ifnum \!tgCode=1
      \ifx\!tgValue\empty % Use default multiplier if user didn't specify one
        \ColumnWidthFactor
      \else
        \!tgValue
      \fi
      \ColumnWidthUnit
    \else
      \!tgValue
    \fi
  \def\!tfWidthText{}%      Override possible prior `w[sample entry]'
  \ReadFormatKeys}


% SET TABSKIP: Invoked by the tabskip keys "t" and "o"
\def\!tfSetTabskip{%
  \ifnum \!tgCode=1
    \skip0 =\InterColumnSpaceUnit
    \multiply\skip0
      \ifx \!tgValue\empty
        \InterColumnSpaceFactor         % Default integer
      \else
       \!tgValue                        % User-specified integer
      \fi
  \else
    \skip0 =\!tgValue                   % User-specified <skip>
  \fi
  \divide\skip0 by 2
  \ifnum\!taColumnNumber=0
    \!thToksEdef\!taRuleColumnTemplate={%
      \the\!taRuleColumnTemplate
      \tabskip \the\skip0 }
  \else
    \!thToksEdef\!taDataColumnTemplate={%
      \the\!taDataColumnTemplate
      \tabskip \the\skip0 }
  \fi
  \if!taOnceOnlyTabskip
  %                               % Tabskip used at right of this col only
  \else
    \!taLastRegularTabskip=\skip0 % Remember this Tabskip, for possible
  \fi                             % restoration after a subsequent"OnceOnly"
  \ReadFormatKeys}


% SET VRULE: Invoked by the key "|"
\def\!tfSetVrule{%
  \!thToksEdef\!taRuleColumnTemplate={%
    \noexpand\hfil
    \noexpand\vrule
    \noexpand\!thWidth
    \ifnum \!tgCode=1
      \ifx \!tgValue\empty
        \the\LineThicknessFactor      % Default integer
      \else
        \!tgValue                     % User-specified integer
      \fi
      \!taLTU                         % \LineThicknessUnit
    \else
      \!tgValue                       % User-specified dimension
    \fi
    ####%
    \noexpand\hfil
    \the\!taRuleColumnTemplate}       % has \tabskips, when column number=0
  \!tfAdjoinPriorColumn}

% SET ALTERNATE VRULE: Invoked by the key "\|", in the form
%   \|{<template for (rule) column>}. The "{" and "}" are mandatory,
% and the <template for column> must contain a "#". The key system
% CAN'T be used to set up this template.  The <template> can have the
% form  "\span\macro".
\def\!tfSetAlternateVrule{%
  \afterassignment\!tfSetAlternateA
  \toks0 =}                           % Put template into \toks0

\def\!tfSetAlternateA{%
  \!thToksEdef\!taRuleColumnTemplate={%
    \the\toks0 \the\!taRuleColumnTemplate} % RCT may have \tabskips
  \!tfAdjoinPriorColumn}

% ADJOIN PRIOR COLUMN
\def\!tfAdjoinPriorColumn{%
  \ifnum \!taColumnNumber=0
    \!taPreamble=\!taRuleColumnTemplate % New \tabskip may have been added
    \ifnum \TracingFormats>0
      \!tfShowRuleTemplate
    \fi
  \else
    \ifx\!tfRowOfWidths\empty  % no "w" keys specified yet, not even this col
    \else
      \!tfUpdateRowOfWidths
    \fi
    % Adjoin positioning glues to left and right of template
    \!thToksEdef\!taDataColumnTemplate={%
      \the \!taLeftGlue
      \the \!taDataColumnTemplate
      \the \!taRightGlue}
    \ifnum \TracingFormats>0
      \!tfShowTemplates
    \fi
    % Adjoin data- and rule-column templates to preamble
    \!thToksEdef\!taPreamble={%
      \the\!taPreamble
      &
      \the\!taDataColumnTemplate
      &
      \the\!taRuleColumnTemplate}
  \fi
%
% START NEW COLUMN
  \advance \!taColumnNumber 1
  % Initialize data-column template, restoring last "regular" tabskip
  % after a "once only" tabskip
  \if!taOnceOnlyTabskip
    \!thToksEdef\!taDataColumnTemplate={%
       ####\tabskip \the\!taLastRegularTabskip}
  \else
    \!taDataColumnTemplate{##}%
  \fi
  % Remaining initializations
  \!taRuleColumnTemplate{}% # is inserted by \SetVrule, or \SetAlternateVrule
  \!taLeftGlue{\hfil}%         % Default positioning is "center"
  \!taRightGlue{\hfil}%
  \!taMinimumColumnWidth=0pt
  \def\!tfWidthText{}%
  \!taOnceOnlyTabskipfalse    % Set true by key "o"
  \ReadFormatKeys}

% UPDATE ROW OF WIDTHS
\def\!tfUpdateRowOfWidths{%
  % If user had a "w[<Text>]" key, set <Text> according to the
  % template for this column, and find the width of the result
  \ifx \!tfWidthText\empty
  \else % set specified text according to current template & find width
    \!tfComputeMinColWidth
  \fi
  \edef\!tfRowOfWidths{%
    \!tfRowOfWidths
    &%
    \omit                                  % Data Column
    \ifdim \!taMinimumColumnWidth>0pt
      \hskip \the\!taMinimumColumnWidth
    \fi
    &
    \omit}}                                % Rule Column

% COMPUTE MINIMUM COLUMN WIDTH (from specified WidthText)
\def\!tfComputeMinColWidth{%
  \setbox0 =\vbox{%
    \ialign{% Plain's initialized \halign; \tabskip=0pt \everycr={}
       \span\the\!taDataColumnTemplate\cr
       \!tfWidthText\cr}}%
  \!taMinimumColumnWidth=\wd0 }

% SHOW (INITIAL) RULE TEMPLATE
\def\!tfShowRuleTemplate{%
  \!thMessage{}
  \!thMessage{TABLE FORMAT}
  \!thMessage{Column: Template}
  \!thMessage{%
    \space *c: ##\tabskip \the\LeftTabskip}
  \!taOldRuleColumnTemplate=\!taRuleColumnTemplate}

% SHOW TEMPLATES
\def\!tfShowTemplates{%
  \!thMessage{%
    \space \space r: \the\!taOldRuleColumnTemplate}
  \!taOldRuleColumnTemplate=\!taRuleColumnTemplate
  \!thMessage{%
    \ifnum \!taColumnNumber<10
      \space
    \fi
    \the\!taColumnNumber c: \the\!taDataColumnTemplate}
  \ifdim\!taMinimumColumnWidth>0pt
    \!thMessage{%
      \space \space w: \the\!taMinimumColumnWidth}
  \fi}


% FINISH UP: Invoked by the keys "." and \EndFormat
\def\!tfFinishFormat{%
  \ifnum \TracingFormats>0
    \!thMessage{%
      \space \space r: \the\!taOldRuleColumnTemplate
        \tabskip \the\RightTabskip}%
    \!thMessage{%
      \space *c: ##\tabskip 0pt}
  \fi
  \ifnum \!taColumnNumber<2
    \!thError{%
      \ifnum \!taColumnNumber=0
        No
      \else
        Only 1
      \fi
      "|"}%
      {\!thReadErrorMsg\!tfTooFewBarsA
       ^^J\!thReadErrorMsg\!tfTooFewBarsB
       ^^J\!thReadErrorMsg\!tkFixIt}%
  \fi
  \!thToksEdef\!taPreamble={%
    ####\tabskip\LeftTabskip
    &
    \the\!taPreamble \tabskip\RightTabskip
    &
    ####\tabskip 0pt \cr}
  \ifnum \TracingFormats>1
    \!thMessage{Preamble=\the\!taPreamble}
  \fi
  \ifnum \TracingFormats>2
    \!thMessage{Row Of Widths="\!tfRowOfWidths"}
  \fi
  \!taBeginFormatfalse % Intercepts "|", tabskips, and "."
  \catcode`\|=13
  \catcode`\"=13
  \!ttDoHalign}

% ERROR MESSAGE FOR NOT ENOUGH "|"'s
\!thStoreErrorMsg\!tfTooFewBarsA{%
  There must be at least 2 "\string|"'s (and/or "\string \|"'s)}
\!thStoreErrorMsg\!tfTooFewBarsB{%
  between \string\BeginFormat\space and \string\EndFormat\space (or ".").}


% REFORMAT [<key letters>]{<text>}: Formats <text> according to
% <key letters>. Used to override the template for a column,
% or columns when used after \use.
\def\ReFormat[{%
  \omit
  \!taDataColumnTemplate{##}%
  \!taLeftGlue{}%
  \!taRightGlue{}%
  \catcode`\|=12  % Inhibit expansion if | immediately follows a <number>
  \catcode`\"=12  % read by \getvalue. Actually, '|' and '"' shouldn't
  \ReadFormatKeys}% appear in a \ReFormat cmd; this is here as a safeguard.

% END REFORMAT: Invoked by the key "]"
\def\!tfEndReFormat{%
  \ifnum \TracingFormats>0
    \!thMessage{ReF:
       \the\!taLeftGlue
       \hbox{\the\!taDataColumnTemplate}%   White lie
       \the\!taRightGlue}
  \fi
  \catcode`\|=13
  \catcode`\"=13
  \!tfReFormat}

\def\!tfReFormat#1{%
  \the \!taLeftGlue
  \vbox{%
    \ialign{%
      \span\the\!taDataColumnTemplate\cr
       #1\cr}}%
  \the \!taRightGlue}



% *********************************************************************
% SECTION G: GET VALUE
% *********************************************************************

% GET_VALUE{<return macro>}<tokens> functions as follows:

% If <tokens> has the form <(stuff)>, then
%    code=2   and   value=<stuff>

% Otherwise <tokens> has the form <DDDXYZ> where <DDD> denotes (a possibly
% empty) string of consecutive digits (0,1,2,...,9) terminated by the first
% character <X> (possibly a blank) that is not a digit. In this case
%    code=1   and   value=<DDD>  (= <null>,  if <DDD> is non-empty).

% Examples:                         Code      Value
%  "\GetValue{\macro} 3"               1       null
%  "\GetValue{\macro}A "               1       null
%  "\GetValue{\macro}1 "               1          1
%  "\GetValue{\macro}25A"              1         25
%  "\GetValue{\macro}25012 "           1      25012
%  "\GetValue{\macro}(10pt)"           2       10pt
%  "\GetValue{\macro}(1in)"            2        1in
%  "\GetValue{\macro} (1in)"           1       null


% GET_VALUE{<macro to execute after value is found>}
\def\!tgGetValue#1{%
  \def\!tgReturn{#1}%          Set return
  \futurelet\!ttemp\!tgCheckForParen}%  Now \!ttemp is the token
                                    %  immediately after {}

% CHECK_PAREN: See if \!ttemp is a (
\def\!tgCheckForParen{%
  \ifx\!ttemp (%
    \!thx \!tgDoParen
  \else
    \!thx \!tgCheckForSpace
  \fi}

% DO_PAREN: Set code to 2, value to stuff inside ( )'s
\def\!tgDoParen(#1){%
  \def\!tgCode{2}%
  \def\!tgValue{#1}%     NOTE #1 MUST BE A LEGITIMATE VALUE
  \!tgReturn}

% CHECK_SPACE: See if \!ttemp is a <blank space>
\def\!tgCheckForSpace{%
  \def\!tgCode{1}%
  \def\!tgValue{}%           Initialize value to <null>
  \ifx\!ttemp\!thSpaceToken
    \!thx \!tgReturn        % <blank space> means no value was specified
  \else
    \!thx \!tgCheckForDigit
  \fi}

% CHECK_DIGIT: \!ttemp is not a <blank space>; if its a digit (0,1,...,9)
% get the <number> starting with that digit.
\def\!tgCheckForDigit{%
  \!taDigitfalse
  \ifx 0\!ttemp
    \!taDigittrue
  \else
    \ifx 1\!ttemp
      \!taDigittrue
    \else
      \ifx 2\!ttemp
        \!taDigittrue
      \else
        \ifx 3\!ttemp
          \!taDigittrue
        \else
          \ifx 4\!ttemp
            \!taDigittrue
          \else
            \ifx 5\!ttemp
              \!taDigittrue
            \else
              \ifx 6\!ttemp
                \!taDigittrue
              \else
                \ifx 7\!ttemp
                  \!taDigittrue
                \else
                  \ifx 8\!ttemp
                    \!taDigittrue
                  \else
                    \ifx 9\!ttemp
                      \!taDigittrue
                    \fi
                  \fi
                \fi
              \fi
            \fi
          \fi
        \fi
      \fi
    \fi
  \fi
  \if!taDigit
    \!thx \!tgGetNumber
  \else
    \!thx \!tgReturn
  \fi}

% GET_NUMBER
\def\!tgGetNumber{%
  \afterassignment\!tgGetNumberA
  \!taCountA=}
\def\!tgGetNumberA{%
  \edef\!tgValue{\the\!taCountA}%
  \!tgReturn}


% ********************************************************************
% MISCELANEOUS "RETURNS" FROM \getvalue
% ********************************************************************

% SET UP PAR BOX: Puts \BeginTableParBox{<user-specified \hsize>}
% to the left of "#" and \EndTableParBox to the right of "#".
\def\!tgSetUpParBox{%
  \edef\!ttemp{%
    \noexpand \ReadFormatKeys
    b{\noexpand \BeginTableParBox{%
      \ifnum \!tgCode=1
        \ifx \!tgValue\empty
          \the\ColumnWidthFactor
        \else
          \!tgValue    % user-specified integer
        \fi
        \!taCWU        % \ColumnWidthUnit
      \else
        \!tgValue      % user-specified dimension
      \fi}}}%
  \!ttemp
  a{\EndTableParBox}}

% SET KERNS
\def\!tgInsertKern{%
  \edef\!ttemp{%
    \kern
    \ifnum \!tgCode=1
      \ifx \!tgValue\empty
        \the\KernFactor
      \else
        \!tgValue    % user-specified integer
      \fi
      \!taKU         % \KernUnit
    \else
      \!tgValue      % user-specified dimension
    \fi}%
  \edef\!ttemp{%
    \noexpand\ReadFormatKeys
    \ifh@            % true if kern goes to left of "#"
      b{\!ttemp}
    \fi
    \ifv@            % true if kern goes to right of "#"
      a{\!ttemp}
    \fi}%
  \!ttemp}

% *********************************************************************
% SECTION K: KEYS
% *********************************************************************

% ****************************************************************
% DEFINING NEW KEYS
% ****************************************************************

% NEW FORMAT KEY <Key Letter>: Must be followed by
%   <Parameter Text> <Replacement Text>
% Sets up a new key letter command by expanding (essentially) to
%   \expandafter \def \csname !tk<Key Letter>\endcsname
%     <Parameter Text>{<Replacement Text>}
% A warning message is issued if <Key Letter> is already in use.
\def\NewFormatKey#1{%
  \!thx\def\!thx\!ttempa\!thx{\string #1}%
  \!thx\def\!thx\!ttempb\!thx{\csname !tk<\!ttempa>\endcsname}%
  \ifnum \TracingKeys>0
    \!tkReportNewKey
  \fi
  \!thx\ifx \!ttempb \relax
    \!thx\!tkDefineKey
  \else
    \!thx\!tkRejectKey
  \fi}

% REPORT NEW KEY
\def\!tkReportNewKey{%
  \!taToksA\!thx{\!ttempa}%
  \!thMessage{NEW KEY: "\the\!taToksA"}}

% DEFINE KEY
\def\!tkDefineKey{%
  \!thx\def\!ttempb}%

% DUPLICATE KEY
\def\!tkRejectKey{%
    \!taToksA\!thx{\!ttempa}%
    \!thError{Key letter "\the\!taToksA" already used}
      {\!thReadErrorMsg\!tkFixIt}
    \def\!tkGarbage}%

% ERROR MESSAGE FOR DUPLICATE KEY
\!thStoreErrorMsg\!tkFixIt{%
  You'd better type \space 'E' \space and fix your file.}


% ****************************************************************
% READING FORMAT KEYS
% ****************************************************************

% READ FORMAT KEYS
\def\ReadFormatKeys#1{%
  \!thx\def\!thx\!ttempa\!thx{\string #1}%
  \!thx\def\!thx\!ttempb\!thx{\csname !tk<\!ttempa>\endcsname}%
  \ifnum \TracingKeys>1
    \!tkReportKey
  \fi
  \!thx\ifx \!ttempb\relax
    \!thx\!tkReplaceKey
  \else
    \!thx\!ttempb
  \fi}

% REPORT KEY
\def\!tkReportKey{%
  \!taToksA\!thx{\!ttempa}%
  \!thMessage{KEY: "\the\!taToksA"}}

% REPLACE KEY
\def\!tkReplaceKey{%
  \!taToksA\!thx{\!ttempa}%
  \!thError {Undefined format key "\the\!taToksA"}
    {\!thReadErrorMsg\!tkUndefined ^^J\!thReadErrorMsg\!tkBadKey}
  \!tkReplaceKeyA}

\def\!tkReplaceKeyA{%
  \!thGetReplacement{\!thReadErrorMsg\!tkReplace}\!tkReplacement
  \!thx\ReadFormatKeys\!tkReplacement}

% ERROR MESSAGES FOR KEY RELACEMENT
\!thStoreErrorMsg\!tkUndefined{%
  The format key in " "'s on the next to top line is undefined.}
\!thStoreErrorMsg\!tkBadKey{%
  Type \space E \space to quit now, or
  \space<CR> \space and respond to next prompt.}
\!thStoreErrorMsg\!tkReplace{%
  Type \space<replacement key><CR> \space,
   or simply \space<CR> \space to skip offending key:}


% ****************************************************************
% PRIMITIVE KEYS
% ****************************************************************

% Key "b":  b{TOKENS} adds TOKENS to the left of (before) the template
\NewFormatKey b#1{%
  \!thx\!tkJoin\!thx{\the\!taDataColumnTemplate}{#1}%
  \ReadFormatKeys}

\def\!tkJoin#1#2{%
  \!taDataColumnTemplate{#2#1}}%

% Key "a":  a{TOKENS} adds TOKENS to the right of (after) the template
\NewFormatKey a#1{%
  \!taDataColumnTemplate\!thx{\the\!taDataColumnTemplate #1}%
  \ReadFormatKeys}

% Key "\{": Enclose template in braces.
\NewFormatKey \{{%
  \!taDataColumnTemplate=\!thx{\!thx{\the\!taDataColumnTemplate}}%
  \ReadFormatKeys}

% Key "*":  "*{N}{KEY LETTERS}" is equivalent to specifying
% <KEY LETTERS>  N  times.
% KEY LETTERS may contain further * specifications
\NewFormatKey *#1#2{%
  \!taCountA=#1\relax
  \!taToksA={}%
  \!thLoop
    \ifnum \!taCountA > 0
    \!taToksA\!thx{\the\!taToksA #2}%
    \advance\!taCountA -1
  \repeat
  \!thx\ReadFormatKeys\the\!taToksA}


% ****************************************************************
% POSITIONING KEYS
% ****************************************************************

% Key "\LeftGlue": Specifies the glue (usually \hfil, or nothing) to be
% added to extreme left of the template to position a column
\NewFormatKey \LeftGlue#1{%
  \!taLeftGlue{#1}%
  \ReadFormatKeys}

% Key "\RightGlue": Specifies the glue (usually \hfil, or nothing) to be
% added to the extreme right of the template to position a column
\NewFormatKey \RightGlue#1{%
  \!taRightGlue{#1}%
  \ReadFormatKeys}

% Key "c": Centered column.
\NewFormatKey c{%
  \ReadFormatKeys
  \LeftGlue\hfil
  \RightGlue\hfil}

% Key "l": Left-adjusted column.
\NewFormatKey l{%
  \ReadFormatKeys
  \LeftGlue{}   % In case more than one positioning key is specified.
  \RightGlue\hfil}

% Key "r": Right-adjusted column.
\NewFormatKey r{%
  \ReadFormatKeys
  \LeftGlue\hfil
  \RightGlue{}}

% Key "k": Adds kerns to left and right of "#"
% This key and the two below use Plain TeX's \if@h as if it were \if@left,
% and \if@v as if it were \if@right. Table making goes on in a group,
% so even in the unlikely circumstance that a \phantom is currently under
% construction, there's no problem.
\NewFormatKey k{%
  \h@true
  \v@true
  \!tgGetValue{\!tgInsertKern}}

% Key "i": Adds a kern to the left of "#"
\NewFormatKey i{%
  \h@true
  \v@false
  \!tgGetValue{\!tgInsertKern}}

% Key "j": Adds a kern to the right of "#"
\NewFormatKey j{%
  \h@false
  \v@true
  \!tgGetValue{\!tgInsertKern}}


% ****************************************************************
% NUMERIC ITEM KEYS
% ****************************************************************

% Key "n": numeric item , non-math mode.
\NewFormatKey n{%
  \def\!tnStyle{}%
   \futurelet\!tnext\!tnTestForBracket}

% Key "N": numeric item, math mode.
\NewFormatKey N{%
  \def\!tnStyle{$}%
   \futurelet\!tnext\!tnTestForBracket}


% ****************************************************************
% ATTRIBUTE KEYS
% ****************************************************************

% Key "m": Math mode.
\NewFormatKey m{%
  \ReadFormatKeys b$ a$}

% Key "M": Displaymath mode.
\NewFormatKey M{%
  \ReadFormatKeys \{ b{$\displaystyle} a$}

% Key "\m": Template ${}#\hfil$
\NewFormatKey \m{%
  \ReadFormatKeys l b{{}} m}

% Key "\M": Template $\displaystyle{{}#\hfil}$
\NewFormatKey \M{%
  \ReadFormatKeys l b{{}} M}

% Key "f":  Set font  (E.g., f\it sets up italic font (assuming \it
% has its usual meaning)
\NewFormatKey f#1{%
  \ReadFormatKeys b{#1}}

% Key "B": abbreviation for f\bf
\NewFormatKey B{%
  \ReadFormatKeys f\bf}

% Key "I": abbreviation for f\it
\NewFormatKey I{%
  \ReadFormatKeys f\it}

% Key "S": abbreviation for f\sl
\NewFormatKey S{%
  \ReadFormatKeys f\sl}

% Key "R": abbreviation for f\rm
\NewFormatKey R{%
  \ReadFormatKeys f\rm}

% Key "T": abbreviation for f\tt
\NewFormatKey T{%
  \ReadFormatKeys f\tt}

% Key "p": ParBox
\NewFormatKey p{%
  \!tgGetValue{\!tgSetUpParBox}}


% ****************************************************************
% MINIMUM COLUMN WIDTH KEY
% ****************************************************************

% Key "w": minimum column width
\NewFormatKey w{%
  \!tkTestForBeginFormat w{\!tgGetValue{\!tfSetWidth}}}


% ****************************************************************
% TABSKIP KEYS
% ****************************************************************

% Key "s": Set tabskip for the inter-column space to the right
% of the current column, and all subsequent spaces, until overriden
% by a new "s" or "o" key.
\NewFormatKey s{%
  \!taOnceOnlyTabskipfalse    % in case same column has a prior "o" key
  \!tkTestForBeginFormat t{\!tgGetValue{\!tfSetTabskip}}}

% Key "o": Apply the \tabskip stated for this column ONLY to the
% inter-column space just to the right of this column; restore the
% the previous \tabskip for subsequent columns.
\NewFormatKey o{%
  \!taOnceOnlyTabskiptrue
  \!tkTestForBeginFormat o{\!tgGetValue{\!tfSetTabskip}}}


% ****************************************************************
% RULE KEYS
% ****************************************************************

% Key "|": Standard rule column designator
\NewFormatKey |{%
  \!tkTestForBeginFormat |{\!tgGetValue{\!tfSetVrule}}}

% Key "\|": Non-standard rule column designator
\NewFormatKey \|{%
  \!tkTestForBeginFormat \|{\!tfSetAlternateVrule}}


% ****************************************************************
% END-OF-FORMAT KEYS
% ****************************************************************

% Key ".":  PERIOD -- end of \BeginFormat section.
\NewFormatKey .{%
  \!tkTestForBeginFormat.{\!tfFinishFormat}}

% Key "\EndFormat": Equivalent to "."
\NewFormatKey \EndFormat{%
  \!tkTestForBeginFormat\EndFormat{\!tfFinishFormat}}

% Key "]": End of \ReFormat section
\NewFormatKey ]{%
  \!tkTestForReFormat ] \!tfEndReFormat}


% ****************************************************************
% VALIDITY CHECKS
% ****************************************************************

% TEST FOR BEGIN FORMAT{<Key>}{Intended Action}: This test is run
% on keys that can only be used by \BeginFormat ---  "s",  "o",
% "|",  "\|",  "w",  ".",  and  "\EndFormat".
\def\!tkTestForBeginFormat#1#2{%
  \if!taBeginFormat
    \def\!ttemp{#2}%
    \!thx \!ttemp
  \else
    \toks0={#1}%
    \toks2=\!thx{\string\ReFormat}%
    \!thx \!tkImproperUse
  \fi}

% TEST FOR RE FORMAT{<Key>}{Intended Action}: This test is run
% on the key "]", which can only be used by \ReFormat.
\def\!tkTestForReFormat#1#2{%
  \if!taBeginFormat
    \toks0={#1}%
    \toks2=\!thx{\string\BeginFormat}%
    \!thx \!tkImproperUse
  \else
    \def\!ttemp{#2}%
    \!thx \!ttemp
  \fi}

% IMPROPER USE OF KEY
\def\!tkImproperUse{%
  \!thError{\!thReadErrorMsg\!tkBadUseA "\the\toks0 "}%
    {\!thReadErrorMsg\!tkBadUseB \the\toks2 \space command.
    ^^J\!thReadErrorMsg\!tkBadKey}%
  \!tkReplaceKeyA}

% ERROR MESSAGES FOR IMPROPER USE OF KEY
\!thStoreErrorMsg\!tkBadUseA{Improper use of key }
\!thStoreErrorMsg\!tkBadUseB{%
  The key mentioned above can't be used in a }



% *********************************************************************
% SECTION n: NUMERIC
% *********************************************************************

% NOTE: THE SPACE BETWEEN A NUMERIC ENTRY AND THE FOLLOWING '|', '"',
%   OR '\|' IS MANDATORY.
% EMPTY NUMERIC ENTRIES ARE NOT ALLOWED: USE '{}' OR '\omit' INSTEAD.

% TEST FOR BRACKET: Invoked by the keys "n" and "N".
\def\!tnTestForBracket{%
  \ifx [\!tnext
    \!thx\!tnGetArgument
  \else
    \!thx\!tnGetCode
  \fi}

% GET CODE: E.g. "4", or "4.0", "0.4", or "10.2"
\def\!tnGetCode#1 {%  NOTE THE BLANK
  \!tnConvertCode #1..!}

% CONVERT CODE: E.g. converts above to [0000], [0000.], [.0000],
% [0000000000.00]
\def\!tnConvertCode #1.#2.#3!{%
  \begingroup
    \aftergroup\edef \aftergroup\!ttemp \aftergroup{%
      \aftergroup[%
      \!taCountA #1
      \!thLoop
        \ifnum \!taCountA>0
        \advance\!taCountA -1
        \aftergroup0
      \repeat
      \def\!ttemp{#3}%
      \ifx\!ttemp \empty
      \else
        \aftergroup.
        \!taCountA #2
        \!thLoop
          \ifnum \!taCountA>0
          \advance\!taCountA -1
          \aftergroup0
        \repeat
      \fi
      \aftergroup]\aftergroup}%
    \endgroup\relax
    \!thx\!tnGetArgument\!ttemp}

% GET ARGUMENT: [<sample left field> <optional .<sample right field>>
\def\!tnGetArgument[#1]{%
  \!tnMakeNumericTemplate\!tnStyle#1..!}

% MAKE NUMERIC TEMPLATE
\def\!tnMakeNumericTemplate#1#2.#3.#4!{%  #1=<empty> or $
  \def\!ttemp{#4}%
  \ifx\!ttemp\empty
    \!taDimenC=0pt
  \else
    \setbox0=\hbox{\m@th #1.#3#1}%
    \!taDimenC=\wd0
  \fi
  \setbox0 =\hbox{\m@th #1#2#1}%
  \!thToksEdef\!taDataColumnTemplate={%
    \noexpand\!tnSetNumericItem
    {\the\wd0 }%
    {\the\!taDimenC}%
    {#1}%
    \the\!taDataColumnTemplate}  % Might have tabskip glue in here
  \ReadFormatKeys}

% SET NUMERIC ITEM
\def\!tnSetNumericItem #1#2#3#4 {%             NOTE THE BLANK
  \!tnSetNumericItemA {#1}{#2}{#3}#4..!}

\def\!tnSetNumericItemA #1#2#3#4.#5.#6!{%
  \def\!ttemp{#6}%
  \hbox to #1{\hss \m@th #3#4#3}%
  \hbox to #2{%
    \ifx\!ttemp\empty
    \else
       \m@th #3.#5#3%
    \fi
    \hss}}



% *********************************************************************
% SECTION S: STRUTS
% *********************************************************************

% The following are in ALLOCATIONS
  %\newdimen\StrutUnit (normal value \normalbaselineskip / 11)
  %\newcount\StrutHeightFactor (normal value 8)
  %\newcount\StrutDepthFactor  (normal value 3)

% MAKE STRUT OF SPECIFIED HEIGHT AND DIMENSION
% \MakeStrut <height><depth>; height and depth are <dimen>'s
\def\MakeStrut#1#2{%
  \vrule width0pt height #1 depth #2}

% STANDARD VERTICAL STRUT
% Makes a strut of height=StrutHeightFactor*StrutUnit
%                  depth =StrutDepthFactor *StrutUnit
\def\StandardTableStrut{%
  \MakeStrut{\StrutHeightFactor\StrutUnit}
    {\StrutDepthFactor\StrutUnit}}


% STANDARD VERTICAL STRUT, WITH EXTRA HEIGHT/DEPTH
% \AugmentedTableStrut<multiple for extra height><multiple for extra depth>
%   makes a strut of height=(StrutHeightFactor+#1)*StrutUnit
%                    depth =(StrutDepthFactor+#2)*StrutUnit
\def\AugmentedTableStrut#1#2{%
  \dimen@=\StrutHeightFactor\StrutUnit
  \advance\dimen@ #1\StrutUnit
  \dimen@ii=\StrutDepthFactor\StrutUnit
  \advance\dimen@ii #2\StrutUnit
  \MakeStrut{\dimen@}{\dimen@ii}}


% ENLARGE<extra height><extra depth><original>
% Enlarges "original" by extra height and extra depth.
% Extra height and extra depth are <dimen>'s.
% Works for various math styles, and takes into account
%   \spacefactor in horizontal mode
\def\Enlarge#1#2{%  3rd argument is picked up later
  % #1=extra height
  % #2=extra depth
  \!taDimenA=#1\relax
  \!taDimenB=#2\relax
  \let\!TsSpaceFactor=\empty
  \ifmmode
    \!thx \mathpalette
    \!thx \!TsEnlargeMath
  \else
    \!thx \!TsEnlargeOther
  \fi}

\def\!TsEnlargeOther#1{%
  \ifhmode
    \setbox\z@=\hbox{#1%
      \xdef\!TsSpaceFactor{\spacefactor=\the\spacefactor}}%
  \else
    \setbox\z@=\hbox{#1}%
  \fi
  \!TsFinishEnlarge}

\def\!TsEnlargeMath#1#2{%
  \setbox\z@=\hbox{$\m@th#1{#2}$}%
  \!TsFinishEnlarge}

\def\!TsFinishEnlarge{%
  \dimen@=\ht\z@
  \advance \dimen@ \!taDimenA
  \ht\z@=\dimen@
  \dimen@=\dp\z@
  \advance \dimen@ \!taDimenB
  \dp\z@=\dimen@
  \box\z@ \!TsSpaceFactor{}}


% ENLARGE BY MULTIPLES OF StrutUnit
% \enlarge<multiple for extra height><multiple for extra depth><original>
% Enlarges by (multiple for extra height)*StrutUnit
%   and       (multiple for extra depth) *StrutUnit
\def\enlarge#1#2{%  3rd argument is picked up later
  \Enlarge{#1\StrutUnit}{#2\StrutUnit}}


% OPENUP#1#2: increases strut height and depth factors by #1 and #2.
\def\OpenUp#1#2{%
  \advance \StrutHeightFactor #1\relax
  \advance \StrutDepthFactor #2\relax}



% *********************************************************************
% SECTION T: TABLES
% *********************************************************************

% Table-making is initiated by \BeginTable. After processing that
% command, TeX absorbs the instructions in the prologue to the table
% until it gets to \BeginFormat. \BeginFormat sets up the preamble
% for the \halign that will be used to create the table. \EndFormat
% initiates the \halign-ment, which is terminated by \EndTable.


% *********************************************************************
% BEGIN TABLE, (DO HALIGN), END TABLE
% *********************************************************************
% BEGIN TABLE
\def\BeginTable{%
  \futurelet\!tnext\!ttBeginTable}

\def\!ttBeginTable{%
  \ifx [\!tnext
    \def\!tnext{\!ttBeginTableA}%
  \else
    \def\!tnext{\!ttBeginTableA[c]}%
  \fi
  \!tnext}

\def\!ttBeginTableA[#1]{%
  \if #1u%                  % "unboxed" table
    \ifmmode
      \def\!ttEndTable{%    % user had better be in display math mode
        \relax}%            %   and have only one table at the outer level
    \else                   % user had better be in vertical mode
      \bgroup
      \def\!ttEndTable{%
        \egroup}%
    \fi
  \else
    \hbox\bgroup $
    \def\!ttEndTable{%
      \egroup %   for the \vtop, \vbox, or \vcenter, yet to come
      $%          for math mode
      \egroup}%   for the \hbox
    \if #1t%
      \vtop
    \else
      \if #1b%
        \vbox
      \else
        \vcenter % math mode was essential for this
      \fi
    \fi
    \bgroup      % for the \vtop, \vbox, or \vcenter
  \fi
  \advance\!taRecursionLevel 1 % RecursionLevel governs initialization
  \let\!ttRightGlue=\relax  % This may be changed by \JustCenter, etc
  \everycr={}
  \ifnum \!taRecursionLevel=1
    \!ttInitializeTable
  \fi}

% INITIALIZE TABLE
\bgroup
  \catcode`\|=13
  \catcode`\"=13
  \catcode`\~=13
  \gdef\!ttInitializeTable{%
    \let\!ttTie=~ %                         Meanings of ~ and \- are
    \let\!ttDH=\- %                           restored by \BeginTableParBox
    \catcode`\|=\active
    \catcode`\"=\active
    \catcode`\~=\active
    \def |{\unskip\!ttRightGlue&&}%         Use rule-column template
    \def\|{\unskip\!ttRightGlue&\omit\!ttAlternateVrule}%
    %                                       Override rule-column template
    \def"{\unskip\!ttRightGlue&\omit&}%     Omit rule-column template
    \def~{\kern .5em}%                      ~ now has the width of a digit
    \def\\{\!ttEndOfRow}%
    \def\-{\!ttShortHrule}%
    \def\={\!ttLongHrule}%
    \def\_{\!ttFullHrule}%
    \def\Left##1{##1\hfill\null}%           \null prevents \unskip from
    \def\Center##1{\hfill##1\hfill\null}%    killing the \hfill
    \def\Right##1{\hfill##1}%
    \def\use{\!ttuse}%
    \def\Use{\!ttUse}%
    \the\EveryTable}
\egroup

\let\!ttRightGlue=\relax  % This may be changed, in a group, by
                          %   \JustCenter, etc

% DO HALIGN: Invoked by END FORMAT (or the key ".")
\def\!ttDoHalign{%
  \baselineskip=0pt \lineskiplimit=0pt \lineskip=0pt %
  \tabskip=0pt
  \halign \the\!taTableSpread \bgroup
   \span\the\!taPreamble
   \ifx \!tfRowOfWidths \empty
   \else
     \!tfRowOfWidths \cr %
   \fi}

% END TABLE
\def\EndTable{%
  \egroup % finishes the \halign
  \!ttEndTable}%    closes off the table envirnoment set up by \BeginTable


% *********************************************************************
% END OF ROW PROCESSING
% *********************************************************************

% END OF ROW: When followed by
%   0, inserts no strut
%   +, inserts an AugmentedTableStrut (with <x-height> and <x-depth>
%      as arguments
%   anything else, inserts a StandardTableStrut,
% and finished off the row with a \cr.
\def\!ttEndOfRow{%
  \futurelet\!tnext\!ttTestForBlank}

% TEST FOR BLANK
\def\!ttTestForBlank{%
%  \!thMessage{At Test For Blank: \meaning\!tnext}
  \ifx \!tnext\!thSpaceToken  % the "usual" case
    \!thx\!ttDoStandard
  \else
    \!thx\!ttTestForZero
  \fi}

% TEST FOR ZERO
\def\!ttTestForZero{%
  \ifx 0\!tnext
    \!thx \!ttDoZero
  \else
    \!thx \!ttTestForPlus
  \fi}

% TEST FOR PLUS
\def\!ttTestForPlus{%
  \ifx +\!tnext
    \!thx \!ttDoPlus
  \else
    \!thx \!ttDoStandard
  \fi}

% DO ZERO: No strut
\def\!ttDoZero#1{% #1 eats the 0
  \cr}

% DO PLUS: Insert "Extra" strut; #2=extra height, #3=extra depth, both
% as integers (units of \StrutUnit)
\def\!ttDoPlus#1#2#3{% #1 eats the +
  \AugmentedTableStrut{#2}{#3}%
  \cr}

% DO STANDARD: Insert standard table strut
\def\!ttDoStandard{%
  \StandardTableStrut
  \cr}


% *********************************************************************
% ALTERNATE VRULES
% *********************************************************************

% A '\|' can appear in a rule-column in place of a '|', '"', or '&'.

% If '\|' is immediately followed by a blank, a string of digits, or
% (...) [... had better be a <dimen>], a \vrule is placed in the
% rule column; the thickness of the \vrule follows TABLE's usual
% conventions. Be sure to put a blank after a string of digits.

% If '\|' is immediately followed by a '*', a user-specified default
% "pseudo"-rule is placed in the rule column. This P.R. is specified
% by the parameterless macro \PseudoVrule. For example,
% to place a "double rule" into a rule column, you could make the definition
%   \def\PseudoVrule{\hfil\vrule \hskip1pt \vrule\hfil}

% If none of the above cases applies, a non-space token follows '\|':
% that token is placed in the rule-column. To put a '*' in a
% rule-column, enter '\|{*}'. '\|\PseudoVrule' has the same effect
% as '\|*'.

% ALTERNATE VRULE
\def\!ttAlternateVrule{%
  \!tgGetValue{\!ttAVTestForCode}}  % AV == Alternate Vrule

% TEST FOR CODE (2)
\def\!ttAVTestForCode{%
  \ifnum \!tgCode=2              % (...) follows "\|"
    \!thx\!ttInsertVrule         % \InsertVrule ends with "&"
  \else
    \!thx\!ttAVTestForEmpty
  \fi}

% TEST FOR EMPTY (VALUE)
\def\!ttAVTestForEmpty{%
  \ifx \!tgValue\empty           % non-digit after "\|"
    \!thx\!ttAVTestForBlank
  \else
    \!thx\!ttInsertVrule         % integer after "\|"
  \fi}

% TEST FOR BLANK
\def\!ttAVTestForBlank{%
  \ifx \!ttemp\!thSpaceToken     % blank after "\|"
    \!thx\!ttInsertVrule
  \else
    \!thx\!ttAVTestForStar
  \fi}

% TEST FOR STAR
\def\!ttAVTestForStar{%
  \ifx *\!ttemp                  % "*" after "\|"
    \!thx\!ttInsertDefaultPR     % PR == pseudo-rule
  \else
    \!thx\!ttGetPseudoVrule       % "Anything else" after "\|"
  \fi}

% INSERT VRULE
\def\!ttInsertVrule{%
  \hfil
  \vrule \!thWidth
    \ifnum \!tgCode=1
      \ifx \!tgValue\empty
        \LineThicknessFactor
      \else
        \!tgValue
      \fi
      \LineThicknessUnit
    \else
      \!tgValue
    \fi
  \hfil
  &}

% INSERT DEFAULT PSEUDO-RULE
\def\!ttInsertDefaultPR*{%
  \PseudoVrule    % User-specified default pseudo-rule
  &}

% GET PSEUDO-RULE
\def\!ttGetPseudoVrule#1{%
  \toks0={#1}%
  #1&}

% DEFAULT PSEUDO-RULE
\def\PseudoVrule{}


% *********************************************************************
% USE: Version of \multispan for rule-&-column tables
% *********************************************************************

% USE
% \use <number> spans the next <number> data columns.
\def\!ttuse#1{%
  \ifnum #1>\@ne
    \omit
    \mscount=#1 %        \mscount is in Plain
    \advance\mscount by \m@ne
    \advance\mscount by \mscount
    \!thLoop
      \ifnum\mscount>\@ne
      \sp@n %            from Plain (\span\omit \advance\mscount\m@ne)
    \repeat
    \span
  \fi}

\def\!ttUse#1[{%
  \!ttuse{#1}%
  \ReFormat[}


% *********************************************************************
% HRULES
% *********************************************************************

% FULL HORIZONTAL RULE: Draws a rule across the table,
% using \noalign{\hrule}
\def\!ttFullHrule{%
  \noalign
  \bgroup
  \!tgGetValue{\!ttFullHruleA}}

\def\!ttFullHruleA{%
  \!ttGetHalfRuleThickness % Sets \dimen0 to half of specified thickness
  \hrule \!thHeight \dimen0 \!thDepth \dimen0
  \penalty0 % so can break an ``unboxed'' table after a horizontal rule.
  \egroup} % ends the \noalign

% SHORT HORIZONTAL RULE: Draws a rule across 1 (or more) columns,
% using \leaders; this rule doesn't extend across the neighboring
% tabskip glues to join up with adjacent rule columns. By contrast
% the LONG HORIZONTAL RULE below does just that.
\def\!ttShortHrule{%
  \omit
  \!tgGetValue{\!ttShortHruleA}}

\def\!ttShortHruleA{%
  \!ttGetHalfRuleThickness % Sets \dimen0 to half of specified thickness
  \leaders \hrule \!thHeight \dimen0 \!thDepth \dimen0 \hfill
  \null    % prevents an \unskip from annihilating the \leaders
  \ignorespaces}

% LONG HORIZONTAL RULE: This rule requires special coding.
% It must be preceded and followed by '&', instead of the usual
% '|' or '"'. However, '\_' can follow '\use' in the usual manner.
% And in fact, to insert long-rules in two or more contiguous columns,
% '\use' MUST be used with an argument = total number of columns involved.
\def\!ttLongHrule{%
  \omit\span\omit\span \!ttShortHrule}

% GET RULE THICKNESS
\def\!ttGetHalfRuleThickness{%
  \dimen0 =
    \ifnum \!tgCode=1
      \ifx \!tgValue\empty
        \LineThicknessFactor
      \else
        \!tgValue    % user-specified integer
      \fi
      \LineThicknessUnit
    \else
      \!tgValue      % user-specified dimension
    \fi
  \divide\dimen0 2 }


% *********************************************************************
% STRETCHING AND SHRINKING A TABLE
% *********************************************************************

% SET TABLE TO WIDTH <dimen>
\def\SetTableToWidth#1{%
  \!taTableSpread={to #1}}

% WIDEN TABLE BY <dimen>
\def\WidenTableBy#1{%
  \ifdim #1=0pt
    \!taTableSpread={}%
  \else
    \!taTableSpread={spread #1}%
  \fi}

\def\Expand{%
  \SetTableToWidth{\hsize}}%

\def\LongLines{%
  \LeftTabskip =0pt plus 1fill
  \RightTabskip=\LeftTabskip
  \Expand}


% *********************************************************************
% REPOSITIONING COMMANDS (\JUSTLEFT, etc.)
% *********************************************************************

\def\JustLeft{%
  \omit \let\!ttRightGlue=\hfill}
\def\JustCenter{%
  \omit \hfill\null \let\!ttRightGlue=\hfill}
\def\JustRight{%
  \omit \hfill\null}


% *********************************************************************
% Restore meaning of \\, and reset category codes
% *********************************************************************
\let\\=\!tacr
\catcode`\!=12
\catcode`\@=12

\endinput
