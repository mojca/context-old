%D \module
%D   [       file=syst-ini,
%D        version=2008.11.04, % 2001.11.16, % 1999.03.17,  % an oldie: 1995.10.10
%D          title=\CONTEXT\ System Macros,
%D       subtitle=Bootstrapping \TEX,
%D         author=Hans Hagen,
%D           date=\currentdate,
%D      copyright={PRAGMA / Hans Hagen \& Ton Otten}]
%C
%C This module is part of the \CONTEXT\ macro||package and is
%C therefore copyrighted by \PRAGMA. See mreadme.pdf for
%C details.

%D We used to load plain \TEX\ in a special way, but redefining
%D a couple of primitives so that for instance font loading was
%D ignored. For those interested, this loader is found in
%D \type {syst-tex.tex}. Some of the comment's are Don Knuth's
%D and more of it can be found in the plain \TEX\ format.

%D Characters can have special states, that can be triggered
%D by setting their category coded. Some are preset, others
%D are to be set as soon as possible, otherwise we cannot
%D define any useful macros.

%catcode`\^^@   =  9 % ascii null is ignored
%catcode`\\     =  0 % backslash is TeX escape character

\catcode`\{     =  1 % left brace is begin-group character
\catcode`\}     =  2 % right brace is end-group character
\catcode`\$     =  3 % dollar sign is math shift
\catcode`\&     =  4 % ampersand is alignment tab
\catcode`\#     =  6 % hash mark is macro parameter character
\catcode`\^     =  7 % circumflex and uparrow are for superscripts
\catcode`\_     =  8 % underline and downarrow are for subscripts
\catcode`\^^I   = 10 % ascii tab is a blank space

%catcode`\^^M   =  5 % ascii return is end-line
%catcode`\%     = 14 % percent sign is comment character
%catcode`\      = 10 % ascii space is blank space
%catcode`\^^?   = 15 % ascii delete is invalid

\catcode`\~     = 13 % tilde is active
\catcode`\^^L   = 13 % ascii form-feed

%catcode`\A     = 11
%.......
%catcode`\Z     = 11

%catcode`\a     = 11
%.......
%catcode`\z     = 11

\chardef\activecatcode = 13 % later this will become a counter

\def ^^L{\par}
\def\^^M{\   } % control <return> = control <space>
\def\^^I{\   } % same for <tab>

%D In \CONTEXT, we simply ignore end||of||file tokens:

\catcode`\^^Z=9

%D It makes sense to know what engine we're running so let's
%D try to deduce it.

\chardef\unknownengine = 0
\chardef\pdftexengine  = 1
\chardef\xetexengine   = 2
\chardef\luatexengine  = 3

\ifx\directlua\undefined
    \ifx\XeTeXversion\undefined
        \ifx\pdftexversion\undefined
            \let\texengine\unknownengine
        \else
            \let\texengine\pdftexengine
        \fi
    \else
        \let\texengine\xetexengine
    \fi
\else
    \let\texengine\luatexengine
\fi

\ifnum\texengine=\unknownengine
    \immediate\write16{>>>}
    \immediate\write16{>>> only pdftex, xetex and luatex are supported}
    \immediate\write16{>>>}
    \let\dump\relax
    \expandafter\end
\fi

%  todo: pdfsave pdfrestore pdfcolor... don't initialize them

\ifnum\texengine=\luatexengine
    \directlua 0 { % this info is stored in the format
        lua.name[0] = "main ctx instance"
        local extraprimitives  = tex.extraprimitives
        local enableprimitives = tex.enableprimitives
        local core   = extraprimitives('core')
        local btex   = extraprimitives('tex')
        local etex   = extraprimitives('etex')
        local pdftex = extraprimitives('pdftex')
        local luatex = extraprimitives('luatex')
        local omega = {
            "textdir", "pagedir", "mathdir", "pardir", "bodydir",
            "leftghost", "rightghost", "localleftbox", "localrightbox",
            "localinterlinepenalty", "localbrokenpenalty",
        }
        local aleph = {
            "boxdir", "pagebottomoffset", "pagerightoffset",
        }
        for _, subset in next, { etex, pdftex, luatex, omega, aleph } do
            enableprimitives('',subset)
        end
        for _, subset in next, { core, btex, etex, pdftex, luatex, omega, aleph } do
            enableprimitives('normal',subset)
        end
    }
\fi

%D \ETEX\ has a not so handy way of telling you the version number,
%D i.e. the revision number has a period in it:

\long\def\gobbleoneargument#1{} % will be defined later on anyway

\mathchardef\etexversion =
    \numexpr\eTeXversion*100+\expandafter\gobbleoneargument\eTeXrevision\relax

%D First we define a simplified version of the \CONTEXT\
%D protection mechanism.

\def\unprotect{\catcode`@=11 }
\def\protect  {\catcode`@=12 }

\unprotect

%D Some pretty important definitions:

\let\bgroup={
\let\egroup=}

%D Allocation of registers is done slightly different than in plain
%D \TEX. First of all we use different reserved counters. We also
%D don't implement a family handler because users are not supposed
%D to implement their own math. We reserve the lowest 31 registers
%D for scratch purposes. Keep in mind that in the core engine
%D some registers are reserved: counters 0 upto 9, and counter 255.
%D
%D As with plain \TEX\ we recommend that macro designers always use
%D \type {\global} assignments with respect to registers numbered 1,
%D 3, 5 \unknown\ 31, and always non||\type {\global} assignments
%D with respect to registers 0, 2, 4, \unknown\ 30. This will prevent
%D \quote {save stack buildup} that might otherwise occur.
%D
%D We reserve some registers for special (management) purposes:

%   0 -  20 : scratch
%  21 - 127 : internal
% 128 - 254 : inserts
% 255       : page
% 256 -     : user

\countdef   \minallocatedregister   =    52 \minallocatedregister   =   256
\countdef   \maxallocatedregister   =    53 \maxallocatedregister   = 32767
\countdef   \minallocatediochannel  =    54 \minallocatediochannel  =    -1
\countdef   \maxallocatediochannel  =    55 \maxallocatediochannel  =    16
\countdef   \minallocatedlanguage   =    56 \minallocatedlanguage   =     0
\countdef   \maxallocatedlanguage   =    57 \maxallocatedlanguage   =   255
\countdef   \maxallocatedinsert     =    58 \maxallocatedinsert     =   254
\countdef   \minallocatedinsert     =    59 \minallocatedinsert     =   128
\countdef   \minallocatedfamily     =    60 \minallocatedfamily     =   128
\countdef   \maxallocatedfamily     =    61 \maxallocatedfamily     =   255

\countdef   \lastallocatedcount     =    32 \lastallocatedcount     = \minallocatedregister
\countdef   \lastallocateddimen     =    33 \lastallocateddimen     = \minallocatedregister
\countdef   \lastallocatedskip      =    34 \lastallocatedskip      = \minallocatedregister
\countdef   \lastallocatedmuskip    =    35 \lastallocatedmuskip    = \minallocatedregister
\countdef   \lastallocatedbox       =    36 \lastallocatedbox       = \minallocatedregister
\countdef   \lastallocatedtoks      =    37 \lastallocatedtoks      = \minallocatedregister
\countdef   \lastallocatedread      =    38 \lastallocatedread      = \minallocatediochannel
\countdef   \lastallocatedwrite     =    39 \lastallocatedwrite     = \minallocatediochannel
\countdef   \lastallocatedmarks     =    40 \lastallocatedmarks     = \minallocatedregister
\countdef   \lastallocatedlanguage  =    41 \lastallocatedlanguage  = \minallocatedlanguage    % not used in context
\countdef   \lastallocatedinsertion =    42 \lastallocatedinsertion = \minallocatedinsert
\countdef   \lastallocatedfamily    =    43 \lastallocatedfamily    = \minallocatedfamily      % not used in context
\countdef   \lastallocatedattribute =    44 \lastallocatedattribute = \minallocatedregister

\countdef   \mincountervalue        =   125 \mincountervalue        = -"7FFFFFFF % beware, we use index 125 at the lua end
\countdef   \maxcountervalue        =   126 \maxcountervalue        =  "7FFFFFFF % beware, we use index 126 at the lua end

%countdef   \minusone               =   127 \minusone               = -1
%chardef    \zerocount              =     0
%chardef    \plusone                =     1

\countdef   \zerocount              =   120 \zerocount              =  0
\countdef   \plusone                =   121 \plusone                =  1
\countdef   \minusone               =   122 \minusone               = -1

%chardef    \normalpagebox                                          = 255
\countdef   \normalpagebox          =   127 \normalpagebox          = 255  % hardcoded in pdftex/xetex

% A few traditional allocations:

\countdef   \count@                 =   255 % hm, used in \newif .. todo: replace it there
\dimendef   \dimen@                 =     0
\dimendef   \dimen@i                =     1 % global only
\dimendef   \dimen@ii               =     2

%D So, effectively we start allocating from 256 and upwards. The
%D inserts sit in the range 128 upto 254. Page numbers use the
%D counters 0 upto 9 and the pagebox is 255. Users can use the
%D scratch registers upto 31 without problem but all others are
%D reserved.

\def\wlog#1{} % \def\wlog{\immediate\write\minusone} % write on log file (only)

%D The allocators share a common helper macro.

\def\newcount   {\allocateregister\lastallocatedcount    \count   \countdef   \maxallocatedregister}
\def\newdimen   {\allocateregister\lastallocateddimen    \dimen   \dimendef   \maxallocatedregister}
\def\newskip    {\allocateregister\lastallocatedskip     \skip    \skipdef    \maxallocatedregister}
\def\newmuskip  {\allocateregister\lastallocatedmuskip   \muskip  \muskipdef  \maxallocatedregister}
\def\newbox     {\allocateregister\lastallocatedbox      \box     \mathchardef\maxallocatedregister}
\def\newtoks    {\allocateregister\lastallocatedtoks     \toks    \toksdef    \maxallocatedregister}
\def\newread    {\allocateregister\lastallocatedread     \read    \chardef    \maxallocatediochannel}
\def\newwrite   {\allocateregister\lastallocatedwrite    \write   \chardef    \maxallocatediochannel}
\def\newmarks   {\allocateregister\lastallocatedmarks    \marks   \mathchardef\maxallocatedregister}
\def\newinsert  {\allocateregister\lastallocatedinsertion\insert  \chardef    \maxallocatedinsert}

%D We don't need these in \CONTEXT:

\def\newlanguage{\allocateregister\lastallocatedlanguage \language\chardef    \maxallocatedlanguage}
\def\newfamily  {\allocateregister\lastallocatedfamily   \fam     \chardef    \maxallocatedfamily}

\let\newfam\newfamily

% watch out, for the moment we disable the check for already being defined
% later we will revert this but first all chardefs must be replaced

\def\newconstant   #1{\ifdefined#1\let#1\undefined\fi\newcount#1}
\def\setnewconstant#1{\ifdefined#1\let#1\undefined\fi\newcount#1#1} % just a number

% matbe setconstant with check

% %D The next definitions are really needed (in \CONTEXT):

%newlinechar=10   \def\outputnewlinechar{\rawcharacter{10}}
\newlinechar=10  \edef\outputnewlinechar{^^J}

%D One reason to start high with allocation is that it permits us to
%D allocate consecutive ranges more easily, for instance in \MPLIB\
%D we want to allocate a continuous range of boxes. It also permits us
%D to do a proper upward allocation for inserts. The current code
%D evolved from code that dealt with older engines but as all engines
%D now provide many registers we removed all traces.

\def\writestatus#1#2{\immediate\write16{#1: #2}} \def\space { }

\def\allocateregisteryes#1#2#3#4#5% last class method max name
  {\ifnum#1<#4\relax
     \global\advance#1\plusone
     \global#3#5=#1\relax
   \else
     \writestatus{warning}{no room for \string#2\space \string#5\space (max: \number#4)}%
   \fi}

\def\allocateregisternop#1#2#3#4#5% last class method max name
  {\writestatus{warning}{\string#2 \string#5 is already defined (\string\relax\space it first)}}

\def\allocateregister#1#2#3#4#5% last class method max name
  {\ifx#5\undefined
     \expandafter\allocateregisteryes
   \else\ifx#5\relax
     \expandafter\expandafter\expandafter\allocateregisteryes
   \else
     \expandafter\expandafter\expandafter\allocateregisternop
   \fi\fi
   #1#2#3#4#5}

%D Since the number of chars exceed 256 now, we can use \type
%D {\chardef} instead of the more limited \type {\mathchardef}.

\ifnum\texengine>\pdftexengine
    \def\newbox  {\allocateregister\lastallocatedbox  \box  \chardef\maxallocatedregister}
    \def\newmarks{\allocateregister\lastallocatedmarks\marks\chardef\maxallocatedregister}
\fi

%D Attributes are something very \LUATEX. In \CONTEXT\ you are not
%D supposed to use the attributes directly but always allocate then
%D first. For instance attribute~0 is reserved for special purposes
%D (this might change).

\ifnum\texengine=\luatexengine
    \let\attributeunsetvalue\mincountervalue % used to be \minusone
    \def\newattribute{\allocateregister\lastallocatedattribute\attribute\attributedef\maxallocatedregister}
\fi

%D Not used by \CONTEXT\ but for instance \PICTEX\ needs it. It's a
%D trick to force strings instead of tokens that take more memory.
%D It's a trick to trick to force strings.

\def\newhelp#1#2{\newtoks#1#1\expandafter{\csname#2\endcsname}}

%D \macros
%D   {scratchcounter,
%D    scratchdimen,scratchskip,scratchmuskip,
%D    scratchbox,
%D    scratchtoks}
%D
%D We now define a few scratch registers, so that successive
%D loads at least have some available.

\newcount   \scratchcounter   \newcount  \globalscratchcounter
\newdimen   \scratchdimen     \newdimen  \globalscratchdimen
\newskip    \scratchskip      \newskip   \globalscratchskip
\newmuskip  \scratchmuskip    \newmuskip \globalscratchmuskip
\newtoks    \scratchtoks      \newtoks   \globalscratchtoks
\newbox     \scratchbox       \newbox    \globalscratchbox

\newcount\scratchcounterone  \newcount\scratchcountertwo  \newcount\scratchcounterthree
\newdimen  \scratchdimenone  \newdimen  \scratchdimentwo  \newdimen  \scratchdimenthree
\newdimen   \scratchskipone  \newdimen   \scratchskiptwo  \newdimen   \scratchskipthree
\newbox   \scratchmuskipone  \newbox   \scratchmuskiptwo  \newbox   \scratchmuskipthree
\newtoks    \scratchtoksone  \newtoks    \scratchtokstwo  \newtoks    \scratchtoksthree
\newbox      \scratchboxone  \newbox      \scratchboxtwo  \newbox      \scratchboxthree

%D More allocations:

\newskip \zeroskip     \zeroskip     = 0pt plus 0pt minus 0pt
\newdimen\zeropoint    \zeropoint    = 0pt
\newdimen\onepoint     \onepoint     = 1pt
\newdimen\maxdimen     \maxdimen     = 16383.99999pt
\newdimen\onebasepoint \onebasepoint = 1bp
\newdimen\scaledpoint  \scaledpoint  = 1sp
\newdimen\thousandpoint\thousandpoint= 1000pt

\let\points\onepoint

\newtoks \emptytoks

%D And even more: (todo: countdefs 60+)

%newcount   \minusone \minusone =    -1
\newcount   \minustwo \minustwo =    -2
%chardef    \zerocount          =     0
%chardef    \plusone            =     1
\chardef    \plustwo            =     2
\chardef    \plusthree          =     3
\chardef    \plusfour           =     4
\chardef    \plusfive           =     5
\chardef    \plussix            =     6
\chardef    \plusseven          =     7
\chardef    \pluseight          =     8
\chardef    \plusnine           =     9
\chardef    \plusten            =    10
\chardef    \plussixteen        =    16
\chardef    \plushundred        =   100
\chardef    \pluscxxvii         =   127
\chardef    \pluscxxviii        =   128
\chardef    \pluscclv           =   255

\ifnum\texengine=\luatexengine
    \chardef    \pluscclvi          =   256
    \chardef    \plusthousand       =  1000
    \chardef    \plustenthousand    = 10000
    \chardef    \plustwentythousand = 20000
    \chardef    \medcard            = 32768
    \chardef    \maxcard            = 65536 % pdftex has less mathchars
\else
    \mathchardef\pluscclvi          =   256
    \mathchardef\plusthousand       =  1000
    \mathchardef\plustenthousand    = 10000
    \mathchardef\plustwentythousand = 20000
    \newcount   \medcard \medcard   = 32768 % pdftex has less mathchars
    \newcount   \maxcard \maxcard   = 65536 % pdftex has less mathchars
\fi

%D We prefer the more readable variant than in plain
%D \TEX. User should only use \type {\emptybox}:

\newbox\voidbox

\def\emptybox {\box   \voidbox}
\def\unvoidbox{\unhbox\voidbox}

\let\leavevmode\unvoidbox % we prefer to use \dontleavehmode

%D Some expected plain variants follow. We don't reuse registers
%D because we don't want clashes.

\newdimen\p@       \p@   \onepoint
\newcount\m@ne     \m@ne \minusone
\newdimen\z@       \z@   \zeropoint
\let     \@ne            \plusone
\let     \tw@            \plustwo
\let     \thr@@          \plusthree
\let     \sixt@@n        \sixteen
\let     \@cclv          \pluscclv
\let     \@cclvi         \pluscclvi
\newbox  \voidb@x
\newtoks \toks@

%D We define \type {\newif} a la plain \TEX, but will
%D redefine it later. As Knuth says:
%D
%D \startnarrower
%D And here's a different sort of allocation: for example,
%D
%D \starttyping
%D \newif\iffoo
%D \stoptyping
%D
%D creates \type {\footrue}, \type {\foofalse} to go
%D with \type {\iffoo}.
%D \stopnarrower

\def\newif#1%
  {\count@\escapechar
   \escapechar\minusone
   \expandafter\expandafter\expandafter\def\@if #1{true}{\let#1\iftrue }%
   \expandafter\expandafter\expandafter\def\@if#1{false}{\let#1\iffalse}%
   \@if#1{false}% the condition starts out false
   \escapechar\count@}

\def\@if#1#2%
  {\csname\expandafter\if@\string#1#2\endcsname}

\bgroup % `if' is required

  \uccode`1=`i \uccode`2=`f \uppercase{\gdef\if@12{}}

\egroup

%D Let's test this one:

\newif\ifdone

%D \macros
%D   {@@escape,@@begingroup,@@endgroup,@@mathshift,@@alignment,
%D    @@endofline,@@parameter,@@superscript,@@subscript,
%D    @@ignore,@@space,@@letter,@@other,@@active,@@comment}
%D
%D In \CONTEXT\ we sometimes manipulate the \CATCODES\ of certain
%D characters. Because we are not that good at remembering numbers,
%D we introduce some symbolic names.

%D As we now have catc-* files, we also have more readable names but
%D We will keep the following around for a while.

\chardef\@@escape      =  0
\chardef\@@begingroup  =  1
\chardef\@@endgroup    =  2
\chardef\@@mathshift   =  3
\chardef\@@alignment   =  4
\chardef\@@endofline   =  5
\chardef\@@parameter   =  6
\chardef\@@superscript =  7
\chardef\@@subscript   =  8
\chardef\@@ignore      =  9
\chardef\@@space       = 10
\chardef\@@letter      = 11
\chardef\@@other       = 12
\chardef\@@active      = 13
\chardef\@@comment     = 14

%D Constants to be used with \type {\currentgrouptype}.

\chardef\@@bottomlevelgroup   =  0
\chardef\@@simplegroup        =  1
\chardef\@@hboxgroup          =  2
\chardef\@@adjustedhboxgroup  =  3
\chardef\@@vboxgroup          =  4
\chardef\@@vtopgroup          =  5
\chardef\@@aligngroup         =  6
\chardef\@@noaligngroup       =  7
\chardef\@@outputgroup        =  8
\chardef\@@mathgroup          =  9
\chardef\@@discretionarygroup = 10
\chardef\@@insertgroup        = 11
\chardef\@@vcentergroup       = 12
\chardef\@@mathchoicegroup    = 13
\chardef\@@semisimplegroup    = 14
\chardef\@@mathshiftgroup     = 15
\chardef\@@mathleftgroup      = 16

\chardef\@@vadjustgroup       = \@@insertgroup

%D Constants to be used with \type {\interactionmode}.

\chardef\@@batchmode     = 0
\chardef\@@nonstopmode   = 1
\chardef\@@scrollmode    = 2
\chardef\@@errorstopmode = 3

%D Constants to be used with \type {\lastnodetype}. The \type
%D {\lastnodetype} primitive is \ETEX\ compliant. The valid range is
%D still -1 .. 15 and glyph nodes have number 0 (used to be char
%D node) and ligature nodes are mapped to 7. That way macro packages
%D can use the same symbolic names as in traditional \ETEX. Keep in
%D mind that the internal node numbers are different and that there
%D are more node types that 15.

\chardef\@@charnode          =  0
\chardef\@@hlistnode         =  1
\chardef\@@vlistnode         =  2
\chardef\@@rulenode          =  3
\chardef\@@insertnode        =  4
\chardef\@@marknode          =  5
\chardef\@@adjustnode        =  6
\chardef\@@ligaturenode      =  7
\chardef\@@discretionarynode =  8
\chardef\@@whatsitnode       =  9
\chardef\@@mathnode          = 10
\chardef\@@gluenode          = 11
\chardef\@@kernnode          = 12
\chardef\@@penaltynode       = 13
\chardef\@@unsetnode         = 14
\chardef\@@mathsnode         = 15

%D Constants to be used with \type {\currentiftype}.

\chardef\@@charif     =  1
\chardef\@@catif      =  2
\chardef\@@numif      =  3
\chardef\@@dimif      =  4
\chardef\@@oddif      =  5
\chardef\@@vmodeif    =  6
\chardef\@@hmodeif    =  7
\chardef\@@mmodeif    =  8
\chardef\@@innerif    =  9
\chardef\@@voidif     = 10
\chardef\@@hboxif     = 11
\chardef\@@vboxif     = 12
\chardef\@@xif        = 13
\chardef\@@eofif      = 14
\chardef\@@trueif     = 15
\chardef\@@falseif    = 16
\chardef\@@caseif     = 17
\chardef\@@definedif  = 18
\chardef\@@csnameif   = 19
\chardef\@@fontcharif = 20

%D Of course we want even bigger log files, so we copied this
%D from the \ETEX\ source files.
%D
%D When watching such logs, beware of nasty side effects of
%D \type {\scantokens}, as in:
%D
%D \starttyping
%D \bgroup
%D   \lccode`a=12\lowercase{\xdef\whatever{a}}\egroup
%D   \def\whatever{test \whatever test}
%D   \scantokens\expandafter{\whatever}
%D \egroup
%D \stoptyping
%D
%D In \LUATEX\ we have ways around this.

\def\tracingall
  {\tracingonline    \plusone
   \tracingcommands  \plusthree
   \tracingstats     \plustwo
   \tracingpages     \plusone
   \tracingoutput    \plusone
   \tracinglostchars \plustwo
   \tracingmacros    \plustwo
   \tracingparagraphs\plusone
   \tracingrestores  \plusone
   \showboxbreadth   \maxdimen
   \showboxdepth     \maxdimen
   \tracinggroups    \plusone
   \tracingifs       \plusone
   \tracingscantokens\plusone
   \tracingnesting   \plusone
   \tracingassigns   \plustwo
   \errorstopmode}

\def\loggingall
  {\tracingall
   \tracingonline    \zerocount}

\def\tracingnone
  {\tracingassigns   \zerocount
   \tracingnesting   \zerocount
   \tracingscantokens\zerocount
   \tracingifs       \zerocount
   \tracinggroups    \zerocount
   \showboxdepth     \plusthree
   \showboxbreadth   \plusfive
   \tracingrestores  \zerocount
   \tracingparagraphs\zerocount
   \tracingmacros    \zerocount
   \tracinglostchars \plusone
   \tracingoutput    \zerocount
   \tracingpages     \zerocount
   \tracingstats     \zerocount
   \tracingcommands  \zerocount
   \tracingonline    \zerocount}

%D Just for tracing purposes we set:

\tracingstats\plusone

%D Here we also save \type {\input}, more will be saved later.

\ifdefined\normalinput \else \let\normalinput\input \fi

%D We don't like outer commands, and we always want access
%D to the original \type {\input} primitive.

\let\normalouter\outer  \def\outer{} % no longer \relax

%D To circumvent dependencies, we can postpone certain
%D initializations to dumping time, by appending them to the
%D \type {\everydump} token register.

\ifdefined\normaldump \else \let\normaldump\dump \fi

\newtoks\everydump

\def\dump{\the\everydump\normaldump}

%D The same applies for the startup actions.

\ifdefined\normaleveryjob \else \let\normaleveryjob\everyjob \fi

\let\everyjob\relax \newtoks\everyjob

\normaleveryjob{\the\everyjob}

\def\appendtotoks #1{\def\temp{#1}\afterassignment\doappendtotoks \scratchtoks=}
\def\prependtotoks#1{\def\temp{#1}\afterassignment\doprependtotoks\scratchtoks=}

\def\doappendtotoks {\expandafter\expandafter\expandafter{\expandafter\the\expandafter\temp\the\scratchtoks}}
\def\doprependtotoks{\expandafter\expandafter\expandafter{\expandafter\the\expandafter\scratchtoks\the\temp}}

%D \macros
%D   {begcsname}
%D
%D Handy for \ETEX-only usage (avoids making \type {\relax}'s:

\def\begcsname#1\endcsname{\ifcsname#1\endcsname\csname#1\endcsname\fi}

%D Now come a few macros that might be needed in successive loading:

\let\endgraf\par
\let\endline\cr

\def\space{ }
\def\empty{}
\def\null {\hbox{}}

%D The following two might be overloaded later on but some modules need
%D then earlier. These functionality is reflected in the name and will not
%D change.

\bgroup
    \catcode`\^^M=\activecatcode%
    \gdef\obeylines{\catcode`\^^M\activecatcode \let^^M\par}%
    \global\let^^M\par%
\egroup

\bgroup
    \gdef\obeyspaces{\catcode`\ \activecatcode}%
    \obeyspaces\global\let =\space%
\egroup

%D Also needed might be a simple loop structure and we borrow
%D plain \TEX's one as it is often expected to be present and
%D it is about the fastest you can get. Beware: this macro
%D does not support nested loops. We use a namespace prefix
%D \type {@@pln}.

\long\def\loop#1\repeat{\long\def\@@plnbody{#1}\@@plniterate}

%D The following makes \type {\loop} \unknown\ \type {\if}
%D \unknown\ \type {\repeat} skippable (clever trick):

\let\repeat\fi % so both \loop and \repeat are reserved words!

%D The original (no \type {@@pln} there):
%D
%D \starttyping
%D \def\@@plniterate{\@@plnbody\let\next\@@plniterate\else\let\next\relax\fi\next}
%D \stoptyping
%D
%D A more efficient alternative:
%D
%D \starttyping
%D \def\@@plniterate{\@@plnbody\expandafter\@@plniterate\else\expandafter\relax\fi}
%D \stoptyping
%D
%D An even more efficient one:

\def\@@plniterate{\@@plnbody\expandafter\@@plniterate\else\fi}

%D We don't define a real output routine yet but at least get rid
%D of pages:

\output{\shipout\box\normalpagebox}

%D Although we don't add pagenumbers yet we alias the default
%D register used for counting pages:

\countdef\pageno=0 \pageno=1 % first page is number 1

%D Beside the raw counter \type {\pageno} the \type {\folio}
%D macro provides the value.

\def\folio{\the\pageno} % kind of expected and therefore reserved

%D The following registers are kind of standard and (for the moment)
%D we define them here. This might change.

\newskip  \bigskipamount          \bigskipamount       = 12pt plus 4pt minus 4pt
\newskip  \medskipamount          \medskipamount       =  6pt plus 2pt minus 2pt
\newskip  \smallskipamount        \smallskipamount     =  3pt plus 1pt minus 1pt

\baselineskip  = 12pt
\lineskip      =  1pt
\lineskiplimit =  0pt

%D Again a few kind-of-extensions the core:

\newskip \hideskip   \hideskip  = -1000pt plus 1fill
\newskip \centering  \centering = 0pt plus 1000pt minus 1000pt

\def\hidewidth % for alignment entries that can stick out
  {\hskip\hideskip}

\def\ialign % initialized \halign
  {\everycr{}%
   \tabskip\zeroskip
   \halign}

\newcount \mscount

\def\spanomit{\span\omit} % bypass error message

\def\multispan#1%
  {\omit
   \mscount#1\relax
   \loop
     \ifnum\mscount>\plusone
       \spanomit \advance\mscount\minusone
   \repeat}

%D The next section deals with selective definitions in
%D later modules. One can of course use the \type {\texengine}
%D number that we defined earlier instead.

\bgroup \obeylines
    \gdef\pickupSOMETEX#1%
      {\expandafter\gdef\csname begin#1\endcsname{\bgroup\obeylines\dopickupSOMETEX{#1}}}
    \gdef\dopickupSOMETEX#1#2
    % {\egroup\immediate\write16{special code for #1 -> [line \the\inputlineno] \detokenize{#2}}}
      {\egroup}
\egroup

\let\endTEX   \relax \long\def\beginTEX   #1\endTEX   {}
\let\endETEX  \relax \long\def\beginETEX  #1\endETEX  {}
\let\endXETEX \relax \long\def\beginXETEX #1\endXETEX {}
\let\endLUATEX\relax \long\def\beginLUATEX#1\endLUATEX{}
\let\endOLDTEX\relax \long\def\beginOLDTEX#1\endOLDTEX{}
\let\endNEWTEX\relax \long\def\beginNEWTEX#1\endNEWTEX{}

\pickupSOMETEX{ETEX}

\ifnum\texengine=\xetexengine
    \pickupSOMETEX{XETEX}
\fi
\ifnum\texengine=\luatexengine
    \pickupSOMETEX{LUATEX}
\fi
\ifnum\texengine<\xetexengine
    \pickupSOMETEX{OLDTEX}
\else
    \pickupSOMETEX{NEWTEX}
\fi

%D \macros
%D    {bindprimitive}
%D
%D We can remap primitives (which is needed because of changes in
%D for instance \PDFTEX).

\def\bindprimitive#1 #2 % new old
  {\ifcsname#1\endcsname \else \ifcsname#2\endcsname
     \expandafter\let\csname#1\expandafter\endcsname\csname#2\endcsname
   \fi \fi}

%D Because \XETEX\ also implements some \PDFTEX\ functionality, we take
%D care of this here instead of a dedicated module. Later modules need
%D to handle the undefined cases.

%D These messy checks will disappear.

% new after 1.10, watch the change in prefix

\bindprimitive quitvmode      ptexquitvmode
\bindprimitive noligatures    ptexnoligatures
\bindprimitive setrandomseed  ptexsetrandomseed
\bindprimitive uniformdeviate ptexuniformdeviate

\bindprimitive quitvmode      pdfquitvmode
\bindprimitive noligatures    pdfnoligatures
\bindprimitive setrandomseed  pdfsetrandomseed
\bindprimitive uniformdeviate pdfuniformdeviate

\bindprimitive resettimer     pdfresettimer
\bindprimitive elapsedtime    pdfelapsedtime

% new per 1.40

\bindprimitive ifprimitive    ifpdfprimitive
\bindprimitive primitive      pdfprimitive
\bindprimitive ifabsdim       ifpdfabsdim
\bindprimitive ifabsnum       ifpdfabsnum

\ifnum\texengine=\xetexengine \else % this test might disappear some day

    \pdfminorversion \plusfive

    \ifdefined\pdfcompresslevel     \else \newcount\pdfcompresslevel      \fi
    \ifdefined\pdfobjcompresslevel  \else \newcount\pdfobjcompresslevel   \fi
    \ifdefined\pdfgentounicode      \else \newcount\pdfgentounicode       \fi \pdfgentounicode      \plusone
    \ifdefined\pdfinclusioncopyfonts\else \newcount\pdfinclusioncopyfonts \fi \pdfinclusioncopyfonts\plusone

    \def\nopdfcompression     {\pdfobjcompresslevel\zerocount \pdfcompresslevel\zerocount}
    \def\maximumpdfcompression{\pdfobjcompresslevel\plusnine  \pdfcompresslevel\plusnine }
    \def\normalpdfcompression {\pdfobjcompresslevel\plusthree \pdfcompresslevel\plusthree}

    \normalpdfcompression

    \let\normalsetrandomseed \setrandomseed
    \let\normaluniformdeviate\uniformdeviate

\fi

%D Handy.

\ifnum\texengine=\luatexengine
    \ifdefined\suppresslongerror % for the moment test
        \suppresslongerror\plusone
    \fi
\fi

%D Basic status stuff.

\newif\ifproductionrun

%D We need to make sure that we start up in \DVI\ mode, so,
%D after testing for running \PDFTEX, we default to \DVI.

\ifx\pdftexversion\undefined \newcount\pdfoutput \fi \pdfoutput=0

%D For those who expect this \unknown

\ifx\fmtname   \undefined \def\fmtname   {ConTeXt Minimized Plain TeX} \fi
\ifx\fmtversion\undefined \def\fmtversion{3.1415926}                   \fi

\let\normalfmtversion\fmtversion

%D A few bonus macros:

\def\modulonumber#1#2{\the\numexpr#2-((((#2+(#1/2))/#1)-1)*#1)\relax}
\def\dividonumber#1#2{\the\numexpr(#2-(#1/2))/#1\relax}

\ifnum\texengine=\xetexengine
    \edef\xetexversion   {\numexpr\XeTeXversion*100+(\expandafter\gobbleoneargument\XeTeXrevision-5)/10\relax}
    \edef\xetexrevision  {\the\numexpr(\expandafter\gobbleoneargument\XeTeXrevision-50)/100\relax}
\fi

\ifcase\texengine
    \def\texenginename    {impossible}
    \edef\texengineversion{0}
\or
    \def\texenginename    {pdfTeX}
    \edef\texengineversion{\dividonumber{100}\pdftexversion.\modulonumber{100}\pdftexversion.\pdftexrevision}
\or
    \def\texenginename    {XeTeX}
    \edef\texengineversion{\dividonumber{100}\xetexversion .\modulonumber{100}\xetexversion .\xetexrevision}
\or
    \def\texenginename    {LuaTeX}
    \edef\texengineversion{\dividonumber{100}\luatexversion.\modulonumber{100}\luatexversion.\luatexrevision}
\else
    \def\texenginename    {impossible}
    \edef\texengineversion{0}
\fi

%D While cleaning this code up a bit I was listening to Heather
%D Nova's \CD\ Redbird. The first song on that \CD\ ends with
%D a few lines suitable for ending this initialization module:
%D
%D \startlines
%D And there's so much I can do for you
%D Given time I know that I can prove
%D Now my world is opened up to you
%D Come inside
%D
%D Welcome to my life
%D Welcome to my world
%D Come inside
%D \stoplines
%D
%D So let's see what \TEX\ can do now that we've opened up
%D the basic machinery.

\protect \endinput
