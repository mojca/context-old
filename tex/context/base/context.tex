%D \module
%D   [       file=context,
%D        version=1995.10.10,
%D          title=\CONTEXT,
%D       subtitle=\CONTEXT\ Format Generation,
%D         author=Hans Hagen,
%D           date=\currentdate,
%D      copyright={PRAGMA / Hans Hagen \& Ton Otten}]
%C
%C This module is part of the \CONTEXT\ macro||package and is
%C therefore copyrighted by \PRAGMA. See mreadme.pdf for
%C details.

\catcode`\{=1 \catcode`\}=2 \catcode`\#=6

%D From the next string (which is set by the script that assembles the
%D distribution) later on we will calculate a number that can be used
%D by use modules to identify the feature level. Starting with version
%D 2004.8.30 the low level interface is english. Watch out and adapt
%D your styles an modules.

\edef\contextformat {\jobname}
\edef\contextversion{2009.07.17 13:16}

%D For those who want to use this:

\let\fmtname   \contextformat
\let\fmtversion\contextversion

\let\showcontextbanner\relax

%D Here we will test for what branch of source files we will take. The
%D file suffix depends on the maintream engine: either lua enhanced,
%D or traditional:
%D
%D \starttyping
%D mki  : low level dutch
%D mkii : low level english
%D mkiv : lua enhanced
%D \stoptyping
%D
%D There is no \type {mkiii} unless you want to tag the transition
%D version that way (going completely etex, code cleanup and such).

\ifx\normalinput\undefined \let\normalinput\input \fi

%D We have two versions, one for \PDFTEX\ and \XETEX, and one for
%D \LUATEX.

\chardef\contextmarkmode = \ifx\directlua\undefined 2 \else 4 \fi

\def\loadcorefile#1{\normalinput#1\relax}
\def\loadmarkfile#1{\normalinput#1.\mksuffix\relax}

\ifnum\contextmarkmode=4
  \def\loadmarkiifile#1{} \let\loadmarkivfile\loadcorefile \edef\mksuffix {mkiv} \edef\contextmark{MKIV}
\else
  \def\loadmarkivfile#1{} \let\loadmarkiifile\loadcorefile \edef\mksuffix {mkii} \edef\contextmark{MKII}
\fi

\let\loadmkiifile\loadmarkiifile
\let\loadmkivfile\loadmarkivfile
\let\loadmkfile  \loadmarkfile

%D Welcome to the main module. When this module is ran through
%D \type{initex} or \type {tex -i} or \type {whatevertex} using
%D \type{whatever switch}, the \CONTEXT\ format file is
%D generated. During this process the user is asked for an
%D interface language. Supplying \type{dutch} will generate a
%D dutch version of \CONTEXT, supplying \type {english} will of
%D course end op in a english version.
%D
%D Another option is to use \TEXEXEC\ (for \MKII):
%D
%D \starttyping
%D texexec --make [--alone] [--engine] [--all]
%D texexec --make [--alone] [--engine] en nl ... metafun mptopdf
%D \stoptyping
%D
%D More information can be found in the \TEXEXEC\ manual.
%D
%D For \MKIV\ you have to do:
%D
%D \starttyping
%D luatools --generate
%D context --make en nl ...
%D \stoptyping

%D When you write modules (or extensions) you should avoid
%D conflicts with existing macro names and mechanisms. If you are
%D coming from another macro package, don't assume that macros
%D with the same name are doing the same! \CONTEXT\ was written
%D from scratch and therefore similarities are often a coincidence
%D (to some extent one always ends up with the same names for
%D similar concepts). The underlying models for elementary subsystems
%D that deal with encodings, languages and fonts probably differ.
%D
%D Time has learned that users coming from \LATEX\ should not fall
%D into thinking that macros like \type {\protect} represent the
%D same functionality in both systems; actually, due to the way
%D \CONTEXT\ is set up, some of those macros do complete different
%D things. Macro packages evolve independent of each other, which
%D means that code written for one system will not work in another
%D system, unless it's real generic code.
%D
%D An API will become available soon (keep an eye on the ConTeXt
%D Wiki www.contextgarden.org) and or listen in to the context
%D mailing list (ntg-context@ntg.nl). Much additional information
%D can be found at the PRAGMA website (www.pragma-ade.com).

%D We now go either the \MKII\ or \MKIV\ route:

\loadmarkfile{context}

%D Let's quit this file when doing a \type {cont-..} generation.

\doifparentfileelse{context}{\donothing}{\endinput}

%D Unless we're generating a \type {cont-..} format, we also
%D do the following.

%D Except from english, no hyphenation patterns are loaded
%D yet. Users can specify their needs in the next module:

\loaduserspecifications

%D Next we default to the same language as the interface.

\unprotect

\installlanguage [\s!us] [\c!state=\v!start]

\startinterface english
  \installlanguage [\s!gb] [\c!state=\v!start]
\stopinterface

\appendtoks \language     [\s!us] \to \everyjob
\appendtoks \mainlanguage [\s!us] \to \everyjob

\startinterface german

  \installlanguage [\s!de] [\c!state=\v!start]

  \appendtoks \language     [\s!de] \to \everyjob
  \appendtoks \mainlanguage [\s!de] \to \everyjob

\stopinterface

\startinterface dutch

  \installlanguage [\s!nl] [\c!state=\v!start]

  \appendtoks \language     [\s!nl] \to \everyjob
  \appendtoks \mainlanguage [\s!nl] \to \everyjob

\stopinterface

\startinterface czech

  \installlanguage [\s!cs] [\c!state=\v!start]

  \appendtoks \language     [\s!cs] \to \everyjob
  \appendtoks \mainlanguage [\s!cs] \to \everyjob

\stopinterface

\startinterface italian

  \installlanguage [\s!it] [\c!state=\v!start]

  \appendtoks \language     [\s!it] \to \everyjob
  \appendtoks \mainlanguage [\s!it] \to \everyjob

\stopinterface

\startinterface romanian

  \installlanguage [\s!ro] [\c!state=\v!start]

  \appendtoks \language     [\s!ro] \to \everyjob
  \appendtoks \mainlanguage [\s!ro] \to \everyjob

\stopinterface

\startinterface french

  \installlanguage [\s!fr] [\c!state=\v!start]

  \appendtoks \language     [\s!fr] \to \everyjob
  \appendtoks \mainlanguage [\s!fr] \to \everyjob

\stopinterface

\startinterface persian

  \installlanguage [\s!pe] [\c!state=\v!start]

  \appendtoks \language     [\s!pe] \to \everyjob
  \appendtoks \mainlanguage [\s!pe] \to \everyjob

\stopinterface

\protect

%D Finally we (pre)load some fonts.

\setupencoding[default=ec]

\usetypescript[fallback][\defaultencoding]

\setupbodyfont[rm,12pt]

%D The next hook can be used to generate a local (extended)
%D format. This file is only searched for at the current
%D path.

% \readlocfile{cont-def.tex}
%   {\writestatus{loading}{adding extensions from cont-def}}
%   {}

%D Now dumping the format is all that's left to be done.

\errorstopmode \dump

\endinput
