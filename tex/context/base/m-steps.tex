%D \module
%D   [       file=m-steps,
%D        version=2001.05.28,
%D          title=\CONTEXT\ Modules,
%D       subtitle=Step Charts \& Tables,
%D         author=Hans Hagen,
%D           date=\currentdate,
%D      copyright={PRAGMA ADE \& \CONTEXT\ Development Team}]
%C
%C This module is part of the \CONTEXT\ macro||package and is
%C therefore copyrighted by \PRAGMA. See mreadme.pdf for
%C details.

%D When I need this module, I will reimplement it by using the new
%D sometxt macro. Anyhow, it reflects the state of 2001.

\unprotect

% temp hack :

% \ifx\v!kleinkorps\undefined \let\v!kleinkorps\setsmallbodyfont \fi

% end of hack

\definecolor [STEPlinecolor]       [s=.5]
\definecolor [STEPframecolor]      [s=.7]
\definecolor [STEPbackgroundcolor] [s=.9]

\def\@@STPF{@@STPF} % frames

\def\@@STPC{@@STPC} % charts
\def\@@STPT{@@STPT} % tables

\def\@@STEC{@@STEC} % cells
\def\@@STET{@@STET} % tables
\def\@@STEL{@@STEL} % lines

\def\setupSTEPcharts{\dodoubleargument\getparameters[\@@STPC]}
\def\setupSTEPtables{\dodoubleargument\getparameters[\@@STPT]}
\def\setupSTEPcells {\dodoubleargument\getparameters[\@@STEC]}
\def\setupSTEPtexts {\dodoubleargument\getparameters[\@@STET]}
\def\setupSTEPlines {\dodoubleargument\getparameters[\@@STEL]}

\setupSTEPcharts
  [\c!before=\blank,
   \c!after=\blank,
  %\c!distance=.25em, % nvt
   \c!hoffset=1em,
   \c!voffset=1ex,
   \c!method=1,
   \c!height=2ex,
   \c!offset=.15\bodyfontsize]

\setupSTEPtables
  [\c!before=\blank,
   \c!after=\blank,
   \c!distance=.25em,
  %\c!hoffset=1em, % nvt
   \c!voffset=1ex,
   \c!method=1,
   \c!width=4em,
   \c!offset=.15\bodyfontsize]

\setupSTEPcells
  [\c!alternative=24,
   \c!background=\v!color,
   \c!backgroundcolor=STEPbackgroundcolor,
   \c!rulethickness=.1\bodyfontsize,
   \c!framecolor=STEPframecolor,
   \c!offset=.25\bodyfontsize,
   \c!style=,
   \c!color=]

\setupSTEPtexts
  [\c!alternative=24,
   \c!background=\v!color,
   \c!backgroundcolor=STEPbackgroundcolor,
   \c!rulethickness=.1\bodyfontsize,
   \c!framecolor=STEPframecolor,
   \c!offset=.25\bodyfontsize,
   \c!style=\v!smallbodyfont,
   \c!color=]

\setupSTEPlines
  [\c!alternative=1,
   \c!rulethickness=.15\bodyfontsize,
   \c!color=STEPlinecolor]

\def\initializeSTEP
  {\initializeSTET \initializeSTEC \initializeSTEL}

\def\initializeSTPC
  {\freezedimenmacro\@@STPCoffset
   \startMPdrawing
     line_v_offset := \@@STPCoffset  ;
     line_method   := \@@STPCmethod ; % only charts
   \stopMPdrawing}

\def\initializeSTPT
  {\freezedimenmacro\@@STPToffset
   \startMPdrawing
     line_h_offset := \@@STPToffset  ;
     line_method   := \@@STPTmethod ; % only charts
   \stopMPdrawing}

\def\initializeSTET
  {\freezedimenmacro\@@STETrulethickness
   \freezedimenmacro\@@STEToffset}

\def\initializeSTEC
  {\freezedimenmacro\@@STECrulethickness
   \freezedimenmacro\@@STECoffset}

\def\initializeSTEL
  {\freezedimenmacro\@@STELrulethickness}

%D ...

\presetlocalframed[\@@STPF]

\def\@@stepcell#1%
  {\doattributes\@@STEC\c!style\c!color
     {\localframed
         [\@@STPF][\c!offset=\@@STECoffset,\c!frame=\v!off]
         {\ignorespaces#1\unskip}}}

\def\@@stepfake#1%
  {\doattributes\@@STEC\c!style\c!color
     {\ignorespaces#1\unskip}}

\def\@@steptext#1%
  {\doattributes\@@STET\c!style\c!color
     {\localframed
         [\@@STPF][\c!offset=\@@STEToffset,\c!frame=\v!off]
         {\ignorespaces#1\unskip}}}

%D The first attempt was purely \METAPOST\ based and spawned
%D the typesetting to the \METAFUN\ handler. This method
%D collects the cells, and directly passes them on to
%D \METAPOST. This method is the cleanest, but has the
%D disadvantage that one cannot embed hyperlinks or document
%D dependent definitions in the cells. The implementation
%D roughly looks as follows:
%D
%D \starttyping
%D \def\startSTEPchart%
%D   {\bgroup
%D    \startMPdrawing
%D      input mp-step ; begin_step_chart ;
%D    \stopMPdrawing
%D    \initializeSTEP
%D    \let\cells\stepchartcells \def\cell{\cells{}}%
%D    \let\texts\stepcharttexts \def\text{\texts{}}}
%D
%D \def\stepchartcells#1#2%
%D   {\setMPtext{tdummy}{\strut\ignorespaces#1\unskip}% beter etex/btex
%D    \setMPtext{bdummy}{\strut\ignorespaces#2\unskip}% beter etex/btex
%D    \startMPdrawing
%D      set_step_chart_cells(\MPstring{tdummy},\MPstring{bdummy}) ;
%D    \stopMPdrawing}
%D
%D \def\stepcharttexts#1#2%
%D   {\setMPtext{tdummy}{\strut\ignorespaces#1\unskip}% beter etex/btex
%D    \setMPtext{bdummy}{\strut\ignorespaces#2\unskip}% beter etex/btex
%D    \startMPdrawing
%D      set_step_chart_texts(\MPstring{tdummy},\MPstring{bdummy}) ;
%D    \stopMPdrawing}
%D
%D \def\stopSTEPchart
%D   {\startMPdrawing
%D      end_step_chart ;
%D    \stopMPdrawing
%D    \MPdrawingdonetrue
%D    \getMPdrawing
%D    \resetMPdrawing
%D    \egroup}
%D \stoptyping
%D
%D This method has the advantage that it does the job in
%D (virtually) one pass, while the next methods need multiple
%D passes: one to build the table, another to synchronize the
%D positions, and a third one beause the dimensions may have
%D changed. The last pass is a result from the fact that
%D positions are related to the page.
%D
%D The second attempt was based on tabulations and used the
%D build in position tracking mechanism, which uses two
%D position nodes per cell.
%D
%D This method collects the content in token list registers
%D and build a table from them. In the collecting pass, the
%D graphics are build stepwise. We need to collect because the
%D order of definitions is not the same as the order of
%D typesetting. We show this alternative too because it
%D demonstrates how to apply backgrounds to table cells.
%D
%D \starttyping
%D \newtoks\stepsonetop \newtoks\stepstwotop
%D \newtoks\stepsonebot \newtoks\stepstwobot
%D \stoptyping
%D
%D During the collecting phase, we temporarily have to
%D increment the name space counter.
%D
%D \starttyping
%D \def\startSTEPchart%
%D   {\bgroup
%D    \resetMPdrawing
%D    \advance\noftabpositions\plusone % begin of preroll
%D    \startMPdrawing
%D      input mp-step ;
%D      begin_step_chart ;
%D    \stopMPdrawing
%D    \initializeSTEP
%D    \newcounter\cellcounter
%D    \stepsonetop\emptytoks \chardef\somesteponetop=1
%D    \stepsonebot\emptytoks \chardef\somesteponebot=1
%D    \stepstwotop\emptytoks \chardef\somesteptwotop=1
%D    \stepstwobot\emptytoks \chardef\somesteptwobot=1
%D    \let\cells\stepchartcells \def\cell{\cells{}}%
%D    \let\texts\stepcharttexts \def\text{\texts{}}}
%D \stoptyping
%D
%D Now we collect the steps and texts, and in the process the
%D graphic is built. Then we continue with building the table.
%D
%D Watch how we anchor the graphic to the main table box. This
%D is needed since the graphic may be larger than the table
%D itself. Actually, these small point took me the most time to
%D digest, even with the right tools (anchors) already in
%D place.
%D
%D \starttyping
%D \def\stopSTEPchart
%D   {\splittabulatefalse
%D    \insidefloattrue
%D    \startMPdrawing
%D      nofcells := \cellcounter ;
%D      end_step_chart ;
%D      anchor_box(\MPpos{\tbPOSprefix origin}) ;
%D    \stopMPdrawing
%D    \MPdrawingdonetrue
%D    \advance\noftabpositions\minusone % end of preroll
%D    \setbox0=\vbox
%D      {\getMPdrawing}
%D    \resetMPdrawing
%D    \increment(\cellcounter,\cellcounter)\decrement\cellcounter
%D    \setbox2=\vbox
%D      {\definetabulate[chart][|*{\cellcounter}{ck0|}] % k0 nills space
%D       \startchart
%D       \ifcase\somesteptwotop \the\stepstwotop \NC \NR \noalign{\kern2ex} \fi
%D       \ifcase\somesteponetop \the\stepsonetop \NC \NR \noalign{\kern2ex} \fi
%D       \ifcase\somesteponebot \the\stepsonebot \NC \NR \noalign{\kern2ex} \fi
%D       \ifcase\somesteptwobot \the\stepstwobot \NC \NR \noalign{\kern2ex} \fi
%D       \noalign{\kern-2ex}%
%D       \stopchart}
%D    \hbox
%D      {\scratchdimen\wd0
%D       \advance\scratchdimen \MPllx bp
%D       \raise\MPlly bp\box0
%D       \hskip-\scratchdimen
%D       \hpos{\tbPOSprefix origin}{\box2}}
%D    \egroup}
%D \stoptyping
%D
%D The steps and texts fill the (at most 4) lines that make up
%D the table. We also feed the (automatically registerd) cell
%D dimensions to the graphic backend.
%D
%D \starttyping
%D \newcounter\cellcounter
%D \newcounter\textcounter
%D
%D \def\stepchartcells#1#2%
%D   {\doloop
%D     {\ifnum\cellcounter>\textcounter
%D        \stepcharttexts{}{}%
%D      \else
%D        \exitloop
%D      \fi}%
%D    \increment\cellcounter
%D    \doifelsenothing{#1}
%D      {\startMPdrawing
%D         cells[t][\cellcounter] := origin ;
%D       \stopMPdrawing
%D       \appendtoks\NC\NC\to\stepsonetop}
%D      {\chardef\somesteponetop=0
%D       \edef\stepidentifier{\cellcounter-t-c}%
%D       \startMPdrawing
%D         initialize_area(\MPpos{b:\tbPOSprefix\stepidentifier},
%D                         \MPpos{e:\tbPOSprefix\stepidentifier}) ;
%D         cells[t][\cellcounter] := pxy ;
%D       \stopMPdrawing
%D       \@EA\appendtoks\@EA\stepidentifierposition\@EA{\stepidentifier}#1\NC\to\stepsonetop}%
%D    \doifelsenothing{#2}
%D      {\startMPdrawing
%D         cells[b][\cellcounter] := origin ;
%D       \stopMPdrawing
%D       \appendtoks\NC\NC\to\stepsonebot}
%D      {\chardef\somesteponebot=0
%D       \edef\stepidentifier{\cellcounter-b-c}%
%D       \startMPdrawing
%D         initialize_area(\MPpos{b:\tbPOSprefix\stepidentifier},
%D                         \MPpos{e:\tbPOSprefix\stepidentifier}) ;
%D         cells[b][\cellcounter] := pxy ;
%D       \stopMPdrawing
%D       \@EA\appendtoks\@EA\stepidentifierposition\@EA{\stepidentifier}#2\NC\to\stepsonebot}}
%D
%D \def\stepcharttexts#1#2% \cellcounter = nofcells
%D   {\increment\textcounter
%D    \doifelsenothing{#1}
%D      {\startMPdrawing
%D         texts[t][\cellcounter][\textcounter] := origin ;
%D       \stopMPdrawing
%D       \appendtoks\NC\NC\to\stepstwotop}
%D      {\chardef\somesteptwotop=0
%D       \edef\stepidentifier{\cellcounter-\textcounter-t-t}%
%D       \startMPdrawing
%D         initialize_area(\MPpos{b:\tbPOSprefix\stepidentifier},
%D                         \MPpos{e:\tbPOSprefix\stepidentifier}) ;
%D         texts[t][\cellcounter][\textcounter] := pxy ;
%D       \stopMPdrawing
%D       \@EA\appendtoks\@EA\NC\@EA\textcellposition\@EA{\stepidentifier}#1\to\stepstwotop}%
%D    \doifelsenothing{#2}
%D      {\startMPdrawing
%D         texts[b][\cellcounter][\textcounter] := origin ;
%D       \stopMPdrawing
%D       \appendtoks\NC\NC\to\stepstwobot}
%D      {\chardef\somesteptwobot=0
%D       \edef\stepidentifier{\cellcounter-\textcounter-b-t}%
%D       \startMPdrawing
%D         initialize_area(\MPpos{b:\tbPOSprefix\stepidentifier},
%D                         \MPpos{e:\tbPOSprefix\stepidentifier}) ;
%D         texts[b][\cellcounter][\textcounter] := pxy ;
%D       \stopMPdrawing
%D       \@EA\appendtoks\@EA\NC\@EA\textcellposition\@EA{\stepidentifier}#2\to\stepstwobot}}
%D \stoptyping
%D
%D Here are the hooks that take care of calculating the cell
%D dimensions.
%D
%D \starttyping
%D \def\textcellposition#1{\GSC[#1:text]}
%D \def\stepidentifierposition#1{\GSC[#1:step]}
%D \stoptyping
%D
%D We abandoned this method after some testing and went for
%D a third one. It was this third method that evolved into the
%D current mechanism.
%D
%D Since this method was not that efficient, a third one was
%D implemented, which used one position per cell. So,
%D
%D \blank {\bf Here starts the real implementation!} \blank
%D
%D Because we want to build one graphic only we need to store
%D the graphic directives. We also need to collect the cells,
%D which are not defined in the order they show up. This
%D solution uses multiple passes over the definitions. First
%D the cells and texts are processed and the associated
%D graphics are defined in the \METAPOST\ file. Next the
%D lines are flushed. We need to do that in a second pass,
%D because in order to determine the nature of the line,
%D \METAPOST\ needs to know if the start and end cells exist.
%D This need comes from the fact that we store the shapes
%D and lines kind of directly with their associated colors and
%D types, so that we can change the settings in between. So,
%D changing for instance the line color, can take place
%D locally.

\newbox\stepboxone \newbox\stepboxtwo
\newbox\textboxone \newbox\textboxtwo

%D We need to define a dedicated name space counter.

\newcounter\currentstepchart

\def\stepchartprefix{@sc@-\currentstepchart-}

%D Next we define the initialization part of the macros.

\newcounter\cellcounter
\newcounter\textcounter

\def\startSTEPchart
  {\dosingleempty\dostartSTEPchart}

\long\def\dostartSTEPchart[#1]#2\stopSTEPchart
  {\ifinsidefloat
   \else
     \whitespace
     \@@STPCbefore
     \startbaselinecorrection
     \setlocalhsize
     \noindent
   \fi
   \vbox\bgroup
   \setupSTEPcharts[#1]%
   \forgetall
   \pushMPdrawing
   \resetMPdrawing
   \doglobal\increment\currentstepchart
   \startMPdrawing
     input mp-step ;
     begin_step_chart ;
   \stopMPdrawing
   \initializeSTEP
   \initializeSTPC
   \global\chardef\somestepboxone\plusone
   \global\chardef\sometextboxone\plusone
   \global\chardef\somestepboxtwo\somestepboxone
   \global\chardef\sometextboxtwo\sometextboxone
   \def\startlines{\bgroup\setupSTEPlines}%
   \def\stoplines {\egroup}%
   \def\cells{\dosingleempty\dostepchartcells}
   \def\texts{\dosingleempty\dostepcharttexts}
   \def\cell {\dosingleempty\docell}%
   \def\text {\dosingleempty\dotext}%
   \def\docell[##1]{\dostepchartcells[##1]{}}%
   \def\dotext[##1]{\dostepcharttexts[##1]{}}
   \doglobal\newcounter\cellcounter
   \doglobal\newcounter\textcounter
   \let\dostepchartcells\doSTEPchartcellsA
   \let\dostepcharttexts\doSTEPcharttextsA
   {#2} % pass one: cells and texts {} keeps setting local
   \startMPdrawing
     nofcells := \cellcounter ;
     analyze_step_chart ;
   \stopMPdrawing
   \doglobal\newcounter\cellcounter
   \doglobal\newcounter\textcounter
   \let\dostepchartcells\doSTEPchartcellsB
   \let\dostepcharttexts\doSTEPcharttextsB
   {#2} % pass two: lines
   \startMPdrawing
     end_step_chart ;
     % if box_found(\MPpos{\stepchartprefix origin}) :
     %   initialize_box(\MPpos{\stepchartprefix origin}) ;
     %   draw pxy ;
     % fi ;
     anchor_box(\MPpos{\stepchartprefix origin}) ;
   \stopMPdrawing
   \MPdrawingdonetrue
   \doifelse\@@STPCmethod{0}
     {\setbox0\null}
     {\setbox0\vbox{\MPstaticgraphictrue\getMPdrawing}}%
   \resetMPdrawing
   \setbox2\vbox
     {\offinterlineskip
      \scratchdimen\@@STPCheight
      \advance\scratchdimen\@@STPCoffset
      \advance\scratchdimen\@@STPCoffset
      \ifcase\sometextboxone \box\textboxone \vskip\scratchdimen  \fi
      \ifcase\somestepboxone \box\stepboxone \vskip\@@STPCvoffset \fi
      \ifcase\somestepboxtwo \box\stepboxtwo \vskip\scratchdimen  \fi
      \ifcase\sometextboxtwo \box\textboxtwo \vskip\@@STPCvoffset \fi
      \global\setbox\stepboxone\emptybox \global\setbox\stepboxtwo\emptybox % needed indeed
      \global\setbox\textboxone\emptybox \global\setbox\textboxtwo\emptybox % needed indeed
     %\kern-\scratchdimen % no, instead:
      \vskip-\lastskip}
   \hbox
     {\scratchdimen\wd0
      \advance\scratchdimen \MPllx bp
      \raise\MPlly bp\box0
      \hskip-\scratchdimen
      \hpos{\stepchartprefix origin}{\box2}}%
   \popMPdrawing
   \egroup
   \ifinsidefloat \else \stopbaselinecorrection \@@STPCafter \fi}

%D The next macro looks more complicated than it is. We collect
%D the cells in boxes. Before adding a new step cell, we padd
%D the text rows. After adding the step cells, we flush text
%D cells that are defined but not yet processed.

\def\doSTEPchartcellsA[#1]#2#3%
  {% synchronize texts
   \doSTEPchartcellsAB[#1]{#2}{#3}%
   % package steps
   \setbox0\hbox{\doifsomething{#2}{\@@stepcell{#2}}}%
   \setbox2\hbox{\doifsomething{#3}{\@@stepcell{#3}}}%
   \ifdim\wd0>\zeropoint \!!doneafalse \else \!!doneatrue \fi
   \ifdim\wd2>\zeropoint \!!donebfalse \else \!!donebtrue \fi
   \ifdim\wd0>\wd2
     \setbox2\hbox to \wd0{\hss\box2\hss}%
   \else
     \setbox0\hbox to \wd2{\hss\box0\hss}%
   \fi
   \if!!donea
     \startMPdrawing
       cells[t][\cellcounter] := nullpicture ;
     \stopMPdrawing
   \else
     \global\chardef\somestepboxone\zerocount
     \edef\stepidentifier{\stepchartprefix\cellcounter-t-c}%
     \setbox0\hbox{\hpos{\stepidentifier}{\box0}}%
     \bgroup
     \iffirstargument\setupSTEPcells[#1]\fi\initializeSTEC
     \startMPdrawing
       initialize_box(\MPpos{\stepidentifier}) ;
       cells[t][\cellcounter] := \MPcellsgraphic ;
     \stopMPdrawing
     \egroup
   \fi
   \if!!doneb
     \startMPdrawing
       cells[b][\cellcounter] := nullpicture ;
     \stopMPdrawing
   \else
     \global\chardef\somestepboxtwo\zerocount
     \edef\stepidentifier{\stepchartprefix\cellcounter-b-c}%
     \setbox2\hbox{\hpos{\stepidentifier}{\box2}}%
     \bgroup
     \iffirstargument\setupSTEPcells[#1]\fi\initializeSTEC
     \startMPdrawing
       initialize_box(\MPpos{\stepidentifier}) ;
       cells[b][\cellcounter] := \MPcellsgraphic ;
     \stopMPdrawing
     \egroup
   \fi
   \global\setbox\stepboxone\hbox
     {\ifdim\wd\stepboxone>\zeropoint
        \box\stepboxone\hskip\@@STPChoffset\else
      \fi\box0}%
   \global\setbox\stepboxtwo\hbox
     {\ifdim\wd\stepboxtwo>\zeropoint
        \box\stepboxtwo\hskip\@@STPChoffset\else
      \fi\box2}%
   % flush saved texts
   \doSTEPchartcellsBA}

\def\doSTEPchartcellsB[#1]#2#3%
  {\doSTEPchartcellsAB[#1]{#2}{#3}%
   \doSTEPchartcellsBA}

\def\doSTEPchartcellsAB[#1]#2#3%
  {\doloop
     {\ifnum\cellcounter>\textcounter
        \texts{}{}\else\exitloop
      \fi}%
   \doglobal\increment\cellcounter}

\def\doSTEPchartcellsBA
  {\scratchtoks\stepchartbuffer
   \stepchartbuffer\emptytoks
   \the\scratchtoks}

\def\MPcellsgraphic
  {image ( drawshape (
     \@@STECalternative, pxy enlarged (-.5*\@@STECoffset),
     \@@STECrulethickness, \MPcolor{\@@STECframecolor},
     \MPcolor{\@@STECbackgroundcolor} ) )}

%D Although each step can have only one associated text, the
%D place where the text is defined determines the starting
%D point of the connecting arrow. Although several methods are
%D possible, we've chosen a funny collector that flushes one
%D step text at a time.

\newtoks\stepchartbuffer

\def\doSTEPcharttextsA[#1]% #2 #3
  {\dodoSTEPcharttextsA{\cellcounter}{#1}}

\def\dodoSTEPcharttextsA#1#2#3#4% #1=number #2=setup
  {\dodoSTEPcharttextsAB{#1}{#2}{#3}{#4}\dodoSTEPcharttextsA
   \ifnum\textcounter>\cellcounter\relax
     \doglobal\decrement\textcounter\relax
   \else
     \setbox0\hbox{\doifsomething{#3}{\@@steptext{#3}}}%
     \setbox2\hbox{\doifsomething{#4}{\@@steptext{#4}}}%
     \ifdim\wd0>\zeropoint \!!doneafalse \else \!!doneatrue \fi
     \ifdim\wd2>\zeropoint \!!donebfalse \else \!!donebtrue \fi
     \if!!donea
       \setbox0\hbox to \@@STPChoffset{\hss}%
       \startMPdrawing
         texts[t][#1][\textcounter] := nullpicture ;
       \stopMPdrawing
     \else
       \global\chardef\sometextboxone\zerocount
       \edef\stepidentifier{\stepchartprefix#1-\textcounter-t-t}%
       \setbox0\hbox to \@@STPChoffset
         {\hss\hpos{\stepidentifier}{\box0}\hss}%
       \bgroup
       \setupSTEPtexts[#2]\initializeSTET
       \startMPdrawing
         initialize_box(\MPpos{\stepidentifier}) ;
         texts[t][#1][\textcounter] := \MPtextsgraphic ;
       \stopMPdrawing
       \egroup
     \fi
     \if!!doneb
       \setbox2\hbox to \@@STPChoffset{\hss}%
       \startMPdrawing
         texts[b][#1][\textcounter] := nullpicture ;
       \stopMPdrawing
     \else
       \global\chardef\sometextboxtwo\zerocount
       \edef\stepidentifier{\stepchartprefix#1-\textcounter-b-t}%
       \setbox2\hbox to \@@STPChoffset
         {\hss\hpos{\stepidentifier}{\box2}\hss}%
       \bgroup
       \setupSTEPtexts[#2]\initializeSTET
       \startMPdrawing
         initialize_box(\MPpos{\stepidentifier}) ;
         texts[b][#1][\textcounter] := \MPtextsgraphic ;
       \stopMPdrawing
       \egroup
     \fi
     \global\setbox\textboxone\hbox
       {\hbox to \wd\stepboxone{\box\textboxone\hss}\box0}
     \global\setbox\textboxtwo\hbox
       {\hbox to \wd\stepboxtwo{\box\textboxtwo\hss}\box2}
   \fi}

\def\doSTEPcharttextsB[#1]% #2 #3
  {\dodoSTEPcharttextsB{\cellcounter}{#1}}

\def\dodoSTEPcharttextsB#1#2#3#4% #1=number #2=setup
  {\dodoSTEPcharttextsAB{#1}{#2}{#3}{#4}\dodoSTEPcharttextsB
   \ifnum\textcounter>\cellcounter\relax
     \doglobal\decrement\textcounter\relax
   \else
     \bgroup
     \initializeSTEL
     \startMPdrawing
       lines[t][#1][\textcounter] := \MPcharttoplinesgraphic{#1}\textcounter ;
       lines[b][#1][\textcounter] := \MPchartbotlinesgraphic{#1}\textcounter ;
     \stopMPdrawing
     \egroup
   \fi}

\def\dodoSTEPcharttextsAB#1#2#3#4#5% #1=number #2=setup
  {\doglobal\increment\textcounter\relax
   \ifnum\textcounter>\cellcounter\relax
     \@EA\appendtoks\@EA#5\@EA{#1}{#2}{#3}{#4}\to\stepchartbuffer
   \fi}

\def\MPtextsgraphic
  {image(drawshape(
     \@@STETalternative, pxy enlarged (-.5*\@@STEToffset),
     \@@STETrulethickness, \MPcolor{\@@STETframecolor},
     \MPcolor{\@@STETbackgroundcolor} ) )}

\def\MPcharttoplinesgraphic#1#2%
  {image(drawline(
     \@@STELalternative, get_step_chart_top_line(#1,#2),
     \@@STELrulethickness, \MPcolor{\@@STELcolor} ) )}

\def\MPchartbotlinesgraphic#1#2%
  {image(drawline(
     \@@STELalternative, get_step_chart_bot_line(#1,#2),
     \@@STELrulethickness, \MPcolor{\@@STELcolor} ) )}

%D Step tables are the vertical counterpart of stepcharts.

\newcounter\currentsteptable

\def\steptableprefix{@st@-\currentsteptable-}

\def\startSTEPtable
  {\dosingleempty\dostartSTEPtable}

\def\dostartSTEPtable[#1]#2\stopSTEPtable
  {\dostartSTEPaligntable[0][#1]#2\stopSTEPaligntable}

\def\startSTEPaligntable
  {\dodoubleempty\dostartSTEPaligntable[1]}

\def\dostartSTEPaligntable[#1][#2]#3\stopSTEPaligntable % flag settings data
  {\ifinsidefloat
   \else
     \whitespace
     \@@STPTbefore
     \startbaselinecorrection
     \setlocalhsize
     \noindent
   \fi
   \vbox\bgroup
   \setupSTEPtables[#2]%
   \forgetall
   \pushMPdrawing
   \doglobal\increment\currentsteptable
   \startMPdrawing
     input mp-step ;
     begin_step_table ;
   \stopMPdrawing
   \initializeSTEP
   \initializeSTPT
   \def\startlines{\bgroup\setupSTEPlines}%
   \def\stoplines {\egroup}%
   \def\prep##1{\ignorespaces##1\unskip\enspace\ignorespaces}%
   \def\cell   {\dosingleempty\docell}%
   \def\cells  {\dosingleempty\docells}%
   \def\text   {\dosingleempty\dotext}%
   % first graphic pass, also trial pass
   \global\dimen1\zeropoint
   \global\dimen3\zeropoint
   \global\dimen5\zeropoint
   \def\docell[##1]%
     {\docells[##1]{}{}}%
   \def\docells[##1]##2##3##4%
     {\doglobal\increment\cellcounter
      \bgroup
      \iffirstargument\setupSTEPcells[##1]\fi
      \initializeSTEC
      \startMPdrawing
        if box_found(\MPpos{\steptableprefix\cellcounter-c}) :
          initialize_box(\MPpos{\steptableprefix\cellcounter-c}) ;
          cells[\cellcounter] := \MPcellsgraphic ;
        fi ;
      \stopMPdrawing
      \egroup
      \def\do####1####2%
        {\setbox\scratchbox\hbox{\@@stepfake{####2}}%
         \ifdim\wd\scratchbox>\dimen####1\global\dimen####1=\wd\scratchbox\fi}%
      \ifcase#1\else\do1{##2}\do3{##3}\fi\do5{##4}}%
   \def\dotext[##1]##2%
     {\bgroup
      \iffirstargument\setupSTEPtexts[##1]\fi
      \initializeSTET
      \startMPdrawing
        if box_found(\MPpos{\steptableprefix\cellcounter-t}) :
          initialize_box(\MPpos{\steptableprefix\cellcounter-t}) ;
          texts[\cellcounter] := \MPtextsgraphic ;
        fi ;
      \stopMPdrawing
      \egroup}
   \doglobal\newcounter\cellcounter#3
   % second graphic pass pass, drawing lines
   \def\docells[##1]##2##3##4%
     {\doglobal\increment\cellcounter}
   \def\dotext[##1]##2%
     {\bgroup
      \initializeSTEL
      \startMPdrawing
        lines[\cellcounter] := \MPtablelinesgraphic ;
      \stopMPdrawing
      \egroup}
   \doglobal\newcounter\cellcounter#3
   % finishing graphic touch
   \startMPdrawing
     nofcells := \cellcounter ;
     end_step_table ;
     anchor_box(\MPpos{\steptableprefix origin}) ;
   \stopMPdrawing
   \MPdrawingdonetrue
   \doifelse\@@STPTmethod{0}
     {\setbox0\null}
     {\setbox0\vbox{\MPstaticgraphictrue\getMPdrawing}}%
   \resetMPdrawing
   % typesetting pass
   \dimen6=\@@STPTdistance \dimen6=2\dimen6
   % cell width
   \dimen8=\dimen1
   \advance\dimen8\dimen3
   \advance\dimen8\dimen5
   % offset width
   \ifcase#1\else \advance\dimen8 \dimen6 \fi
   % arrow width
   \advance\dimen8 \@@STPTwidth
   \advance\dimen8 \@@STPToffset
   \advance\dimen8 \@@STPToffset
   \def\docells[##1]##2##3##4%
     {\doglobal\increment\cellcounter
      \def\do####1####2####3####4% % strut really needed there !
        {\hbox to \dimen####1{####2\@@stepfake{####3}\strut####4}}%
      \setbox8\hbox
        {\ifcase#1\else
           \do1\hss{##2}\relax \hskip\@@STPTdistance
           \do3\hss{##3}\hss   \hskip\@@STPTdistance
         \fi
         \do5\relax{##4}\hss}%
     \hpos{\steptableprefix\cellcounter-c}{\@@stepcell{\box8}}
     \endgraf
     \nointerlineskip
     \kern\@@STPTvoffset}
   \def\dotext[##1]##2%
     {\bgroup
      \hskip\dimen8
      \advance\hsize-\dimen8
      \advance\hsize-\dimen6 % twice the offset
      \setbox0\hbox{\@@steptext{##2}}%
% to do
%     \ifdim\wd0>\hsize
%       \setbox0=\vbox{\@@steptext{##2}}%
%     \fi
% align
      \hpos{\steptableprefix\cellcounter-t}{\box0}%
      \endgraf
      \egroup
      \nointerlineskip
      \kern\@@STPTvoffset}
   \setbox2\vbox
     {\doglobal\newcounter\cellcounter
      #3\kern-\@@STPTvoffset}
   \hbox
     {\scratchdimen\wd0
      \advance\scratchdimen \MPllx bp
      \raise\MPlly bp\box0
      \hskip-\scratchdimen
      \hpos{\steptableprefix origin}{\box2}}
  \popMPdrawing
  \egroup
  \ifinsidefloat \else \stopbaselinecorrection \@@STPTafter \fi}

\def\MPtablelinesgraphic
  {image ( drawline (
     \@@STELalternative, get_step_table_line(\cellcounter),
     \@@STELrulethickness, \MPcolor{\@@STELcolor} ) )}

\protect

\continueifinputfile{m-steps.tex}

% A simple paragraph-flow test:

\starttext

\startbuffer
\startSTEPchart
\cells {A}     {B}
\cells {one}   {five}  \texts{$+2$}{$-2$}
\cells {two}   {four}  \texts{$+3$}{$-3$}
\cells {three} {three} \texts{$+4$}{$-4$}
\cells {four}  {two}   \texts{$+5$}{$-5$}
\cells {five}  {one}
\stopSTEPchart
\stopbuffer

\getbuffer

\startnarrower \getbuffer \stopnarrower

\placefigure[left]{}{\getbuffer}

\stoptext
