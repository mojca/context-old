%D \module
%D   [       file=meta-tex,
%D        version=2006.06.07,
%D          title=\CONTEXT\ Support Macros,
%D       subtitle=\METAPOST\ fast text insertion,
%D         author=Hans Hagen,
%D           date=\currentdate,
%D      copyright=\PRAGMA]
%C
%C This module is part of the \CONTEXT\ macro||package and is
%C therefore copyrighted by \PRAGMA. See mreadme.pdf for
%C details.

%D Many thanks to Fabrice Popineau and Taco Hoekwater in helping me
%D figure out some aspects of the text inclusion method implemented
%D here. The following code is derived from a more advanced (and to
%D be used) mechanism where \TEX, \METAPOST\ and \LUA\ play together.

%D Much of this mechanism was written with running live DVD's of
%D the Dave Matthews band in the background (or the corner of my
%D screen).

% todo: testmacro for empty pic

\unprotect

\newwrite\TeXtextwrite

\def\openTeXtexts {\immediate\openout \TeXtextwrite\currentTeXtext.mpb\relax}
\def\closeTeXtexts{\immediate\closeout\TeXtextwrite}

\def\currentTeXtext {\jobname-mpgraph}
\def\currentTeXstack{mpgtxt}

\initializeboxstack\currentTeXstack

\newtoks\collectedmptexts
\newtoks\everyTeXtexts

\chardef\TeXtextsmode\zerocount % no inheritance

\long\def\startTeXtexts#1\stopTeXtexts
  {\global\collectedmptexts\expandafter{\the\collectedmptexts#1}}

\def\dostartTeXtexts
  {\global\setfalse\TeXtextdone
   \startnointerference
   \openTeXtexts
   \ifcase\TeXtextsmode
     % normally there is no need for this (faster anyway)
   \or
     \scantokens\expandafter{\the\everyMPTEXgraphic}% brr
   \or
     \the\everyTeXtexts
   \fi
   \ifrunMPgraphics
     \initializeboxstack\currentTeXstack
   \else
     \global\let\openTeXtexts\relax
     \global\let\finishTeXtexts\closeTeXtexts
   \fi}

\def\dostopTeXtexts
  {\ifrunMPgraphics
      \closeTeXtexts
   \fi
   \stopnointerference}

\let\finishTeXtexts\relax

\appendtoks
  \finishTeXtexts
\to \everystoptext

\newconditional\TeXtextdone

% \long\def\TeXtext#1%
%   {\dowithnextboxcontent
%      {\setnormalcatcodes}
%      {\global\settrue\TeXtextdone
%       \immediate\write\TeXtextwrite{savetxt(#1,\the\wd\nextbox,\the\ht\nextbox,\the\dp\nextbox);}%
%       \savebox\currentTeXstack{#1}{\box\nextbox}}
%     \hbox}

\long\def\TeXtext
  {\dosingleempty\doTeXtext}

% currently, colors in the converter don't use the color stack
%
% 0 = nothing, withcolor works ok, but nested colors fail
% 1 = local color stack ok
% 2 = obey color stack (not yet supported)

\chardef\TeXtextcolormode\plusone

\def\definetextext[#1]#2{\setvalue{textext@@#1}{#2}}

% \definetextext[framed]{\framed}
%
% \startMPcode
%     draw \sometxt[framed]{black} rotated 45 ;
% \stopMPcode

\long\def\doTeXtext[#1]#2#3%
  {\begingroup
   \setbox\nextbox\hbox
     {\setnormalcatcodes
      \endlinechar\minusone
      \everyeof\emptytoks
     %\def\ascii{#3}\scantokens\expandafter{\ascii}}%
      \ifcase\TeXtextcolormode
        \scantokens{\executeifdefined{textext@@#1}\firstofoneargument{#3}}%
      \else
        \localcolortrue
        \startcurrentcolor
        \scantokens{\executeifdefined{textext@@#1}\firstofoneargument{#3}}%
        \stopcurrentcolor
      \fi}%
   \global\settrue\TeXtextdone
   \edef\currenttextxt{\number#2}%
   \executeifdefined{textext::#1}{\getvalue{textext::depth}}%
   \savebox\currentTeXstack\currenttextxt{\box\nextbox}%
   \endgroup}

\setvalue{textext::depth}{\immediate\write\TeXtextwrite{savetxt(\currenttextxt,\the\nextboxwd,\the\nextboxht,\the\nextboxdp) shifted (0,-\the\nextboxdp);}}
\setvalue{textext::nodepth}{\immediate\write\TeXtextwrite{savetxt(\currenttextxt,\the\nextboxwd,\the\nextboxht,\the\nextboxdp);}}

\setvalue{textext::d}{\getvalue{textext::depth}}
\setvalue{textext::n}{\getvalue{textext::nodepth}}

\newbox\mptextbox

% \loadmapfile[lm-texnansi.map] % the font is not really used, i.e. nothing ends up in the file
\definefontsynonym[MPtxtfont][texnansi-lmtt10]
\definefont[localMPtxtfont][MPtxtfont at 10bp]

\ifx\getTeXtext\undefined

    % this took a while to figure out

   \let\MPtextdata\empty

   \def\getTeXtext
     {\ifx\MPtextdata\empty\else
        \localMPtxtfont
        \setbox\mptextbox\hbox{\foundbox\currentTeXstack{\number\nofTeXtexts}}%
        \setbox\scratchbox\hbox{\MPtextdata}% set in meta-pdf.mkii/mkiv
        \edef\mpwd{\the\dimexpr\MPtextsize\dimexpr\wd\scratchbox/10\relax\relax}%
        \edef\mpht{\the\dimexpr\MPtextsize\dimexpr\ht\scratchbox/10\relax\relax}%
        \setbox\mptextbox\hbox{\raise\dp\mptextbox\box\mptextbox}%
        \dp\mptextbox\zeropoint
        \scale[\c!width=\mpwd,\c!height=\mpht]{\box\mptextbox}%
      \fi}

\fi

\let\nofTeXtexts\!!zerocount

\setvalue{handleMPtext00001}% only height in tag (00001)
  {\setbox\scratchbox\hbox
     {\obeyMPspecials
      \edef\nofTeXtexts{\number\MPtextnumber}%
      \getTeXtext}%
   \setbox\scratchbox\hbox
     {\hskip\lastMPmoveX\onebasepoint\raise\lastMPmoveY\onebasepoint
      \box\scratchbox}%
   \ht\scratchbox\zeropoint
   \dp\scratchbox\zeropoint
   \wd\scratchbox\zeropoint
   \box\scratchbox}

\startMPextensions
    string txtfile ; txtfile := "\currentTeXtext.mpb" ;
    string txtfont ; txtfont := "\truefontname{MPtxtfont}" ;
    string txtpref ; txtpref := "00001::::" ;
\stopMPextensions

\long\def\dodofiltersometxt#1#2#3%
  {\ifx#2\empty
   \else
     \increment\txtcounter
     \TeXtext\txtcounter{#1}%
     \expandafter\filtersometxt
   \fi#2#3}

\long\def\redofiltersometxt[#1]#2%
  {\increment\txtcounter
   \TeXtext[#1]\txtcounter{#2}%
   \filtersometxt}

\long\def\filtersometxt#1\sometxt
  {\doifnextcharelse[\redofiltersometxt\dodofiltersometxt}

% cleaner in mkiv
%
% \filtersometxt abc\sometxt{def};hij\sometxt{klm};\sometxt{}\empty\relax

\long\def\flushTeXtexts#1%
  {\newcounter\txtcounter
   \dostartTeXtexts
     \the\collectedmptexts
     \filtersometxt#1\sometxt{}\empty\relax
   \dostopTeXtexts
   \ifconditional\TeXtextdone
     \immediate\write\MPwrite{loadtxts ; txtnext := 0 ;}%
     \global\collectedmptexts\emptytoks
   \fi
   \newcounter\txtcounter}

% \long\def\sometxt#1{sometxt(nexttxt)} % to be used in mp definitions, no ; here

\long\def\sometxt #1#{\dosometxt}       % grab optional [args]
\long\def\dosometxt#1{sometxt(nexttxt)} % to be used in mp definitions, no ; here

% we redefine the writer:

\long\def\writecheckedMPgraphic#1%
  {\ifforceMPTEXgraphic
     \global\MPTEXgraphictrue
   \else
     \global\MPTEXgraphicfalse
     \edef\ascii{#1}\defconvertedcommand\MPascii\ascii
     \the\MPTEXgraphicchecks\relax % \relax is end condition!
   \fi
   \flushMPTEXgraphic% % verbatimtex etc
   \flushTeXtexts{#1}% added
   \writeMPgraphic{#1}} % potential optimization: pass \ascii

\protect \endinput

% torture test (will move)

\startMPpage
    numeric a_b_c ;
    picture p ; pickup pencircle scaled .1pt ;
    p := \sometxt{Just a \color[blue]{simple} example text.} ;
    p := image(draw p; draw boundingbox p withcolor red; ) ;
    p := p rotatedaround(center p,  360*(5/100)) ;
    draw p ; draw boundingbox p withcolor blue ;
    currentpicture := currentpicture scaled 20 ;
    draw boundingbox currentpicture  withcolor .5white ;
    setbounds currentpicture to boundingbox currentpicture enlarged 10pt ;
\stopMPpage

\startMPpage
    picture p ;
    p := \sometxt{\framed[width=fit,align=middle]{\input tufte\relax}} ;
    draw p rotatedaround(center p, 30) ;
\stopMPpage

\startMPpage
    picture p ;
    p := \sometxt{\framed[width=fit,align=middle]{\input tufte\relax}} ;
    draw p slanted .5 ;
\stopMPpage

\dorecurse{10} {
    \startTeXtexts
        \TeXtext{\recurselevel}{\ruledhbox{I must be {\green crazy} to implement this}}
    \stopTeXtexts
    \startMPpage
        picture p ; pickup pencircle scaled .1pt ;
        numeric i ; i := \recurselevel ;
        p := sometxt(i) ;
        p := p rotatedaround(center p,  360*(i*5/100)) ;
        draw p ; draw boundingbox p withcolor blue ;
        currentpicture := currentpicture scaled 20 ;
        draw boundingbox currentpicture  withcolor .5white ;
    \stopMPpage
}

\startTeXtexts
  \dorecurse{100}{\TeXtext{\recurselevel}{\ruledhbox{\strut interesting \recurselevel}}}
\stopTeXtexts

\startMPpage
    picture p ; pickup pencircle scaled .1pt ;
    for i = 1 upto 100:
        p := sometxt(i) ;
        p := p rotatedaround(center p,  360*(i*5/100)) ;
        draw p ; draw boundingbox p withcolor blue ;
    endfor ;
    currentpicture := currentpicture scaled 20 ;
    draw boundingbox currentpicture  withcolor .5white ;
\stopMPpage

\startTeXtexts
    \dorecurse{100}{\TeXtext{\recurselevel}{\ruledhbox{\strut interesting \recurselevel}}}
\stopTeXtexts

\startMPpage
    picture p ; pickup pencircle scaled .1pt ;
    for i = 1 step 5 until 100 :
        p := sometxt(i) ;
        p := p rotatedaround(center p,  360*(i/100)) ;
        draw p ; draw boundingbox p withcolor blue ;
    endfor ;
    currentpicture := currentpicture scaled 20 ;
    draw boundingbox currentpicture  withcolor .5white ;
\stopMPpage

\startTeXtexts
    \dorecurse{20}{\TeXtext{\recurselevel}{\externalfigure[t:/sources/cow.pdf][width=1cm]}}
\stopTeXtexts

\startMPpage
    picture p ; pickup pencircle scaled .1pt ;
    for i = 1 upto 20 :
        p := sometxt(i) ;
        p := p shifted (2.5cm,0) rotated (360*(i/20)) ;
        draw p ; draw boundingbox p withcolor blue  ;
    endfor ;
    currentpicture := currentpicture scaled 10 ;
    draw boundingbox currentpicture  withcolor .5white ;
\stopMPpage

\startTeXtexts
    \dorecurse{200}{\TeXtext{\recurselevel}{\ruledhbox{\strut I must be {\green crazy} \recurselevel}}}
\stopTeXtexts

\startMPpage
    picture p ; pickup pencircle scaled .1pt ;
    numeric i ; i := 100 ;
    p := sometxt(i) ;
    p := p rotatedaround(center p,  360*(i*36/100)) ;
    draw p ; draw boundingbox p withcolor blue  ;
    currentpicture := currentpicture scaled 20 ;
    draw boundingbox currentpicture  withcolor .5white ;
\stopMPpage

\dorecurse{10}{
    \startTeXtexts
        \dorecurse{200}{\TeXtext{\recurselevel}{\ruledhbox{\strut I must be {\green crazy} \recurselevel}}}
    \stopTeXtexts
    \startMPpage
        picture p ; pickup pencircle scaled .1pt ;
        j := 10*\recurselevel-9;
        k := 10*\recurselevel;
        for i = j upto k:
            p := sometxt(i) ;
            p := p rotatedaround(center p,  360*(i/100)) ;
            draw p ; draw boundingbox p withcolor blue ;
        endfor ;
        currentpicture := currentpicture scaled 20 ;
        draw boundingbox currentpicture  withcolor red ;
    \stopMPpage
}
