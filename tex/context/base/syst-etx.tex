%D \module
%D   [       file=syst-etx,
%D        version=1999.03.17, % some time ...
%D          title=\CONTEXT\ System Macros,
%D       subtitle=Efficient \PLAIN\ \TEX\ loading,
%D         author=Hans Hagen,
%D           date=\currentdate,
%D      copyright={PRAGMA / Hans Hagen \& Ton Otten}]
%C
%C This module is part of the \CONTEXT\ macro||package and is
%C therefore copyrighted by \PRAGMA. See mreadme.pdf for
%C details.

%D This module prepares \CONTEXT\ for \ETEX. We don't use
%D the definition files that come with this useful \TEX\
%D extension, but implement our own alternatives.

%D \CONTEXT\ was one of the first systems that had support for \ETEX\
%D built in. In the process we found out that the extensions were not
%D as bug free as the rest of \TEX. Especially the bugs in \type
%D {lastnode}, flushing of token lists with an index in the extension
%D range, and spurious box behaviour of boxes adressed in the extended
%D box space made us a bit careful. It's hard to to track down such
%D bugs, especially if one has a mind set of \TEX\ being bug free. If
%D you encounter unexpected behaviour let me know. Currently the
%D scantokens mechanism can handle only one||liners, but Taco will
%D provide an alternative some day.
%D
%D \starttyping
%D \bgroup
%D   \lccode`a=12\lowercase{\xdef\whatever{a}}\egroup
%D   \def\whatever{test \whatever test}
%D   \scantokens\expandafter{\whatever}
%D \egroup
%D \stoptyping

\unprotect

%D \ETEX\ has a not so handy way of telling you the version number,
%D i.e. the revision number has a period in it:

\long\def\gobbleoneargument#1{}

\beginETEX
  \mathchardef\etexversion=\numexpr\eTeXversion*100+\expandafter\gobbleoneargument\eTeXrevision\relax
\endETEX

\beginTEX
  \mathchardef\etexversion=0
\endTEX

%D Constants to be used with \type {\grouptype}.

\chardef\@@bottomlevelgroup   =  0
\chardef\@@simplegroup        =  1
\chardef\@@hboxgroup          =  2
\chardef\@@adjustedhboxgroup  =  3
\chardef\@@vboxgroup          =  4
\chardef\@@vtopgroup          =  5
\chardef\@@aligngroup         =  6
\chardef\@@noaligngroup       =  7
\chardef\@@outputgroup        =  8
\chardef\@@mathgroup          =  9
\chardef\@@discretionarygroup = 10
\chardef\@@insertgroup        = 11
\chardef\@@vcentergroup       = 12
\chardef\@@mathchoicegroup    = 13
\chardef\@@semisimplegroup    = 14
\chardef\@@mathshiftgroup     = 15
\chardef\@@mathleftgroup      = 16

\chardef\@@vadjustgroup       = \@@insertgroup

%D Constants to be used with \type {\interactionmode}.

\chardef\@@batchmode     = 0
\chardef\@@nonstopmode   = 1
\chardef\@@scrollmode    = 2
\chardef\@@errorstopmode = 3

%D Constants to be used with \type {\lastnodetype}.

\chardef\@@charnode          =  0
\chardef\@@hlistnode         =  1
\chardef\@@vlistnode         =  2
\chardef\@@rulenode          =  3
\chardef\@@insertnode        =  4
\chardef\@@marknode          =  5
\chardef\@@adjustnode        =  6
\chardef\@@ligaturenode      =  7
\chardef\@@discretionarynode =  8
\chardef\@@whatsitnode       =  9
\chardef\@@mathnode          = 10
\chardef\@@gluenode          = 11
\chardef\@@kernnode          = 12
\chardef\@@penaltynode       = 13
\chardef\@@unsetnode         = 14
\chardef\@@mathsnode         = 15

%D Constants to be used with \type {\iftype}.

\chardef\@@charif     =  1
\chardef\@@catif      =  2
\chardef\@@numif      =  3
\chardef\@@dimif      =  4
\chardef\@@oddif      =  5
\chardef\@@vmodeif    =  6
\chardef\@@hmodeif    =  7
\chardef\@@mmodeif    =  8
\chardef\@@innerif    =  9
\chardef\@@voidif     = 10
\chardef\@@hboxif     = 11
\chardef\@@vboxif     = 12
\chardef\@@xif        = 13
\chardef\@@eofif      = 14
\chardef\@@trueif     = 15
\chardef\@@falseif    = 16
\chardef\@@caseif     = 17
\chardef\@@definedif  = 18
\chardef\@@csnameif   = 19
\chardef\@@fontcharif = 20

%D Just in case we are not using \ETEX, we define some out of
%D range constants.

\beginTEX

\chardef\grouptype       = 255
\chardef\interactionmode = 255
\chardef\nodetype        = 255
\chardef\iftype          = 255

\endTEX

%D Of course we want even bigger log files, so we copied this
%D from the \ETEX\ source files.

\beginETEX \tracing...

\def\tracingall
  {\tracingonline    \@ne
   \tracingcommands  \thr@@
   \tracingstats     \tw@
   \tracingpages     \@ne
   \tracingoutput    \@ne
   \tracinglostchars \tw@
   \tracingmacros    \tw@
   \tracingparagraphs\@ne
   \tracingrestores  \@ne
   \showboxbreadth   \maxdimen
   \showboxdepth     \maxdimen
   \tracinggroups    \@ne
   \tracingifs       \@ne
   \tracingscantokens\@ne
   \tracingnesting   \@ne
   \tracingassigns   \tw@
   \errorstopmode}

\def\loggingall
  {\tracingall
   \tracingonline    \z@}

\def\tracingnone
  {\tracingassigns   \z@
   \tracingnesting   \z@
   \tracingscantokens\z@
   \tracingifs       \z@
   \tracinggroups    \z@
   \showboxdepth     \thr@@
   \showboxbreadth   5
   \tracingrestores  \z@
   \tracingparagraphs\z@
   \tracingmacros    \z@
   \tracinglostchars \@ne
   \tracingoutput    \z@
   \tracingpages     \z@
   \tracingstats     \z@
   \tracingcommands  \z@
   \tracingonline    \z@ }

\endETEX

%D Just to be sure:

\ifx\eTeX\undefined

  \def\eTeX{$\varepsilon$-\TeX}

\fi

%D In \ETEX\ we have lots of registers, so we redefine a few
%D low level macros. We reserve some extra space for inserts
%D and as soon as we near the end of the first register
%D memory bank (often some 10 less than 255), we switch to the
%D slower range \type {\@@medallocation}||\type {\@@maxallocation}.

\beginETEX \new...

%D First we redefine the plain \TEX\ register allocation macros.

\def\newcount   {\myalloc@0\count   \countdef   \@@maxallocation}
\def\newdimen   {\myalloc@1\dimen   \dimendef   \@@maxallocation}
\def\newskip    {\myalloc@2\skip    \skipdef    \@@maxallocation}
\def\newmuskip  {\myalloc@3\muskip  \muskipdef  \@@maxallocation}
\def\newbox     {\myalloc@4\box     \mathchardef\@@maxallocation}
\def\newtoks    {\myalloc@5\toks    \toksdef    \@@maxallocation}
\def\newread    {\myalloc@6\read    \chardef    \@@minallocation}
\def\newwrite   {\myalloc@7\write   \chardef    \@@minallocation}
\def\newmarks   {\myalloc@8\marks   \mathchardef\@@maxallocation}
\def\newlanguage{\myalloc@9\language\chardef    \@@minallocation}

\def\topofboxstack{\number\count24 }

%D Since in \CONTEXT\ we only have one math family left we
%D redefine \type {\newfam}.

\def\newfam#1{\chardef#1=15 }

%D Therefore we should reset the related counter.

\count18=1

%D We use some constants in the tests.

\mathchardef\@@minallocation =    16
\mathchardef\@@medallocation =   256
\mathchardef\@@maxallocation = 32767

%D I cannot imagine that more than~8 extra insert classes
%D are needed, but, for critical editions, we may need many
%D more, so:

\chardef\@@insallocation     =     32

%D However, there's a bug in \ETEX\ versions smaller than 2.2,
%D so we need to play safe:

\ifnum\etexversion<202 \chardef\@@insallocation=8 \fi

%D My low level allocation macro now comes down to:

\def\myalloc@#1#2#3#4#5%
  {\global\advance\count1#1by\@ne
   \ifnum\count1#1>\@@medallocation \else
     \ifnum\count1#1<\numexpr\@@medallocation-\@@insallocation\relax\else
       \global\count1#1=\numexpr\@@medallocation+\@ne\relax % \wait
     \fi
   \fi
   \ifnum\count1#1>#4%
     \global\count1#1=#4%
     \errmessage{No room for (\string#2) \string#5}%
   \fi
   \allocationnumber=\count1#1%
   \global#3#5=\allocationnumber
   \wlog{\string#5=\string#2\the\allocationnumber}}

\def\newinsert#1%
  {\ifnum\insc@unt>\numexpr\@@medallocation-\@@insallocation\relax
     \global\advance\insc@unt by\m@ne
     \allocationnumber=\insc@unt
     \global\chardef#1=\allocationnumber
     \wlog{\string#1=\string\insert\the\allocationnumber}%
   \else
     \errmessage{No room for a new insert \string#1 (\number\insc@unt)}%
   \fi}

\endETEX

%D These macros can be checked by tests like:
%D
%D \starttyping
%D \let\wlog\message \dorecurse{1000}{\newcount\dummy}
%D \stoptyping

%D A few bonus bindings.

\ifx\normalprotected \undefined \let\normalprotected \protected  \fi
\ifx\normalunexpanded\undefined \let\normalunexpanded\unexpanded \fi
\ifx\normalexpanded  \undefined \let\normalexpanded  \expanded   \fi

%D \macros
%D   {begcsname}
%D
%D Handy for \ETEX-only usage:

\beginETEX \ifcsname

  \def\begcsname#1\endcsname{\ifcsname#1\endcsname\csname#1\endcsname\fi}

\endETEX

\beginTEX

  \def\begcsname#1\endcsname{\csname#1\endcsname}

\endTEX

\protect \endinput
