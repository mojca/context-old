%D \module
%D   [       file=back-ini,
%D        version=2009.04.15,
%D          title=\CONTEXT\ Backend Macros,
%D       subtitle=Initialization,
%D         author=Hans Hagen,
%D           date=\currentdate,
%D      copyright=\PRAGMA]
%C
%C This module is part of the \CONTEXT\ macro||package and is
%C therefore copyrighted by \PRAGMA. See mreadme.pdf for
%C details.

\writestatus{loading}{ConTeXt Backend Macros / Initialization}

\registerctxluafile{back-ini}{1.001}

%D We currently have a curious mix between tex and lua backend
%D handling but eventually most will move to lua.

\unprotect

%D Right from the start \CONTEXT\ had a backend system based on
%D runtime pluggable code. As most backend issues involved specials
%D and since postprocessors had not that much in common, we ended up
%D with a system where we could switch backend as well as output code
%D for multiple backends at the same time.
%D
%D Because \LUATEX\ has the backend built in, and since some backend
%D issues have been moved to the frontend I decided to provide new
%D backend code for \MKIV, starting with what was actually used.
%D
%D At this moment \DVI\ is no longer used for advanced document
%D output and we therefore dropped support for this format. Future
%D versions might support more backends again, but this has a low
%D priority.
%D
%D The big question is: what is to be considered a backend issue and
%D what not. For the moment we treat image inclusion, object reuse,
%D position tracking and color as frontend issues, if only because we
%D deal with them via \LUA\ code and as such we don't depend too much
%D on macro calls that need to inject code for the backend.
%D
%D Not everything here makes sense and the content of this file will
%D definitely change.

%D We use a couple of (global) variables because it saves us the
%D trouble of dealing with arguments.

\letempty \@@DriverFieldName
\letempty \@@DriverFieldWidth
\letempty \@@DriverFieldHeight
\letempty \@@DriverFieldDefault
\letempty \@@DriverFieldNumber
\letempty \@@DriverFieldNumber
\letempty \@@DriverFieldStyle
\letempty \@@DriverFieldColor
\letempty \@@DriverFieldBackgroundColor
\letempty \@@DriverFieldFrameColor
\letempty \@@DriverFieldLayer
\letempty \@@DriverFieldOption
\letempty \@@DriverFieldAlign
\letempty \@@DriverFieldClickIn
\letempty \@@DriverFieldClickOut
\letempty \@@DriverFieldRegionIn
\letempty \@@DriverFieldRegionOut
\letempty \@@DriverFieldAfterKey
\letempty \@@DriverFieldFormat
\letempty \@@DriverFieldValidate
\letempty \@@DriverFieldCalculate
\letempty \@@DriverFieldFocusIn
\letempty \@@DriverFieldFocusOut

\letempty \@@DriverCommentLayer
\letempty \@@DriverAttachmentLayer

\letempty \@@DriverImageBox
\letempty \@@DriverImageOptions
\letempty \@@DriverImageWidth
\letempty \@@DriverImageHeight
\letempty \@@DriverImageFile
\letempty \@@DriverImageLabel
\letempty \@@DriverImageType
\letempty \@@DriverImageMethod
\letempty \@@DriverImagePage

\newif\ifcollectreferenceactions

%D \macros
%D   {dostartgraymode,dostopgraymode,
%D    dostartrgbcolormode,dostartcmykcolormode,dostartgraycolormode,dostopcolormode}
%D
%D Switching to and from color can be done in two ways:
%D
%D \startitemize[packed,n]
%D \item  insert driver specific commands
%D \item  pass instructions to the output device
%D \stopitemize
%D
%D The first approach is more general and lays the
%D responsibility at the driver side. Probably due to the fact
%D that \TEX\ does not directly support color, we have been
%D confronted for the last few years with changing special
%D definitions. The need for support depends on how a macro
%D package handles colored text that crosses the page boundary.
%D Again, there are two approaches.
%D
%D \startitemize[packed,n]
%D \item  let \TEX\ do the job
%D \item  let the driver handle things
%D \stopitemize
%D
%D The first approach is as driver independant as possible and
%D can easily be accomplished by using \TEX's mark mechanism.
%D In \CONTEXT\ we follow this approach. More and more, drivers
%D are starting to support color, including stacking them.
%D
%D Colors as well as grayscales can be represented in scales
%D from~0 to~1. When drivers use values in the range 0..255,
%D this value has to be adapted in the translation process.
%D Technically it's possible to get a grayscale from combining
%D colors. In the \cap{RGB} color system, a color with Red,
%D Green and Blue components of 0.80 show the same gray as a
%D Gray Scale specified 0.80. The \cap{CMYK} color system
%D supports a Black component apart from Cyan, Magenta and
%D Yellow.
%D
%D Depending on the target format, color support differs from
%D gray support. PostScript for example offers different
%D operators for setting gray and color. This is because
%D printing something using three colors is someting else than
%D printing with just black.
%D
%D In \CONTEXT\ we have implemented a color subsystem that
%D supports the use of well defined colors that, when printed
%D in black and white, still can be distinguished. This
%D approach enables us to serve both printed and electronic
%D versions, using colored text and illustrations. More on the
%D fundamentals of this topic can be found in the \cap{MAPS} of
%D the Dutch User Group, 14 (95.1).
%D
%D To satisfy all those needs, we define four specials which
%D supply enough information for drivers to act upon. We
%D could have used more general commands with the keywords
%D 'rgb' and 'gray', but because these specials are used often,
%D we prefer the more direct and shorter alternative.
%D
%D We start with the installation of color and grayscale
%D specials. The values are in the range 0..1 (e.g. 0.25).
%D
%D \starttyping
%D \dostartgraymode      {gray} ... \dostopgraymode
%D \dostartrgbcolormode  {red} {green} {blue} ... \dostopcolormode
%D \dostartcmykcolormode {cyan} {magenta} {yellow} {black} ... \dostopcolormode
%D \dostartgraycolormode {gray} ... \dostopcolormode
%D \stoptyping
%D
%D Because we can expect conflicts between drivers, we
%D implement them as category \type{or}. In previous versions
%D of \DVIPSONE\ the use of their color||specials did not
%D interfere with the PostScript ones, but recent versions do.

\let \dostartgraymode          \gobbleoneargument
\let \dostopgraymode           \donothing
\let \dostartrgbcolormode      \gobblethreearguments
\let \dostartcmykcolormode     \gobblefourarguments
\let \dostartgraycolormode     \gobbleoneargument
\let \dostopcolormode          \donothing
\let \dostartspotcolormode     \gobbletwoarguments
\let \doregisterrgbspotcolor   \gobblesevenarguments
\let \doregistercmykspotcolor  \gobbleeightarguments
\let \doregistergrayspotcolor  \gobblefourarguments
\let \doregisterrgbindexcolor  \gobblesevenarguments
\let \doregistercmykindexcolor \gobbleeightarguments
\let \doregistergrayindexcolor \gobblefourarguments
\let \doregisterspotcolorname  \gobbletwoarguments
\let \dostartnonecolormode     \donothing
\let \doregisternonecolor      \donothing

%D \macros
%D   {doinsertsoundtrack}
%D
%D Sounds are (for the moment) just files with
%D associated options.
%D
%D \starttyping
%D \doinsertsoundtrack {file} {label} {options}
%D \stoptyping

\let \doinsertsoundtrack \gobblethreearguments

%D \macros
%D   {dostartrotation,dostoprotation,
%D    dostartscaling,dostopscaling,
%D    dostartmirroring,dostopmirroring,
%D    dostartnegative,dostopnegative}
%D    dostartoverprint,dostopoverprint}
%D
%D We support a couple of transformations and renderings:
%D
%D \starttyping
%D \dostartrotation {angle} ... \dostoprotation
%D \dostartscaling {x} {y} ... \dostopscaling
%D \dostartmirroring {x} {y} ... \dostopmirroring
%D \stoptyping

\let \dostartrotation  \gobbleoneargument
\let \dostoprotation   \donothing
\let \dostartscaling   \gobbletwoarguments
\let \dostopscaling    \donothing
\let \dostartmirroring \donothing
\let \dostopmirroring  \donothing

\let \dostartnegative  \donothing
\let \dostopnegative   \donothing
\let \dostartoverprint \donothing
\let \dostopoverprint  \donothing

%D The following two specials are used in for instance \type
%D {\vadjust}'d margin material inside colored paragraphs.

\let \dostartgraphicgroup \donothing
\let \dostopgraphicgroup  \donothing

%D \macros
%D   {doselectfirstpaperbin,
%D    doselectsecondpaperbin}
%D
%D Here are some very printer||specific ones. No further
%D comment.

\let \doselectfirstpaperbin  \donothing
\let \doselectsecondpaperbin \donothing

%D \macros
%D   {doovalbox}
%D
%D When we look at the implementation, this is a complicated
%D one. There are seven arguments.
%D
%D \starttyping
%D \doovalbox {w} {h} {d} {linewidth} {radius} {stroke} {fill} {variant}
%D \stoptyping
%D
%D This command has to return a \type{\vbox} which can be used
%D to lay over another one (with text). The radius is in
%D degrees, the stroke and fill are~\type{1} (true) of~\type{0}
%D (false).

\let \doovalbox \gobbleeightarguments

%D \macros
%D   {dostartclipping,dostopclipping}
%D
%D Clipping is implemented in such a way that an arbitrary code
%D can be fed.
%D
%D \starttyping
%D \dostartclipping {pathname} {width} {height}
%D \dostopclipping
%D \stoptyping

\let \dostartclipping \gobblethreearguments
\let \dostopclipping  \donothing

%D \macros
%D   {dosetupidentity}
%D
%D We can declare some characteristics of the document with
%D
%D \starttyping
%D \dosetupidentity {title} {subject} {author} {creator} {date} {keys}
%D \stoptyping
%D
%D All data is in string format.

\let \dosetupidentity \gobblesixarguments

%D \macros
%D   {dosetuppaper}
%D
%D This special can be used to tell the driver what page size
%D to use. The special takes three arguments.
%D
%D \starttyping
%D \dosetuppaper {type} {width} {height}
%D \stoptyping
%D
%D The type is one of the common identifiers, like A4, A5 or
%D B2.

\let \dosetuppaper \gobblethreearguments

%D \macros
%D   {dosetupprinter}
%D
%D Some drivers enable the user to specify the paper type
%D used and/or page dimensions to be taken into account.
%D
%D \starttyping
%D \dosetupprinter {type} {hoffset} {voffset} {width} {height}
%D \stoptyping
%D
%D The first argument is one of \type{letter}, \type{legal},
%D \type{A4}, \type{A5} etc. The dimensions are in
%D basepoints.

\let \dosetupprinter \gobblefourarguments

%D \macros
%D   {dosetupopenaction, dosetupclosaction,
%D    dosetupopenpageaction, dosetupclospageaction,
%D    dosetupinteraction,
%D    dosetupscreen,
%D    dosetupviewmode}
%D
%D Here come some obscure interactive commands. Probably the
%D specs will change with the development of the macros that
%D use them.
%D
%D The first ones can be used to set up the interaction.
%D
%D \starttyping
%D \dosetupinteraction
%D \stoptyping
%D
%D Normally this command does nothing but giving a message
%D that some scheme is supported.
%D
%D \starttyping
%D \dosetupstartaction
%D \dosetupstopaction
%D \stoptyping
%D
%D These two setup the actions to be executed when the document
%D is opened and closed.
%D
%D The next commands sets up the page and screen. They are
%D kind of related.
%D
%D \starttyping
%D \dosetuppage   {hoffset} {voffset} {width} {height} {options}
%D \dosetupscreen {hoffset} {voffset} {width} {height} {options}
%D \stoptyping
%D
%D The first four arguments are in points. Option~1 results in a
%D full screen launch.
%D
%D \starttyping
%D \dosetuppageview {keyword}
%D \stoptyping
%D
%D For the moment we only support \type{fit}.

\let \dosetupinteraction              \donothing
\let \dosetupopenaction               \donothing
\let \dosetupscreen                   \gobblefourarguments
\let \dosetuppageview                 \gobbleoneargument
\let \dosetupcloseaction              \donothing
\let \dosetupopenpageaction           \donothing
\let \dosetupclosepageaction          \donothing
\let \dosetuprenderingopenpageaction  \donothing
\let \dosetuprenderingclosepageaction \donothing
\let \dosetupcropbox                  \gobblefourarguments
\let \dosetuptrimbox                  \gobblefourarguments
\let \dosetupartbox                   \gobblefourarguments
\let \dosetupbleedbox                 \gobblefourarguments

%D \macros
%D   {dostarthide,
%D    dostophide}
%D
%D Not every part of the screen is suitable for paper. Menus
%D for instance have no meaning on an non||interactive medium.
%D These elements are hidden by means of:
%D
%D \starttyping
%D \dostarthide .. \dostophide
%D \stoptyping

\let \dostarthide \donothing
\let \dostophide  \donothing

%D \macros
%D   {dostartgotolocation, dostopgotolocation,
%D    dostartgotorealpage, dostopgotorealpage}
%D
%D When we want to support hypertext buttons, again we have
%D to deal with two concepts.
%D
%D \startitemize[packed,n]
%D \item let \TEX\ highlight the text
%D \item let the driver show us where to click
%D \stopitemize
%D
%D The first approach is the most secure one. It gives us
%D complete control over the visual appearance of hyper
%D buttons. The second alternative lets the driver guess what
%D part of the text needs highlighting. As long as we deal with
%D not too complicated textual buttons, this is no problem.
%D It's even a bit more efficient when we take long mid
%D paragraph active regions into account. When we let \TEX\
%D handle active sentences {\em for instance marked like this
%D one}, we have to take care of line- and pagebreaks ourselve.
%D However, it's no trivial matter to let a driver find out
%D where things begin and end. Because most hyperlinks can be
%D found in tables of contents and registers, the saving in
%D terms of bytes can be neglected and the first approach is a
%D clear winner.
%D
%D The most convenient way of cross||referencing is using named
%D destinations. A more simple scheme is using page numbers as
%D destinations. Because the latter alternative can often be
%D implemented more efficient, and because we cannot be sure
%D what scheme a driver supports, we always have to supply a
%D pagenumber, even when we use named destinations.
%D
%D To enable a driver to find out what to make active, we have
%D to provide begin and endpoints, so like with color, we use
%D pairs of specials. The first scheme can be satisfied with
%D proper dimensions of the areas to be made active.
%D
%D The interactive real work is done by the following four
%D specials. The reason for providing the first one with both
%D a label and a number, is a result of the quite poor
%D implementation of \type{pdfmarks} in version 1.0 of
%D Acrobat. Because only pagenumbers were supported as
%D destination, we had to provide both labels (\DVIWINDO) and
%D pagenumbers (\PDF). Some drivers use start stop pairs.
%D
%D \starttyping
%D \dostartgotolocation {w} {h} {url} {file} {label} {page}
%D \dostartgotorealpage {w} {h} {url} {file}         {page}
%D \stoptyping
%D
%D Their counterparts are:
%D
%D \starttyping
%D \dostopgotolocation
%D \dostopgotorealpage
%D \stoptyping
%D
%D The internal alternative is used for system||generated
%D links, the external one for user||generated links. The
%D Uniform Resource Locator can be used to let the reader
%D surf the net.

\let \dostartgotolocation \gobblesixarguments
\let \dostopgotolocation  \donothing
\let \dostartgotorealpage \gobblefourarguments
\let \dostopgotorealpage  \donothing

%D One may wonder why jumps to page and location are not
%D combined. By splitting them, we enable macro||packages to
%D force the prefered alternative, while on the other hand
%D drivers can pick up the alternative desired most.

%D \macros
%D  {dostartgotoJS, doflushJSpreamble}
%D
%D Rather special is the option to include and execute
%D JavaScript code. This is a typical \PDF\ option.
%D
%D \starttyping
%D \dostartgotoJS {w} {h} {script}
%D \stoptyping
%D
%D This not so standard \TEX\ feature should be used with
%D care. Preamble scripts are flushed by
%D
%D \doflushJSpreamble {script}

\let \dostartgotoJS      \gobblethreearguments
\let \dostopgotoJS       \donothing
\let \doflushJSpreamble  \gobbleoneargument

%D \macros
%D   {dostartthisislocation, dostopthisislocation,
%D    dostartthisisrealpage, dostopthisisrealpage}
%D
%D Before we can goto some location or page, we have to tell
%D the system where it can be found. Because some drivers
%D follow the \SGML\ approach of begin||end tags, we have to
%D support pairs. A possible extension to this scheme is
%D supplying coordinates for viewing the text.
%D
%D The opposite commands of \type{\dogotosomething} have only
%D one argument:
%D
%D \starttyping
%D \dostartthisislocation {label}
%D \dostartthisisrealpage {page}
%D \stoptyping
%D
%D These commands are accompanied by:
%D
%D \starttyping
%D \dostopthisislocation
%D \dostopthisisrealpage
%D \stoptyping
%D
%D As with all interactive commands's they are installed as
%D \type{and} category specials.

\let \dostartthisislocation \gobbleoneargument
\let \dostopthisislocation  \donothing
\let \dostartthisisrealpage \gobbleoneargument
\let \dostopthisisrealpage  \donothing

%D In \CONTEXT\ we don't use the \type{\stopsomething}
%D macros because we let \TEX\ take care of typographic
%D issues.

%D \macros
%D   {doresetgotowhereever}
%D
%D These and others need:

\let \doresetgotowhereever \donothing

%D \macros
%D   {dostartexecutecommand, dostopexecutecommand}
%D
%D The actual behavior of the next pair of commands depends
%D much on the viewing engine. Therefore one cannot depend
%D too much on their support.
%D
%D \starttyping
%D \dostartexecutecommand {w} {h} {command} {options}
%D \stoptyping
%D
%D At least the next commands are supported (more examples
%D can be found in \type {spec-fdf.tex}:
%D
%D \startlinecorrection\setupalign[middle]\leavevmode
%D \starttable[|l|l|]
%D \HL
%D \NC \bf command  \NC \bf action                  \NC\SR
%D \HL
%D \NC first        \NC go to the first page        \NC\FR
%D \NC previous     \NC go to the previous page     \NC\MR
%D \NC next         \NC go to the next page         \NC\MR
%D \NC last         \NC go to the last page         \NC\MR
%D \NC backward     \NC go back to the  link list   \NC\MR
%D \NC forward      \NC go forward in the link list \NC\MR
%D \NC print        \NC enter print mode            \NC\MR
%D \NC exit         \NC exit viewer                 \NC\MR
%D \NC close        \NC close document              \NC\MR
%D \NC enter        \NC enter viewer                \NC\MR
%D \NC help         \NC show help on the viewer     \NC\LR
%D \HL
%D \stoptable
%D \stoplinecorrection
%D
%D Options are to be passed as a comma separated list of
%D assignments.

\let \dostartexecutecommand \gobblefourarguments
\let \dostopexecutecommand  \donothing

%D \macros
%D   {dostartobject,
%D    dostopobject,
%D    doresetobjects,
%D    doinsertobject}
%D
%D Reuse of object can reduce the output filesize
%D considerably. Reusable objects are implemented with:
%D
%D \starttyping
%D \dostartobject{class}{name}{width}{height}{depth}
%D some typeset material
%D \dostopobject
%D \stoptyping
%D
%D \starttyping
%D \doinsertobject{class}{name}
%D \stoptyping
%D
%D The savings can be huge in interactive texts. The next macro needs
%D to be called after a graphic is inserted (in order to clean up
%D global references).
%D
%D \starttyping
%D \doresetobjects
%D \stoptyping

\let \dostartobject   \gobblefourarguments
\let \dostopobject    \donothing
\let \doinsertobject  \gobbletwoarguments
\let \doresetobjects  \donothing

%D \macros
%D   {doregisterfigure, doregisterfigurecolor}
%D
%D Images can be objects as well and it's up to the driver to
%D handle this. Alternative images are also up to the driver,
%D and the next macro tells the driver that the previous image
%D is somehow followed by another and that both have to be
%D handled together. This is a rather fuzzy model, but for the
%D moment it suits its purpose: low res screen versions combined
%D with high res printable ones.

\let \doregisterfigure      \gobbletwoarguments
\let \doregisterfigurecolor \gobbleoneargument

% %D \macros
% %D   {dogetobjectreference}
% %D
% %D For very special purposes, one can ask for the internal
% %D reference to the object. Beware!
%
% \let \dogetobjectreference \gobblethreearguments
%
% %D The first argument is the name, the second a macro that
% %D gets the associated value.

%D \macros
%D   {dostartrunprogram, dostoprunprogram,
%D    dostartgotoprofile, dostopgotoprofile,
%D    dobeginofprofile,
%D    doendofprofile}
%D
%D These specials are still experimental. They are not yet
%D supported by the programs the way they should be.
%D
%D {\em --- still undocumented ---}

\let \dostartrunprogram  \gobblefourarguments
\let \dostoprunprogram   \donothing
\let \dostartgotoprofile \gobblethreearguments
\let \dostopgotoprofile  \donothing
\let \dobeginofprofile   \gobblefourarguments
\let \doendofprofile     \donothing

%D \macros
%D   {doinsertbookmark}
%D
%D Bookmarks, that is viewer generated tables of contents, are
%D a strange phenomena, mainly because \TEX\ can provide
%D whatever kind of table in much better quality.

\let \doinsertbookmark \gobblefourarguments

%D This special is called as:
%D
%D \starttyping
%D \doinstallbookmark {level} {nofsubentries} {text} {page} {open}
%D \stoptyping
%D
%D This definition is very \PDF\ oriented, so for more
%D information we kindly refer to the \PDF\ manuals.

%D \macros
%D   {dosetpagetransition}
%D
%D In presentations, fancy page transitions can, at least for a
%D short moment, let the audience focus at the screen. Like the
%D previous one, this special is very \PDF.
%D
%D \starttyping
%D \dosetpagetransition{dissolve}{0}
%D \stoptyping
%D
%D Transitions have symbolic names, like dissolve, box, split,
%D blinds, wipe and glitter. The second argument determines
%D the wait time (unless zero).

\let \dosetpagetransition \gobbletwoarguments

%D \macros
%D   {dopresettextfield,dopresetlinefield,
%D    dopresetchoicefield,dopresetpopupfield,dopresetcombofield,
%D    dopresetbuttonfield,dopresetcheckfield,
%D    dopresetradiofield,dopresetradiorecord}
%D
%D The special drivers are programmed independant from their
%D calling macros are thereby use the standard \TEX\ way of
%D passing parameters. Unfortunately fields often have more
%D than nine characteristics, so we pack some arguments in one.
%D
%D \starttyping
%D \dopresettextfield / \dopresetlinefield
%D   {name} {width} {height} {default} {length}
%D   {style,color} {options} {alignment} {actions}
%D
%D \dopresetchoicefield / \dopresetpopupfield / \dopresetcombofield
%D   {name} {width} {height} {default}
%D   {style,color} {options} {values} {actions}
%D
%D \dopresetpushfield
%D   {name} {width} {height} {default}
%D   {options} {values} {actions}
%D
%D \dopresetcheckfield
%D   {name} {width} {height} {default}
%D   {options} {values} {actions}
%D
%D \dopresetradiofield
%D   {name} {width} {height} {default}
%D   {options} {parent} {values} {actions}
%D
%D \dopresetradiorecord
%D   {name} {top} {options} {kids} {actions}
%D \stoptyping

\let \dopresetlinefield   \gobbleninearguments
\let \dopresettextfield   \gobbleninearguments
\let \dopresetchoicefield \gobbleeightarguments
\let \dopresetpopupfield  \gobbleeightarguments
\let \dopresetcombofield  \gobbleeightarguments
\let \dopresetpushfield   \gobblesevenarguments
\let \dopresetcheckfield  \gobblesevenarguments
\let \dopresetradiofield  \gobbleeightarguments
\let \dopresetradiorecord \gobblefourarguments

%D \macros
%D   {dodefinefieldset,dogetfieldset,doiffieldset}
%D
%D Field sets, used in resetting and submitting, are handled
%D by:

\let \dodefinefieldset \gobbletwoarguments
\let \dogetfieldset    \gobbleoneargument
\let \doiffieldset     \gobbletwoarguments

%D \macros
%D   {dosetfieldstatus}
%D
%D For practical reasons we set some field characteristics
%D using:
%D
%D \starttyping
%D \dosetfieldstatus {mode} {parent} {kids} {root}
%D \stoptyping

\let \dosetfieldstatus \gobblefourarguments

%D with:

\def\fieldlonermode {0} % no \chardef here
\def\fieldparentmode{1} % no \chardef here
\def\fieldchildmode {2} % no \chardef here
\def\fieldcopymode  {3} % no \chardef here

%D \macros
%D   {doregistercalculationset}
%D
%D We can define a calculation order list with:
%D
%D \starttyping
%D \doregistercalculationset {set identifier}
%D \stoptyping

\let \doregistercalculationset \gobbleoneargument

%D \macros
%D   {doinsertcomment, doflushcomments}
%D
%D Not so much out of need, but to be complete, we also
%D implement text annotations, so called  comment:
%D
%D \starttyping
%D \doinsertcomment
%D   {title} {width} {height} {color} {open} {symbol} {collect} {data}
%D \stoptyping
%D
%D When enables, comments can be collected and flushed:
%D
%D \starttyping
%D \doflushcomments
%D \stoptyping

\let \doinsertcomment \gobbleeightarguments
\let \doflushcomments \donothing

%D \macros
%D   {dostarttransparency,dostoptransparency}
%D
%D \starttyping
%D \dostarttransparency{fraction}{type}
%D \dostoptransparency
%D \stoptyping
%D
%D Although in \CONTEXT\ transparency is closely integrated
%D in the color drivers, in the end it is an independent
%D feature.

\let \dostarttransparency \gobbletwoarguments
\let \dostoptransparency  \donothing

%D \macros
%D   {doattachfile}
%D
%D \starttyping
%D \doattachfile{title}{width}{height}{depth}{color}{symbol}{filename}{source}
%D \stoptyping

\let \doattachfile \gobbleeightarguments

%D Experimental (properties):

\let \dostartviewerlayer      \gobbleoneargument
\let \dostopviewerlayer       \donothing
\let \dodefineviewerlayer     \gobblefivearguments
\let \domakeviewerlayerlist   \gobbleoneargument

\let \doinsertrenderingwindow \gobblefourarguments
\let \doinsertrendering       \gobblefourarguments
\let \doinsertrenderingobject \gobblefourarguments
\let \doinsertrenderingobject \gobblefourarguments

\let \dostartfonteffect       \gobblethreearguments
\let \dostopfonteffect        \donothing

%D From now on, mapfile loading is also a special; we assume the
%D more or less standard dvips syntax.

\let \doresetmapfilelist \donothing
\let \doloadmapfile      \gobbletwoarguments % + - = | filename
\let \doloadmapline      \gobbletwoarguments % + - = | fileline

%D \macros
%D   {ifusepagedestinations}
%D
%D In \PDF\ version 1.0 only page references were supported,
%D while in \DVIWINDO\ 1.N only named references were accepted.
%D Therefore \CONTEXT\ supports both methods of referencing. In
%D \PDF\ version 1.1 named destinations arrived. Lack of
%D continuous support of version 1.1 viewers for \MSDOS\
%D therefore sometimes forces us to prefer page references. As
%D a bonus, they are faster too and have no limitations. How
%D fortunate we were having both mechanisms available when the
%D version 3.0 (\PDF\ version 1.2) viewers proved to be too
%D bugged to support named destinations.

\newif\ifusepagedestinations

%D \macros
%D   {ifhighlighthyperlinks}
%D
%D The next switch can be used to make user hyperlinks are
%D not highlighted when clicked on.

\newif\ifhighlighthyperlinks

%D \macros
%D   {ifgotonewwindow}
%D
%D To make the {\em goto previous jump} feature more
%D convenient when using more than one file, it makes sense
%D to force the viewer to open a new window for each file
%D opened.

\newif\ifgotonewwindow

%D \macros
%D   {jobsuffix}
%D
%D By default, \TEX\ produces \DVI\ files which can be
%D converted to other filetypes. Sometimes it is handy to
%D know what the target file will be. In other driver
%D modules we wil set \type {\jobsuffix} to \type {pdf}.

% this will become a mode

\def\jobsuffix{pdf}

\ifdefined\resetsystemmode \else
    \let\setsystemmode  \gobbleoneargument
    \let\resetsystemmode\gobbleoneargument
\fi

\def\setjobsuffix#1%
  {\resetsystemmode\jobsuffix
   \edef\jobsuffix{#1}%
   \setsystemmode\jobsuffix}

%D \macros
%D   {everyresetspecials}
%D
%D Now what will this one do? We'll see in a few lines.

\newtoks\everyresetspecials

\appendtoksonce
  \ifdefined\setjobsuffix\setjobsuffix{pdf}\fi
\to \everyresetspecials

\def\defineoutput{\dodoubleargument\dodefineoutput}

\def\usespecials       [#1]{}
\def\dodefineoutput[#1][#2]{}
\def\setupoutput       [#1]{}

\protect \endinput
