%D \module
%D   [       file=t-bib,
%D        version=2006.07.14,
%D          title=\CONTEXT\ Publication Module,
%D       subtitle=Publications,
%D         author=Taco Hoekwater,
%D           date=\currentdate,
%D      copyright=Public Domain]
%C
%C Donated to the public domain.


%D The original was developed independantly by Taco Hoekwater while still working for Kluwer
%D Academic publishers (it still used the dutch interface then). Development continued after
%D he left Kluwer, and in Januari 2005, the then already internationalized file was merged
%D with the core distribution by Hans Hagen.  The current version is once again by Taco.
%D
%D More documentation and additional resources can be found on the contextgarden:
%D \hyphenatedurl{http://wiki.contextgarden.net//Bibliography}.

%D \subject{DONE (dd/mm/yyyy)}
%D
%D \startitemize
%D \item add author definition (and associated system variable) (26/05/2005)
%D \item add finalnamesep support for Oxford comma (17/09/2005)
%D \item add \type{\insert...} for: doi, eprint, howpublished (19/09/2005)
%D \item allow a defaulted \type{\setupcite} (19/11/2005)
%D \item renamed citation type 'number' to 'serial' (19/11/2005)
%D \item better definition of \type{\inverted...author} (19/11/2005)
%D \item don't reset [numbercommand] in \type {\setuppublication} by default (20/11/2005)
%D \item don't disable other \type {\setuppublication} keys if alternative is present (20/11/2005)
%D \item drop \type{\sanitizeaccents} (20/11/2005)
%D \item added \type{\nocite} and \type{\cite[none]} (21/11/2005)
%D \item added headtext for it  (23/11/2005)
%D \item make \type{\cite[url]} and \type{\cite[doi]} interactive (23/11/2005)
%D \item make right-aligned labels in the list work even when autohang=no
%D \item use 'et al.' instead of 'et.al.'. Pointed out by Peter M�nster (30/12/2005)
%D \item added headtext for cz (31/12/2005)
%D \item Keep whitespace after \type{\cite} with single argument (31/12/2005)
%D \item Fix broken \type{\cite{}} support (31/12/2005)
%D \item Use \type{\readfile} inside \type{\usepublications} instead of \type{\readsysfile} (12/01/2006)
%D \item Use \type{\currentbibyear} and \type{\currentbibauthor} instead of \type{\YR} and \type{\AU} (05/02/2006)
%D \item Fix compressed version of authoryear style (05/02/2006)
%D \item Rename the clashing data fields \type{\url} and \type{\type} to \type{\biburl} and \type{\bibtype} (05/02/2006)
%D \item Added two french bibl files from Renaud Aubin (06/02/2006)
%D \item Five new bib class and eight extra bib fields, for IEEEtran (07/02/2006)
%D \item French keyword translation, provided by Renaud (08/02/2006)
%D \item fix underscores in undefined keys (22/02/2006)
%D \item Destroy interactivity in labels of the publication list (13/03/2006)
%D \item fix multi-cite list compression (11/4/2006)
%D \item fix \type{\getcitedata} (11/4/2006)
%D \item magic for chapter bibs (18-25/4/2006)
%D \item language setting (25/4/2006)
%D \item use \type{\hyphenatedurl} for \type{\inserturl} (25/4/2006)
%D \item Add \type{\docitation} to \type{\nocite}(26/4/2006)
%D \item patents can have numbers, added to bst files (26/4/2006)
%D \item \type{\docitation} needs a \type{\iftrialtypesetting} (27/4/2006)
%D \item \type{\filllocalpublist}'s loop is bound by definedness, not resolvedness (27/4/2006)
%D \item \type{\setuppublications[monthconversion=]} added (15/5/2006)
%D \item use \type{\undefinedreference} instead of bare question marks (15/5/2006)
%D \item add grouping around \type{\placepublications} commands (16/5/2006)
%D \item fix a bug in \type{\cite{<item>}} (17/5/2006)
%D \item support \type{\cite[authornum]} (18/5/2006)
%D \item make \type{\cite} unexpandable (20/6/2006)
%D \item allow hyperlinks in author\&year combo's 
%D       (cite list compression has to be off) (20/6/2006)
%D \item fix duplicate labels for per-chapter style (20/6/2006)
%D \item allow \type{\setupcite[interaction=(start|stop)]}
%D \item fix the item number in the publication list with 'numbering=yes' (22/6/2006)
%D \item make the default criterium for \type{\placepublications} be \type{previous} (23/6/2006)
%D \item fix \type{\normalauthor} and \type{\normalshortauthor} spacing (29/6/2006)
%D \item do not typeset empty arguments to \type{\typesetapublication} (29/6/2006)
%D \item add \type{symbol=none} to \type{\setuplist} in unnumbered 
%D       mode to prevent typesetting of bare numbers (29/6/2006)
%D \item remove two incorrect spaces from bibl-num.tex (1/7/2006)
%D \item reset font styles within \type{\cite}, so that font switches
%D       in \type{left} stay in effect (12/7/2006)
%D \item guard added against loading bbl files multiple times (13/7/2006)
%D \item fix \type{\cite[num]} with compression is on. (14/7/2006)
%D \stopitemize
%D
%D \subject{WISHLIST}
%D
%D \startitemize
%D \item link back from publication list to citation
%D \item export \type {\citation{<cited item>}}
%D \item sort out different APS versions: PR A/B/L vs. RPM
%D \item don't load the whole lot, but filter entries instead
%D \stopitemize

\unprotect

%D A few new shortcuts:

\definesystemvariable  {pv}  % PublicationVariable
\definesystemvariable  {pb}  % PuBlication

\definemessageconstant {bib}
\definefileconstant    {bibextension} {bbl}

%D Some user information messages.

\startmessages all library: bib
    title: publications
    1: file -- not found, unknown style ignored
    2: file -- not found, don't forget to run bibtex
    3: wrote a new auxiliary file \jobname.aux
    4: loading database from --
    5: warning: cite argument -- on \the\inputlineno
    6: loading formatting style from --
\stopmessages

%D Some constants for the multi-lingual interface

\startconstants        dutch                english

             database: database             database
             sorttype: sorttype             sorttype
             compress: compress             compress
             autohang: autohang             autohang
            artauthor: artauthor            artauthor
               editor: editor               editor
      authoretallimit: authoretallimit      authoretallimit
   artauthoretallimit: artauthoretallimit   artauthoretallimit
      editoretallimit: editoretallimit      editoretallimit
    authoretaldisplay: authoretaldisplay    authoretaldisplay
 artauthoretaldisplay: artauthoretaldisplay artauthoretaldisplay
    editoretaldisplay: editoretaldisplay    editoretaldisplay
       authoretaltext: authoretaltext       authoretaltext
    artauthoretaltext: artauthoretaltext    artauthoretaltext
       editoretaltext: editoretaltext       editoretaltext
           otherstext: otherstext           otherstext
              andtext: andtext              andtext
          totalnumber: totalnumber          totalnumber
         firstnamesep: firstnamesep         firstnamesep
               vonsep: vonsep               vonsep
            juniorsep: juniorsep            juniorsep
           surnamesep: surnamesep           surnamesep
          lastnamesep: lastnamesep          lastnamesep
         finalnamesep: finalnamesep         finalnamesep
              namesep: namesep              namesep
               pubsep: pubsep               pubsep
           lastpubsep: lastpubsep           lastpubsep
           refcommand: refcommand           refcommand
           samplesize: samplesize           samplesize

\stopconstants

\startvariables       dutch                 english
                      german                czech
                      italian               romanian
                      french
               title: titel                 title
                      titel                 titul
                      titolo                titlu
                      titre
               short: kort                  short
                      kurz                  short
                      short                 short
                      short
                cite: cite                  cite
                      cite                  cite
                      cite                  cite
                      cite
                 bbl: bbl                   bbl
                      bbl                   bbl
                      bbl                   bbl
                      bbl
                 bib: bib                   bib
                      bib                   bib
                      bib                   bib
                      bib
              author: auteur                author
                      autor                 autor
                      autore                autor
                      auteur

\stopvariables

\def\biblistname{pubs} % for compatibility

%D how to load the references. There is some new stuff here
%D to support Idris' (incorrect :-)) use of projects

\let\preloadbiblist\relax


\ifx\currentcomponent\v!text
  % single file
  \edef\temp{\the\everystarttext}%
  \ifx\temp\empty
    % post-starttext
    \def\preloadbiblist{\dousepublications\jobname }%
  \else
    % pre-starttext
    \appendtoks \dousepublications\jobname \to \everystarttext
  \fi
  %
\else \ifx\currentcomponent\v!project
  % a project file, have to set up the partial products!
  \def\startproduct #1 %
    {\doateverystarttext
     \dousepublications{#1}%
     \donextlevel\v!product\currentproduct
     \doexecutefileonce\doexecutefileonce
     \donotexecutefile\doexecutefile#1\\}%
  %
\else \ifx\currentcomponent\v!product
  % a product file
  \def\preloadbiblist{\dousepublications\jobname }%
  %
\else
  % a component? not sure what to do
  \def\preloadbiblist{\dousepublications\jobname }%
  %
\fi \fi \fi

\definelist[pubs]
\setuplist[pubs][\c!width=]


%D The text string for the publication list header

\setupheadtext[en][pubs=References]
\setupheadtext[nl][pubs=Literatuur]
\setupheadtext[de][pubs=Literatur]
\setupheadtext[it][pubs=Bibliografia]
\setupheadtext[sl][pubs=Literatura]
\setupheadtext[fr][pubs=Bibliographie]

%D \macros{bibdoif,bibdoifnot,bibdoifelse}
%D
%D Here are a few small helpers that are used a lot
%D in all the typesetting commands
%D (\type{\insert...}) we will encounter later.

\long\def\bibdoifelse#1%
  {\@EA\def\@EA\!!stringa\@EA{#1}%
   \ifx\!!stringa\empty
     \expandafter\secondoftwoarguments
   \else
     \expandafter\firstoftwoarguments
   \fi}

\long\def\bibdoifnot#1%
  {\@EA\def\@EA\!!stringa\@EA{#1}%
   \ifx\!!stringa\empty
     \expandafter\firstofoneargument
   \else
     \expandafter\gobbleoneargument
   \fi}

\long\def\bibdoif#1%
  {\@EA\def\@EA\!!stringa\@EA{#1}%
   \ifx\!!stringa\empty
     \expandafter\gobbleoneargument
   \else
     \expandafter\firstofoneargument
   \fi}


%D Bibtex settings separated out

%D No point in writing the aux file if there is no database...

\def\setupbibtex{\dosingleempty\dosetupbibtex}

\def\dosetupbibtex[#1]%
  {\let\@@pbdatabase\empty
   \let\@@pbsort    \empty
   \getparameters[\??pb][#1]
   \expanded{\processaction[\@@pbsort]}
        [      \v!no=>\def\bibstyle{cont-no},
           \v!author=>\def\bibstyle{cont-au},
            \v!title=>\def\bibstyle{cont-ti},
            \v!short=>\def\bibstyle{cont-ab},
          \s!default=>\def\bibstyle{cont-no},
          \s!unknown=>\def\bibstyle{cont-no}]%
   \ifx\@@pbdatabase\empty\else \writeauxfile \fi}

%D \macros{writeauxfile}
%D
%D Unfortunately, \BIBTEX\ is not the best configurable program
%D around. The names of the commands it parses as well as the \type{.aux}
%D extension to the file name are both hardwired.
%D
%D This means \CONTEXT\ has to write a \LATEX-style auxiliary file, yuk!
%D The good news is that it can be rather short. We'll just ask
%D \BIBTEX\ to output the entire database(s) into the \type{bbl} file.
%D
%D The \type{\bibstyle} command controls how the \type{bbl} file will
%D be sorted. The possibilities are:
%D
%D \startitemize[packed]
%D \item by author (+year, title): cont-au.bst
%D \item by title  (+author, year): cont-ti.bst
%D \item by short key as in abbrev.bst: cont-ab.bst
%D \item not sorted at all: cont-no.bst
%D \stopitemize

\def\writeauxfile
  {\openout \scratchwrite \jobname.aux
   \write   \scratchwrite {\string\citation{*}}%
   \write   \scratchwrite {\string\bibstyle{\bibstyle}}%
   \write   \scratchwrite {\string\bibdata{\@@pbdatabase}}%
   \closeout\scratchwrite
   \showmessage\m!bib{3}{}}

%D \macros{ifsortbycite,iftypesetall,ifautohang,ifbibcitecompress}
%D
%D The module needs some new \type{\if} statements.

%D Default sort order of the reference list is by citation.

\newif\ifsortbycite        \sortbycitetrue

%D By default, only referenced publications are typeset

\newif\iftypesetall        \typesetallfalse

%D Hanging indentation of the publication list
%D will not adjust itself according to the width of the label.

\newif\ifautohang          \autohangfalse

%D Cite lists are compressed, if possible.

\newif\ifbibcitecompress   \bibcitecompresstrue

\def\setuppublications
  {\dosingleargument\dosetuppublications}

\def\bibleftnumber#1%
  {#1\hfill~}


\def\dosetuppublications[#1]%
  {\getparameters
     [\??pb]
     [\c!alternative=,#1]%
   \doifsomething\@@pbalternative
      {\readsysfile
         {bibl-\@@pbalternative.tex}
         {\showmessage\m!bib{6}{bibl-\@@pbalternative}\let\@@pbalternative\empty}
         {\showmessage\m!bib{1}{bibl-\@@pbalternative}\let\@@pbalternative\empty}}%
  \getparameters
     [\??pb]
     [#1]%
    \processaction
     [\@@pbcriterium]
     [    \v!all=>\typesetalltrue,
      \s!unknown=>\typesetallfalse]%
   \processaction
     [\@@pbautohang]
     [    \v!yes=>\autohangtrue,
      \s!unknown=>\autohangfalse]%
   \processaction
     [\@@pbsorttype]
     [   \v!cite=>\sortbycitetrue,
          \v!bbl=>\sortbycitefalse,
      \s!default=>\sortbycitetrue,
      \s!unknown=>\sortbycitefalse]%
   \processaction
     [\@@pbnumbering]
     [    \v!yes=>\let\@@pbinumbercommand\firstofoneargument,
           \v!no=>\let\@@pbinumbercommand\gobbleoneargument,
        \v!short=>\def\@@pbinumbercommand##1{\getvalue{pbds-\@@pbk}},
          \v!bib=>\def\@@pbinumbercommand##1{\getvalue{pbdn-\@@pbk}},
      \s!unknown=>\let\@@pbinumbercommand\firstofoneargument]%
   \processaction
     [\@@pbrefcommand]
     [\s!default=>\edef\@@citedefault{\@@pbrefcommand},
      \s!unknown=>\edef\@@citedefault{\@@pbrefcommand}]}

% initialize

\def\@@pbrefcommand{num}
\def\@@pbnumbercommand{\bibleftnumber}

%D \macros{usepublications}
%D
%D We need \type{\usereferences} so that it is possible to
%D refer to page and/or appearance number for publications
%D in the other document.

\def\usepublications[#1]%
  {\usereferences[#1]\processcommalist[#1]\dousepublications}

\def\dousepublications#1%
  {\doonlyonce
     {#1.\f!bibextension}
     {\readfile{#1.\f!bibextension}
       {\showmessage\m!bib{4}{#1.\f!bibextension}}
       {\showmessage\m!bib{2}{#1.\f!bibextension}}}}

%D \macros{setuppublicationlist}
%D
%D This will be the first command in (\BIBTEX-generated) \type{bbl}
%D files. `samplesize' is a sample value (in case of \BIBTEX-generated
%D files, this will be the longest `short' key). `totalnumber'
%D is the total number of entries that will follow in this
%D file.

%D Both values are only needed for the label calculation
%D if `autohang' is `true', so by default the command is
%D not even needed, and therefore I saw no need to give
%D it it's own system variable and it just re-uses \type{pb}.

\def\setuppublicationlist
  {\dosingleempty\dosetuppublicationlist}

\def\dosetuppublicationlist[#1]%
  {\getparameters[\??pv data][#1]%
   \setuplist
    [pubs]
    [\c!samplesize={AA99},\c!totalnumber={99},
     \c!alternative=a,\c!interaction=,\c!pagenumber=\v!no,#1]}

\def\setuppublicationlayout[#1]#2%
  {\setvalue{\??pv data#1}{#2\unskip}}

%D \macros{bibalternative}
%D
%D A nice little shorthand that will be used so we don't have to
%D key in the weird \type{\@@pv} parameter names all the time.

\def\bibalternative#1%
  {\getvalue{\??pv\@@currentalternative#1}}

%D \macros{simplebibdef,bibcommandlist}
%D
%D \type{\simplebibdef} defines \type{bib@#1}, which in turn will
%D use one argument that is stored in \type{@@pb@#1}.
%D
%D \type{\simplebibdef} also defines \type{insert#1}, which can be
%D used in the argument of \type{\setuppublicationlayout} to fetch
%D one of the \type{@@pb@} data entries. \type{insert#1} then has
%D three arguments: \type{#1} are commands to be executed before the
%D data, \type{#2} are commands to be executed after the data, and
%D \type{#3} are commands to be executed if the data is not found.

%D \type{\bibcommandlist} is the list of commands that is affected
%D by this approach. Later on, it will be used to do a series
%D of assignments from \type{#1} to \type{bib@#1}: e.g
%D \type{\title} becomes \type{\bib@title} when used within
%D a publication.

\def\simplebibdef#1% hh: funny expansion ?
  {\@EA\long\@EA\def\csname bib@#1\endcsname##1%
     {\setvalue{\??pb @#1}{##1}%
      \ignorespaces}%
      \@EA\def\csname insert#1\endcsname##1##2##3%
        {\@EA\bibdoifelse
           \@EA{\csname @@pb@#1\endcsname}%
           {##1\csname @@pb@#1\endcsname##2}%
           {##3}%
      }}

\def\bibcommandlist
  {abstract, annotate, arttitle, assignee, bibnumber, bibtype, biburl, chapter, city,
   comment, country, day, dayfiled, doi, edition, eprint, howpublished, isbn, issn,
   issue, journal, keyword, keywords, lastchecked, month, monthfiled, names, nationality,
   note, notes, organization, pages, pubname, pubyear, revision, series, size, thekey,
   title, volume, yearfiled}


\processcommacommand[\bibcommandlist]\simplebibdef

\def\insertbiburl#1#2#3%
    {{\bibdoifelse{\@@pb@biburl}%
       {\edef\ascii{\@EA\detokenize\@EA{\@@pb@biburl}}%
        #1\@EA\hyphenatedurl\@EA{\ascii}#2}{#3}}}

\def\insertmonth#1#2#3%
  {\bibdoifelse{\@@pb@month}%
    {#1\doifnumberelse{\@@pb@month}%
       {\doifconversiondefinedelse\@@pbmonthconversion
          {\convertnumber\@@pbmonthconversion{\@@pb@month}}{\@@pb@month}}%
       {\@@pb@month}#2}{#3}}

\let\inserturl \insertbiburl  % for backward compat.
\let\inserttype\insertbibtype % for backward compat.

\def\newbibfield[#1]%
  {\simplebibdef{#1}%
   \edef\bibcommandlist{\bibcommandlist,#1}}

%D \macros{bib@crossref}
%D
%D \type{\crossref} is used in database files to point to another
%D entry. Because of this special situation, it has to be defined
%D separately. Since this command will not be seen until at
%D \type{\placepublications}, it may force extra runs. The same is
%D true for \type{\cite} commands inside of publications.

\def\bib@crossref#1%
  {\setvalue{\??pb @crossref}{#1}\ignorespaces}

\def\insertcrossref#1#2#3%
  {\bibdoifelse{\@@pb@crossref}
     {#1\@EA\cite\@EA[\@@pb@crossref]#2}
     {#3}}

%D \macros{complexbibdef,specialbibinsert}
%D
%D The commands \type{\artauthor}, \type{\author} and
%D \type{\editor} are more complex than the other commands.
%D Their argument lists have this form:
%D
%D \type{\author[junior]{firstnames}[inits]{von}{surname}}
%D
%D (bracketed stuff should become optional someday)
%D
%D And not only that, but there also might be more than one of each of
%D these commands. This is why a special command is needed to insert
%D them, as well as one extra counter for each command.

%D All of these \type{\@EA}'s and \type{\csnames} make this code
%D look far more complex than it really is. For example, the argument
%D \type{author} defines the macro \type{\bib@author} to do two
%D things: increment the counter \type{\author@num} (let's say to 2)
%D and next store it's arguments in the macro \type{\@@pb@author2}.
%D And it defines \type{\insertauthors} to expand into
%D \starttyping
%D \specialbibinsert{author}{\author@num}{<before>}{<after>}{<not>}
%D \stoptyping

\def\complexbibdef#1%
  {\@EA\newcounter\csname #1@num\endcsname
   \@EA\def\csname bib@#1\endcsname[##1]##2[##3]##4##5%
     {\@EA\increment\csname #1@num\endcsname
      \setvalue{\??pb @#1\csname #1@num\endcsname}%
        {{##2}{##4}{##5}{##3}{##1}}\ignorespaces}%
   \@EA\def\csname insert#1s\endcsname##1##2##3%
     {\specialbibinsert{#1}{\csname #1@num\endcsname}{##1}{\unskip ##2}{##3}}}

\processcommalist[author,artauthor,editor]\complexbibdef

%D Another level of indirection is needed to control the
%D typesetting of all of these arguments, which explains the usage
%D of \type{\tempa} below.

%D There is some sneaky stuff with \type{\tempa} and \type{\tempb}
%D going on here to resolve the \type{\csname}'s. It probably could
%D be done a little bit more elegant, but it works. The basic goal
%D is to get the command that will actually typeset the name into
%D the macro \type{\tempb}, and to make sure that that command will actually
%D recieve five arguments (see the definition of
%D e.g. \type{\invertedauthor} below).

%D There is a conflict between `author' and the predefined interface
%D variable `auteur'. The old version is overruled `auteur' is
%D overruled by the systemconstant definition at the top of this file!

%D The increment/decrement trick on \type{\scratchcounter} is needed
%D to decide what name the last one is.

\newcount\etallimitcounter
\newcount\etaldisplaycounter
\newcount\todocounter

\def\specialbibinsert#1#2#3#4#5%
  {\bgroup
   \ifnum#2>\zerocount
     \letcscsname\tempa\csname \??pv data#1\endcsname
     \def\tempb{\@EA\tempa}%
     \etallimitcounter  =0\bibalternative{#1etallimit}\relax
     \etaldisplaycounter=0\bibalternative{#1etaldisplay}\relax
     \ifnum #2>\etallimitcounter
       \todocounter\etaldisplaycounter
       % just in case ...
	   \ifnum\todocounter>\etallimitcounter
         \todocounter\etallimitcounter
       \fi
     \else
       \todocounter#2\relax
     \fi
     \scratchcounter\zerocount
     \ifnum\todocounter>\zerocount
       #3%
       \doloop
         {\ifnum \scratchcounter < \todocounter
            \advance\scratchcounter \plusone
            \ifnum \scratchcounter = \todocounter
              \@EA\tempb \csname @@pb@#1\the\scratchcounter\endcsname
              \ifnum\etallimitcounter<#2 \bibalternative{#1etaltext}\fi #4%
            \else
              \@EA\tempb \csname @@pb@#1\the\scratchcounter\endcsname
              \advance\scratchcounter \plusone
              \ifnum \scratchcounter = \todocounter
                 \ifnum \todocounter > \plustwo
                    \bibalternative\c!finalnamesep
                 \else
                    \bibalternative\c!lastnamesep
                 \fi
              \else
                 \bibalternative\c!namesep
              \fi
              \advance\scratchcounter \minusone
            \fi
          \else
            \exitloop
          \fi}%
      \else
        #5%
      \fi
   \else
     #5%
   \fi
   \egroup}

%D \macros{invertedauthor,normalauthor,invertedshortauthor,normalshortauthor}
%D
%D Just some commands that can be used in \type{\setuppublicationparameters}
%D If you want to write an extension to the styles, you might
%D as well define some of these commands yourself.
%D
%D The argument liust has been reordered here, and the meanings
%D are:
%D
%D {\obeylines\parskip0pt
%D \type{#1} firstnames
%D \type{#2} von
%D \type{#3} surname
%D \type{#4} inits
%D \type{#5} junior
%D }
%D

\def\normalauthor#1#2#3#4#5%
  {\bibdoif{#1}{#1\bibalternative\c!firstnamesep}%
   \bibdoif{#2}{#2\bibalternative\c!vonsep}%
   #3%
   \bibdoif{#5}{\bibalternative\c!surnamesep#5\unskip}}

\def\normalshortauthor#1#2#3#4#5%
  {\bibdoif{#4}{#4\bibalternative\c!firstnamesep}%
   \bibdoif{#2}{#2\bibalternative\c!vonsep}%
   #3%
   \bibdoif{#5}{\bibalternative\c!surnamesep#5\unskip}}

\def\invertedauthor#1#2#3#4#5%
  {\bibdoif{#2}{#2\bibalternative\c!vonsep}%
   #3%
   \bibdoif{#5}{\bibalternative\c!juniorsep #5}%
   \bibdoif{#1}{\bibalternative\c!surnamesep #1\unskip}}

\def\invertedshortauthor#1#2#3#4#5%
  {\bibdoif{#2}{#2\bibalternative\c!vonsep}%
   #3%
   \bibdoif{#5}{\bibalternative\c!juniorsep #5}%
   \bibdoif{#4}{\bibalternative\c!surnamesep #4\unskip}}

%D \macros{clearbibitem,clearbibitemtwo,bibitemdefs}
%D
%D These are used in \type{\typesetapublication} to do
%D initializations and cleanups.

\def\clearbibitem#1{\setvalue{\??pb @#1}{}}%

\def\clearbibitemtwo#1%
  {\letvalue{#1@num}\!!zerocount
   \scratchcounter\plusone
   \doloop
     {\doifdefinedelse{\??pb @#1\the\scratchcounter}
        {\letvalue{\??pb @#1\the\scratchcounter}\empty
         \advance\scratchcounter\plusone}%
        {\exitloop}}}

\def\bibitemdefs#1%
   {\@EA\let\@EA\tempa \csname bib@#1\endcsname
    \@EA\let\csname #1\endcsname \tempa }

%D \macros{startpublication}
%D
%D We are coming to the end of this module, to the macros that
%D do typesetting and read the \type{bbl} file.

\newcounter\bibcounter

%D Just a \type{\dosingleempty} is the most friendly
%D of doing this: there need not even be an argument
%D to \type{\startpublication}. Of course, then there
%D is no key either, and it had better be an
%D article (otherwise the layout will be all screwed up).

\def\startpublication{\dosingleempty\dostartpublication}
\def\stoppublication {}

%D Only specifying the key in the argument is also
%D legal. In storing this stuff into macros, some trickery with
%D token registers is needed to fix the expansion problems. Even so,
%D this appears to not always be 100\% safe, so people are
%D urgently advised to use \ETEX\ instead of traditional \TEX.
%D
%D In \ETEX, all expansion problems are conviniently solved by
%D the primitive \type{\protected}. To put that another way:
%D
%D It's not a bug in this module if it does not appear in \ETEX!

\long\def\dostartpublication[#1]#2\stoppublication%
  {\increment\bibcounter
   \bgroup
   \doifassignmentelse{#1}%
     {\getparameters[\??pb][k=,t=article,n=,s=,a=,y=,o=,u=,#1]}%
     {\getparameters[\??pb][k=#1,t=article,n=,s=,a=,y=,o=,u=]}%
   \@EA\toks\@EA2\@EA{\@@pba}%
   \@EA\toks\@EA4\@EA{\@@pbs}%
   \toks0={\ignorespaces #2}%
   \setxvalue{pbdk-\@@pbk}{\@@pbk}
   \setxvalue{pbda-\@@pbk}{\the\toks2}
   \setxvalue{pbdy-\@@pbk}{\@@pby}
   \setxvalue{pbds-\@@pbk}{\the\toks4}
   \setxvalue{pbdn-\@@pbk}{\@@pbn}
   \setxvalue{pbdt-\@@pbk}{\@@pbt}
   \setxvalue{pbdo-\@@pbk}{\@@pbo}
   \setxvalue{pbdu-\@@pbk}{\@@pbu}
   \setxvalue{pbdd-\@@pbk}{\the\toks0}
   \xdef\allrefs{\allrefs,\@@pbk}%
   \egroup }

% intialization of the order-list:

\let\allrefs\empty

%D The next macro is needed because the number command of the
%D publist sometimes needs to fetch something from the current
%D item (like the 'short' key). For this, the ID of the current
%D item is passed in the implict parameter \type{\@@pbk}

\def\makepbkvalue#1{\def\@@pbk{#1}}

%D

\newif\ifinpublist

% from Hans

\def\ignoresectionconversion
  {\let\@@sectionconversion\secondoftwoarguments}

\let\normaldosetfilterlevel\dosetfilterlevel

\def\patcheddosetfilterlevel#1#2% beware: this one is \let
  {\bgroup
   \ignoresectionconversion
   \edef\askedlevel{#1}%
   \edef\askedfilter{#2}%
%   \message{ASKD: \meaning\askedlevel}%
%   \message{PREV: \meaning\v!previous}%
   \ifx\askedlevel\v!current
     \dosetcurrentlevel\askedlevel
   \else\ifx\askedlevel\v!previous
     \dosetpreviouslevel\askedlevel
   \else\ifx\askedlevel\v!all
     \global\chardef\alltoclevels\plusone
   \else\ifx\askedlevel\v!text
     \global\chardef\alltoclevels\plusone
   \else
     \edef\byaskedlevel{\csname\??by\askedlevel\endcsname}%
     \ifx\byaskedlevel\v!text
       \dosettextlevel\askedlevel
     \else
       \dosetotherlevel\askedlevel
     \fi
   \fi\fi\fi\fi
   % experiment
   \ifx\askedfilter\empty \else
     \xdef\currentlevel{\currentlevel\sectionseparator\askedfilter}%
   \fi
   \egroup}


\unless\ifcsname currentlocationrefence\endcsname

\let\currentlocationreference\empty

\def\dogetreferenceelements#1#2#3#4#5%
  {\chardef\currentreferencetype=\ifx#1\relax0\else#1\fi\relax
   \ifnum\currentreferencetype<2
     \edef\currentpagereference{#2}%
     \let \currentdatareference\empty
     \edef\currentlocationreference{#2}%
     \ifx\currentpagereference   \empty
       \let\currentfolioreference\folio
     \else
       \def \currentpagereference {\referencepagenumber[#2]}%
       \edef\currentfolioreference{\dosplitofffoliopart[#2]}%
     \fi
     \edef\currentrealreference{#3}%
     \settextreferences#4\end
     \ifnum0#5<\crossreferencenumber
       \forwardreferencetrue
     \else
       \forwardreferencefalse
     \fi
   \else
     \let \currentlocationreference\empty
     \edef\currentrealreference {#3}%
     \def \currentdatareference {#2}%
     \let \currentfolioreference\folio
     \settextreferences#4\end
     \forwardreferencefalse
   \fi
   \ifodd\currentreferencetype
     \realreferencepagefalse
   \else
     \docheckrealreferencepage\currentrealreference
     \ifrealreferencepage \else
       \docheckrealreferencepage\currentdatareference
     \fi
   \fi}

\fi

\def\filllocalpublist%
  {\let\dosetfilterlevel\patcheddosetfilterlevel
   \dosettoclevel\??li{pubs}%
   \let\dosetfilterlevel\normaldosetfilterlevel
   \global\let\glocalpublist\empty
   \doloop
     {\doifdefinedelse
        {\r!cross cite-\jobname-\recurselevel}
        {\doifreferencefoundelse
           {cite-\jobname-\recurselevel}
           {\@EA\doifreglevelelse\@EA[\currentlocationreference]
              {\@EA\doglobal\@EA\addtocommalist\@EA
                {\currenttextreference}\glocalpublist}{}}
           {}}%
        {\exitloop}}%
    \let\localpublist\glocalpublist}

% 
\def\typesetpubslist
  {\dobeginoflist 
   \edef\askedlevel{\csname \??li pubs\c!criterium\endcsname}%
   \ifx\askedlevel\v!all
      \def\bibrefprefix{}%
   \else %
      \preparebibrefprefix
   \fi
   \ifsortbycite
     \filllocalpublist 
     \iftypesetall
       \let\localallrefs\allrefs
        \processcommacommand[\localpublist]\typesetapublication
        \def\removefromallrefs##1%
           {\removefromcommalist{##1}\localallrefs }%
        \processcommacommand[\localpublist]\removefromallrefs
        \processcommacommand[\localallrefs]\typesetapublication
      \else
        \processcommacommand[\localpublist]\typesetapublication
     \fi
   \else
     \iftypesetall
       \processcommacommand[\allrefs]\typesetapublication
     \else
       % 
       \filllocalpublist
       \processcommacommand[\allrefs]\maybetypesetapublication
     \fi
   \fi 
   \doendoflist }

\newif\ifinpublist
\def\maybetypesetapublication#1%
  {\global\inpublistfalse
   \def\test{#1}%
   \def\runtest##1%
      {\def\tempa{##1}\ifx \test\tempa \global\inpublisttrue \fi}%
   \processcommacommand[\localpublist]\runtest
   \ifinpublist \typesetapublication{#1}\fi}

\def\initializepubslist
  {\edef\@@pbnumbering{\@@pbnumbering}%
   \ifautohang
     \ifx\@@pbnumbering\v!short
       \setbox\scratchbox\hbox{\@@pbnumbercommand{\getvalue{\??li pubs\c!samplesize}}}%
     \else\iftypesetall
       \setbox\scratchbox\hbox{\@@pbnumbercommand{\getvalue{\??li pubs\c!totalnumber}}}%
     \else
       \setbox\scratchbox\hbox{\@@pbnumbercommand{\numreferred}}%
     \fi\fi
     \edef\samplewidth{\the\wd\scratchbox}%
     \setuplist[pubs][\c!width=\samplewidth,\c!distance=0pt]%
     \def\@@pblimitednumber##1{\hbox to \samplewidth{\@@pbnumbercommand{##1}}}%
   \else
     \doifemptyelse
        {\getvalue{\??li pubs\c!width}}
        {\def\@@pblimitednumber##1{\hbox{\@@pbnumbercommand{##1}}}}%
        {\def\@@pblimitednumber##1{\hbox to \getvalue{\??li pubs\c!width}{\@@pbnumbercommand{##1}}}}%
   \fi
   \ifx\@@pbnumbering\v!no
     \setuplist[pubs][\c!numbercommand=,\c!symbol=\v!none,\c!textcommand=\outdented]
   \else
     \setuplist[pubs][\c!numbercommand=\@@pblimitednumber]%
   \fi
   \forgetall % bugfix 2005/03/18
}


\def\outdented#1% move to supp-box ?
  {\hskip -\hangindent #1}


%D The full list of publications

\def\completepublications
  {\dosingleempty\docompletepublications}

\def\docompletepublications[#1]%
  {%\getparameters[\??pv data][#1]%
   \begingroup
   \setuplist[pubs][\c!criterium=\v!previous,#1]   
   \expanded{\systemsuppliedtitle[pubs]{\noexpand\headtext{pubs}}}%
   \dodoplacepublications   }

%D And the portion with the entries only.

\def\placepublications
  {\dosingleempty\doplacepublications}

\def\doplacepublications[#1]%
  {%\getparameters[\??pv data][#1]
   \begingroup
   \setuplist[pubs][\c!criterium=\v!previous,#1]
   \dodoplacepublications }%

\def\dodoplacepublications%
  {\initializepubslist
   \global\let\bibcounter\!!zerocount
   \inpublisttrue
   \typesetpubslist
   \inpublistfalse  
   \endgroup }


%D \subsubject{What's in a publication}
%D

\unexpanded\def\typesetapublication#1%
  {\doifsomething{#1}{\doglobal\increment\bibcounter 
    \dodolistelement{pubs}{}{\bibcounter}%
      {\expanded{\reference[\bibrefprefix#1]{\bibcounter}}%
       \strut \dotypesetapublication{#1}\strut }{}{}}}

\def\dotypesetapublication#1%
  {\bgroup
   \def\@@currentalternative{data}%
   \makepbkvalue{#1}%
   \processcommacommand[\bibcommandlist,crossref]\clearbibitem
   \processcommalist   [artauthor,author,editor]\clearbibitemtwo
   \processcommacommand[\bibcommandlist]\bibitemdefs
   \processcommalist   [artauthor,author,editor,crossref]\bibitemdefs
   \let\biblanguage\empty
   \getvalue{pbdd-#1}%
   \ifcsname pbdt-#1\endcsname \bibalternative{\getvalue{pbdt-#1}}\fi
   \egroup }


%D An afterthought

\def\maybeyear#1{}

%D An another

\def\noopsort#1{}

%D This is the result of bibtex's `language' field.

\def\setbiblanguage#1#2{\setvalue{\??pb @lang@#1}{#2}}

\def\lang#1%
   {\def\biblanguage{#1}%
    \ifcsname \??pb @lang@#1\endcsname
        \expanded{\language[\getvalue{\??pb @lang@#1}]}%
     \fi \ignorespaces}


%D \subject{Citations}

%D \macros{cite,bibref}
%D
%D The indirection with \type{\dobibref} allows \LATEX\ style
%D \type{\cite} commands with a braced argument (these might appear
%D in included data from the \type{.bib} file).

\unexpanded\def\cite
  {\doifnextcharelse{[}
     {\dodocite}
     {\dobibref}}

\def\dobibref#1%
  {\docite[#1][]}

\def\dodocite[#1]%
  {\startstrictinspectnextcharacter
   \dodoubleempty\dododocite[#1]}

\def\dododocite[#1][#2]{%
   \stopstrictinspectnextcharacter
   \docite[#1][#2]}

\def\docite[#1][#2]%
  {\begingroup
   \setupinteraction[\c!style=]%
   \edef\temp{#2}%
   \ifx\empty\temp   \secondargumentfalse   
   \else             \secondargumenttrue   \fi
   \ifsecondargument
     \processcommalist[#2]\docitation
     \doifassignmentelse
       {#1}%
       {\getparameters[LO][\c!alternative=,\c!extras=,#1]%
        \edef\@@currentalternative{\LOalternative}%
	    \ifx\@@currentalternative\empty
          \edef\@@currentalternative{\@@citedefault}%
        \fi
	    \ifx\LOextras\empty
          \setupcite[\@@currentalternative][#1]%
        \else
	      \expandafter\ifx\csname LOright\endcsname \relax
              \edef\LOextras{{\LOextras\bibalternative\c!right}}%
          \else
              \edef\LOextras{{\LOextras\LOright}}%
          \fi
          \expanded{\setupcite[\@@currentalternative][#1,\c!right=\LOextras]}%
        \fi
       }%
       {\def\@@currentalternative{#1}}%
      \expanded{%
         \processaction[\csname @@pv\@@currentalternative compress\endcsname]}
             [    \v!yes=>\bibcitecompresstrue,
                   \v!no=>\bibcitecompressfalse,
              \s!default=>\bibcitecompresstrue,
              \s!unknown=>\bibcitecompresstrue]%
      \getvalue{bib\@@currentalternative ref}[#2]%
   \else
     \processcommalist[#1]\docitation
     \expanded{\processaction[\csname @@pv\@@citedefault compress\endcsname]}
       [    \v!yes=>\bibcitecompresstrue,
             \v!no=>\bibcitecompressfalse,
        \s!default=>\bibcitecompresstrue,
        \s!unknown=>\bibcitecompresstrue]%
     \edef\@@currentalternative{\@@citedefault}%
     \getvalue{bib\@@citedefault ref}[#1]%
   \fi
   \endgroup}

%D \macros{nocite}

\def\nocite[#1]%
  {\processcommalist[#1]\addthisref
   \processcommalist[#1]\docitation }

%D \macros{setupcite}

\def\setupcite{\dodoubleempty\dosetupcite}

\def\dosetupcite[#1][#2]%
  {\ifsecondargument
     \def\dodosetupcite##1{\getparameters[\??pv##1][#2]}%
     \processcommalist[#1]\dodosetupcite
   \else % default case
     \getparameters[\??pv\@@citedefault][#1]%
   \fi }


%D Low-level stuff

\def\getcitedata#1[#2]#3[#4]#5to#6%
    {\bgroup
     \addthisref{#4}%
     \dofetchapublication{#4}%
     \doifdefinedelse{@@pb@bib#2}%
       {\xdef#6{\getvalue{@@pb@bib#2}}}%
       {\xdef#6{\getvalue{@@pb@#2}}}%
     \egroup }


\def\dofetchapublication#1%
  {\makepbkvalue{#1}%
   \processcommacommand[\bibcommandlist,crossref]\clearbibitem
   \processcommalist   [artauthor,author,editor]\clearbibitemtwo
   \processcommacommand[\bibcommandlist]\bibitemdefs
   \processcommalist   [artauthor,author,editor,crossref]\bibitemdefs
   \getvalue{pbdd-#1}}

%D This new version writes a reference out to the tui file for every
%D \type{\cite}. This will allow backlinking

\newcounter\citationnumber

\def\docitation#1{%
  \iftrialtypesetting \else 
    \doglobal\increment\citationnumber
    \expanded{\rawreference{}{cite-\jobname-\citationnumber}{#1}}%
  \fi}


%D \macros{numreferred,doifreferredelse,addthisref,publist}
%D
%D The interesting command here is \type{\addthisref}, which maintains
%D the global list of references.
%D
%D \type{\numreferred} is needed to do automatic calculations on
%D the label width, and \type{\doifreferredelse} will be used
%D to implement \type{criterium=cite}.

\newcounter\numreferred

\long\def\doifreferredelse#1{\doifdefinedelse{pbr-#1}}

\def\addthisref#1%
  {\doifundefinedelse{pbr-#1}
     {\setxvalue{pbr-#1}{\citationnumber}%
      \doglobal\increment\numreferred
      \ifx\publist\empty \gdef\publist{#1}\else\appended\gdef\publist{,#1}\fi}
     {\setxvalue{pbr-#1}{\getvalue{pbr-#1},\citationnumber}}}

\let\publist\empty

%D \macros{doifbibreferencefoundelse}
%D
%D Some macros to fetch the information provided by
%D \type{\startpublication}.

\def\doifbibreferencefoundelse#1#2#3%
  {\doifdefinedelse{pbdk-#1}
     {#2}
     {\showmessage\m!bib{5}{#1 is unknown}#3}}

%D \macros{ixbibauthoryear,thebibauthors,thebibyears}
%D
%D If compression of \type{\cite}'s argument expansion is on,
%D the macros that deal with authors and years call this internal
%D command to do the actual typesetting.
%D
%D Two entries with same author but with different years may
%D be condensed into ``Author (year1,year2)''. This is about the
%D only optimization that makes sense for the (author,year)
%D style of citations (years within one author have to be unique
%D anyway so no need to test for that, and ``Author1, Author2 (year)''
%D creates more confusion than it does good).
%D
%D In the code below,
%D the macro \type{\thebibauthors} holds the names of the alternative
%D author info fields for the current list. This is a commalist,
%D and \type{\thebibyears} holds the (collection of) year(s) that go with
%D this author (possibly as a nested commalist).
%D
%D There had better be an author for all cases, but there
%D does not have to be year info always. \type{\thebibyears} is
%D pre-initialized because this makes the insertion macros simpler.
%D
%D In `normal' \TeX, of course there are expansion problems again.

\def\ixbibauthoryear#1#2#3#4%
  {\bgroup
   \gdef\ixlastcommand  {#4}%
   \gdef\ixsecondcommand{#3}%
   \gdef\ixfirstcommand {#2}%
   \glet\thebibauthors  \empty
   \glet\thebibyears    \empty
   \getcommalistsize[#1]%
   \ifbibcitecompress
     \dorecurse\commalistsize{\xdef\thebibyears{\thebibyears,}}%
     \processcommalist[#1]\docompressbibauthoryear
   \else
     \processcommalist[#1]\donormalbibauthoryear
   \fi
   \egroup
   \dobibauthoryear}

%D \macros{dodobibauthoryear}
%D
%D This macro only has to make sure that the lists
%D \type{\thebibauthors} and \type{\thebibyears} are printed.

\def\dobibauthoryear
  {\scratchcounter\zerocount
   \getcommacommandsize[\thebibauthors]%
   \@EA\processcommalist\@EA[\thebibauthors]\dodobibauthoryear}

\def\dodobibauthoryear#1%
  {\advance\scratchcounter\plusone
   \edef\wantednumber{\the\scratchcounter}%
   \getfromcommacommand[\thebibyears][\wantednumber]%
   \@EA\def\@EA\currentbibyear\@EA{\commalistelement}%
   \setcurrentbibauthor{#1}%
   \ifnum\scratchcounter=\plusone
     \ixfirstcommand
   \else\ifnum \scratchcounter=\commalistsize\relax
     \ixlastcommand
   \else
     \ixsecondcommand
   \fi\fi}

\def\setcurrentbibauthor#1%
 {\getcommacommandsize[#1]%
  \ifcase\commalistsize
   % anonymous?
   \def\currentbibauthor{}%
  \or
   \def\currentbibauthor{#1}%
  \or
   \expanded{\docurrentbibauthor#1}%
  \else
   % this can't happen/
   \def\currentbibauthor{}%
  \fi }


\def\docurrentbibauthor#1,#2%
  {\doifemptyelse{#2}
      {\def\currentbibauthor{#1\bibalternative{otherstext}}}
      {\def\currentbibauthor{#1\bibalternative{andtext}#2}}}

%D This is not the one Hans made for me, because I need a global
%D edef, and the \type{\robustdoifinsetelse} doesn't listen to
%D \type{\doglobal }

\def\robustaddtocommalist#1#2% {item} \cs
  {\robustdoifinsetelse{#1}#2\resetglobal
     {\dodoglobal\xdef#2{\ifx#2\empty\else#2,\fi#1}}}


%D \macros{donormalbibauthoryear}
%D
%D Now we get to the macros that fill the two lists.
%D The `simple' one really is quite simple.

\def\donormalbibauthoryear#1%
  {\addthisref{#1}%
   \def\myauthor{Xxxxxxxxxx}%
   \def\myyear{0000}%
   \doifbibreferencefoundelse{#1}
     {\def\myauthor{{\getvalue{pbda-#1}}}%
      \def\myyear  {\getvalue{pbdy-#1}}}%
     {}%
   \@EA\doglobal\@EA\appendtocommalist\@EA{\myauthor}\thebibauthors
   \@EA\doglobal\@EA\appendtocommalist\@EA{\myyear  }\thebibyears}

%D \macros{docompressbibauthoryear}
%D
%D So much for the easy parts. Nothing at all will be done if
%D the reference is not found or the reference does not contain
%D author data. No questions marks o.s.s. (to be fixed later)

\def\docompressbibauthoryear#1%
  {\addthisref{#1}%
   \def\myauthor{Xxxxxxxxxx}%
   \def\myyear  {0000}%
   \doifbibreferencefoundelse{#1}
     {\xdef\myauthor{\csname pbda-#1\endcsname }%
      \xdef\myyear  {\csname pbdy-#1\endcsname }}
     {}%
    \ifx\myauthor\empty\else
      \checkifmyauthoralreadyexists
      \findmatchingyear
    \fi}

%D two temporary counters. One of these two can possibly be replaced
%D by \type{\scratchcounter}.

\newcount\bibitemcounter
\newcount\bibitemwanted

%D The first portion is simple enough: if this is the very first author
%D it is quite straightforward to add it. \type{\bibitemcounter} and
%D \type{\bibitemwanted} are needed later to insert the year
%D information in the correct item of \type{\thebibyears}

\def\checkifmyauthoralreadyexists
  {\doifemptyelsevalue{thebibauthors}
     {\global\bibitemwanted  \plusone
      \global\bibitemcounter \plusone
      \xdef\thebibauthors{{\myauthor}}}
     {% the next weirdness is because according to \getcommalistsize,
      % the length of \type{[{{},{}}]} is 2.
      \@EA\getcommalistsize\@EA[\thebibauthors,]%
      \global\bibitemcounter\commalistsize
      \global\advance\bibitemcounter\minusone
      \global\bibitemwanted \zerocount
      \processcommacommand[\thebibauthors]\docomparemyauthor}}

%D The outer \type{\ifnum} accomplishes the addition of
%D a new author to \type{\thebibauthors}. The messing about with
%D the two counters is again to make sure that \type{\thebibyears}
%D will be updated correctly.If the author {\it was} found,
%D the counters will stay at their present values and everything
%D will be setup properly to insert the year info.

\def\docomparemyauthor#1%
  {\global\advance\bibitemwanted \plusone
   \def\mytempc{#1}%
%   \message{authors: \myauthor <=>\mytempc \ifx\mytempc\myauthor :Y \else :N
%          \meaning \myauthor, \meaning\mytempc\fi (\the\bibitemwanted = \the\bibitemcounter)}%
   \ifx\mytempc\myauthor
     \quitcommalist
   \else
     \ifnum\bibitemwanted = \bibitemcounter\relax
       \global\advance\bibitemwanted \plusone
       \global\bibitemcounter\bibitemwanted\relax
       \@EA\doglobal\@EA\robustaddtocommalist\@EA{{\myauthor}}\thebibauthors
     \fi
   \fi}

%D This macro should be clear now.

\def\findmatchingyear
  {\edef\wantednumber{\the\bibitemwanted}%
   \getfromcommacommand[\thebibyears][\wantednumber]%
   \ifx\commalistelement\empty
     \edef\myyear{{\myyear}}%
   \else
     \edef\myyear{{\commalistelement, \myyear}}%
   \fi
   \edef\newcommalistelement{\myyear}%
   \doglobal\replaceincommalist \thebibyears \wantednumber}


%D \macros{preparebibrefprefix}
%D
%D The reference list only writes bare references when the criterium
%D is `all'. Otherwise, a prefix is added to make sure that pdfTeX 
%D does not encounter duplicate named references. On the generation
%D side, this is not a big problem. \type{\preparebibrefprefix}
%D creates a suitable string to prepend if a prefix is needed.

\def\preparebibrefprefix
   {\let\dosetfilterlevel\patcheddosetfilterlevel
    \dosettoclevel\??li{pubs}%
    \let\dosetfilterlevel\normaldosetfilterlevel
    \edef\bibrefprefix{\@@sectiontype\currentlevel\sectionseparator}}%

%D \macros{preparebibreflist}
%D
%D But this optional prefixing is a bit of a problem on the 
%D other side. We would like to do \type{\goto{}[article-full]}
%D but can't do it like that, because the actual label may be 
%D \type{1:2:0:3:4:article-full]} or so. The problem is solved
%D by building a commalist that looks like this:
%D \starttyping
%D \def\bibreflist%
%D    {1:2:0:3:4:article-full,
%D     1:2:0:3:article-full,
%D     1:2:0:article-full,
%D     1:2:article-full,
%D     1:article-full,
%D     article-full}
%D \stoptyping

\def\preparebibreflist#1%
   {\let\bibreflist\empty
    \def\storeitem##1%
      {\ifx\bibreflist\empty 
         \edef\prefix{##1\sectionseparator}%
         \edef\bibreflist{\prefix#1,#1}%
      \else
         \edef\prefix{\prefix##1\sectionseparator}%
         \edef\bibreflist{\prefix#1,\bibreflist}%
      \fi}%
	\expanded{\processseparatedlist[\bibrefprefix][\sectionseparator]}\storeitem }
  
%D \macros{gotobiblink,inbiblink,atbiblink}
%D
%D The final task is looping over that list until a match is found.

\newif\ifbibreffound

\def\gotobiblink#1[#2]%
   {\bgroup
    \preparebibrefprefix
    \preparebibreflist{#2}%
    \global\bibreffoundfalse
    \def\setuplink##1%
          {\ifbibreffound\else
           \doifreferencefoundelse
               {##1}
               {\global\bibreffoundtrue \goto{#1}[##1]}%
               {}\fi}%
    \processcommacommand[\bibreflist]\setuplink
    \ifbibreffound \else \unknownreference{#2}\fi
    \egroup }

\def\atbiblink[#1]%
   {\bgroup
    \preparebibrefprefix
    \preparebibreflist{#1}%
    \global\bibreffoundfalse
    \def\setuplink##1%
          {\ifbibreffound\else
           \doifreferencefoundelse
               {##1}
               {\global\bibreffoundtrue \at[##1]}%
               {}\fi}%
    \processcommacommand[\bibreflist]\setuplink
    \ifbibreffound \else \unknownreference{#1}\fi
    \egroup }

\def\inbiblink[#1]%
   {\bgroup
    \preparebibrefprefix
    \preparebibreflist{#1}%
    \global\bibreffoundfalse
    \def\setuplink##1%
          {\ifbibreffound\else
           \doifreferencefoundelse
               {##1}
               {\global\bibreffoundtrue \in[##1]}%
               {}\fi}%
    \processcommacommand[\bibreflist]\setuplink
    \ifbibreffound \else \unknownreference{#1}\fi
    \egroup }

%D \macros{bibauthoryearref,bibauthoryearsref,bibauthorref,bibyearref}
%D
%D Now that all the hard work has been done, these are simple.
%D \type{\ixbibauthoryearref} stores the data in the macros
%D \type{\currentbibauthor} and \type{\currentbibyear}.

\def\ifbibinteractionelse%
  {\edef\test{\bibalternative\c!interaction}%
   \ifx\test\v!stop
     \expandafter\secondoftwoarguments
    \else
     \expandafter\firstoftwoarguments
    \fi}

\def\bibmaybeinteractive#1#2%
  {\ifbibcitecompress #2\else
   \ifbibinteractionelse{\gotobiblink{#2}[#1]}{#2}\fi }

\def\bibauthoryearref[#1]%
 {\ixbibauthoryear{#1}%
   {\bibmaybeinteractive{#1}{{\currentbibauthor}\bibalternative\c!inbetween
      \bibalternative\v!left{\currentbibyear}\bibalternative\v!right}}
   {\bibalternative\c!pubsep
    \bibmaybeinteractive{#1}{{\currentbibauthor}\bibalternative\c!inbetween
      \bibalternative\v!left  {\currentbibyear}\bibalternative\v!right}}
   {\bibalternative\c!lastpubsep
    \bibmaybeinteractive{#1}{{\currentbibauthor}\bibalternative\c!inbetween
      \bibalternative\v!left {\currentbibyear}\bibalternative\v!right}}}

\def\bibauthoryearsref[#1]%
  {\bibalternative\v!left
   \ixbibauthoryear{#1}
     {\bibmaybeinteractive{#1}{{\currentbibauthor}\bibalternative\c!inbetween{\currentbibyear}}}
     {\bibalternative\c!pubsep    
      \bibmaybeinteractive{#1}{{\currentbibauthor}\bibalternative\c!inbetween{\currentbibyear}}}
     {\bibalternative\c!lastpubsep
      \bibmaybeinteractive{#1}{{\currentbibauthor}\bibalternative\c!inbetween{\currentbibyear}}}%
   \bibalternative\v!right}

\def\bibauthorref[#1]%
  {\bibalternative\v!left
   \ixbibauthoryear{#1}%
    {\bibmaybeinteractive{#1}{{\currentbibauthor}}}
    {\bibalternative\c!pubsep    \bibmaybeinteractive{#1}{{\currentbibauthor}}}
    {\bibalternative\c!lastpubsep\bibmaybeinteractive{#1}{{\currentbibauthor}}}%
   \bibalternative\v!right}

\def\bibyearref[#1]%
  {\bibalternative\v!left
   \ixbibauthoryear{#1}%
      {\bibmaybeinteractive{#1}{{\currentbibyear}}}
      {\bibalternative\c!pubsep    \bibmaybeinteractive{#1}{{\currentbibyear}}}
      {\bibalternative\c!lastpubsep\bibmaybeinteractive{#1}{{\currentbibyear}}}%
   \bibalternative\v!right}

%D ML problems:

%D \macros{bibshortref,bibkeyref,bibpageref,bibtyperef,bibserialref}
%D
%D There is hardly any point in trying to compress these. The only
%D thing that needs to be done is making sure that
%D the separations are inserted correctly. And that is
%D what \type{\refsep} does.

\newif\iffirstref

\def\refsep{\iffirstref\firstreffalse\else\bibalternative\c!pubsep\fi}

\def\bibshortref[#1]%
  {\bibalternative\v!left
   \firstreftrue\processcommalist[#1]\dobibshortref
   \bibalternative\v!right}

\def\dobibshortref#1%
  {\addthisref{#1}\refsep
   \doifbibreferencefoundelse{#1}{\gotobiblink{\getvalue{pbds-#1}}[#1]}
	 {\unknownreference{#1}}}


\def\bibserialref[#1]%
  {\bibalternative\v!left
   \firstreftrue\processcommalist[#1]\dobibserialref
   \bibalternative\v!right}

\def\dobibserialref#1%
  {\addthisref{#1}\refsep
   \doifbibreferencefoundelse{#1}{\gotobiblink{\getvalue{pbdn-#1}}[#1]}
      {\unknownreference{#1}}}

\def\bibkeyref[#1]%
  {\bibalternative\v!left
   \firstreftrue\processcommalist[#1]\dobibkeyref
   \bibalternative\v!right}

\def\dobibkeyref#1%
  {\addthisref{#1}\refsep\gotobiblink{#1}[#1]}

\def\gotoDOI#1#2%
  {\ifbibinteractionelse
     {\useURL[bibfoo#1][http://dx.doi.org/#2]\goto{\url[bibfoo#1]}[url(bibfoo#1)]}
     {\hyphenatedurl{#2}}}

\def\bibdoiref[#1]%
  {\bibalternative\v!left
   \firstreftrue\processcommalist[#1]\dobibdoiref
   \bibalternative\v!right}

\def\dobibdoiref#1%
  {\addthisref{#1}\refsep
   \doifbibreferencefoundelse{#1}{\expanded{\gotoDOI{#1}{\getvalue{pbdo-#1}}}}
        {\unknownreference{#1}}}


\def\biburlref[#1]%
  {\bibalternative\v!left
   \firstreftrue\processcommalist[#1]\dobiburlref
   \bibalternative\v!right}

\def\gotoURL#1#2%
   {\ifbibinteractionelse
       {\useURL[bibfoo#1][#2]\goto{\url[bibfoo#1]}[url(bibfoo#1)]}
       {\hyphenatedurl{#2}}}

\def\dobiburlref#1%
  {\addthisref{#1}\refsep
   \doifbibreferencefoundelse{#1}{\expanded{\gotoURL{#1}{\getvalue{pbdu-#1}}}}
      {\unknownreference{#1}}}

\def\bibtyperef[#1]%
  {\bibalternative\v!left
   \firstreftrue\processcommalist[#1]\dobibtyperef
   \bibalternative\v!right}

\def\dobibtyperef#1%
  {\addthisref{#1}\refsep
   \doifbibreferencefoundelse{#1}{\gotobiblink{\getvalue{pbdt-#1}}[#1]}
       {\unknownreference{#1}}}

\def\bibpageref[#1]%
  {\bibalternative\v!left
   \firstreftrue\processcommalist[#1]\dobibpageref
   \bibalternative\v!right}

\def\dobibpageref#1%
  {\addthisref{#1}\refsep
   \ifbibinteractionelse{\atbiblink[#1]}{{\referencingfalse\at[#1]}}}

\def\bibdataref[#1]%
  {\bibalternative\v!left
   \firstreftrue\processcommalist[#1]\dobibdata
   \bibalternative\v!right}

\def\dobibdata#1%
  {\addthisref{#1}\refsep
   \doifbibreferencefoundelse{#1}{\dotypesetapublication{#1}}
       {\unknownreference{#1}}}

\let\bibnoneref\nocite

%D \macros{bibnumref}
%D
%D It makes sense to try and compress the argument list of
%D \type{\bibnumref}. There are two things involved: the actual
%D compression, and a sort routine. The idea is to store the
%D found values in a new commalist called \type{\therefs}.

%D But that is not too straight-forward, because \type{\in} is
%D not expandable,
%D so that the macro \type{\expandrefs} is needed.

\def\expandrefs#1%
   {\bgroup
    \preparebibrefprefix
    \preparebibreflist{#1}%
    \global\bibreffoundfalse
    \def\setuplink##1%
          {\ifbibreffound\else
           \doifreferencefoundelse
               {##1}
               {\global\bibreffoundtrue 
                \@EA\doglobal\@EA\addtocommalist\@EA{\reftypet}\therefs }%
               {}\fi}%
    \processcommacommand[\bibreflist]\setuplink
    \ifbibreffound \else \showmessage\m!bib{5}{#1 unknown}%
       \doglobal\addtocommalist{0}\therefs\fi
    \egroup }


%D But at least the actual sorting code is simple (note that sorting
%D a list with exactly one entry fails to return anything, which
%D is why the \type{\ifx} is needed).

\def\bibnumref[#1]%
  {\bibalternative\v!left
   \penalty\!!tenthousand
   \processcommalist[#1]\addthisref
   \firstreftrue
   \ifbibcitecompress
     \glet\therefs\empty
     \processcommalist[#1]\expandrefs
     \sortcommacommand[\therefs]\donumericcompare
     \ifx\empty\sortedcommalist\else
       \let\therefs\sortedcommalist
     \fi
     \compresscommacommandnrs[\therefs]%
     \processcommacommand[\compressedlist]\verysimplebibnumref
   \else
     \processcommalist[#1]\dosimplebibnumref
   \fi
   \bibalternative\v!right}

\def\dosimplebibnumref  #1%
  {\refsep\ifbibinteractionelse
     {\inbiblink[#1]}{{\referencingfalse\inbiblink[#1]}}}

\def\verysimplebibnumref#1{\doverysimplebibnumref#1}

\def\doverysimplebibnumref#1#2%
  {\refsep
   \ifcase#1\relax \unknownreference{#1}\else
     \def\tempa{#2}\ifx\empty\tempa#1\else#1\bibalternative\c!inbetween#2\fi
   \fi}

%D By request from Sanjoy. This makes it easier to implement
%D \type{\citeasnoun}.

\def\bibauthornumref[#1]%
  {\getcommalistsize[#1]
   \global\bibitemcounter\commalistsize
   \firstreftrue
   \processcommalist[#1]\dobibauthornumref }

\def\dobibauthornumref#1%
  {\addthisref{#1}\refsep
   \doifbibreferencefoundelse{#1}
      {\getvalue{pbda-#1}%
       \bibalternative\c!inbetween
       \bibalternative\v!left
       \ifbibinteractionelse{\inbiblink[#1]}
            {{\referencingfalse\inbiblink[#1]}}%
       \bibalternative\v!right}
      {\unknownreference{#1}}}

%D And some defaults are loaded from bibl-apa:

\setuppublications
  [\v!month\v!conversion=,\c!alternative=apa]

\preloadbiblist

\protect \endinput